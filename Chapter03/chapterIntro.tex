\chapter{A description of Kent's Theory of Quantum Physics\label{kentchapterdesc}}

In this chapter, I will describe Kent's theory of quantum physics, but before doing this, it is worth briefly reminding ourselves of the problem in quantum physics that we wish to address. 

In chapter \ref{BellChapter}, we discussed the EPR-Bohm paradox and the problem of trying to account for the mysterious correlation of spin measurements on two spatially separated particles. We saw that the Copenhagen interpretation of quantum physics is unable to satisfactorily resolve this paradox because it posits that there is an instantaneous collapse of the state upon measurement, but the idea of an instantaneous collapse does not make sense in special relativity where there is no such thing as an instant of time. 

We also saw that any local hidden variables theory, that is, any theory in which parameter independence (PI) and outcome independence (OI) hold, will imply Bell's inequality, and this inequality is known to be experimentally violated. Shimony proposed that quantum theory and special relatively could peacefully coexist if we accepted PI and rejected OI, but as Butterfield\footnote{See \cite{Butterfield}} points out, Shimony's proposal  does not address the problem of what an outcome is despite his proposal relying on there being outcomes. 

As discussed in chapter \ref{measprobchap}, the problem of outcomes remains an unresolved part of the measurement problem, and the many-worlds interpretation that attempts to sidestep the problem of outcomes is deeply unsatisfactory. But as well as critiquing Shimony's proposal, Butterfield thinks that a suitable interpretation of quantum physics could provide what is missing in Shimony's account. It is for this reason that Butterfield highlights Kent's theory of quantum physics.

In this chapter, we will just focus on describing Kent's theory, and we will postpone our evaluation of whether Kent's theory can adequately resolve the EPR-Bohm paradox until chapter \ref{KentEval}. In describing Kent's theory of quantum physics, we will focus on the ideas Kent presents in his 2014 paper.\footnote{\cite{Kent2014}.} 

Kent's theory of quantum physics has some similarities in common with the pilot wave interpretation. Firstly,  there is no quantum state collapse in Kent's theory. Secondly, some additional values beyond standard quantum theory (i.e. in addition to the quantum state) are included in Kent's theory. And thirdly, Kent's theory is a one-world interpretation of quantum physics. I'll consider these three features of Kent's theory in some detail as I describe his theory. I'll then present an account of his toy model that provides a simple example of how the ideas of his theory fit together. 
\chapter{Kent's Interpretation of Quantum Physics\label{kentchapter}}
In this chapter, I will present and evaluate Kent's interpretation of quantum physics, but before beginning this presentation, it is worth briefly reminding ourselves of the problem in quantum physics that we wish to address. 

In chapter \ref{BellChapter}, we discussed the EPR-Bohm paradox and the problem of trying to account for the mysterious correlation of spin measurements on two spatially separated particles. I argued that the Copenhagen interpretation of quantum physics is unable to satisfactorily resolve this paradox because it posits that there is an instantaneous collapse of the state upon measurement which is incompatible with special relativity. We also saw that any local hidden variables theory, that is, any theory in which parameter independence (PI) and outcome independence (OI) hold, will imply Bell's inequality, and this inequality is known to be experimentally violated. Shimony proposed that quantum theory and special relatively could peacefully coexist if we accepted PI and rejected OI, but as Butterfield\footnote{See \cite{Butterfield}} points out, Shimony's proposal  does not address the problem of what an outcome is despite his proposal relying on there being outcomes. 

As discussed in chapter \ref{measprobchap}, the problem of outcomes remains an unresolved part of the measurement problem, and the many worlds interpretation that attempts to sidestep the problem of outcomes is deeply unsatisfactory. But as well as critiquing Shimony's proposal, Butterfield thinks that a suitable interpretation of quantum physics could provide what is missing in Shimony's account. It is for this reason that Butterfield highlights Kent's interpretation of quantum physics.

In this chapter, we will therefore discuss whether Kent's interpretation can live up to the hopes that Butterfield has for it. We thus aim to show that the predictions of Kent's interpretation do not contradict the  predictions of quantum theory that have been experimentally validated. We also aim to show that Kent's interpretation possesses the symmetries (known as Lorentz invariance) that belong to a theory that is consistent with Einstein's theory of special relativity. And we will also consider Kent's interpretation in the light of decoherence theory and show how Kent succeeds in giving us a convincing account of what an outcome is. 

Butterfield has also emphasized the need to understand Kent's interpretation in the light of an important theorem proved in recent years, the so-called Collbeck-Renner Theorem\footnote{See \cite{LeegwaterGijs2016Aitf}, \cite{ColbeckRoger2011Neoq}, \cite{ColbeckRoger2012Tcoq}, \cite{LandsmanK2015OtCt}, and \cite{Landsman}.} which says that if a theory satisfies PI together with what is called a `no conspiracy' criterion, then this theory is reducible to standard quantum theory without any hidden variables. This suggests that if Kent's interpretation satisfactorily addresses the problem of outcomes in a way that  ensures peaceful coexistence between special relativity and quantum physics, then it can't be a hidden-variables theory. We will thus explain how Kent's model can be an interpretation of quantum physics without it being a hidden-variables theory.
\chapter{A Description of Kent's Theory of Quantum Physics\label{kentchapterdesc}}

In this chapter, I will describe Kent's theory of quantum physics, but before doing this, it is worth briefly reminding ourselves of the problem in quantum physics that we wish to address. 

In chapter \ref{BellChapter}, we discussed the EPR-Bohm paradox and the problem of trying to account for the mysterious correlation of spin measurements on two spatially separated particles. We saw that the Copenhagen interpretation of quantum physics is unable to satisfactorily resolve this paradox because it posits that there is an instantaneous collapse of the state upon measurement, but the idea of an instantaneous collapse does not make sense in special relativity where there is no such thing as an instant of time.\footnote{Since instantaneous spatially extended collapse makes no sense in special relativity, some philosophers have entertained the possibility of there being some additional spacetime structure so that instantaneous collapse does make sense. This involves defining a foliation, that is a partition of spacetime into a continuous series of three-dimensional hypersurfaces, where each hypersurface defines what it means to be an instant. The state of the universe at any one instant is a state of one of these hypersurfaces. Sometimes the state will change from one hypersurface to another via Schr\"{o}dinger evolution, but at other times, when there are state collapses, there will be a discontinuity in the transition between the hypersurfaces. Maudlin discusses some of the issues associated with adding a foliated structure to spacetime -- see \cite{Maudlin3}.} 

We also saw that any hidden variables theory that makes the same predictions as quantum mechanics when averaged over the hidden variables (and hence violates Bell's inequality) cannot satisfy both  parameter independence (PI) and outcome independence (OI). Shimony proposed that quantum theory and special relativity could peacefully coexist if we accepted PI and rejected OI, but as Butterfield\footnote{See \cite{Butterfield}} points out, Shimony's proposal  does not address the problem of what an outcome is despite his proposal relying on there being outcomes. 

As discussed in chapter \ref{measprobchap}, the problem of outcomes remains an unresolved part of the measurement problem, and the many-worlds interpretation that attempts to sidestep the problem of outcomes is deeply unsatisfactory. But as well as critiquing Shimony's proposal, Butterfield thinks that a suitable interpretation of quantum physics could provide what is missing in Shimony's account. It is for this reason that Butterfield highlights Kent's theory of quantum physics.

In this chapter, we will just focus on describing Kent's theory, and we will postpone our evaluation of whether Kent's theory can adequately resolve the EPR-Bohm paradox until chapter \ref{KentEval}. In describing Kent's theory of quantum physics, we will focus on the ideas Kent presents in his 2014 paper.\footnote{\cite{Kent2014}.} 
 
Kent's theory of quantum physics has some similarities in common with the Bohmian interpretation. Firstly,  there is no quantum state collapse in Kent's theory. Secondly, some additional variables beyond standard quantum theory (i.e. in addition to the quantum state) are included in Kent's theory. And thirdly, Kent's theory is a one-world interpretation of quantum physics. I'll consider these three features of Kent's theory in some detail as I describe his theory. I'll also explain how to perform the expectation calculations that are central to Kent's theory. And finally, I'll present an account of Kent's toy model that provides a simple example of how the ideas of his theory fit together. 

I don't claim any significant degree of originality in this chapter, but I do provide some details regarding my own understanding of Kent's interpretation. For example, in section \ref{AdditionalVariablesDetails}, I discuss the additional variables of Kent's interpretation in far more detail that Kent does, and in section \ref{kentcalculation}, I state explicitly the equations (\ref{rProb}), (\ref{qrnProb}), and (\ref{beable1}) that are used in the calculation of Kent's beables.

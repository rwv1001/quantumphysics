\chapter{Evaluating Kent's Theory}
In this section I will consider how well Kent's theory allows for peaceful coexistence between standard quantum theory and special relativity. I'll begin by showing that Kent's theory is consistent with standard quantum theory. I'll then show that Kent's theory is Lorentz invariant. The problems of outcomes will be addressed in a section that considers how Kent's theory ties in with decoherence theory and d'Espagnat's objection about improper mixtures. I'll then consider PI in Kent's theory and the consistency of Kent's theory with Colbeck and Renner's theorem. Finally, I will raise some issues about the nature of Kent's beables.

\section{Consistency of Kent's Theory with Standard Quantum Physics\textsuperscript{*}\label{kentinterpretationconsistency}}
\renewcommand{\thefootnote}{\fnsymbol{footnote}}
\footnotetext[1]{As mentioned in the introduction on page \pageref{asteriskmeaning}, sections marked with an asterisk may be challenging to readers who do not have a mathematics or physics background.}\label{LorentzInvarianceSection}\renewcommand*{\thefootnote}{\arabic{footnote}}If we are to take Kent's theory seriously, it should be just as good at making predictions as standard quantum theory, and it had better not contradict empirical observations. Over the last century, standard quantum theory has been firmly established scientifically, and so far, it has not been contradicted by any experimental observations. Standard quantum theory allows us form a quantum state description of a physical system based on how the system was set up in an experimental environment, and then Schr\"{o}dinger's equation can be used to evolve this state forwards in time, and finally, we can calculate expectation values for various physical quantities belonging to this physical system, and these agree with the average values measured on the system when the experiment is performed many times. In other words, standard quantum theory is empirically adequate in its domain of applicability. Thus, if we can show that Kent's theory is just as good at making predictions as standard quantum theory, then it too will be empirically adequate to the same degree. This doesn't necessarily mean that Kent's theory will make exactly the same predictions as standard quantum physics, for the additional information that Kent's theory requires beyond standard quantum theory may alter these predictions. Indeed, if this additional information made absolutely no difference to the predictions of standard quantum theory, then it seem rather redundant. But we should nevertheless be able to derive the predictions of standard quantum theory from Kent's theory by averaging over the unknown variables that describe the additional information in Kent's theory. 

In order to show that Kent's theory is just as good as standard quantum theory, we will need to express $\ev*{T^{\mu\nu}(y)}_{\tau_S}$ in terms of the observable $\hat{T}^{\mu\nu}(y)$ and the initial state $\ket{\Psi_0}$. To find such an expression, we would ideally like to find a spacelike hypersurface $S'$ that contains both $S^1(y)$ and $y$. Then we could consider how the observables $\hat{T}_S(x)$ and $\hat{T}^{\mu\nu}(y)$ act on the state $\ket{\Psi'}=U_{S'S_0}\ket{\Psi_0}$ and use this action to determine the probabilities $P(q(\tau),r)$ and $P(r)$ needed to define the conditional probability $P(q(\kappa)|r)$ as defined in (\ref{conditionalprobability}). However, since by definition, a spacelike hypersurface must be continuous with any two locations on it being spacelike-separated, it is going to be impossible to find a spacelike hypersurface $S'$ with the desired properties of containing both $S^1(y)$ and $y$. Nevertheless, what we can do is find a sequence of spacelike hypersurfaces $S_n(y)$ such that\label{siydef} $S_n(y)\subset S_{n'}(y)$ for $n<n'$, and such that for any $x\in S^1(y)$, there exists $n$ such that $x\in S_n(y)$. An example of one such $S_n(y)$ is shown in figure \ref{S3}. When there is no ambiguity, we will drop the $y$ and write $S_n$ instead of $S_n(y)$. 

 \begin{figure}[ht!]
\captionsetup{justification=justified}
\centering

\tikzmath{
\a= 1;  
\e = 0.1;
\lam=0.9;
\h=-1;
\hae=(3*\a*\a+6*\a*\e+7*\e*\e-3*\a*sqrt(\a*\a+4*\e*\e)-4*\e*sqrt(\a*\a+4\e*\e))/(4*\a+4*\e-2*sqrt(\a*\a+4*\e*\e));
\hae=0.205678;
\circsize=1.2;
\md = (\a+\h)/2;
\lrange = 4;
\rrange=2;
\fictlabel=(\rrange-\lrange)/2;
\ss=(-\lrange-\a)/2;
\sss=\a+(\rrange-\a)/2;
\tlen=0.75;
\labelpos=(-\lrange-\a)/2;
} 

\begin{tikzpicture}[thick, scale=2]

\def\dotsize{0.7}

\definecolor{tempcolor}{RGB}{0,151,76}
\draw[<->] (-\lrange, \h) node[left] {$S_0$} -- (\rrange, \h) node[right] {$S_0$};
\filldraw (0,0) circle (\dotsize pt) node [below right] {$y$} ;
              
\draw[->] (\rrange,\md-\tlen/2) --  (\rrange,\md+\tlen/2) node[midway,right]{time}; 

\draw[->,blue, thick] [domain=-\a/2:-\lrange, samples=150]   plot (\x, {\a-\e/2-1/2*(sqrt(\lam*\lam*pow(\x+\hae+\a,2)+\e*\e)-\e+\lam*(\x+\hae+\a))}) node[left, black]{$S_n(y)$}  ;
\draw[blue, thick] [domain=-\a/2:\a/2, samples=150] plot (\x, {sqrt(\lam*\lam*\x*\x+\e*\e)-\e})   node[right, black]{$S_n(y)$};
\draw[->,blue, thick] [domain=\a/2:\rrange, samples=150]   plot (\x, {\a-\e/2-1/2*(sqrt(\lam*\lam*pow(\x-\hae-\a,2)+\e*\e)-\e-\lam*(\x-\hae-\a))}) node[right, black]{$S_n(y)$}  ;
\draw[gray] (-\lrange, \a)  -- (\rrange, \a)  {};
\draw[gray, dashed] (-\a, \a) -- (0,0) {};
\draw[gray, dashed](0,0) -- (\a, \a) {};
\draw[gray](\a, \a) --  (\rrange, \a)  ;         
\coordinate (B) at (\a,\a);
\node at (B)[red,circle,fill,inner sep=\circsize pt]{};
\coordinate (A) at (-\a,\a);
\node at (A)[red,circle,fill,inner sep=\circsize pt]{};
\coordinate (C) at (0,0);
\node at (C)[black,circle,fill,inner sep=\circsize pt]{};

\coordinate[label = above:$S$]  (D) at (0,\a);

\coordinate[label = above:$S$]  (D) at (\sss,\a);

\filldraw (\ss,\a) circle (\dotsize pt) node [above] {$x\in S_n(y)\cap S$} ;


 
\node (start) at (\labelpos,\h) [below] {Initial State $\ket{\Psi_0}$};
\node (evolution) at (\labelpos,\md+0.05) [below] {Unitary Evolution $\ket{\Psi_n}=U_{S_nS_0}\ket{\Psi_0}$};
\node (final) at (\labelpos,\a) [below] {Unitary Evolution $U_{SS_0}\ket{\Psi_0}$};
%\node at (-\ss+0.17,\mn-0.18){$-a_0$};
\draw [->, shorten <= 5pt] (start) [above] -- (evolution); 
\draw [->] (evolution) -- (final); 
\end{tikzpicture}

\vspace*{2px}
\caption{$S_n\myeq S_n(y)$ is a spacelike hypersurface containing $y$ and all of $S^1(y)$ in the limit as $n\rightarrow\infty$.   }
\label{S3}
\end{figure}
 
 Now if $r_n$ is the statement that  $T_S(x)=\tau_S(x)$ for all $x\in S_n(y)\cap S$, then so long as $\tau_S(x)$ is chosen by the Born Rule so that $P(r)\neq 0$, it will follow that
 \begin{equation}
	P(q(\tau)|r)=\lim_{n\rightarrow\infty}P(q(\tau)|r_n).
 \end{equation}
 Therefore, from the definition of the beable $\ev*{T^{\mu\nu}(y)}_{\tau_S}$ given on page \pageref{Kentbeable} together with the definition of conditional expectation given in equation (\ref{conditionalexpectation}), we have 
\begin{equation}\label{beable1}
\ev*{T^{\mu\nu}(y)}_{\tau_S}=\lim_{n\rightarrow\infty}\sum_\tau\frac{P(q(\tau),r_n)\tau}{P(r_n)}.
\end{equation} 
 To calculate $P(q(\tau)|r_n)$, we note that since $S_n$ is a spacelike hypersurface,   there will exist a unitary operator $U_{S_nS_0}$ which maps the Hilbert space of states $H_{S_0}$ describing $S_0$ to the Hilbert space of states $H_{S_n}$\label{HSidef} describing $S_n$ in accord with how the states of $H_{S_0}$  evolve over time. Now let $H_{S_n,\tau_S}\subset H_{S_n}$ be the subspace of states $\ket{\xi}$ for which  $\hat{T}_S(x)\ket{\xi}=\tau_S(x)\ket{\xi}$  for all $x\in S_n\cap S$, and  let $\{\ket{\xi_1},\ket{\xi_2},\ldots\}$ be an orthonormal basis of $H_{S_n,\tau_S}$. Given that the initial state of the world is $\ket{\Psi_0}$, the probability $P(r_n)$ of ``measuring'' the value of $T_S(x)$ on $S_n\cap S$ to be $\tau_S(x)$ will be 
\begin{equation}\label{Pri}
P(r_n)=\sum_j \abs{ \ip{\xi_j}{\Psi_n}}^2,
\end{equation}
where $\ket{\Psi_n}=U_{S_nS_0}\ket{\Psi_0}$, and this probability will be independent of the particular orthonormal basis  $\{\ket{\xi_j}:j\}$ of $H_{S_n,\tau_S}$.\footnote{To see why this is, we note that we can extend the orthonormal set $\{\ket{\xi_1},\ket{\xi_2},\ldots\}$ to an orthonormal basis  $\{\ket{\xi_1},\ket{\xi_2},\ldots\}\cup\{\ket{\zeta_1},\ket{\zeta_2},\ldots\}$ of $H_{S_n}$ which consists entirely of $\hat{T}_S$-eigenstates. We can think of each of the states of this orthonormal basis as the possible measurement outcomes when making the notional measurement of $T_S(x)$ on $S_n\cap S$. By the Born Rule, it therefore follows that $P(r_n)=\sum_j \abs{ \ip{\xi_j}{\Psi_n}}^2$. But to see that this probability is independent of the particular basis, we can uniquely write $\ket{\Psi_n}$ as a sum $\ket{\Psi_n}=\ket{\xi}+\ket{\zeta}$ where $\ket{\xi}$ belongs to the span of $\{\ket{\xi_j}:j\}$ and $\ket{\zeta}$ belongs to the span of $\{\ket{\zeta_j}:j\}$.  Then since $\ket{\xi}=\sum_j\ip{\xi_j}{\Psi_n}\ket{\xi_j}$, it follows that $$\ip{\xi}{\xi}=\sum_j\abs{ \ip{\xi_j}{\Psi_n}}^2=P(r_n).$$ Therefore, since  $\ip{\xi}{\xi}$ is independent of the particular basis chosen of $H_{S_n,\tau_S}$, then so is $P(r_n)$.  \label{priproof} } If we define 
\begin{equation}\label{tauprojection}
\pi_n=\sum_j\dyad{\xi_j},
\end{equation}
then it is easy to see that
\begin{equation}\label{Prn}
	P(r_n)=\ev*{\pi_n}{\Psi_n}.
\end{equation}
We also see that $\pi_n$ is Hermitian (i.e. has real eigenvalues) and that $\pi_n \pi_n = \pi_n$. Any Hermitian operator $\pi$ with $\pi^2=\pi$ is called a \textbf{projection}\index{projection}. We thus see that $\pi_n$ is a projection.

Turning to the calculation of $P(q(\tau), r_n)$, note that for the Tomonaga-Schwinger formulation of relativistic quantum physics, the operators $\hat{T}_S(x)$ and $\hat{T}^{\mu\nu}(y)$ for fixed $\mu,\nu$ commute when $x$ and $y$ are spacelike-separated. It therefore follows that we can express any state of $H_{S_n}$ as a superposition of simultaneous eigenstates of $\hat{T}^{\mu\nu}(y)$ and $\hat{T}_S(x)$ for $x\in S_n\cap S$.\footnote{We make the same approximation as mentioned on page \pageref{simultaneous} in footnote \ref{glosssim}.}  For a particular choice of $\mu,\nu$, we can then form an orthonormal basis $\{\ket{\eta_j}:j\}$ of $H_{S_n}$ consisting of simultaneous $\hat{T}^{\mu\nu}(y)$, $\hat{T}_S(x)$-eigenstates so that $\hat{T}^{\mu\nu}(y)\ket{\eta_j}=\tau^{(j)}\ket{\eta_j}$ and $\hat{T}_{S}(x)\ket{\eta_j}=\tau_S^{(j)}(x)\ket{\eta_j}$ for $x\in S_n\cap S$, where $\tau^{(j)}$ and $\tau_S^{(j)}(x)$ are the corresponding eigenstates. If we define  $\pi_{n,\tau}=\sum_j\dyad{\chi_{j,\tau}}$ where $\{\ket{\chi_{j,\tau}}:j\}$ is the subset of $\{\ket{\eta_j}:j\}$ such that $\hat{T}^{\mu\nu}(y)\ket{\chi_{j,\tau}}=\tau\ket{\chi_{j,\tau}}$ and $\hat{T}_S(x)\ket{\chi_{j,\tau}}=\tau_S(x)\ket{\chi_{j,\tau}}$ for all $x\in S_n\cap S$, then 
\begin{equation}\label{pqtauri}
P(q(\tau), r_n)=\sum_j \abs{ \ip{\chi_{j,\tau}}{\Psi_n}}^2=\ev*{\pi_{n,\tau}}{\Psi_n}.\protect\footnotemark
\end{equation}
\footnotetext{The proof of this is very similar to the proof given in footnote \ref{priproof}.}But if we define $\pi_\tau=\sum_j\dyad{\eta_{j,\tau}}$ where $\{\ket{\eta_{j,\tau}}:j\}$ is the subset of  $\{\ket{\eta_j}:j\}$ with $\hat{T}^{\mu\nu}(y)\ket{\eta_{j,\tau}}=\tau\ket{\eta_{j,\tau}}$, then we also have  $\pi_{n,\tau}=\pi_n\pi_\tau$.\footnote{To see why this is, 
we first show that $\pi_n=\sum_j\dyad{h_{n,j}}$ where $\{\ket{h_{n,j}}:j\}$ is the subset of $\{\ket{\eta_j}:j\}$  for which $\ket{h_{n,j}}\in H_{S_n,\tau_S}$. 
Note that $\pi_n\ket{h_{n,j}}=\ket{h_{n,j}}$    since $\{\ket{\xi_j}:j\}$ is a basis for $H_{S_n,\tau_S}$ and $\ket{h_{n,j}}\in H_{S_n,\tau_S}$. 
Therefore, $\pi_n\pi_{n,h}=\pi_{n,h}$  where  $\pi_{n,h}=\sum_j\dyad{h_{n,j}}$. 
But  $\pi_{n,h}\ket{\xi_j}=\ket{\xi_j}$ since $\{\ket{h_{n,j}}:j\}$ is a basis for $H_{S_n,\tau_S}$ and $\ket{\xi_j}\in H_{S_n,\tau_S}$. 
Therefore, $\pi_{n,h}\pi_n=\pi_n.$ But $\pi_{n,h}\pi_n= \pi_n\pi_{n,h}$ since $\pi_n$ and $\pi_{n,h}$ are Hermitian. Hence, $\pi_n= \pi_{n,h}$. Now the summands of $\pi_n\pi_\tau$ are only going to consist of those $\dyad{\eta_j}$ for which $\hat{T}^{\mu\nu}(y)\ket{\eta_j}=\tau\ket{\eta_j}$ and for which $\hat{T}_S(x)\ket{\eta_j}=\tau_S(x)\ket{\eta_j}$ for all $x\in S_n\cap S$, and these are just the $\dyad*{\chi_{j,\tau}}$ which are the summands of  $\pi_{n,\tau}$. Hence,  $\pi_n\pi_\tau=\pi_{n,\tau}.$} 
  Hence,
\begin{equation}\label{pqtauri2}
P(q(\tau), r_n)=\ev*{\pi_n\pi_\tau}{\Psi_n}.
\end{equation}
But clearly $\hat{T}^{\mu\nu}(y)=\sum_\tau \tau \pi_\tau.$ Therefore, combining (\ref{beable1}), (\ref{pqtauri}), and (\ref{pqtauri2}), we have 
\begin{equation}\label{kentconsistency0}
\ev*{T^{\mu\nu}(y)}_{\tau_S}=\lim_{n\rightarrow\infty}\frac{\sum_\tau \ev*{\pi_n\pi_\tau}{\Psi_n}\tau}{\ev*{\pi_n}{\Psi_n}}=\lim_{n\rightarrow\infty}\frac{\ev*{\pi_n\hat{T}^{\mu\nu}(y)}{\Psi_n}}{\ev*{\pi_n}{\Psi_n}}.
\end{equation}
We are now in a position to show that Kent's theory is consistent with standard quantum theory. First let us consider what we need to show. 

In the pilot wave interpretation, its consistency with standard quantum theory requires that if one averages the expectation values of an observable over the hidden variables (i.e. the positions and the momenta of all the particles) then one obtains the expectation value of the observable given by standard quantum theory as indicated in equation (\ref{bohmconsistency}). 

Now in Kent's theory, the hidden variables on which his beables $\ev*{T^{\mu\nu}(y)}_{\tau_S}$ depend are the values $\tau_S(x)$ of $T_S(x)$ for $x\in S^1(y)\cap S$. The operator $\pi_n$ in equation (\ref{kentconsistency0}) in the limit as $n\rightarrow\infty$ encapsulates this hidden information. To remind ourselves of $\pi_n$'s dependency on $\tau_S$ restricted to $S_n\cap S$, we will now write $\pi_n(\tau_{S_n\cap S})$ for $\pi_n$ where $\tau_{S_n\cap S}$ is the function $\tau_S$ restricted to $S_n\cap S$. Likewise, we will write  $r_n(\tau_{S_n\cap S})$ for $r_n$, the statement that $T_S(x)=\tau_S(x)$ for all $x\in S_n(y)\cap S$. If we let $j$ index all possible functions $\tau^{(j)}_{S_n\cap S}$ taking real values on $S_n\cap S$, then the analogue of (\ref{bohmconsistency}) requires us to show that 
\begin{equation}\label{kentconsistency}
\ev*{\hat{T}^{\mu\nu}(y)}=\lim_{n\rightarrow\infty}\sum_{j}P\big(r_n(\tau^{(j)}_{S_n\cap S})\big)\ev*{T^{\mu\nu}(y)}_{\tau^{(j)}_{S_n\cap S}}
\end{equation}
for all $y$ lying between $S_0$ and $S$, where the left-hand side of (\ref{kentconsistency}) is just the expectation value of $\hat{T}^{\mu\nu}(y)$ predicted by standard quantum theory. Equation (\ref{kentconsistency}) is sufficient to establish consistency with standard quantum theory because ultimately, all observables are going to be reducible to expressions dependent on $\hat{T}^{\mu\nu}(y)$, since once we know what to expect for $\hat{T}^{\mu\nu}(y)$, we will know what to expect for the energy and momentum densities for all measuring apparatus readouts etc. and hence what to expect for all measurement outcomes. But from (\ref{Prn}) and (\ref{kentconsistency0}), we have 
\begin{equation}\label{kentconsistency1}
\lim_{n\rightarrow\infty}\sum_jP\big(r_n(\tau^{(j)}_{S_n\cap S})\big)\ev*{T^{\mu\nu}(y)}_{\tau^{(j)}_{S_n\cap S}}=\lim_{n\rightarrow\infty}\sum_{j}{\ev*{\pi_{n}(\tau^{(j)}_{S_n\cap S})\hat{T}^{\mu\nu}(y)}{\Psi_n}}
\end{equation}
Since there is an orthonormal basis $\{\ket{\eta_j}:j\}$ of $H_{S_n}$ consisting of simultaneous $\hat{T}_S(x)$-eigenstates so that $\hat{T}_S(x)\ket{\eta_j}=\tau^{(j)}_{S_n\cap S}(x)\ket{\eta_j}$ for all $x\in S_n\cap S$, it follows that $\sum_j \pi_{n}(\tau^{(j)}_{S_n\cap S})=I$. Therefore, equation (\ref{kentconsistency}) follows from (\ref{kentconsistency1}) which is what we were aiming to show for standard quantum consistency to hold.




\section{Kent's Theory and Lorentz Invariance\label{LorentzInvariance}\textsuperscript{*}}
In order to explain what it means for Kent's theory to be Lorentz invariant, we first need to explain how spacetime coordinates look to different observers. 

A spacetime location is represented by a four-tuple $(x^0, x^1, x^2, x^3)$ where $(x^i)_{n=1}^3$ are spatial coordinates, and where $x^0=ct$ with $c$ being equal to the speed of light and $t$ being the time. If $(1,0,0,0)$ corresponds to the spacetime location $\hat{e}_0$, and $(0,1,0,0)$ corresponds to the spacetime location $\hat{e}_1$, etc., then we can express any other spacetime location  as a sum $\sum_{\mu=0}^3x^\mu\hat{e}_\mu$. We will use the so-called \textbf{Einstein summation convention}\index{Einstein summation convention}\label{Einsteinsum} of dropping the summation sign and implicitly assuming that there is a summation whenever an upper index and a lower index are the same so that we can write $x^\mu\hat{e}_\mu$ instead of $\sum_{\mu=0}^3x^\mu\hat{e}_\mu$.  

Now suppose an observer $O$ expresses spacetime locations in terms of $\{\hat{e}_\mu:\mu=0,\ldots,3\}$ and hence uses the coordinates $(x^0, x^1, x^2, x^3)$ to describe various spacetime locations. For another observer $O'$, it may be more natural to express spacetime locations in terms of a different set of spacetime locations $\{\hat{e}'_\mu:\mu=0,\ldots,3\}$ so that the location described by $O$ as $(x^0, x^1, x^2, x^3)$ would be described by $O'$ as $({x'}^0, {x'}^1, {x'}^2, {x'}^3)$ where ${x'}^\mu{\hat{e}'}_\mu=x^\mu\hat{e}_\mu$.  For instance if $O$ and $O'$ are moving with respect to each other, they may both want to use coordinates in which their own spatial coordinates are fixed and in which the spatial coordinates of the other observer are changing. As another example, figure \ref{rotfigure} shows how the $(x^1, x^2)$-coordinates transform under a spatial rotation. 


\begin{figure}[ht!]
\captionsetup{justification=justified}
\centering
\tikzmath{
\textscale = 0.7;
\picscale = 0.58;
\circsize=3.5;
\px=3;
\py=3.5;
\pr=sqrt(\px*\px+\py*\py);
\th=33;
\thd=atan(\py/\px)-\th;
\pyd=\pr*sin(\thd);
\pxd=\pr*cos(\thd);
\rrange =7; 
\labelx= \rrange/2;
\labely=-0.3;
\vr=3;
\vth=75;
\vx=\vr*cos(\vth);
\vy=\vr*sin(\vth);
\vxt=\vx+\px;
\vyt=\vy+\py;
\vrt=sqrt(\vxt*\vxt+\vyt*\vyt);
\vthd=atan(\vyt/\vxt)-\th;
\vxtd=\vrt*cos(\vthd);
\vytd=\vrt*sin(\vthd);
\vxd=\vr*cos(\vth-\th);
\vyd=\vr*sin(\vth-\th);
} 
\begin{tikzpicture}[scale=\picscale] 
    \draw[-latex] (0,0) -- (\rrange,0)  node[right,scale=\textscale, text width=5em] {$\hat{e}_1$};
    \draw[-latex] (0,0) -- (0,\rrange)  node[right,scale=\textscale, text width=5em] {$\hat{e}_2$};
    \draw[dotted, thick] (\px,0) --  (\px,\py) node[midway,right, scale=\textscale]{${x}^2$}; 
     \draw[dotted, thick] (0,\py) --  (\px,\py)  node[midway,above, scale=\textscale]{${x}^1$};  

    \begin{scope}[rotate=\th,draw=red, text=red]
       \draw[-latex] (0,0) -- (\rrange,0)  node[right,scale=\textscale, text width=5em]  {$\hat{e}'_1$};
   	   \draw[-latex] (0,0) -- (0,\rrange)  node[right,scale=\textscale, text width=5em] {$\hat{e}'_2$};
   	   \draw[dotted, thick] (\pxd,0) --  (\pxd,\pyd) node[midway, above=4,right=1,scale=\textscale]{${x'}^2$}; 
   	   \draw[dotted, thick] (0,\pyd) --  (\pxd,\pyd)  node[midway,above, scale=\textscale]{${x'}^1$};  
    \end{scope}

  \draw [black,fill] (\px,\py) circle [radius=\circsize pt] ;
    
    \coordinate[label = below: (a)]  (D) at (\labelx,\labely); 
\end{tikzpicture}% pic 1
\vspace*{2px}
\caption{Shows how a location (marked as $\bullet$) can be expressed either  in coordinates $(x^1, x^2)$ with respect to the basis $\{\hat{e}_1,\hat{e}_2\}$ or in coordinates $({x'}^1,{x'}^2)$ with respect to the basis $\{\hat{e}'_1,\hat{e}'_2\}$.}\label{rotfigure}
\end{figure}

Now the key fact about all observers is that they must always observe light in a vacuum to have a constant speed $c$. Thus,  for a photon that goes through the spacetime locations $(0,0,0,0)$ and $(x^0, x^1, x^2, x^3)$ in the coordinates of $O$, we must have $(x^0, x^1, x^2, x^3)=(ct,tv^1,tv^2,tv^3)$ where 
$$\sqrt{(v^1)^2 +(v^2)^2+(v^3)^2}=c.$$ But if $(0,0,0,0)$ and $(x^0, x^1, x^2, x^3)$ corresponds to $(0,0,0,0)$ and $({x'}^0, {x'}^1, {x'}^2, {x'}^3)$ respectively in the coordinates of another observer $O'$, then we must also have $({x'}^0, {x'}^1, {x'}^2, {x'}^3)=(ct',t'{v'}^1,t'{v'}^2,t'{v'}^3)$ where 
$$\sqrt{({v'}^1)^2 +({v'}^2)^2+({v'}^3)^2}=c.$$ 
In either case, we must have 
\begin{equation}\label{invariant}
(x^0)^2- (x^1)^2- (x^2)^2 - (x^3)^2=({x'}^0)^2- ({x'}^1)^2- ({x'}^2)^2 - ({x'}^3)^2=0.
\end{equation}
If we define $\eta_{00}=1$, $\eta_{ni}=-1$ for $i=1,2,3$ and $\eta_{\mu\nu}=0$ for $\mu\neq\nu$, then using the Einstein summation convention as well as the convention of lowering indices so that we define $x_\mu\myeq\eta_{\mu\nu}x^\nu$, then (\ref{invariant}) is equivalent to 
$$x_\mu x^\mu={x'}_\mu {x'}^\mu=0.$$ 
Thus, for any coordinate transformation $x\rightarrow x'$ such that  $x_\mu x^\mu={x'}_\mu{x'}^\mu$,  if the speed of light is $c$ in the $x$-coordinates, then the speed of light is also guaranteed to be $c$ in the $x'$-coordinates.  A \textbf{Lorentz transformation}\index{Lorentz transformation} $\Lambda$ is any coordinate transformation of the form ${x'}^\mu=\Lambda\indices{^\mu_\nu}x^\nu$ such that $x_\mu x^\mu={x'}_\mu{x'}^\mu$. Since a Lorentz transformation must satisfy
$$x_\mu x^\mu=\eta_{\mu\rho}\Lambda\indices{^\rho_\sigma}x^\sigma\Lambda\indices{^\mu_\nu}x^\nu$$
for all $x$, it follows that  
\begin{equation}\label{lorentztrans}
\Lambda\indices{^\rho_\mu}\eta_{\rho\sigma}\Lambda\indices{^\sigma_\nu}=\eta_{\mu\nu}.\protect\footnotemark
\end{equation}
\footnotetext{To see why this is, note that if  $x_\mu x^\mu={x'}_\mu{x'}^\mu$ for all $x$, then  for any other spacetime location $y$, we have $(x+y)_\mu (x+y)^\mu={(x'+y')}_\mu{(x'+y')}^\mu$. If we expand this out and cancel $x_\mu x^\mu$ with ${x'}_\mu{x'}^\mu$ and cancel $y_\mu y^\mu$ with ${y'}_\mu{y'}^\mu$, and using the fact that $y_\mu x^\mu=x_\mu y^\mu$, etc. we find that  $x_\mu y^\mu={x'}_\mu{y'}^\mu$ for all $x$ and $y$. Hence,
$$\eta_{\nu\mu}x^\mu y^\nu = x_\mu y^\mu=\eta_{\sigma\rho}\Lambda\indices{^\rho_\mu}\Lambda\indices{^\sigma_\nu}x^\mu y^\nu.$$  
Since we can choose $x$ such that $x^\mu=1$ and $x^\alpha = 0$ for $\alpha\neq\mu$, and can choose $y$ such that $y^\nu=1$ and $y^\beta=0$ for $\beta\neq\nu$. Then we get
$$\eta_{\mu\nu} = \eta_{\sigma\rho}\Lambda\indices{^\rho_\mu}\Lambda\indices{^\sigma_\nu}, $$ and hence the result follows.}Having considered how the coordinates of a spacetime location viewed by one observer relate to the coordinates of the same spacetime location viewed by a different observer, we can now consider how physical quantities viewed by different observers relate to each other. The simplest kind of physical quantity is called a \textbf{scalar}\index{scalar}. A scalar defined at a particular spacetime location has the same value no matter what frame of reference an observer uses. One example of a scalar is an object's \textbf{rest mass}\index{rest mass} which is the mass an object would have if it had no velocity. There is still a transformation rule for scalars since the spacetime location at which the scalar is measured is usually expressed in terms of an observer's coordinate system, and the coordinates of  such a location  will differ for different observers. Thus, if $\phi(x)\myeq\phi(x^0, x^1, x^2, x^3)$ is the value of a scalar defined at the spacetime location $(x^0, x^1, x^2, x^3)$ as described by an observer $O$, then another observer $O'$ using a different set of coordinate $({x'}^0, {x'}^1, {x'}^2, {x'}^3)$ to describe the location $(x^0, x^1, x^2, x^3)$ will describe this same scalar as $\phi'(x')\myeq\phi'({x'}^0, {x'}^1, {x'}^2, {x'}^3)$  where $\phi'(x')=\phi(x)$. Since $\phi'$ is just a function of the four numbers ${x'}^0, {x'}^1, {x'}^2,$ and ${x'}^3$, we can rename these numbers ${x}^0, {x}^1, {x}^2,$ and ${x}^3$, and then 
\begin{equation}\label{lorentzscalar}
\phi'(x)=\phi(\Lambda^{-1}x)
\end{equation} 
where $\Lambda^{-1}$ is the inverse Lorentz transformation that takes the coordinates $x'=({x'}^0, {x'}^1, {x'}^2, {x'}^3)$ of a location to the coordinates $x=(x^0, x^1, x^2, x^3)$ describing that location. Thus, equation (\ref{lorentzscalar}) shows us how a scalar transforms under a Lorentz transformation $\Lambda$. 

Many physical quantities, however, are not scalars and so will look different to different observers. For instance, the energy of an object has a kinetic component that depends on the velocity the object has relative to an observer. However, it turns out that if an observer $O$ considers an object's energy $E$ together with its three components of momentum $p^1, p^2$, and $p^3$ (in the directions $\hat{e}_1$, $\hat{e}_2$, and $\hat{e}_3$ respectively) to form the four-tuple $p\myeq(E/c, p^1, p^2, p^3)$ known as the object's \textbf{four-momentum}\index{four-momentum}, then $p$ transforms in the same way as spacetime coordinates transform between different observers. In other words, a different observer $O'$ whose coordinates are given by ${x'}^\mu=\Lambda\indices{^\mu_\nu}x^\nu$ would observe the object's four-momentum to be ${p'}^\mu=\Lambda\indices{^\mu_\nu}p^\nu$.\footnote{In order for $p$ to transform in this way, we have to redefine what we mean by energy and momentum. In classical mechanics, the momentum of an object is the product of the object's mass and its velocity. In the context of special relativity, however, the four-momentum of an object is defined to be the product of its rest mass $m_0$ and its \textbf{four-velocity}\index{four-velocity} where the four velocity of an object is a four-tuple $(u^0, u^1, u^2, u^3)$ with $u_\mu u^\mu=c^2$ such that the object's velocity (in the classical sense) is the vector $(c \frac{u^1}{u^0}, c \frac{u^2}{u^0}, c\frac{u^3}{u^0})$. The motivation for this definition can be seen by considering an object whose classical velocity is $\vb{v}=(v^1,v^2,v^3)$ that goes through $(0,0,0,0)$. It will have a spacetime trajectory $x(t)=(ct, v^1 t, v^2 t, v^3 t)$. $u$ is just the four-vector proportional to $x(1)$ with $u_\mu u^\mu=c^2$.  We can easily work out the 
four-velocity $u$ of an object whose classical velocity is $\vb{v}$. For we must have $u^i=\frac{v^i u^0}{c}$, for $i=1$ to $3$. Therefore, since $u_\mu u^\mu=c^2$, we must have $(u^0)^2\big(1-\frac{v^2}{c^2}\big)=c^2$ where $v=\sqrt{({v}^1)^2+({v}^2)^2+({v}^3)^2}$. Thus, if we define $\beta={v}/{c}$ and $\gamma=\frac{1}{\sqrt{1-\beta^2}}$, then $u^0=c\gamma$ and $u^i=\gamma v^i$ for $i=1$ to $3$, and hence the four-velocity of the object must be $u=\gamma(c,v^1,v^2,v^3).$ From this, we see that the object's four-momentum will be $\gamma m_0(c,v^1,v^2,v^3).$ If the object's velocity is very small compared to the speed of light, then $\gamma\approx 1+\frac{v^2}{2c^2}$, and hence the object's four-momentum $(E/c, p^1, p^2, p^3)$ will be approximately $(m_0c+\frac{1}{2}m_0{v^2}/c, m_0v^1,m_0v^2,m_0v^3)$. Therefore, $(p^1, p^2, p^3)$ is approximately equal to the classical momentum. However, the energy is now $E=m_0c^2+\frac{1}{2}m_0{v^2}$. Thus, in addition to the kinetic energy term $\frac{1}{2}m_0{v^2}$, there is a rest mass energy $m_0c^2$. If we define the \textbf{relativistic mass}\index{relativistic mass} $m=\gamma m_0$, then we obtain Einstein's famous formula $E=mc^2$.  } More generally, any list of four physical quantities $(\varphi^0, \varphi^1, \varphi^2, \varphi^3)$ that transforms as $\varphi\rightarrow\varphi'$ with  ${\varphi'}^\mu=\Lambda\indices{^\mu_\nu}\varphi^\nu$ is called a \textbf{four-vector}\index{four-vector}.
\begin{figure}[ht!]
	\captionsetup{justification=justified}
	\centering
	\tikzmath{
	\textscale = 0.7;
	\picscale = 0.58;
	\circsize=3.5;
	\px=3;
	\py=3.5;
	\pr=sqrt(\px*\px+\py*\py);
	\th=33;
	\thd=atan(\py/\px)-\th;
	\pyd=\pr*sin(\thd);
	\pxd=\pr*cos(\thd);
	\rrange =7; 
	\labelx= \rrange/2;
	\labely=-0.3;
	\vr=3;
	\vth=75;
	\vx=\vr*cos(\vth);
	\vy=\vr*sin(\vth);
	\vxt=\vx+\px;
	\vyt=\vy+\py;
	\vrt=sqrt(\vxt*\vxt+\vyt*\vyt);
	\vthd=atan(\vyt/\vxt)-\th;
	\vxtd=\vrt*cos(\vthd);
	\vytd=\vrt*sin(\vthd);
	\vxd=\vr*cos(\vth-\th);
	\vyd=\vr*sin(\vth-\th);
	} 
	\begin{tikzpicture}[scale=\picscale ] 
		\draw[-latex] (0,0) -- (\rrange,0)  node[right,scale=\textscale, text width=5em] {$\hat{e}_1$};
		\draw[-latex] (0,0) -- (0,\rrange)  node[right,scale=\textscale, text width=5em] {$\hat{e}_2$};
		\draw[dotted, thick] (\px,0) --  (\px,\py) node[midway,right, scale=\textscale]{${x}^2$}; 
		\draw[dotted, thick] (0,\py) --  (\px,\py)  node[midway,above, scale=\textscale]{${x}^1$};  
		
		\draw[-latex] (\px,\py)  -- (\px+\vx,\py+\vy)  node[right=22,above=-4,scale=\textscale, text width=5em] {${\varphi}$};
		\draw[dotted, thick] (\px,\py) --  (\px,\py+\vy) node[midway,left, scale=\textscale]{${\varphi}^2$}; 
		\draw[dotted, thick] (\px,\py+\vy) --  (\px+\vx,\py+\vy) node[midway,above, scale=\textscale]{${\varphi}^1$};  
	
		\begin{scope}[rotate=\th,draw=red, text=red]
		   \draw[-latex] (0,0) -- (\rrange,0)  node[right,scale=\textscale, text width=5em]  {$\hat{e}'_1$};
			  \draw[-latex] (0,0) -- (0,\rrange)  node[right,scale=\textscale, text width=5em] {$\hat{e}'_2$};
		\draw[dotted, thick] (\pxd,\pyd) --  (\vxtd,\pyd) node[midway,right=3,below=-3, scale=\textscale]{${\varphi'}^1$}; 
		\draw[dotted, thick] (\vxtd,\pyd) --  (\vxtd,\vytd) node[midway,right=6,above=-4, scale=\textscale]{${\varphi'}^2$};  
		\end{scope}
	
	  \draw [black,fill] (\px,\py) circle [radius=\circsize pt];    
		\coordinate[label = below: (b)]  (D) at (\labelx,\labely); 
	\end{tikzpicture}% pic 1
	\vspace*{2px}
	\caption{Shows how a four-vector ${\varphi}$ (of which only two components are shown) defined at a spacetime location (indicated by $\bullet$) can be expressed either as $(\varphi^1, \varphi^2)$ with respect to the basis $\{\hat{e}_1,\hat{e}_2\}$ or as $({\varphi'}^1,{\varphi'}^2)$ with respect to the basis $\{\hat{e}'_1,\hat{e}'_2\}$.}\label{rotfigure2}
	\end{figure}
Figure \ref{rotfigure2} shows how (two of) the components of a four-vector $\varphi$ at a particular location will differ for different observers under a spatial rotation of the coordinates. A four-vector $\varphi^\mu(x)$ defined at every spacetime location $x$ is called a \textbf{four-vector field}\index{four-vector field}. If $O$ observes this vector-field $\varphi^\mu(x)$, and $O'$ is another observer whose coordinates are related to the coordinates $O$ via the Lorentz transformation $\Lambda$, then $O'$ will describe this vector-field as ${\varphi'}^\mu(x')\myeq\Lambda\indices{^\mu_\nu}\varphi^\nu(x).$ Hence, under the Lorentz transformation $\Lambda$,  a vector field $\varphi^\mu(x)$ transforms as ${\varphi}^\mu(x)\rightarrow {\varphi'}^\mu(x')$ where
\begin{equation}\label{lorentzvector}
{\varphi'}^\mu(x)=\Lambda\indices{^\mu_\nu}\varphi^\nu(\Lambda^{-1}x).
\end{equation} 



From a four-vector $\varphi^\mu$, we can also define the so-called \textbf{four-covector}\index{four-covector}: 
\begin{equation}\label{covector}
	\varphi_\mu\myeq\eta_{\mu\nu}\varphi^\nu.
\end{equation}
To see how four-covectors transform under a Lorentz transformation $\Lambda$, it will be helpful to define 
\begin{equation}\label{colambda}
	\Lambda\indices{_\mu^\nu}\myeq\eta_{\mu\rho}\eta^{\nu\sigma}\Lambda\indices{^\rho_\sigma}
\end{equation}
where $\eta^{\nu\sigma}=\eta_{\nu\sigma}$. If we also define the \textbf{Kronecker-delta}\index{Kronecker-delta} $\delta^\nu_\mu$ such that $\delta^\nu_\mu=1$ when $\mu=\nu$ and $\delta^\nu_\mu=0$ otherwise, then using the fact that $\eta_{\mu\rho}\eta^{\nu\rho}=\delta^\nu_\mu$ together with equation (\ref{lorentztrans}), we have 
\begin{equation}\label{lambdainverse}
\Lambda\indices{^\rho_\mu}\Lambda\indices{_\rho^\nu}=\delta^\nu_\mu.
\end{equation}
Since by definition, the inverse of $\Lambda^{-1}$ satisfies 
$(\Lambda^{-1})\indices{^\nu_\rho}\Lambda\indices{^\rho_\mu}=\delta^\nu_\mu,$
 we have $(\Lambda^{-1})\indices{^\nu_\rho}=\Lambda\indices{_\rho^\nu}.$  From (\ref{lorentzvector}), (\ref{covector}), and (\ref{colambda}), we therefore see that under a Lorentz transformation $\Lambda$, a four-covector field $\varphi_\mu(x)$ transforms as $\varphi_\mu(x)\rightarrow\varphi'_\mu(x')$
where
\begin{equation}\label{lorentzcovector}
\varphi'_\mu(x)=\Lambda\indices{_\mu^\nu}\varphi_\nu(\Lambda^{-1}x)
\end{equation}
 

Besides scalars, four-vectors, and four-covectors, we also need to consider physical quantities called rank-two tensors. The stress-energy tensor $T^{\mu\nu}$ mentioned on page \pageref{stressenergy} is an example of a rank-two tensor. The defining property of a rank-two tensor field $\varphi^{\mu\nu}(x)$ is that under a Lorentz transformation $\Lambda$, it transforms as $\varphi^{\mu\nu}(x)\rightarrow{\varphi'}^{\mu\nu}(x')$ where
\begin{equation}\label{lorentztensor}
{\varphi'}^{\mu\nu}(x)=\Lambda\indices{^\mu_\rho}\Lambda\indices{^\nu_\sigma}\varphi^{\rho\sigma}(\Lambda^{-1}x).
\end{equation}
On page \pageref{massenergydensity}, we introduced the mass-energy density $T_S(x)$ on a spacelike hypersurface $S$. As explained in section \ref{massenergydensity}, the values of $T_S(x)$ are the additional values that Kent uses to supplement standard quantum theory.  It was mentioned in passing that $T_S(x)$ does not depend on which frame of reference one is in. In other words, $T_S(x)$ is a scalar. I will now explain why this is so. 

We first need to consider the precise definition of $T_S(x)$. At each spacetime location on the spacelike hypersurface $S$ which an observer $O$ describes as having coordinates $x=(x^\mu)_{\mu=0}^3$, we define  $\eta^\mu(x)$ to be the future-directed  unit four-vector at $x$ that is orthogonal to $S$. In other words, $\eta^0(x)>0$, $\eta_\mu(x)\eta^\mu(x)=1$, and if $y\in S$ is very close to $x$, then $\frac{(x-y)_\mu\eta^\mu(x)}{\sqrt{(x-y)_\nu(x-y)^\nu}}\approx 0.$  $T_S(x)$ is then given by the formula 
\begin{equation}\label{TSdef}
T_S(x)=T^{\mu\nu}(x)\eta_{\mu}(x)\eta_{\nu}(x).
\end{equation}
For example, if $S$ was the spacelike hypersurface consisting of all spacetime locations $x = (0,x^1,x^2,x^3)$, then $\big(\eta^{0}(x),\eta^{1}(x),\eta^{3}(x),\eta^{3}(x)\big) =(1,0,0,0),$ and hence $T_S(x)=T^{00}(x)$ which is the density of relativistic mass at $x$, i.e. the energy density at $x$ divided by $c^2$. 

To see why $T_S(x)$ is a scalar, suppose that $\Lambda$ is a Lorentz transformation such that $\Lambda\indices{^0_\mu}\eta^\mu>0$ for any future-directed  unit four-vector vector $\eta^\mu$. We refer to a $\Lambda$ with this property as an \textbf{orthochronous}\index{Lorentz transformation!orthochronous} Lorentz transformation. Also, suppose that $O$ and $O'$ are two observers such that spacetime locations that observer $O$ describes as having coordinates $x=(x^\mu)_{\mu=0}^3$ are described by $O'$ as having coordinates $x'=(\Lambda\indices{^\mu_\nu}x^\nu)_{\mu=0}^3$. Then since ${x'}_\mu{y'}^\mu= x_\mu y^\mu$, it follows that the future-directed unit four-vector orthogonal to $S$ at $x$ which $O$ describes as $\eta^\mu(x)$ will be described by $O'$ as  ${\eta'}^\mu(x')=\Lambda\indices{^\mu_\nu}\eta^\nu(x)$. Thus, for any location in $S$ that $O'$ describes as having coordinates $x'$ with corresponding  future-directed $S$-orthogonal unit four-vector ${\eta'}^\mu(x')$, $O'$ can construct a function $T'_S(x')$  with 
\begin{equation}\label{TSprimedef}
T'_S(x')=T'^{\mu\nu}(x')\eta'_{\mu}(x')\eta'_{\nu}(x').
\end{equation}
Then using  (\ref{lorentzcovector}) and (\ref{lorentztensor}) on the right-hand side of (\ref{TSprimedef}),  we have
\begin{equation}\label{invariantTS1}
\begin{split}
T'_S(x')&=\Lambda\indices{^\mu_\rho}\Lambda\indices{^\nu_\sigma}T^{\rho\sigma}(x)\Lambda\indices{_\mu^\alpha}\eta_{\alpha}(x)\Lambda\indices{_\nu^\beta}\eta_{\beta}(x)\\
&=\Lambda\indices{^\mu_\rho}\Lambda\indices{_\mu^\alpha} \Lambda\indices{^\nu_\sigma}\Lambda\indices{_\nu^\beta}T^{\rho\sigma}(x)\eta_{\alpha}(x)\eta_{\beta}(x)\\
&=\delta^\alpha_\rho\delta^\beta_\sigma T^{\rho\sigma}(x)\eta_{\alpha}(x)\eta_{\beta}(x)\\
&=T^{\alpha\beta}(x)\eta_{\alpha}(x)\eta_{\beta}(x)\\
&=T_S(x)
\end{split}
\end{equation}
where on the third line we have used (\ref{lambdainverse}), and on the last line we have used (\ref{TSdef}). To obtain (\ref{invariantTS1}), we assumed that $\Lambda$ is orthochronous, but if $\Lambda$ is non-orthochronous, we would need to take the negations of ${\eta'}^\mu(x')$ to get the future-directed $S$-orthogonal unit four-vector. But clearly this will not affect the equality in (\ref{invariantTS1}), so (\ref{invariantTS1}) holds for all Lorentz transformations, whether they are orthochronous or non-orthochronous.  We thus see that   $T_S(x)$ is a scalar.

Let us now consider the Hilbert space $H_{S_n}$ as defined on page \pageref{HSidef} for a spacelike hypersurface $S_n$. 
Given that $\hat{T}^{\mu\nu}(x)$ is the observable whose eigenstates with eigenvalues $\tau$ are the states of $S_n$ for which an observer $O$ observes the stress-energy tensor $T^{\mu\nu}(x)$ to take the value $\tau$ at $x$, it follows from (\ref{TSdef}) that 
\begin{equation}\label{TShat}
	\hat{T}_S(x)\myeq \hat{T}^{\mu\nu}(x)\eta_{\mu}(x)\eta_{\nu}(x)
	\end{equation}
	will be the observable whose eigenstates with eigenvalues $\tau_S(x)$ are the states of $S_n$ for which an observer $O$ observes $T_S(x)$ to take the value $\tau_S(x)$ at $x$.
Two observers $O$ and $O'$ will typically assign different physical states to $S_n$ based on their frame of reference. E.g. if $O$ and $O'$ are traveling at different speeds, they will attribute different energy levels and momenta to the spacetime locations of $S_n$. For the Lorentz transformation that relates the coordinates of $O'$ to the coordinates of $O$, i.e. $x'=\Lambda x$,  there  will then be a unitary operator $U(\Lambda):H_{S_n}\rightarrow H_{S_n}$ such that if
$O$ observes  $S_n$ to be in the state $\ket{\psi_n}\in H_{S_n}$, then $O'$ will observe $S_n$ to be in the state $U(\Lambda)\ket{\psi_n}$. But in order for $U(\Lambda)\ket{\psi_n}$ to be meaningful, we need to specify how the Hermitian operators that act on $U(\Lambda)\ket{\psi_n}$ correspond to the physical quantities that $O'$ observes. So we specify that the Hermitian operator 
\begin{equation}\label{Thatprime}
\hat{T}'^{\mu\nu}(x')=\hat{T}^{\mu\nu}(x')
\end{equation}
will be the observable whose eigenstates with eigenvalues $\tau'$ are the states of $S_n$ for which $O'$ observes the stress-energy tensor $T^{\prime\mu\nu}(x')$ to take the value $\tau'$ at $x'$. Since $T^{\mu\nu}(x)$ transforms according to (\ref{lorentztensor}), it will follow that
\begin{equation}\label{TUrelation}
U(\Lambda)^{-1}\hat{T}^{\mu\nu}(x)U(\Lambda)=\Lambda\indices{^\mu_\rho}\Lambda{^\nu_\sigma}\hat{T}^{\rho\sigma}(\Lambda^{-1}x).
\end{equation}

We also insist that $U(\Lambda)$ is unitary because this means that if $O$ calculates the probability $S_n$ transitions from state $\ket{\psi_n}$ to state $\ket{\chi_n}$, then $O'$ would calculate the same probability for the corresponding transition from the state $\ket{\psi_n'}=U(\Lambda)\ket{\psi_n}$ to the state $\ket{\chi_n'}=U(\Lambda)\ket{\chi_n}.$\footnote{This follows from (\ref{unitarycond}) which implies 
$\abs{\ip{\chi'_n}{\psi'_n}}^2=\abs{\ip{\chi_n}{\psi_n}}^2$, together with the Born Rule given on page \pageref{bornrule}.}

Now to say that Kent's model is Lorentz invariant, is to say that (\ref{kentconsistency0}) defines a rank-two tensor, for then this quantity and the quantities on which it depends will transform in the way that physical quantities should transform under a Lorentz transformation. Thus, in order to show that Kent's model is Lorentz invariant, we need to show that if $\{\ket{\xi_j}:j\}$ is an orthonormal basis of the Hilbert space of states $H_{S_n,\tau_S}$ for which $O$ observes $T_S(x)$ to be $\tau_S(x)$ for all $x\in S_n(y)\cap S$, and if $\{\ket{\xi_j'}:j\}$ is an orthonormal basis of the Hilbert space of states $H_{S_n,\tau_S'}$ for which $O'$ observes $T'_S(x')$ to be $\tau'_S(x')$ for all $x'\in S_n(y')\cap S$, then
\begin{equation}\label{kentlorentz}
\lim_{n\rightarrow\infty}\frac{\ev{\pi_n'\hat{T}^{\mu\nu}(y')}{\Psi_n'}}{\ev{\pi_n'}{\Psi_n'}}=\Lambda\indices{^\mu_\rho}\Lambda\indices{^\nu_\sigma} \lim_{n\rightarrow\infty}\frac{\ev{\pi_n\hat{T}^{\rho\sigma}(y)}{\Psi_n}}{\ev{\pi_n}{\Psi_n}}
\end{equation}
where $\pi_n=\sum_j\dyad{\xi_j}$, $\pi_n'=\sum_j\dyad{\xi_j'}$, and $\ket{\Psi_n'}=U(\Lambda)\ket{\Psi_n}$. 

To see why (\ref{kentlorentz}) holds, we first recall that $\pi_n'$ will be independent of which orthonormal basis we choose for $H_{S_n,\tau_S'}$.\footnote{We showed this was the case for $\pi_n$ in footnote \ref{priproof} on page \pageref{priproof}.} Therefore, if we can show that $\{\ket{\xi_j'}\myeq U(\Lambda)\ket{\xi_j}:j\}$ is an orthonormal basis of $H_{S_n,\tau_S'}$, it will follow that $\pi_n'=U(\Lambda)\pi_nU(\Lambda)^{-1}$. 

That the elements of $\{U(\Lambda)\ket{\xi_j}:j\}$ are orthonormal follows from the unitarity of $U(\Lambda)$ together with the orthonormality of  $\{\ket{\xi_j}:j\}$.  Since $\hat{T}^{\mu\nu}(x')$ is the observable whose eigenstates with eigenvalue $\tau'$ are the states of $S_n(y')$ for which $O'$ observes the stress-energy tensor $T^{\prime\mu\nu}(x')$ to take the value $\tau'$ at $x'$, it follows from (\ref{TSprimedef}) and (\ref{Thatprime})  that 
\begin{equation}\label{TShatprime}
	\hat{T}'_S(x')\myeq\eta_\mu'(x')\eta_\nu'(x')\hat{T}^{\mu\nu}(x')
\end{equation}
 will be the observable whose eigenstates with eigenvalue $\tau'_S$ are the states of $S_n(y')$ for which $O'$ observes $T'_S(x')$ to take the value $\tau'_S$ at $x'$, where as usual, $\eta^{\prime\mu}(x')$ is the unit four-vector orthogonal to $S_n(y')$ at $x'$. Now if $x'\in S_n(y')\cap S$ in the coordinates of $O'$, then $x=\Lambda^{-1}x'\in S_n(y)\cap S$ in the coordinates of $O$. Using the same calculation as in (\ref{invariantTS1}) together with (\ref{lorentzcovector}), we have
\begin{equation}\label{TSLambda}
\begin{split}
\hat{T}_S(x)&\myeq\eta_\mu(x)\eta_\nu(x)\hat{T}^{\mu\nu}(x)\\
&=\eta_{\mu}'(x')\eta_{\nu}'(x') \Lambda\indices{^\mu_\rho}\Lambda\indices{^\nu_\sigma}\hat{T}^{\rho\sigma}(x).
\end{split}
\end{equation}
By (\ref{TUrelation}) we have
\begin{equation}
	U(\Lambda)^{-1}\hat{T}^{\mu\nu}(x')=\Lambda\indices{^\mu_\rho}\Lambda{^\nu_\sigma}\hat{T}^{\rho\sigma}(x)U(\Lambda)^{-1},
\end{equation}
so using this with (\ref{TSLambda}) and (\ref{TShatprime}), we have
\begin{equation}\label{TSU}
\begin{split}
\hat{T}_S(x)U(\Lambda)^{-1}&=\eta_{\mu}'(x')\eta_{\nu}'(x')\Lambda\indices{^\mu_\rho}\Lambda\indices{^\nu_\sigma}\hat{T}^{\rho\sigma}(x)U(\Lambda)^{-1}\\
&=U(\Lambda)^{-1} \eta_{\mu}'(x')\eta_{\nu}'(x')\hat{T}^{\mu\nu}(x')\\
&=U(\Lambda)^{-1} \hat{T}_S'(x').
\end{split}
\end{equation}
Now suppose that $\ket{\xi'}$ is a state for which $O'$ observes $T'_S(x')$ to be $\tau'_S(x')$ for all $x'\in S_n(y')\cap S$. Then $\hat{T}_S'(x')\ket{\xi'}=\tau_S'(x'),$ and so by (\ref{TSU}), 
\begin{equation}\label{TSUxi}
\begin{split}
\hat{T}_S(x)U(\Lambda)^{-1}\ket{\xi'}&= U(\Lambda)^{-1} \hat{T}_S'(x')\ket{\xi'}\\
&=\tau_S'(x')U(\Lambda)^{-1}\ket{\xi'}\\
&=\tau_S(x)U(\Lambda)^{-1}\ket{\xi'}
\end{split}
\end{equation}
where on the last line we have used the fact that $T_S(x)$ is a scalar. Therefore, $U(\Lambda)^{-1}\ket{\xi'}$ can be expressed as a linear combination of the basis elements $\{\ket{\xi_j}:j\}$ of $H_{S_n,\tau_S}$, and hence  $\ket{\xi'}$  can be expressed as a linear combination of $\{U(\Lambda)\ket{\xi_j}:j\}$. From (\ref{TSU}) we also see that  $U(\Lambda)\hat{T}_S(x)=\hat{T}_S'(x')U(\Lambda)$, so
$$
\hat{T}_S'(x')U(\Lambda)\ket{\xi_j}=U(\Lambda)\hat{T}_S(x)\ket{\xi_j}=\tau_S(x)U(\Lambda)\ket{\xi_j}=\tau_S'(x')U(\Lambda)\ket{\xi_j}
$$
for all $x'\in S_n(y')\cap S$. Therefore, $U(\Lambda)\ket{\xi_j}\in H_{S_n,\tau_S'}$.
Since $\{\ket{\xi_j'}\myeq U(\Lambda)\ket{\xi_j}:j\}$ is a spanning orthonormal subset of $H_{S_n,\tau_S'}$, it must therefore be an orthonormal basis of $H_{S_n,\tau_S'}$. From this it follows that $\pi_n'=U(\Lambda)\pi_nU(\Lambda)^{-1}$. Therefore, 
\begin{equation}\label{kentlorentz2}
\begin{split}
\frac{\ev{\pi_n'\hat{T}^{\mu\nu}(y')}{\Psi_n'}}{\ev{\pi_n'}{\Psi_n'}}&=\frac{\ev{U(\Lambda)^{-1}U(\Lambda)\pi_nU(\Lambda)^{-1}\hat{T}^{\mu\nu}(y')U(\Lambda)}{\Psi_n}}{\ev{U(\Lambda)^{-1}U(\Lambda)\pi_n U(\Lambda)^{-1}U(\Lambda)}{\Psi_n}}\\
&=\frac{\ev{\pi_nU(\Lambda)^{-1}\hat{T}^{\mu\nu}(y')U(\Lambda)}{\Psi_n}}{\ev{\pi_n}{\Psi_n}}\\
&=\frac{\ev{\pi_n\Lambda\indices{^\mu_\rho}\Lambda{^\nu_\sigma}\hat{T}^{\rho\sigma}(y)}{\Psi_n}}{\ev{\pi_n }{\Psi_n}}
\end{split}
\end{equation}
where on the last line we have used (\ref{TUrelation}). Thus, equation (\ref{kentlorentz}) holds, and hence Kent's model is Lorentz invariant.

Note that in this proof of Lorentz invariance, we don't need to take the limit of $S_n$ as $n\rightarrow\infty$. That is, we could remove the $\lim_{n\rightarrow\infty}$ from equation (\ref{kentconsistency0}) and consider a particular $S_n$, and the corresponding $\ev*{T^{\mu\nu}(y)}_{\tau_S}$ would still be a rank-two tensor. Butterfield tells us that Kent's theory is  Lorentz invariant because his algorithm respects the light cone structure of $y$.\footnote{See \cite[30]{Butterfield}.} However, this statement could be slightly misleading because we don't need to consider the subset $S^1(y)\subset S$ of locations outside the light cone of $y$ in order to obtain a Lorentz invariant model. Doing the calculation on any Tomonaga-Schwinger spacelike hypersurface is sufficient to guarantee Lorentz invariance since any such spacelike hypersurface (e.g. $S_n$) is not altered at all by a Lorentz transformation -- only its coordinate description changes under a Lorentz transformation, and so the additional information of the scalar $\tau_S(x)$ on $S_n\cap S$ is Lorentz invariant. The only reason we need to consider the limit $\lim_{n\rightarrow \infty}S_n$ and hence $S^1(y)=\lim_{n\rightarrow \infty}S_n\cap S$ is that it is only in the limit that we use all the available information in $\tau_S(x)$ to calculate $\ev*{T^{\mu\nu}(y)}_{\tau_S}$. 



\section{Kent's Theory\label{Kentdecoherencesection} and Decoherence Theory\textsuperscript{*}}
In section \ref{probOutcomes} we saw that decoherence theory by itself does not offer a solution to the problem of outcomes. In this section, we consider how the additional information in Kent's theory is sufficient to address this problem. We will explain this by again considering  Kent's toy model discussed in section \ref{toysection}.

We thus suppose that a system is in a superposition $\psi_0^\text{sys} = c_1\psi_1^\text{sys}+c_2\psi_2^\text{sys}$ of two local states $\psi_1^\text{sys}$ and $\psi_2^\text{sys}$ where $\abs{c_1}^2+\abs{c_2}^2=1$, and that there is a photon coming in from the left that interacts with the system. We also suppose that  $y_1$ is a spacetime location with spatial location $z_1$ between the two spacelike hypersurfaces $S_0$ and $S$, and we consider a spacelike hypersurface $S_n=S_n(y_1)$  in a sequence of spacelike hypersurfaces that each contain $y_1$ as described on page \pageref{siydef}. 

In order to obtain a sufficiently simple description of the state $\ket{\Psi_n}\in H_{S_n}$ of $S_n$ for which we can use the formula (\ref{kentconsistency0}) to calculate Kent's beable, we will  
use a coarse-grained model so that $S_n$ is treated as a mesh of tiny cells labeled by a sequence $(y_k)_{k=1}^\infty$. Thus, for each cell $y_k$ there will be a Hilbert space $H_k$ describing the state of that cell. We can think of each of these $y_k$ as systems that can become entangled with one another, but we will assume that $y_1$ is entangled with only a finite number $M$ of the other $y_k$. What this means is that the most general expression for $\ket{\Psi_n}$ will be of the form
\begin{equation}\label{Sistate}
\ket{\Psi_n}=\Big(\sum_j\sum_{n\in\mathbb{N}^M} c_{j,n}\ket{\xi_{1,j}}\prod_{l=1}^{M}\ket{\xi_{k_{l},n_l}}\Big)\Xi.
\end{equation}
In this expression, $\{\ket{\xi_{1,j}}:j\}$ is an orthonormal basis of $H_1$, $\mathbb{N}^M$ means the set of all lists $(n_1,\ldots,n_M)$ with each $n_l\in\mathbb{N}$ where $\mathbb{N}$ is the set of positive integers greater than 0. The set of states $\{\ket{\xi_{k_{l},n_l}}:n_l\in\mathbb{N}\}$ form an orthonormal basis of $H_{k_{l}}$ for each $k_l$, and the $k_{l}$ are all distinct from each other and from $1$. Also, $M$ is chosen to be as small as possible so that any common factors of $\ket{\Psi_n}$ belong to $\Xi$ which is a sum of states of the form $\prod_l\ket{\xi_{\kappa_l}}$ where the states $\ket{\xi_{\kappa_l}}\in H_{\kappa_l}$ ranging over all the cells of $S_n$ not included in the set $\{k_{l}:l=1,\ldots,M\}.$ We also assume that each summand $c_{j,n}\ket{\xi_{1,j}}\prod_{l=1}^{M}\ket{\xi_{k_{l},n_l}}\Xi$ of $\Psi_n$ contains a state in each $H_k$ for every cell $k$ of $S_n$. In other words if $k\neq 1$ and does not belong to the set $\{k_{l}:l=1,\ldots,M\}$ then $k$ belongs to the set $\{\kappa_l:l\}$. Also, we will give $H_{S_n}$ an inner product so that if 
$$ \ket{\Psi_n'}=\Big(\sum_j\sum_{n\in\mathbb{N}^M} c'_{j,n}\ket{\xi_{1,j}}\prod_{l=1}^{M}\ket{\xi_{k_{l},n_l}}\Big)\Xi',$$ then
$$\ip{\Psi_n'}{\Psi_n}=\Big(\sum_j\sum_{n\in\mathbb{N}^M} \overline{c'_{j,n}}c_{j,n} \Big) \ip{\Xi'}{\Xi}$$
where $\ip{\Xi'}{\Xi}$ is defined in the obvious way. With this inner product, we will assume that $\ket{\Psi_n}$ is appropriately normalized so that $\ip{\Psi_n}{\Psi_n}=1$. If we also assume that  $\ip{\Xi}{\Xi}=1$, it will follow that $\sum_j\sum_{n\in\mathbb{N}^M}\abs{c_{j,n}}^2=1.$

We now consider several scenarios from Kent's toy model. In each scenario, we will use the decomposition (\ref{Sistate}) of $\ket{\Psi_n}$ to calculate the partial trace encapsulating all the information needed to calculate expectation values at different spacetime locations. 

First, consider Figure \ref{kentdeco1} which depicts the spacelike hypersurface $S_n(y^a_1)$ for a spacetime location $y^a_1$ that occurs before the photon has interacted with the system.

\begin{figure}[ht!]
\captionsetup{justification=justified}
\centering
\tikzmath{
\a=5.3;  
\psione=0;
\psitwo=1.5;
\h=1;
\labelx=(\psione+\psitwo)/2;
\labely=-1.5;
\phstartx=-1.7;
\phstarty=\h;
\tbeg=\phstarty;
\ca=\phstartx-\tbeg;
\ttwo=\psitwo-\ca;
\cb=\ttwo+\ca;
\xend=\ttwo-\a+\cb;
\tone=\psione-\ca;
\cc=\tone+\ca;
\xendz=\tone-\a+\cc;
\tthree=\cc+\tone-\psitwo;
\circsize=0.08;
\md = (\a+\h)/2;
\tlen=0.75;
\textscale = 0.7;
\picscale = 0.89;
\nudge=0.1;
\tnudge=(\ttwo-\tone)/2+0.8;
\ttesta=2*\tone-\ttwo-\tnudge;
\ttestb=2*\tone-\ttwo+\tnudge;
\ttestc=\ttwo-\tnudge;
\ttestd=\ttwo+\tnudge;
\sonea=\a+\psione-\ttesta;
\sadd=0.5;
\lrange = -(\psione-\a+\ttesta)+\sadd;
\rrange =\lrange; 
\timex=\rrange-1.5;
\e = 0.4;
\ttestx=2.4;
\at=\a-\ttestx;
\ct=\ttestx;
\lam = 0.87;
\e = 0.2;
\hae=(2*pow(\at,3)*\lam*(2+\lam)+\e*\e*(2*\e-sqrt(4*\e*\e+\at*\at*\lam*\lam))-8*\at*\e*(-2*\e+sqrt(4*\e*\e+\at*\at*\lam*\lam))-2*\at*\at*(2+\lam)*(-2*\e+sqrt(4*\e*\e+\at*\at*\lam*\lam)))/(4*\at*\lam*(2*\at+2*\e-sqrt(4*\e*\e+\at*\at*\lam*\lam)))-(\e*\e/4+\at*\at*(-1+\lam))/(\at*\lam);
\mmu=(\tone-\tbeg)/(\psione-\phstartx);
\nnu=(\psione*\tbeg-\phstartx*\tone)/(\psione-\phstartx);
} 
\tikzstyle{scorestars}=[star, star points=5, star point ratio=2.25, draw, inner sep=1pt]%
\tikzstyle{scoresquare}=[draw, rectangle, minimum size=1mm,inner sep=0pt,outer sep=0pt]%
\begin{tikzpicture}[scale=\picscale,
declare function={
	testxonep(\ps,\t)=\a+\ps-\t;
	testxonem(\ps,\t)=\ps-\a+\t; 
	xa(\m,\n)=(-\ct*\m+\e*\m+\m*\n-sqrt(\ct*\ct*\lam*\lam+\e*\e*\m*\m+2*\e*\lam*\lam*\n+\lam*\lam*\n*\n-2*\ct*\lam*\lam*(\e+\n)))/(\lam*\lam-\m*\m);
	xb(\m,\n)=(-\ct*\m+\e*\m+\m*\n+sqrt(\ct*\ct*\lam*\lam+\e*\e*\m*\m+2*\e*\lam*\lam*\n+\lam*\lam*\n*\n-2*\ct*\lam*\lam*(\e+\n)))/(\lam*\lam-\m*\m);
	bl(\x)=\ct+\at-\e/2-1/2*(sqrt(\lam*\lam*pow(\x+\hae+\at,2)+\e*\e)-\e+\lam*(\x+\hae+\at));
	br(\x)=\ct+\at-\e/2-1/2*(sqrt(\lam*\lam*pow(\x-\hae-\at,2)+\e*\e)-\e-\lam*(\x-\hae-\at));
	bc(\x)=\ct+sqrt(\lam*\lam*\x*\x+\e*\e)-\e;
},
 ] 
\definecolor{tempcolor}{RGB}{250,190,0}
\definecolor{darkgreen}{RGB}{40,190,40}

\draw[->,blue, thick] [domain=-\at/2:-\lrange, samples=150]   plot (\x, {bl(\x)})  ;
\draw[blue, thick] [domain=-\at/2:\at/2, samples=150] plot (\x, {bc(\x)})   ;
\draw[->,blue, thick] [domain=\at/2:\rrange, samples=150]   plot (\x, {br(\x)})   ;
 \node[scale=\textscale]  at (2.7,4.1) {$S_n(y^a_1)$}; 

\draw[<->] (-\lrange, \h) node[left, scale=\textscale] {$S_0$} -- (\rrange, \h) node[right, scale=\textscale] {$S_0$};
\draw[<->] (-\lrange, \a) node[left, scale=\textscale] {$S$} -- (\rrange, \a) node[right, scale=\textscale] {$S$};
              
\draw[->, shorten <= 5pt,  shorten >= 1pt] (\psione,\h) node[below, scale=\textscale]{$\psi_1^{\text{sys}}$} node[above left, scale=\textscale]{$z_1$}-- (\psione,\a) ;
\draw[->,  shorten <= 5pt,  shorten >= 1pt] (\psitwo,\h) node[below, scale=\textscale]{$\psi_2^{\text{sys}}$} node[above right, scale=\textscale]{$z_2$}-- (\psitwo,\a);

\draw[dashed, tempcolor,  thick](\phstartx,\tbeg) node[below left=-2,scale=\textscale]{}--(\psione,\tone) node[above, pos=0.3, rotate=45,scale=0.7,black] {} node[below, pos=0.3, rotate=45,scale=0.7,black] {};
\draw[dashed, tempcolor](\psione,\tone) node[below left=-2,scale=\textscale]{}--(\psitwo,\ttwo );
\draw[dashed, tempcolor](\psitwo,\ttwo)--(\xend,\a); 
\draw[dashed, tempcolor,  thick](\psione,\tone)--(\xendz,\a);



%\draw[dashed, darkgreen, ultra thick](\psione,\tone)--(\psitwo,\tthree);
%\draw [black,fill] (\psitwo,\tthree) circle [radius=\circsize] node [black,below=4,right,scale=\textscale] {$2t_1-t_2$}; 


%\draw [black,fill] (\psione,\tone) circle [radius=\circsize] node [black,left=4,scale=\textscale] {$t_1$}; 
%\draw [black,fill](\psitwo,\ttwo)circle [radius=\circsize] node [black,above right,scale=\textscale, align=left] {$t_2$}; 

%\draw [decorate, decoration = {calligraphic brace}] (\psitwo+\nudge,\ttwo) --  (\psitwo+\nudge,\tone+\nudge/4) node[midway,right, scale=\textscale]{$t_2-t_1$}; 
%\draw [decorate, decoration = {calligraphic brace}] (\psitwo+\nudge,\tone-\nudge/4) --  (\psitwo+\nudge,2*\tone-\ttwo) node[midway,right, scale=\textscale]{$t_2-t_1$}; 

\draw[->] (\timex,\md-\tlen/2) --  (\timex,\md+\tlen/2) node[midway,right, scale=\textscale]{time}; 
 
\draw[dotted](\psione,\ttestx)--({testxonep(\psione,\ttestx)},\a);
\draw[dotted](\psione,\ttestx)--({testxonem(\psione,\ttestx)},\a);
%\draw [black,fill](\psione,\ttestx) circle [radius=\circsize] node [black,below=6,left,scale=\textscale] {$y^a_1$}; 
\draw (\psione,\ttestx) node[ scoresquare, fill=gray]  {} node [black,below=6,right,scale=\textscale] {$y^a_1$};
\draw (\psitwo, {bc(\psitwo)}) node[ scoresquare, fill=gray]  {} node [black,below=6,right,scale=\textscale] {$y^a_2$};
\draw ({xa(\mmu,\nnu)}, {xa(\mmu,\nnu)*\mmu+\nnu}) node[ scoresquare, fill=white]  {} node [black,left=7,below=-4,scale=\textscale] {$y^a_3$};

\draw[magenta,->,ultra thick] ({testxonem(\psione,\ttestx)},\a)--(-\lrange,\a);
\draw[magenta,->,ultra thick] ({testxonep(\psione,\ttestx)},\a)--(\rrange,\a);
%\draw [black,fill] (\xendz,\a) circle [radius=\circsize] node [black,above,scale=\textscale] {$\gamma_1$}; 
\end{tikzpicture}% pic 1

\vspace*{2px}
\caption{Depiction of a superposition of two local states at $z_1$ and $z_2$ before the photon has interacted with them. The gray squares indicate cells in $S^1(y^a_1)$ whose states are among the summands in (\ref{Sistate}) rather than in $\Xi$. The white square indicates a cell in $S_n(y^a_1)$ whose state is a factor in $\Xi$.}
\label{kentdeco1}
\end{figure}

The gray squares correspond to the summands that appear in (\ref{Sistate}). If the system were in the $\psi_1^\text{sys}$-state, then the state describing $S_n(y^a_1)$ would have a factor $\ket{\psi_1^\text{sys}}\in H_1$ indicating that there is a 
non-zero mass at the $y^a_1$-cell, and there would also be a factor $\ket{0_2}\in H_2$ which we use to indicate that there is zero mass/energy at $y^a_2$. 
There is also an incoming photon at the $y^a_3$-cell, and so we use $\ket{\gamma_3}$ to indicate that there is a photon there.
 Thus, if  the system  were in the $\psi_1^\text{sys}$-state, we would write the state of $S_n(y^a_1)$ 
 as $\ket{\Psi_n}=\ket{\psi_1^\text{sys}}\ket{0_2}\ket{\gamma_3}\Xi'$, where $\Xi'$ describes the states of all the other cells of $S_n(y^a_1)$. In this very simple scenario, $\Xi'=\sum_{k\neq 1,2,3}\ket{0_k}$ indicating that there is zero mass/energy at all the other $y_k$.

 Alternatively, if the system were in the state $\psi_2^\text{sys}$, then the state describing $S_n(y^a_1)$ would have a factor $\ket{\psi_2^\text{sys}}\in H_2$ 
 indicating that there is a non-zero mass at the $y^a_2$-cell, and there would also be a factor $\ket{0_1}\in H_1$ which we use to indicate that there is zero mass at $y^a_1$,
  and again the $y^a_3$-cell would be in the $\ket{\gamma_3}$, and every other cell would be described by  $\Xi'$  just as if the system had been in the $\psi_1^\text{sys}$-state. Therefore, when the system is in the state $\psi_2^\text{sys}$, we would write the state of $S_n(y^a_1)$ as $\ket{\Psi_n}=\ket{0_1}\ket{\psi_2^\text{sys}}\ket{\gamma_3}\Xi'$. 
 
 Now since the system is actually in a supposition $\psi_0^\text{sys} = c_1\psi_1^\text{sys}+c_2\psi_2^\text{sys}$, the state of $S_n(y^a_1)$ will be 
 \begin{equation*}
 \ket{\Psi_n}=\big(c_1\ket{\psi_1^\text{sys}}\ket{0_2}+c_2\ket{0_1}\ket{\psi_2^\text{sys}}\big)\ket{\gamma_3}\Xi'=\big(c_1\ket{\psi_1^\text{sys}}\ket{0_2}+c_2\ket{0_1}\ket{\psi_2^\text{sys}}\big)\Xi
 \end{equation*}
where we have absorbed the $\ket{\gamma_3}$-state into $\Xi$ (i.e. $\Xi= \ket{\gamma_3}\Xi'$).

Now as it stands, the state $\ket{\Psi_n}$ describing $S_n(y^a_1)$ has a definite mass-energy density $\tau_S(x)$ for $x\in S_n(y^a_1)\cap S$, namely $0$. Thus, if $\pi_n$ is the operator featuring in (\ref{kentconsistency0}) that corresponds to this definite mass-energy density, then $\pi_n\ket{\Psi_n}=\ket{\Psi_n}$. Therefore, equation (\ref{kentconsistency0}) for Kent's beables tells us that
$$\ev*{T^{\mu\nu}(y^a_1)}_{\tau_S}=\ev*{\hat{T}^{\mu\nu}(y^a_1)}{\Psi_n},$$ 
where we have also used the fact that $\ip*{\Psi_n}{\Psi_n}=1.$

Now as we saw in section \ref{decotheory}, if we are interested only in the expectation values of observables for a system $\mathcal{S}$ contained within a universe $\mathcal{U}=\mathcal{S}+\mathcal{E}$, then the information needed to do this can be encapsulated in the reduced density matrix for $\mathcal{S}$. Thus, if the universe is described by a state 
$\ket{\Psi}=\sum_j c_j \ket{\psi_j}_\mathcal{S}\ket{E_j}$ with corresponding density matrix $\hat{\rho}=\dyad{\Psi}\in M(H_\mathcal{U})$, then the reduced density matrix $\hat{\rho}_\mathcal{S}\in M(H_\mathcal{S})$ is the Hermitian operator acting on the state space $H_\mathcal{S}$ with the property that 
\begin{equation}\tag{\ref{reducedev} revisited}
\ev*{\hat{O}_\mathcal{U}}_\rho=\Tr_\mathcal{S}(\hat{\rho}_\mathcal{S}\hat{O}_\mathcal{S})
\end{equation}
where $\hat{O}_\mathcal{S}$ is an observable on $H_\mathcal{S}$,  and $\hat{O}_\mathcal{U}$ is the corresponding observable on $H_\mathcal{U}$. Furthermore, we also have
\begin{equation}\label{reduced2}
\hat{\rho}_\mathcal{S}=\sum_j \abs{c_j}^2\dyad{\psi_j}+\sum_{j\neq k} c_j\overline{c_k}\ip{E_k}{E_j}\dyad{\psi_j}{\psi_k}.\protect\footnotemark
\end{equation}
\footnotetext{cf. (\ref{reduced})}We can thus apply this to the situation at hand by taking $S_n$ to be our universe $\mathcal{U}$ and $y^a_1$ to be the system $\mathcal{S}$, and $S_n\setminus \{y^a_1\}$ to be the environment $\mathcal{E}$. If we assume that $\ip{0_2}{\psi_2^\text{sys}}= 0$, then by (\ref{reduced2}), the corresponding reduced density matrix $\hat{\rho}_{y^a_1}$ takes the form of an improper mixture
\begin{equation}\label{kentred}
\hat{\rho}_{y^a_1}= \abs{c_1}^2\dyad{\psi_1^\text{sys}}+\abs{c_2}^2\dyad{0_1}.
\end{equation}
Kent's beables at $y^a_1$ will thus take the form 
\begin{equation}\label{kentbe}
\begin{split}
\ev*{T^{\mu\nu}(y^a_1)}_{\tau_S}&=\Tr_{y^a_1}(\hat{\rho}_{y^a_1}\hat{T}^{\mu\nu}(y^a_1))\\
&=\abs{c_1}^2\ev*{\hat{T}^{\mu\nu}(y^a_1)}{\psi_1^\text{sys}}+\abs{c_2}^2\ev*{\hat{T}^{\mu\nu}(y^a_1)}{0_1}.
\end{split}
\end{equation}

Let us now consider Kent's beables at the spacetime location $y^b_1$ depicted in figure \ref{kentdecoh2}.
 
\begin{figure}[ht!]
\captionsetup{justification=justified}
\centering
\tikzmath{
\a=5.3;  
\psione=0;
\psitwo=1.5;
\h=1;
\labelx=(\psione+\psitwo)/2;
\labely=-1.5;
\phstartx=-1.7;
\phstarty=\h;
\tbeg=\phstarty;
\ca=\phstartx-\tbeg;
\ttwo=\psitwo-\ca;
\cb=\ttwo+\ca;
\xend=\ttwo-\a+\cb;
\tone=\psione-\ca;
\cc=\tone+\ca;
\xendz=\tone-\a+\cc;
\tthree=\cc+\tone-\psitwo;
\circsize=0.08;
\md = (\a+\h)/2;
\tlen=0.75;
\textscale = 0.7;
\picscale = 0.89;
\nudge=0.1;
\tnudge=(\ttwo-\tone)/2+0.8;
\ttesta=2*\tone-\ttwo-\tnudge;
\ttestb=2*\tone-\ttwo+\tnudge;
\ttestc=\ttwo-\tnudge;
\ttestd=\ttwo+\tnudge;
\sonea=\a+\psione-\ttesta;
\sadd=0.5;
\lrange = -(\psione-\a+\ttesta)+\sadd;
\rrange =\lrange; 
\timex=\rrange-1.5;
\e = 0.4;
\ttestx=3.0;
\at=\a-\ttestx;
\ct=\ttestx;
\lam = 0.87;
\e = 0.2;
\hae=(2*pow(\at,3)*\lam*(2+\lam)+\e*\e*(2*\e-sqrt(4*\e*\e+\at*\at*\lam*\lam))-8*\at*\e*(-2*\e+sqrt(4*\e*\e+\at*\at*\lam*\lam))-2*\at*\at*(2+\lam)*(-2*\e+sqrt(4*\e*\e+\at*\at*\lam*\lam)))/(4*\at*\lam*(2*\at+2*\e-sqrt(4*\e*\e+\at*\at*\lam*\lam)))-(\e*\e/4+\at*\at*(-1+\lam))/(\at*\lam);
\mmu=(\tone-\tbeg)/(\psione-\phstartx);
\nnu=(\psione*\tbeg-\phstartx*\tone)/(\psione-\phstartx);
\muf=(\a-\tone)/(\xendz-\psione);
\nuf=(\xendz*\tone-\psione*\a)/(\xendz-\psione);
} 
\tikzstyle{scorestars}=[star, star points=5, star point ratio=2.25, draw, inner sep=1pt]%
\tikzstyle{scoresquare}=[draw, rectangle, minimum size=1mm,inner sep=0pt,outer sep=0pt]%
\begin{tikzpicture}[scale=\picscale,
declare function={
	testxonep(\ps,\t)=\a+\ps-\t;
	testxonem(\ps,\t)=\ps-\a+\t; 
	xa(\m,\n)=(-\ct*\m+\e*\m+\m*\n-sqrt(\ct*\ct*\lam*\lam+\e*\e*\m*\m+2*\e*\lam*\lam*\n+\lam*\lam*\n*\n-2*\ct*\lam*\lam*(\e+\n)))/(\lam*\lam-\m*\m);
	xb(\m,\n)=(-\ct*\m+\e*\m+\m*\n+sqrt(\ct*\ct*\lam*\lam+\e*\e*\m*\m+2*\e*\lam*\lam*\n+\lam*\lam*\n*\n-2*\ct*\lam*\lam*(\e+\n)))/(\lam*\lam-\m*\m);
	bl(\x)=\ct+\at-\e/2-1/2*(sqrt(\lam*\lam*pow(\x+\hae+\at,2)+\e*\e)-\e+\lam*(\x+\hae+\at));
	br(\x)=\ct+\at-\e/2-1/2*(sqrt(\lam*\lam*pow(\x-\hae-\at,2)+\e*\e)-\e-\lam*(\x-\hae-\at));
	bc(\x)=\ct+sqrt(\lam*\lam*\x*\x+\e*\e)-\e;
},
 ] 
\definecolor{tempcolor}{RGB}{250,190,0}
\definecolor{darkgreen}{RGB}{40,190,40}

\draw[->,blue, thick] [domain=-\at/2:-\lrange, samples=150]   plot (\x, {bl(\x)})  ;
\draw[blue, thick] [domain=-\at/2:\at/2, samples=150] plot (\x, {bc(\x)})  node[right=20, above=10,black,scale=\textscale]{}  ;
\draw[->,blue, thick] [domain=\at/2:\rrange, samples=150]   plot (\x, {br(\x)})   ;

 \node[scale=\textscale]  at (2.7,4.6) {$S_n(y^b_1)$}; 
\draw[<->] (-\lrange, \h) node[left, scale=\textscale] {$S_0$} -- (\rrange, \h) node[right, scale=\textscale] {$S_0$};
\draw[<->] (-\lrange, \a) node[left, scale=\textscale] {$S$} -- (\rrange, \a) node[right, scale=\textscale] {$S$};
              
\draw[->, shorten <= 5pt,  shorten >= 1pt] (\psione,\h) node[below, scale=\textscale]{$\psi_1^{\text{sys}}$} node[above left, scale=\textscale]{$z_1$}-- (\psione,\a) ;
\draw[->,  shorten <= 5pt,  shorten >= 1pt] (\psitwo,\h) node[below, scale=\textscale]{$\psi_2^{\text{sys}}$} node[above right, scale=\textscale]{$z_2$}-- (\psitwo,\a);

\draw[dashed, tempcolor,  thick](\phstartx,\tbeg) node[below left=-2,scale=\textscale]{}--(\psione,\tone) node[above, pos=0.3, rotate=45,scale=0.7,black] {} node[below, pos=0.3, rotate=45,scale=0.7,black] {};
\draw[dashed, tempcolor,  thick](\psione,\tone) node[below left=-2,scale=\textscale]{}--(\psitwo,\ttwo );
\draw[dashed, tempcolor,  thick](\psitwo,\ttwo)--(\xend,\a); 
\draw[dashed, tempcolor, thick](\psione,\tone)--(\xendz,\a);





\draw[->] (\timex,\md-\tlen/2) --  (\timex,\md+\tlen/2) node[midway,right, scale=\textscale]{time}; 
 
\draw[dotted](\psione,\ttestx)--({testxonep(\psione,\ttestx)},\a);
\draw[dotted](\psione,\ttestx)--({testxonem(\psione,\ttestx)},\a);
%\draw [black,fill](\psione,\ttestx) circle [radius=\circsize] node [black,below=6,left,scale=\textscale] {$y^b_1$}; 
\draw (\psione,\ttestx) node[ scoresquare, fill=gray]  {} node [black,below=6,right,scale=\textscale] {$y^b_1$};
\draw (\psitwo, {bc(\psitwo)}) node[ scoresquare, fill=gray]  {} node [black,below=6,right,scale=\textscale] {$y^b_2$};
\draw ({xa(\mmu,\nnu)}, {xa(\mmu,\nnu)*\mmu+\nnu}) node[ scoresquare, fill=gray]  {} node [black,below=3,scale=\textscale] {$y^b_4$};
\draw ({xb(\muf,\nuf)}, {xb(\muf,\nuf)*\muf+\nuf}) node[ scoresquare, fill=gray]  {} node [black,below=3,scale=\textscale] {$y^b_3$};

\draw[magenta,->,ultra thick] ({testxonem(\psione,\ttestx)},\a)--(-\lrange,\a);
\draw[magenta,->,ultra thick] ({testxonep(\psione,\ttestx)},\a)--(\rrange,\a);
%\draw [black,fill] (\xendz,\a) circle [radius=\circsize] node [black,above,scale=\textscale] {$\gamma_1$}; 
\end{tikzpicture}% pic 1

\vspace*{2px}
\caption{Depiction of a superposition of two local states at $z_1$ and $z_2$ with $S_n(y^b_1)$ being after the photon has interacted without the photon intersecting $S_n(y^b_1)\cap S$. The gray squares indicate cells in $S^1(y^b_1)$ whose states are among the summands in (\ref{Sistate}).}
\label{kentdecoh2}
\end{figure}
The state of $S_n(y^b_1)$ will then be
 \begin{equation*}
 \ket{\Psi_n}=\big(c_1\ket{\psi_1^\text{sys}}\ket{0_2}\ket{\gamma_3}\ket{0_4}+c_2\ket{0_1}\ket{\psi_2^\text{sys}}\ket{0_3}\ket{\gamma_4}\big)\Xi
 \end{equation*}
 where the notation is analogous to that in the previous example. Since no photon detections are registered on $S_n(y^b_1)\cap S$, we again have $\pi_n\ket{\Psi_n}=\ket{\Psi_n}$ so that the reduced density matrix $\hat{\rho}_{y_1^b}$ will again be given by  (\ref{kentred}) with $y^a_1$ replaced by $y^b_1$. However, in this case, Kent's beables $\ev*{T^{\mu\nu}(y^b_1)}_{\tau_S}$ will not be given by (\ref{kentbe}) because in the limit as $n\rightarrow\infty$, the photon \emph{will} be registered on $S_n(y^b_1)\cap S$. 
 
 To deal with the case when a photon is registered on $S_n(y^b_1)\cap S$, we consider a third example as depicted in figure \ref{kentdecoh3}.


\begin{figure}[ht!]
\captionsetup{justification=justified}
\centering
\tikzmath{
\a=5.3;  
\psione=0;
\psitwo=1.5;
\h=1;
\labelx=(\psione+\psitwo)/2;
\labely=-1.5;
\phstartx=-1.7;
\phstarty=\h;
\tbeg=\phstarty;
\ca=\phstartx-\tbeg;
\ttwo=\psitwo-\ca;
\cb=\ttwo+\ca;
\xend=\ttwo-\a+\cb;
\tone=\psione-\ca;
\cc=\tone+\ca;
\xendz=\tone-\a+\cc;
\tthree=\cc+\tone-\psitwo;
\circsize=0.08;
\md = (\a+\h)/2;
\tlen=0.75;
\textscale = 0.7;
\picscale = 0.89;
\nudge=0.1;
\tnudge=(\ttwo-\tone)/2+0.8;
\ttesta=2*\tone-\ttwo-\tnudge;
\ttestb=2*\tone-\ttwo+\tnudge;
\ttestc=\ttwo-\tnudge;
\ttestd=\ttwo+\tnudge;
\sonea=\a+\psione-\ttesta;
\sadd=0.5;
\lrange = -(\psione-\a+\ttesta)+\sadd;
\rrange =\lrange; 
\timex=\rrange-1.5;
\e = 0.4;
\ttestx=4;
\at=\a-\ttestx;
\ct=\ttestx;
\lam = 0.87;
\e = 0.2;
\hae=(2*pow(\at,3)*\lam*(2+\lam)+\e*\e*(2*\e-sqrt(4*\e*\e+\at*\at*\lam*\lam))-8*\at*\e*(-2*\e+sqrt(4*\e*\e+\at*\at*\lam*\lam))-2*\at*\at*(2+\lam)*(-2*\e+sqrt(4*\e*\e+\at*\at*\lam*\lam)))/(4*\at*\lam*(2*\at+2*\e-sqrt(4*\e*\e+\at*\at*\lam*\lam)))-(\e*\e/4+\at*\at*(-1+\lam))/(\at*\lam);
\mmu=(\tone-\tbeg)/(\psione-\phstartx);
\nnu=(\psione*\tbeg-\phstartx*\tone)/(\psione-\phstartx);
\muf=(\a-\ttwo)/(\xend-\psitwo);
\nuf=(\xend*\ttwo-\psitwo*\a)/(\xend-\psitwo);
} 
\tikzstyle{scorestars}=[star, star points=5, star point ratio=2.25, draw, inner sep=1pt]%
\tikzstyle{scoresquare}=[draw, rectangle, minimum size=1mm,inner sep=0pt,outer sep=0pt]%
\begin{tikzpicture}[scale=\picscale,
declare function={
	testxonep(\ps,\t)=\a+\ps-\t;
	testxonem(\ps,\t)=\ps-\a+\t; 
	xa(\m,\n)=(-\ct*\m+\e*\m+\m*\n-sqrt(\ct*\ct*\lam*\lam+\e*\e*\m*\m+2*\e*\lam*\lam*\n+\lam*\lam*\n*\n-2*\ct*\lam*\lam*(\e+\n)))/(\lam*\lam-\m*\m);
	xb(\m,\n)=(-\ct*\m+\e*\m+\m*\n+sqrt(\ct*\ct*\lam*\lam+\e*\e*\m*\m+2*\e*\lam*\lam*\n+\lam*\lam*\n*\n-2*\ct*\lam*\lam*(\e+\n)))/(\lam*\lam-\m*\m);
	bl(\x)=\ct+\at-\e/2-1/2*(sqrt(\lam*\lam*pow(\x+\hae+\at,2)+\e*\e)-\e+\lam*(\x+\hae+\at));
	br(\x)=\ct+\at-\e/2-1/2*(sqrt(\lam*\lam*pow(\x-\hae-\at,2)+\e*\e)-\e-\lam*(\x-\hae-\at));
	bc(\x)=\ct+sqrt(\lam*\lam*\x*\x+\e*\e)-\e;
},
 ] 
\definecolor{tempcolor}{RGB}{250,190,0}
\definecolor{darkgreen}{RGB}{40,190,40}

\draw[->,blue, thick] [domain=-\at/2:-\lrange, samples=150]   plot (\x, {bl(\x)})  ;
\draw[blue, thick] [domain=-\at/2:\at/2, samples=150] plot (\x, {bc(\x)})  node[right=20, above=10,black,scale=\textscale]{}  ;
\draw[->,blue, thick] [domain=\at/2:\rrange, samples=150]   plot (\x, {br(\x)})   ;

 \node[scale=\textscale]  at (1,4.1) {$S_n(y^c_1)$}; 
\draw[<->] (-\lrange, \h) node[left, scale=\textscale] {$S_0$} -- (\rrange, \h) node[right, scale=\textscale] {$S_0$};
\draw[<->] (-\lrange, \a) node[left, scale=\textscale] {$S$} -- (\rrange, \a) node[right, scale=\textscale] {$S$};
              
\draw[->, shorten <= 5pt,  shorten >= 1pt] (\psione,\h) node[below, scale=\textscale]{$\psi_1^{\text{sys}}$} node[above left, scale=\textscale]{$z_1$}-- (\psione,\a) ;
\draw[->,  shorten <= 5pt,  shorten >= 1pt] (\psitwo,\h) node[below, scale=\textscale]{$\psi_2^{\text{sys}}$} node[above right, scale=\textscale]{$z_2$}-- (\psitwo,\a);

\draw[dashed, tempcolor,  thick](\phstartx,\tbeg) node[below left=-2,scale=\textscale]{}--(\psione,\tone) node[above, pos=0.3, rotate=45,scale=0.7,black] {} node[below, pos=0.3, rotate=45,scale=0.7,black] {};
\draw[dashed, tempcolor,  thick](\psione,\tone) node[below left=-2,scale=\textscale]{}--(\psitwo,\ttwo );
\draw[dashed, tempcolor,  thick](\psitwo,\ttwo)--(\xend,\a); 
\draw[dashed, tempcolor,  thick](\psione,\tone)--(\xendz,\a);

\draw[magenta,->,ultra thick] ({testxonem(\psione,\ttestx)},\a)--(-\lrange,\a);
\draw[magenta,->,ultra thick] ({testxonep(\psione,\ttestx)},\a)--(\rrange,\a);



\draw[->] (\timex,\md-\tlen/2) --  (\timex,\md+\tlen/2) node[midway,right, scale=\textscale]{time}; 
 
\draw[dotted](\psione,\ttestx)--({testxonep(\psione,\ttestx)},\a);
\draw[dotted](\psione,\ttestx)--({testxonem(\psione,\ttestx)},\a);
%\draw [black,fill](\psione,\ttestx) circle [radius=\circsize] node [black,below=6,left,scale=\textscale] {$y^c_1$}; 
\draw (\psione,\ttestx) node[ scoresquare, fill=gray]  {} node [black,below=6,right,scale=\textscale] {$y^c_1$};
\draw (\psitwo, {bc(\psitwo)}) node[ scoresquare, fill=gray]  {} node [black,below=6,right,scale=\textscale] {$y^c_2$};
\draw ({xb(\muf,\nuf)}, {xb(\muf,\nuf)*\muf+\nuf}) node[ scoresquare, fill=gray]  {} node [black,right=7, below=-4,scale=\textscale] {$y^c_4$};
\draw (\xendz,\a) node[ scoresquare, fill=gray]  {} node [black,below=3,scale=\textscale] {$y^c_3$};


%\draw [black,fill] (\xendz,\a) circle [radius=\circsize] node [black,above,scale=\textscale] {$\gamma_1$}; 
\end{tikzpicture}% pic 1

\vspace*{2px}
\caption{Depiction of a superposition of two local states at $z_1$ and $z_2$ with $y^c_1$ sufficiently late that the photon intersects $S_n(y^c_1)\cap S$. The gray squares indicate cells in $S^1(y^c_1)$ whose states are among the summands in (\ref{Sistate})}
\label{kentdecoh3}
\end{figure}

In this case, the state of $S_n(y^c_1)$ will be
 \begin{equation*}
 \ket{\Psi_n}=\big(c_1\ket{\psi_1^\text{sys}}\ket{0_2}\ket{\gamma_3}\ket{0_4}+c_2\ket{0_1}\ket{\psi_2^\text{sys}}\ket{0_3}\ket{\gamma_4}\big)\Xi
 \end{equation*}
 but now we have to consider the fact that the photon intersects $S_n(y^c_1)\cap S$. There are two possible (notional) measurement outcomes that can occur on $S_n(y^c_1)\cap S$: either $T_S=\tau_{S,1}$ where $\tau_{S,1}(y^c_3)\neq 0$, or $T_S=\tau_{S,2}$ where $\tau_{S,2}(y^c_3)=0.$ 
 
 The case  $T_S=\tau_{S,1}$ indicates that there is a photon detection at $y^c_3$ so that the local state at the $y^c_3$-cell is $\ket{\gamma_3}$. Therefore, if we write $\pi_{n,1}$ for the operator $\pi_n$, we have 
 $$\pi_{n,1}\ket{\Psi_n}=c_1\ket{\psi_1^\text{sys}}\ket{0_2}\ket{\gamma_3}\ket{0_4}\Xi.$$
 Therefore, 
 $\ev*{\pi_{n,1}\hat{T}^{\mu\nu}(y^c_1)}{\Psi_n}=\abs{c_1}^2\ev*{\hat{T}^{\mu\nu}(y^c_1)}{\psi_1^\text{sys}}$ and  $\ev*{\pi_{n,1}}{\Psi_n}=\abs{c_1}^2$. Hence, by (\ref{kentconsistency0}), Kent's beables at $y^c_1$ will be 
 $$\ev*{T^{\mu\nu}(y^c_1)}_{\tau_{S,1}}=\ev*{\hat{T}^{\mu\nu}(y^c_1)}{\psi_1^\text{sys}}.$$ 
 From this, it follows that the reduced density matrix at $y^c_1$ will take the form of a pure state:
 \begin{equation}\label{purerho}
\hat{\rho}_{y^c_1}= \dyad{\psi_1^\text{sys}}.
\end{equation} 
 On the other hand, for the case when  $T_S=\tau_{S,2}$, this indicates that there is no photon detection at $y^c_3$, so that the local state at the $y^c_3$-cell will be $\ket{0_3}$. So if we now  write $\pi_{n,2}$ for the operator $\pi_n$, we have 
 $$\pi_{n,2}\ket{\Psi_n}=c_2\ket{0_1}\ket{\psi_2^\text{sys}}\ket{0_3}\ket{\gamma_4}\Xi.$$
 Therefore, 
 $\ev*{\pi_{n,2}\hat{T}^{\mu\nu}(y^c_1)}{\Psi_n}=\abs{c_2}^2\ev*{\hat{T}^{\mu\nu}(y^c_1)}{0_1}$ and  $\ev*{\pi_{n,2}}{\Psi_n}=\abs{c_2}^2$,  
  and so by (\ref{kentconsistency0}), Kent's beables at $y^c_1$ will be 
 $$\ev*{T^{\mu\nu}(y^c_1)}_{\tau_{S,2}}=\ev*{\hat{T}^{\mu\nu}(y^c_1)}{0_1}.$$
 In this case, the reduced density matrix at $y^c_1$  will be
  \begin{equation}
\hat{\rho}_{y^c_1}= \dyad{0_1},
\end{equation} 
which is again a pure state.

In these examples we have therefore seen how the additional information concerning photon detection on $S_n(y_1)\cap S$ is able to determine whether the reduced density matrix at $y_1$ is a pure state or an improper mixture. Hence, Kent's theory offers an answer to d'Espagnat's problem of outcomes. As mentioned in section \ref{probOutcomes}, d'Espagnat noticed that with decoherence theory alone, we are not entitled to give an ignorance interpretation to the reduced density matrix for a system that is an improper mixture, and thus we are not able to conclude from the reduced density matrix alone that an outcome has occurred. However, if the reduced density matrix of a system goes from being an improper mixture to a pure state of the form $\dyad{\psi}$ as it does when Kent's additional information is taken into account, then we can say that an outcome has occurred, namely the outcome of the system being in the state $\ket{\psi}.$  




\section{Butterfield's Analysis of Outcome Independence  in Kent's theory}
Let us now consider Kent's theory in the light of Shimony's notion of Outcome Independence (OI)  as defined in section \ref{OISec}. 

Butterfield\footnote{See \cite[30-32]{Butterfield}} tries to answer the question of whether OI holds in Kent's theory by considering an example that builds on Kent's toy model. Butterfield's example is designed to capture the salient features of a Bell experiment where two spatially separated observers always observe opposite outcomes of some measurement. Following Kent, Butterfield thus considers a universe in one spatial dimensional. In this universe, there are two entangled systems, a left-system and a right-system as depicted in figure \ref{ButterfieldOI}.

\begin{figure}[ht!]
	\captionsetup{justification=justified}
	\centering
	\tikzmath{
	\a=5.3;  
	\psione=-1;
	\psitwo=-0.5;
	\psioneq=2;
	\psitwoq=2.5;
	\psioner=0;
	\psitwor=1;
	\e = 0.1;
	\h=-0.8;
	\phstartx=-3.9;
	\phstarty=-.8;
	\tbeg=\phstarty;
	\ca=\phstartx-\tbeg;
	\ttwo=\psitwo-\ca;
	\cb=\ttwo+\ca;
	\xend=\ttwo-\a+\cb;
	\tone=\psione-\ca;
	\cc=\tone+\ca;
	\xendz=\tone-\a+\cc;
	\tthree=\cc+\tone-\psitwo;
	\phstartxq=\psitwoq+2.9;
	\phstartyq=-.8;
	\tbegq=\phstartyq;
	\ttwoq=\tbegq-(\psitwoq-\phstartxq);
	\xendq=-\ttwoq+\a+\psitwoq;
	\toneq=\tbegq-(\psioneq-\phstartxq);
	\xendzq=-\toneq+\a+\psioneq;
	\circsize=0.05;
	\md = (\a+\h)/2;
	\tlen=0.75;
	\textscale = 0.8;
	\picscale = 0.95;
	\nudge=0.1;
	\tnudge=(\ttwo-\tone)/2;
	\ttesta=2*\tone-\ttwo-\tnudge;
	\ttestb=2*\tone-\ttwo+\tnudge;
	\ttestc=\ttwo-\tnudge;
	\ttestd=\ttwo+\tnudge;
	\sonea=\a+\psione-\ttesta;
	\sadd=0.5;
	\lrange = -(\psioner-\a+\ttesta)+\sadd;
	\rrange =\a+\psitwor-\ttesta+\sadd+0.5; 
	\timex=\rrange-0.7;
	} 
	\begin{tikzpicture}[scale=1.2,
	declare function={
		testxonep(\ps,\t)=\a+\ps-\t;
		testxonem(\ps,\t)=\ps-\a+\t; 
	},
	 ] 
	
	 \definecolor{tempcolor}{RGB}{250,190,0}
	 \definecolor{darkgreen}{RGB}{40,190,40}
	 \draw[<->] (-\lrange, \h) node[left, scale=\textscale] {$S_0$} -- (\rrange, \h) node[right, scale=\textscale] {$S_0$};
	 \draw[<->] (-\lrange, \a) node[left, scale=\textscale] {$S$} -- (\rrange, \a) node[right, scale=\textscale] {$S$};
	 
				   
	 \draw[->, shorten <= 5pt,  shorten >= 1pt] (\psione,\h) node[below, scale=\textscale]{$\psi_1^{\text{sys}}$} node[above left, scale=\textscale]{$Z=z_1$}-- (\psione,\a);
	 \draw[->,  shorten <= 5pt,  shorten >= 1pt] (\psitwo,\h) node[below, scale=\textscale]{$\psi_2^{\text{sys}}$} node[above right, scale=\textscale]{$Z=z_2$}-- (\psitwo,\a);
	 
	 \draw[->, shorten <= 5pt,  shorten >= 1pt] (\psioneq,\h) node[below, scale=\textscale]{$\psi_3^{\text{sys}}$} node[above left, scale=\textscale]{$Z=z_3$}-- (\psioneq,\a);
	 \draw[->,  shorten <= 5pt,  shorten >= 1pt] (\psitwoq,\h) node[below, scale=\textscale]{$\psi_4^{\text{sys}}$} node[above right, scale=\textscale]{$Z=z_4$}-- (\psitwoq,\a);
	 
	 %\draw[dashed, tempcolor, ultra thick](\phstartx,\tbeg) node[below left=-2,scale=\textscale]{}--(\psitwo,\ttwo );
	 \draw[dashed, tempcolor, ultra thick](\phstartx,\tbeg) node[below left=-2,scale=\textscale]{}--(\psione,\tone );
	 \draw[dashed, gray, ultra thick](\psitwo,\ttwo)--(\xend,\a);
	 \draw[dashed, gray, ultra thick](\psione,\tone)--(\psitwo,\ttwo) ;
	 \draw[dashed, tempcolor, ultra thick](\psione,\tone)--(\xendz,\a) ;
	 
	 \draw [red,fill] (\xendz,\a) circle [radius=\circsize] node [black,above,scale=\textscale] {$\gamma_1$}; 
	 \draw [black,fill] (\xend,\a) circle [radius=\circsize] node [black,above,scale=\textscale] {$\gamma_2$}; 
	 
	 \draw[dashed, tempcolor, ultra thick](\phstartxq,\tbegq) node[below left=-2,scale=\textscale]{}--(\psitwoq,\ttwoq );
	 \draw[dashed, tempcolor, ultra thick](\psitwoq,\ttwoq)--(\xendq,\a)  ;
	 \draw[dashed, gray, ultra thick](\psioneq,\toneq)--(\psitwoq,\ttwoq) ;
	 \draw[dashed, gray, ultra thick](\psioneq,\toneq)--(\xendzq,\a) ;
	 
	 \draw [black,fill] (\xendzq,\a) circle [radius=\circsize] node [black,above,scale=\textscale] {$\gamma_3$}; 
	 \draw [red,fill] (\xendq,\a) circle [radius=\circsize] node [black,above,scale=\textscale] {$\gamma_4$}; 
	 %\draw[dashed, darkgreen, ultra thick](\psione,\tone)--(\psitwo,\tthree);
	 %\draw [black,fill] (\psitwo,\tthree) circle [radius=\circsize] node [black,below=4,right,scale=\textscale] {$t=2t_1-t_2$}; 
	 
	 
	 \draw [red,fill] (\psione,\tone) circle [radius=\circsize]; 
	 \draw [black,fill](\psitwo,\ttwo)circle [radius=\circsize]; 
	 
	 \draw [black,fill] (\psioneq,\toneq) circle [radius=\circsize]; 
	 \draw [red,fill](\psitwoq,\ttwoq)circle [radius=\circsize]; 
	 
	 %\draw [decorate, decoration = {calligraphic brace}] (\psitwo+\nudge,\ttwo) --  (\psitwo+\nudge,\tone+\nudge/4) node[midway,right, scale=\textscale]{$t_2-t_1$}; 
	 %\draw [decorate, decoration = {calligraphic brace}] (\psitwo+\nudge,\tone-\nudge/4) --  (\psitwo+\nudge,2*\tone-\ttwo) node[midway,right, scale=\textscale]{$t_2-t_1$}; 

	\draw[->] (\timex,\md-\tlen/2) --  (\timex,\md+\tlen/2) node[midway,right, scale=\textscale]{time}; 
	 
	%\draw[dotted](\psione,\ttesta)--({testxonep(\psione,\ttesta)},\a);
	%\draw[dotted](\psione,\ttesta)--({testxonem(\psione,\ttesta)},\a);
	%\draw [black,fill](\psione,\ttesta) circle [radius=\circsize] node [black,below=5,right,scale=\textscale] {$y_a$}; 
	
	%\draw[magenta,->,ultra thick] ({testxonem(\psione,\ttesta)},\a)--(-\lrange,\a);
	%\draw[magenta,->,ultra thick] ({testxonep(\psione,\ttesta)},\a)--(\rrange,\a);
	
	\end{tikzpicture}% pic 1
	
	\vspace*{2px}
	\caption{Butterfield's thought experiment for analyzing OI}
	\label{ButterfieldOI}
	\end{figure}

Two locations $z_1$ and $z_2$ with $z_2>z_1$ belong to a left-system, and there are two possible outcomes for a measurement on the left-system: either all the mass/energy of the left-system is localized at $z_1$ or all the mass/energy of the left-system  is localized at $z_2$. These two possibilities are analogous to a spin up or a spin down measurement outcome in a Stern-Gerlach statement. Likewise, two locations $z_3$ and $z_4$ with $z_3<z_4$ and $z_3\gg z_2$ belong to a right-system, and again, there are two possible measurement outcomes: either all the mass/energy of the right-system is localized at $z_3$ or all the mass/energy of the right-system  is localized at $z_4$. 

The initial joint state of the two systems is 
$a \psi_1\psi_4 +b \psi_2\psi_3.$
This means that the left-system will be found to be localized at $z_1$ with probability $\abs{a}^2$, and at $z_2$ with probability $\abs{b}^2$, and if the left-system is localized at $z_1$, the  right system must be localized at $z_4$, whereas if the left-system is localized at $z_2$, then the right system must be localized at $z_3$.  

Now Butterfield supposes that there are two photons, one coming in from the left that interacts with the left system, and one coming in from the right that interacts with the right system. As in Kent's toy model, there is a late time spacelike hypersurface $S$, on which the photons are ``measured''. Since the joint state of the two systems  is in superposition, there will be two possible measurement outcomes for the two photons that arrive at $S$. Either the left-photon is measured at $\gamma_1$ and the right-photon is measured at $\gamma_4$, or the left-photon is measured at $\gamma_2$ and the right photon is measured at $\gamma_3$. Thus, if we suppose that the (notional) measurement for $T_S(x)$ yields an energy distribution $\tau_S(x)$ that is nonzero at $\gamma_1$ and $\gamma_4$, but is zero at $\gamma_2$ and $\gamma_3$, then we can say that the outcome of the measurement on the two systems is that the left system is localized at $z_1$ and the right system is localized at $z_4$. Moreover, the probability of this outcome is $1$ given that the (notional) measurement of $T_S(x)$ on $S$ is $\tau_S(x)$. In other words, this model is deterministic. But as we saw on page \pageref{OIdet}, if a model is deterministic, then OI must hold. This is the conclusion that Butterfield draws. 

Now if Kent's theory is to be consistent with special relativity, OI being satisfied might initially seem concerning. Indeed, we saw in section \pageref{OISec} that OI implies the negation of PI, and the negation of PI is not consistent with special relativity. However, there is one salient feature of a Bell experiment that is not captured in Butterfield's scenario, namely, in a Bell experiment, one can perform different measurements.  PI and its negation only make sense when there are parameters that can be changed. Furthermore, in the proof that OI implies the negation of PI,\footnote{The proof  that determinism implies the negation of PI (on pages \pageref{bellinequality2} to \pageref{PIdeterminism}), also assumes that the choice of parameter is not determined by the hidden variable $\lambda$.} it is assumed that the choice of parameter is not determined by the hidden variable $\lambda$. If the choice of parameters did depend on $\lambda$, then for $\hat{a}\neq\hat{b}$, at least one of the probabilities $P_{\lambda,\bm{\hat{a}},\bm{\hat{c}}}(\uvbp{a};\uvbp{c}),$ $P_{\lambda,\bm{\hat{c}},\bm{\hat{b}}}(\uvbp{c};\uvbp{b})$ or $P_{\lambda,\bm{\hat{a}},\bm{\hat{b}}}(\uvbp{a};\uvbp{b})$ would not be well-defined.\footnote{For example, if we thought of $P_{\lambda,\bm{\hat{a}},\bm{\hat{b}}}(X;Y)$ as a conditional probability $P(X;Y|\lambda,\bm{\hat{a}},\bm{\hat{b}})$ and the probability $P(\lambda,\bm{\hat{a}},\bm{\hat{b}})=0$, then according to the definition of conditional probability,  $P(X;Y|\lambda,\bm{\hat{a}},\bm{\hat{b}})=\frac{0}{0}.$} Even though Butterfield is only considering OI in his thought experiment, a proper analysis of OI shouldn't be undertaken without considering an experiment with parameters (e.g. knob settings that correspond to measurement axes of a Stern-Gerlach experiment). This is because the determination of whether OI holds will depend on what one counts as being the hidden variable of a system, and we need the hidden variable of a system to be such that the notion of PI is well-defined. Otherwise, one's verdict on OI will be irrelevant to Shimony's analysis of why Bell's inequality fails to hold. 


\section{Hidden variables and the Colbeck-Renner theorem}
Butterfield assumes that the hidden variables in Kent's theory consist in the outcome $\tau_S(x)$ of $T_S(x)$ over the whole of $S$, and so far, I haven't questioned this assumption. However, this assumption is going to cause difficulties in the context of Shimony's analysis. This is because in Kent's theory, the information in $\tau_S(x)$ over the whole of $S$ clearly would determine which parameters are chosen in a Bell experiment, for this information would determine where a silver atom coming out of a Stern-Gerlach apparatus would be detected on a detection screen (as depicted in figure \ref{rotate}), and from the position of this detection, one could determine the orientation of the magnetic field used in the Stern-Gerlach experiment. So if we stipulated that $\lambda=\tau_S$ is the hidden variable of every system in Kent's theory, then Kent's theory wouldn't satisfy the preconditions necessary for defining OI and PI. This would make Kent's theory radically different from the pilot wave interpretation where one can define OI and PI because the hidden variables, being the positions and momenta of the particles, are independent of the measurement choices. An unfortunate consequence of not being able to define OI and PI is that we wouldn't be able to evaluate Kent's theory in the light of Shimony's analysis of why Bell's inequality fails to hold. 

But it is not obvious that we should stipulate that $\lambda=\tau_S$ is the hidden variable of every system in Kent's theory. Just because we give $\tau_S$ a single label $\lambda$, it doesn't follow that $\tau_S$ is a single piece of information. There is typically going to be a huge amount of information in $\tau_S$, and for a given system $\mathcal{S}$, we should  discern carefully what collection of information in $\tau_S$ should be stipulated as being the hidden variable $\lambda$ of $\mathcal{S}$. The criteria on which we should make such a decision should at least include the following:
\begin{enumerate}
	\item all the information of $\lambda$ is about $\mathcal{S}$ so that a change in $\lambda$ corresponds to a change in the system $\mathcal{S}$.\label{hidden1}
\end{enumerate} 
In the pilot wave interpretation, the positions and momenta of the particles that constitute a system would fulfil this criterion. On the other hand, all the information in $\tau_S$ of Kent's theory would not fulfil this criterion unless of course $\mathcal{S}$ was the whole universe. 

Note, however, that we don't insist that a difference in $\mathcal{S}$ entails a difference in $\lambda$. This is because a hidden-variables theory is envisaged as augmenting standard quantum theory. So in the case when $\mathcal{S}$ is not entangled with any other system, there will be a quantum state describing $\mathcal{S}$, and this quantum state can be other than it is (indicating that $\mathcal{S}$ can be in a different physical state)  whilst the hidden variable remains the same. We thus impose a second criterion for a hidden-variables theory:
\begin{enumerate}
	\setcounter{enumi}{1}
	\item \label{hidden3} If $\lambda$ is the hidden variable of a system $\mathcal{S}$ and if $\ket{\phi}$ is the quantum state of $\mathcal{S}$ or of some composite system $\mathcal{U}$ that contains $\mathcal{S}$ as a subsystem, then it is possible for there to be a different quantum state $\ket{\phi'}$ of $\mathcal{S}$ (or $\mathcal{U}$) while the hidden variable $\lambda$ remains unchanged, and it is possible for there to be a different hidden variable $\lambda$ while $\ket{\phi}$ remains unchanged.
\end{enumerate}
This criterion is satisfied in the pilot wave interpretation, since the quantum state is the pilot wave itself. The pilot could be other than it is without any of the positions and momenta of the particles changing, but changing the pilot wave would result in a physical change of the system since the pilot wave governs how  the positions and the momenta of the particles subsequently evolve over time.

Another criterion for a collection of information $\lambda$ to constitute the hidden variable of a system $\mathcal{S}$ is the following: 
\begin{enumerate}
	\setcounter{enumi}{2}
\item \label{hidden2} it should be possible to change the measurement parameters when measuring $\mathcal{S}$ without this having any affect on $\lambda$. 
\end{enumerate} 
If this criterion doesn't hold, we cannot even begin to consider whether PI holds in a given theory. In the pilot wave interpretation, the positions and momenta of the particles that constitute a system would fulfil this criterion, whereas all the information in $\tau_S$ of Kent's theory would not. We used this criterion when showing that OI implies the negation of PI. 

Closely related to criterion \ref{hidden2} is the following criterion: 
\begin{enumerate}
	\setcounter{enumi}{3}
	\item \label{hidden5} If $p_\lambda$ is the probability that a system $\mathcal{S}$ has hidden variable $\lambda$, then $p_\lambda$ must be independent of any choice of measurement made on $\mathcal{S}$.
\end{enumerate} 
We are thus assuming there is a whole range of possibilities for the hidden variable $\lambda$, but because we don't know what the hidden variable $\lambda$ is, we can only assign it a probability. Knowledge of the quantum state of the system may help us assign such a probability, but this probability cannot depend on the choice of any future measurement we might make on the system. Butterfield refers to criterion \ref{hidden5} as the 'no-conspiracy' assumption, though he adds that this is a rather unfair label since there wouldn't necessarily be anything conspiratorial if this assumption was violated.\footnote{See \cite[34]{Butterfield}.}

We should also state explicitly a fifth criterion:
\begin{enumerate}
	\setcounter{enumi}{4}
	\item \label{hidden4} Suppose  $\mathcal{A}$ is any system that is entangled with $\mathcal{S}$, and that the quantum state of the composite system $\mathcal{S}+\mathcal{A}$  is $\ket{\phi}_{\mathcal{S}+\mathcal{A}}$. Then for any measurement $O_\mathcal{S}$ on $\mathcal{S}$ and $O_\mathcal{A}$ on $\mathcal{A}$, there is a probability $P_\lambda^{\ket{\phi}_{\mathcal{S}+\mathcal{A}}}(O_\mathcal{S}=o_\mathcal{S},O_\mathcal{A}=o_\mathcal{A})$ for the joint measurement of $O_\mathcal{S}$ and $O_\mathcal{A}$ on $\mathcal{S}+\mathcal{A}$ that is a function of $\lambda$  despite $\lambda$ only referring to the system $\mathcal{S}$.	
\end{enumerate}

In addition to these five criteria for a hidden variable $\lambda$ of a system $\mathcal{S}$, it is also desirable for a hidden-variables theory to satisfy PI and empirical adequacy. We defined PI for a two-outcome measurement on page \pageref{PIdef}, but it is easy to generalize the definition of PI for measurements with more than two outcomes. Thus, using the notation of criterion \ref{hidden4} and letting $O_\mathcal{A}'$ denote a second choice of measurement  on $\mathcal{A}$,  PI states that
\begin{equation}\tag{PI}
\sum_{o_\mathcal{A}}P_\lambda^{\ket{\phi}_{\mathcal{S}+\mathcal{A}}}(O_\mathcal{S}=o_\mathcal{S},O_\mathcal{A}=o_\mathcal{A}) = \sum_{o'_\mathcal{A}}P_\lambda^{\ket{\phi}_{\mathcal{S}+\mathcal{A}}}(O_\mathcal{S}=o_\mathcal{S},O'_\mathcal{A}=o'_\mathcal{A}) 
\end{equation}
where the summations on both sides are over all the possible measurement outcomes of $O_\mathcal{A}$ and $O_\mathcal{A}'$ respectively.

As for the definition of \textbf{empirical adequacy}\index{empirical adequacy} (EA), using the notation of criteria \ref{hidden5} and \ref{hidden4}, this states that
\begin{equation}\tag{EA}\label{adeq}
	\sum_{\lambda\in\Lambda}p_\lambda P_\lambda^{\ket{\phi}_{\mathcal{S}+\mathcal{A}}}(O_\mathcal{S}=o_\mathcal{S},O_\mathcal{A}=o_\mathcal{A})=P^{\ket{\phi}_{\mathcal{S}+\mathcal{A}}}(O_\mathcal{S}=o_\mathcal{S},O_\mathcal{A}=o_\mathcal{A})
\end{equation}
where $\Lambda$ is the set of all hidden variables so that $\sum_{\lambda\in\Lambda} p_\lambda = 1$, and where 
$$P^{\ket{\phi}_{\mathcal{S}+\mathcal{A}}}(O_\mathcal{S}=o_\mathcal{S},O_\mathcal{A}=o_\mathcal{A})$$
 is the standard probability calculated using the Born Rule with the eigenstates of the observables $\hat{O}_\mathcal{S}$ and $\hat{O}_\mathcal{A}$ and the quantum state $\ket{\phi}_{\mathcal{S}+\mathcal{A}}$. EA is essentially the same as equation (\ref{bohmconsistency}). It also has some similarities with (\ref{kentconsistency}), though the main difference is the range of the summation -- the index of the summands of (\ref{kentconsistency}) does not parametrize hidden variables that satisfy criteria \ref{hidden1} to \ref{hidden4} above.

Now it turns out that criteria \ref{hidden1} to \ref{hidden4} together with the conditions of PI and EA are very restrictive. In his 2016 paper, Leegwater proves a version of the Colbeck-Renner theorem.\footnote{See \cite{LeegwaterGijs2016Aitf}.} Leegwater's version takes the following form: if one defines hidden variables according to criteria \ref{hidden1} to \ref{hidden4}, then in any hidden-variables theory for which PI and EA hold, the hidden variables are redundant. In other words, in the notation of criterion \ref{hidden4},
\begin{equation}\label{colbeckrenner}
P_\lambda^{\ket{\phi}_{\mathcal{S}+\mathcal{A}}}(O_\mathcal{S}=o_\mathcal{S},O_\mathcal{A}=o_\mathcal{A})=P^{\ket{\phi}_{\mathcal{S}+\mathcal{A}}}(O_\mathcal{S}=o_\mathcal{S},O_\mathcal{A}=o_\mathcal{A})
\end{equation}
for any measurement $O_\mathcal{S}$ on $\mathcal{S}$ and $O_\mathcal{A}$ on $\mathcal{A}$.\footnote{Strictly speaking, we should say that equation (\ref{colbeckrenner}) holds for almost all $\lambda$, but we need not concern ourselves here with the details of measure theory that would be needed to make sense of this qualification.} 

Thus, the Colbeck-Renner theorem means that we cannot hope to make Kent's theory into a hidden-variables theory that satisfies PI and AE by simply defining more carefully what the hidden variables should be, for the information in Kent's theory is clearly non-redundant.

But nevertheless, it still seems that we should be able to make some kind of sense of PI and AE in Kent's theory and that we should be able to evaluate Kent's theory on the basis of whether these notions of PI and EA are true in this context. To achieve this aim, one strategy would be to relax one of the five criteria for a hidden variable. Since we still want to be able to make sense of PI and AE,  a process of elimination suggests that the most obvious hidden variable criterion to drop would be criterion \ref{hidden3}. In other words, instead of thinking of $\tau_S$ as an augmentation of standard quantum theory, we could instead think of $\tau_S$ as a rather elaborate way of stipulating the initial quantum states of experiments as well as the quantum states of measurement outcomes. The information of $\tau_S$ would then be non-redundant. Moreover, if we could appropriately partition the information in $\tau_S(x)$ on the basis of whether it determined the quantum state of the particle, or the quantum state of the apparatus, or the quantum state of the rest of the universe, we could then consider whether Kent's theory gave the same predictions as standard quantum theory. If it did, then PI and AE would hold in Kent's theory, since these both hold in standard quantum theory. And since Kent's theory is formulated in the Lorentz invariant setting of Schwinger and Tomonaga, this would mean that Kent's theory is a solution to the measurement problem!

\section{Kent's Theory and standard quantum theory\textsuperscript{*}}
In this section, I will show that Kent's theory does indeed give the same predictions as standard quantum theory in the case of an experimental apparatus $\mathcal{A}$ measuring the properties of a particle $\mathcal{S}$.

Let's assume that the  apparatus has already interacted with many photons during its existence up until the time $t_i$. Likewise, let's assume the particle has interacted with many photons up until this time. Now suppose that the spacelike hypersurface $S$ has energy density $\tau_S(x)$ indicating that some photons have been ``measured'' on $S$ to be in a state $\ket*{\gamma_i^{(\mathcal{A})}}$ which is correlated with the apparatus being in a state $\ket{a}$ shortly before time $t_i$ and in the vicinity of spatial location $z_1$ as depicted in figure \ref{pisolution}. Similarly, we suppose that $\tau_S(x)$ also indicates that some photons have been ``measured'' on $S$ to be in a state $\ket*{\gamma_i^{(\mathcal{S})}}$ which is correlated with the particle being in a state $\ket{s}$ also shortly before time $t_i$ and in the vicinity of spatial location $z_1$. We assume that the time $t_i$ is just before the particle enters the apparatus (given the measurement of $\ket*{\gamma_i^{(\mathcal{S})}}$ and  $\ket*{\gamma_i^{(\mathcal{A})}}$ on $S$) and that no more photons are measured on $S$ that have become entangled with particle  or the apparatus until the particle emerges from the apparatus at time $t_f$. Then some more photons interact with the apparatus $\mathcal{A}$ and get entangled with it shortly before time $t_m$, and these photons are detected on $S$ to be in a state  $\ket*{\gamma_f^{\prime\prime}}$, and this state is correlated with the apparatus now being in a state $\ket{a_{f}}$, and hence the particle being in state $\ket{s_f}$. 

\begin{figure}[ht!]
	\captionsetup{justification=justified}
	\centering
	\tikzmath{
	\a=5.3;  
	\qa=5.3;  
	\psione=0;
	\qpsione=0;
	\psitwo=1.5;
	\h=-1;
	\labelx=(\psione+\psitwo)/2;
	\labely=-1.5;
	\phstartx=-1.1;
	\qphstartx=-3.8;
	\wphstartx=-0.8;
	\phstarty=\h;
	\tbeg=\phstarty;
	\ca=\phstartx-\tbeg;
	\ttwo=\psitwo-\ca;
	\cb=\ttwo+\ca;
	\xend=\ttwo-\a+\cb;
	\tone=\psione-\ca;
	\cc=\tone+\ca;
	\xendz=\tone-\a+\cc;
	\qca=\qphstartx-\tbeg;
	\qtone=\qpsione-\qca;
	\qqcc=\qtone+\qca;
	\qxendz=\qtone-\qa+\qqcc;
	\wa=5.3;  
	\wpsione=0;
	\wttestx=3.7;
	\wca=\wphstartx-\tbeg;
	\wtone=\wpsione-\wca;
	\wwcc=\wtone+\wca;
	\wxendz=\wtone-\wa+\wwcc;
	\wat=\wa-\wttestx;
	\wct=\wttestx;
	\wlam = 0.87;
		\we=0.1;
	\tthree=\cc+\tone-\psitwo;
	\circsize=0.08;
	\md = (\a+\h)/2;
	\tlen=0.75;
	\textscale = 0.7;
	\picscale = 0.78;
	\nudge=0.1;
	\tnudge=(\ttwo-\tone)/2+0.8;
	\ttesta=2*\tone-\ttwo-\tnudge;
	\sonea=\a+\psione-\ttesta;
	\sadd=0.5;
	\lrange = -(\psione-\a+\ttesta)+\sadd;
	\rrange =\lrange; 
	\timex=\rrange-1.5;
	\ttestx=2.4;
	\at=\a-\ttestx;
	\ct=\ttestx;
	\qttestx=1.2;
	\qat=\qa-\qttestx;
	\qct=\qttestx;
	\qlam = 0.87;
	\lam = 0.87;
	\e = 0.1;
	\qe=0.1;
	\hae=(2*pow(\at,3)*\lam*(2+\lam)+\e*\e*(2*\e-sqrt(4*\e*\e+\at*\at*\lam*\lam))-8*\at*\e*(-2*\e+sqrt(4*\e*\e+\at*\at*\lam*\lam))-2*\at*\at*(2+\lam)*(-2*\e+sqrt(4*\e*\e+\at*\at*\lam*\lam)))/(4*\at*\lam*(2*\at+2*\e-sqrt(4*\e*\e+\at*\at*\lam*\lam)))-(\e*\e/4+\at*\at*(-1+\lam))/(\at*\lam);
	\qhae=(2*pow(\qat,3)*\qlam*(2+\qlam)+\qe*\qe*(2*\qe-sqrt(4*\qe*\qe+\qat*\qat*\qlam*\qlam))-8*\qat*\qe*(-2*\qe+sqrt(4*\qe*\qe+\qat*\qat*\qlam*\qlam))-2*\qat*\qat*(2+\qlam)*(-2*\qe+sqrt(4*\qe*\qe+\qat*\qat*\qlam*\qlam)))/(4*\qat*\qlam*(2*\qat+2*\qe-sqrt(4*\qe*\qe+\qat*\qat*\qlam*\qlam)))-(\qe*\qe/4+\qat*\qat*(-1+\qlam))/(\qat*\qlam);
		\whae=(2*pow(\wat,3)*\wlam*(2+\wlam)+\we*\we*(2*\we-sqrt(4*\we*\we+\wat*\wat*\wlam*\wlam))-8*\wat*\we*(-2*\we+sqrt(4*\we*\we+\wat*\wat*\wlam*\wlam))-2*\wat*\wat*(2+\wlam)*(-2*\we+sqrt(4*\we*\we+\wat*\wat*\wlam*\wlam)))/(4*\wat*\wlam*(2*\wat+2*\we-sqrt(4*\we*\we+\wat*\wat*\wlam*\wlam)))-(\we*\we/4+\wat*\wat*(-1+\wlam))/(\wat*\wlam);
	} 
	\tikzstyle{scorestars}=[star, star points=5, star point ratio=2.25, draw, inner sep=1pt]%
	\tikzstyle{scoresquare}=[draw, rectangle, minimum size=1mm,inner sep=0pt,outer sep=0pt]%
	\begin{tikzpicture}[scale=\picscale,
	declare function={
		testxonep(\ps,\t)=\a+\ps-\t;
		testxonem(\ps,\t)=\ps-\a+\t; 
		bl(\x)=\ct+\at-\e/2-1/2*(sqrt(\lam*\lam*pow(\x+\hae+\at,2)+\e*\e)-\e+\lam*(\x+\hae+\at));
		br(\x)=\ct+\at-\e/2-1/2*(sqrt(\lam*\lam*pow(\x-\hae-\at,2)+\e*\e)-\e-\lam*(\x-\hae-\at));
		bc(\x)=\ct+sqrt(\lam*\lam*\x*\x+\e*\e)-\e;
		qbl(\qx)=\qct+\qat-\qe/2-1/2*(sqrt(\qlam*\qlam*pow(\qx+\qhae+\qat,2)+\qe*\qe)-\qe+\qlam*(\qx+\qhae+\qat));
		qbr(\qx)=\qct+\qat-\qe/2-1/2*(sqrt(\qlam*\qlam*pow(\qx-\qhae-\qat,2)+\qe*\qe)-\qe-\qlam*(\qx-\qhae-\qat));
		qbc(\qx)=\qct+sqrt(\qlam*\qlam*\qx*\qx+\qe*\qe)-\qe;
		wbl(\wx)=\wct+\wat-\we/2-1/2*(sqrt(\wlam*\wlam*pow(\wx+\whae+\wat,2)+\we*\we)-\we+\wlam*(\wx+\whae+\wat));
		wbr(\wx)=\wct+\wat-\we/2-1/2*(sqrt(\wlam*\wlam*pow(\wx-\whae-\wat,2)+\we*\we)-\we-\wlam*(\wx-\whae-\wat));
		wbc(\wx)=\wct+sqrt(\wlam*\wlam*\wx*\wx+\we*\we)-\we;
	},
	 ] 
	\definecolor{tempcolor}{RGB}{250,190,0}
	\definecolor{darkgreen}{RGB}{40,190,40}
	
	\draw[->,blue, thick] [domain=-\at/2:-\lrange, samples=150]   plot (\x, {bl(\x)})  ;
	\draw[blue, thick] [domain=-\at/2:\at/2, samples=150] plot (\x, {bc(\x)})   ;
	\draw[->,blue, thick] [domain=\at/2:\rrange, samples=150]   plot (\x, {br(\x)})   ;
	
	 
	 \draw[->,gray, thick] [domain=-\at/2:-\lrange, samples=150]   plot (\x, {qbl(\x)})  ;
	\draw[gray, thick] [domain=-\at/2:\at/2, samples=150] plot (\x, {qbc(\x)})   ;
	\draw[->,gray, thick] [domain=\at/2:\rrange, samples=150]   plot (\x, {qbr(\x)})   ;
	
	\draw[->,black, thick] [domain=-\at/2:-\lrange, samples=150]   plot (\x, {wbl(\x)})  ;
	 \draw[black, thick] [domain=-\at/2:\at/2, samples=150] plot (\x, {wbc(\x)})   ;
	 \draw[->,black, thick] [domain=\at/2:\rrange, samples=150]   plot (\x, {wbr(\x)})   ;

	  
	  
	
	
	 \node[scale=\textscale]  at (2.5,4.1) {$S_{n,f}$}; 
	  \node[scale=\textscale]  at (3.9,4.1) {$S_{n,i}$}; 
	  	  \node[scale=\textscale]  at (1.1,4.1) {$S_{n,m}$}; 
	
	\draw[<->] (-\lrange, \h) node[left, scale=\textscale] {$S_0$} -- (\rrange, \h) node[right, scale=\textscale] {$S_0$};
	\draw[<->] (-\lrange, \a) node[left, scale=\textscale] {$S$} -- (\rrange, \a) node[right, scale=\textscale] {$S$};
				  
	\draw[->, shorten <= 5pt,  shorten >= 1pt] (\psione,\h)  node[above right, scale=\textscale]{$z_1$}-- (\psione,\a) ;
	
	
	\draw[dashed, tempcolor,  ultra thick](\phstartx,\tbeg) node[below left=-2,scale=\textscale]{}--(\psione,\tone) node[above, pos=0.3, rotate=45,scale=0.7,black] {} node[below, pos=0.3, rotate=45,scale=0.7,black] {};
	\draw[dashed, tempcolor,  ultra thick](\psione,\tone)--(\xendz,\a);
	
	
	\draw[dashed, tempcolor, ultra thick](\qphstartx,\tbeg) node[below left=-2,scale=\textscale]{}--(\psione,\qtone) node[above, pos=0.3, rotate=45,scale=0.7,black] {} node[below, pos=0.3, rotate=45,scale=0.7,black] {};
	\draw[dashed, tempcolor, ultra thick](\psione,\qtone)--(\qxendz,\a);

	\draw[dashed, orange](\wphstartx,\tbeg) node[below left=-2,scale=\textscale]{}--(\psione,\wtone) node[above, pos=0.3, rotate=45,scale=0.7,black] {} node[below, pos=0.3, rotate=45,scale=0.7,black] {};
	\draw[dashed, orange](\psione,\wtone)--(\wxendz,\a);
	
	\draw[->] (\timex,\md-\tlen/2) --  (\timex,\md+\tlen/2) node[midway,right, scale=\textscale]{time}; 
	 
	\draw[dotted](\psione,\ttestx)--({testxonep(\psione,\ttestx)},\a);
	\draw[dotted](\psione,\ttestx)--({testxonem(\psione,\ttestx)},\a);
	\draw[dotted](\psione,\qttestx)--({testxonep(\psione,\qttestx)},\a);
	\draw[dotted](\psione,\qttestx)--({testxonem(\psione,\qttestx)},\a);
	
	\draw (\psione,\ttestx) node[ scoresquare, fill=gray]  {} node [black,below=6,right,scale=\textscale] {$t_f$};
	\draw (\psione,\qttestx) node[ scoresquare, fill=gray]  {} node [black,below=6,right,scale=\textscale] {$t_i$};
	\draw (\psione,\wttestx) node[ scoresquare, fill=gray]  {} node [black,below=6,right,scale=\textscale] {$t_m$};
	
%	\draw[magenta,->,ultra thick] ({testxonem(\psione,\qttestx)},\a)--(-\lrange,\a);
%	\draw[magenta,ultra thick] ({testxonem(\psione,\ttestx)},\a)-- ({testxonem(\psione,\qttestx)},\a);
%	\draw[magenta,ultra thick] ({testxonep(\psione,\ttestx)},\a)-- ({testxonep(\psione,\qttestx)},\a);
%	\draw[magenta,->,ultra thick] ({testxonep(\psione,\qttestx)},\a)--(\rrange,\a);
	
	\draw [black,fill](\xendz,\a) circle [radius=\circsize] node [black,above=9,right=-4,scale=\textscale] {$\gamma_i^{(\mathcal{A})}$}; 
	\draw [black,fill](\wxendz,\a) circle [radius=\circsize] node [black,above=9,left=-7,scale=\textscale] {$\gamma_i^{(\mathcal{S})}$}; 
	\draw [black,fill](\qxendz,\a) circle [radius=\circsize] node [black,above,scale=\textscale] {$\gamma_f^{\prime\prime}$}; 
	\draw [black,fill](-3.7,\a) circle [radius=\circsize] node [black,above=9,left=-7,scale=\textscale] {$\gamma_0'$}; 
	%\draw [black,fill] (\xendz,\a) circle [radius=\circsize] node [black,above,scale=\textscale] {$\gamma_1$}; 
	\end{tikzpicture}% pic 1
	
	\vspace*{2px}
	\caption{Depicts an experiment where the state of some photons $\gamma_i^{(\mathcal{S})}$ and $\gamma_i^{(\mathcal{A})}$ on the spacelike hypersurface $S$ determines the initial conditions of an experimental setup of a particle $\mathcal{S}$ and apparatus $\mathcal{A}$ in the vicinity of the spacetime location $(z_1, t_i)$. The state of the photons $\gamma_f^{\prime\prime}$ on the spacelike hypersurface $S$ determines the final state of the apparatus after the particle has left it at time $t_f$ so that the apparatus at time $t_m$ displays a definite measurement outcome. It is assumed that no incoming photons have become entangled with the experiment after the $\gamma_i^{(\mathcal{S})}$ and $\gamma_i^{(\mathcal{A})}$ photons and before the $\gamma_f^{\prime\prime}$ photons have become entangled with the experiment.  }
	\label{pisolution}
	\end{figure}


We aim to show that within Kent's theory, we can calculate the probability the particle emerges from the measuring apparatus $\mathcal{A}$ in state $\ket{s_{f}}$ given that it enters $\mathcal{A}$ in state $\ket{s}$, and that this probability is the same as if one ignored $S$ and just applied the Born Rule to $\ket{s}$ and $\ket{s_f}$. 

In order to show this, let us choose a sequence of spacelike hypersurfaces $S_{n,i}$ which go through the spacetime location $y_i=(t_i, z_1)$ such that $\lim_{n\rightarrow\infty} S_{n,i}\cap S=S^1(y_i),$ where as usual,  $S^1(y_i)$ consists of all the spacetime locations of $S$ outside the light cone of $y_i$. Let us assume that $n$ is sufficiently large so that the photons described by $\ket*{\gamma_i^{(\mathcal{S})}}$ and $\ket*{\gamma_i^{(\mathcal{A})}}$ belong to $S_{n,i}$. The spacelike hypersurface $S_{n,i}$ and the photons being reflected from the vicinity of $z_1$ just before time $t_i$ are also depicted in figure \ref{pisolution}.

Typically, the quantum state $\ket{\Psi_{n,i}}=U_{S_{n,i},S_0}\ket{\Psi_0}$ of the spacelike hypersurface $S_{n,i}$ (where  $U_{S_{n,i},S_0}$ is the unitary operator relating the states of two spacelike hypersurfaces as discussed on page \pageref{SchwingerUnitaryOP}) will also include photon correlations with $\mathcal{S}$ and $\mathcal{A}$ corresponding to other possible ``measurements'' of $T_S(x)$ besides $\tau_S(x)$. So in general, we would expect the state of $S_{n,i}$ to be of the form
$$ \ket{\Psi_{n,i}}=\sum_{j,k}c_{j,k}\ket{\sigma_j}\ket{\alpha_k}\ket*{\gamma_j^{(\mathcal{S})}}\ket*{\gamma_k^{(\mathcal{A})}},$$
where $\{\ket{\sigma_j}:j\}$ 
is an orthonormal basis of states for the particle $\mathcal{S}$ with $\ket{s}\in\{\ket{\sigma_j}:j\}$,   $\{\ket{\alpha_k}:k\}$ 
is an orthonormal basis of states describing the apparatus $\mathcal{A}$ with $\ket{a}\in\{\ket{\alpha_k}:k\}$,  $\{\ket*{\gamma_j^{(\mathcal{S})}}:j\}$ 
are normalized states of photons in $S_{n,i}\cap S$ that are entangled with the particle $\mathcal{S}$ such that $\ip*{\gamma_j^{(\mathcal{S})}}{\gamma_{j'}^{(\mathcal{S})}}\approx 0$ for $j\neq j'$,   $\{\ket*{\gamma_k^{(\mathcal{A})}}:k\}$  are normalized states of photons in $S_{n,i}\cap S$ that are entangled with the apparatus $\mathcal{A}$ such that $\ip*{\gamma_k^{(\mathcal{A})}}{\gamma_{k'}^{(\mathcal{A})}}\approx 0$ for $k\neq k'$, and for clarity, we have absorbed any other environmental information into the states $\ket{\alpha_k}$. 

If we now define the projection $\pi_{n,i}$ corresponding to the ``measurement outcome''  $\tau_S(x)$ on $S_{n,i}\cap S$ as in equation (\ref{tauprojection}), and if we also assume that the bases are indexed so that $\ket{s}=\ket{\sigma_i}$
 and $\ket{a}=\ket{\alpha_i}$ then
\begin{equation}\label{piphi}
	\pi_{n,i}\ket{\Psi_{n,i}}\approx c\ket{s}\ket{a}\ket*{\gamma_i^{(\mathcal{S})}}\ket*{\gamma_i^{(\mathcal{A})}}
\end{equation}
 where $c=c_{i,i}$. For convenience, we will omit the reference to $n$ and write $S_i$ for $S_{n,i}$ and $\pi_i$ for $\pi_{n,i}$. We will also write $\ket{\Phi_i}$ for the normalized state of $\pi_i\ket{\Psi_i}$ so that 
 \begin{equation}\label{phii}
	\ket{\Phi_i}\approx \ket{s}\ket{a}\ket*{\gamma_i^{(\mathcal{S})}}\ket*{\gamma_i^{(\mathcal{A})}}.
\end{equation}

 We now suppose that at time $t_i$, $\ket{a}$ is the ready state of the apparatus with pointer states $\{\ket{a_j}:j\}$ so that if $\ket{s}=\sum_i c_j\ket{s_j}$, then under Schr\"{o}dinger evolution from time $t_i$ to $t_f$,
 $$\ket{s}\ket{a}\rightarrow\sum_j c_j\ket{s_j}\ket{a_j}.$$
 We assume that before time $t_f$, no photons have had a chance to get entangled with $\mathcal{S}+\mathcal{A}$. It is only by time $t_m$ that we assume a measurement of photons in state $\ket{\gamma_f^{\prime\prime}}$ on $S$ outside the light cone of $(t_m,z_1)$ is able to determine that the apparatus is in state $\ket{a_f}$ and hence that the particle is in state $\ket{s_f}$. Although the measurement outcome $\tau_S(x)$ on the whole of $S$ determines with probability $1$ that the apparatus and the particle will be in the states $\ket{a_f}$ and $\ket{s_f}$ respectively at time $t_m$, if we consider the probability $P(f|\ket{\Phi_i})$ that this outcome occurs based just on the state $\ket{\Phi_i}$, then typically this probability is going to be less than $1$.
 
 To calculate this probability, we first consider the evolution of $\ket{\Phi_i}$ to $ U_{S_f,S_i}\ket{\Phi_{i}}$. It's possible that there may be photons ``measured'' on $S$ between the spacelike hypersurfaces $S_i$ and $S_f$ as indicated in figure \ref{pisolution} (i.e. on $(S\cap S_f)\setminus(S\cap S_i)$), but we are assuming that they do not get entangled with the different pointer states of the apparatus. In other words, $\ket{\Phi_i}$ will evolve to a state of the form 
\begin{equation}\label{USfievolve1}
	U_{S_f,S_i}\ket{\Phi_{i}}\approx \sum_j c_j\ket{s_j}\ket{a_j}\ket*{\gamma_i^{(\mathcal{S})}}\ket*{\gamma_i^{(\mathcal{A})}}\sum_kg_k\ket{\gamma'_k},
\end{equation}
where $\ket{\gamma_k'}$ correspond to the possible measurements of $T_S(x)$  on $(S\cap S_f)\setminus(S\cap S_i)$, and $\sum_k\abs{g_k}^2=1$.

But from time $t_f$ to $t_m$, we assume that the apparatus does get entangled with photons which are measured on $S\cap S_m$. Thus, if $\{\ket{\gamma_j^{\prime\prime}}:j\}$ are the normalized states representing the possible measurements outcomes of these photons such that $\ip{\gamma_j^{\prime\prime}}{\gamma_k^{\prime\prime}}\approx 0$ for $j\neq k$, then 
$$U_{S_m,S_f} U_{S_f,S_i}\ket{\Phi_{i}}\approx \sum_j c_j\ket{s_j}\ket{a_j}\ket*{\gamma_i^{(\mathcal{S})}}\ket*{\gamma_i^{(\mathcal{A})}}\sum_kg_k\ket{\gamma'_k}\ket{\gamma_j^{\prime\prime}}.$$
Since we are assuming that at time $t_m$ a measurement of photons on $S\cap S_m$ is able to determine that the apparatus is in state $\ket{a_f}$, this can only happen if $U_{S_m,S_f} U_{S_f,S_i}\ket{\Phi_{i}}$ is found to be in one of the states $\ket{\Phi_{k,f}}$ for some $k$ where
$$\ket{\Phi_{k,j}}=\ket{s_j}\ket{a_j}\ket*{\gamma_i^{(\mathcal{S})}}\ket*{\gamma_i^{(\mathcal{A})}}\ket{\gamma'_k}\ket{\gamma_j^{\prime\prime}}.$$
By the Born Rule, the probability $\ket{\Phi_i}$ will be found to be in state $\ket{\Phi_{k,j}}$ will be
$$\abs{\ip{\Phi_{k,j}}{\Phi_i}}=\abs{c_j}^2\abs{g_k}^2.$$
Therefore,
$$P(f|\ket{\Phi_i})=\sum_k \abs{\ip{\Phi_{k,f}}{\Phi_i}}=\abs{c_f}^2=\abs{\ip{s_f}{s}}^2.$$
Hence, the probability that a complete measurement of $T_S(x)$ on $S$ will give a measurement outcome of the particle being in state $\ket{s_f}$ given the partial measurement of $T_S(x)$ on $S_i\cap S$ determines the particle to be initially in the state $\ket{s}$ will be the same as the standard Born Rule probability $\abs{\ip{s_f}{s}}^2$ of $\ket{s}$ being found to be in state $\ket{s_f}$.

We can also recover this probability using Kent's conditional expectation. To do this, we recall that in standard quantum theory, if for some state $\ket{\psi}$ of a system we define the operator $[\psi]=\dyad{\psi}$, then when the system is in some initial state $\ket{\chi}$, the Born Rule implies that $\ev{[\psi]}{\chi}=P(\psi|\chi)$, where $P(\psi|\chi)$ is the probability that the system will be found to be in state $\ket{\psi}$ given that it was initially in state $\ket{\chi}$. But by (\ref{evev}),  $\ev{[\psi]}{\chi}$ is just the expectation $\ev*{\psi}_\chi$ of $[\psi]$ when $[\psi]$ is treated as an observable. 

Now in equation (\ref{kentconsistency0}), we saw how to calculate the expectation value $\ev*{T^{\mu\nu}(y)}_{\tau_S}$ of the observable $\hat{T}^{\mu\nu}(y)$ given the notional measurement $\tau_S$ on $S$ outside the light cone of $y$. This suggests that the expectation value of any observable $\hat{O}$ defined at spacetime location $(t_i,z_1)$ given the notional measurement $\tau_S$ on $S$ outside the light cone of $(t_i,z_1)$ is going to be
$$\ev*{\hat{O}}_{\tau_S}=\frac{\ev*{\pi_i\hat{O}}{\Psi_i}}{\ev*{\pi_i}{\Psi_i}}.$$ 
By (\ref{piphi}), $\ev*{\pi_i}{\Psi_i}=\abs{c}^2$, and so taking $\hat{O}$ to be $[s_f]$ we have 
$$ \ev*{[s_f]}_{\tau_S}=\frac{\abs{c}^2\abs{\ip{s_f}{s}}^2}{\abs{c}^2}=\abs{\ip{s_f}{s}}^2.$$
Thus, Kent's conditional expectation $\ev*{[s_f]}_{\tau_S}$ gives us the same probability $\abs{\ip{s_f}{s}}^2$ for a particle transitioning from state $\ket{s}$ to state $\ket{s_f}$ as in standard quantum theory.

Also note that we can typically expect the $\ket*{\gamma_i^{(\mathcal{S})}}$-state to be independent of the $\ket*{\gamma_i^{(\mathcal{A})}}$-state. Therefore, since $\ket*{\gamma_i^{(\mathcal{A})}}$ will determine the measurement choice, and since $\ket*{\gamma_i^{(\mathcal{S})}}$ determines the initial state of the particle, we can expect the state of the particle to be independent of the measurement choice in Kent's theory. Thus, we can fulfil one of the necessary criteria (i.e. criterion \ref{hidden2}) for PI to be a well-defined notion.

\section{Kent's Theory and Parameter Independence\label{kentpi}}
In addition to criterion \ref{hidden2} being satisfied, criterion \ref{hidden4} must also be true if PI is to be a well-defined notion. In the previous section, we saw how we can generalize Kent's beable $\ev*{\hat{T}^{\mu\nu}(y)}_{\tau_S}$ to calculate conditional expectations $\ev*{\hat{O}}_{\tau_S}$ for any observable $\hat{O}$ defined at a particular spacetime location $(t_i, z_1)$. Calculating the probability for two measurements requires calculating the conditional expectation of an observable that depends on two spacetime locations. In order to do this, we need to make a further adaption to Kent's theory. In this section, we will describe this adaption and show that with it, Kent's theory allows us to calculate probabilities for Bell-type experiments, and that these probabilities are the same as in standard quantum theory. Since PI holds in standard quantum theory,  a consequence of Kent's theory agreeing with standard quantum theory is that PI will also hold in Kent's theory.


So let's consider figure \ref{bellsolution} which depicts a one-dimensional view of a Bell-type experiment.
\begin{figure}[ht!]
	\captionsetup{justification=justified}
	\centering
	\tikzmath{
	\a=5.3;  
	\qa=5.3;  
	\ja=5.3;  
	\psione=-1;
	\jpsione=1;
	\qpsione=-1;
	\psitwo=1.5;
	\h=-1;
	\labelx=(\psione+\psitwo)/2;
	\labely=-1.5;
	\phstartx=-2.1;
	\offset=-\psione;
	\joffset=-\jpsione;
	\qphstartx=-3.8;
	\jphstartx=-3.8;
	\wphstartx=-1.8;
	\phstarty=\h;
	\tbeg=\phstarty;
	\ca=\phstartx-\tbeg;
	\ttwo=\psitwo-\ca;
	\cb=\ttwo+\ca;
	\xend=\ttwo-\a+\cb;
	\tone=\psione-\ca;
	\cc=\tone+\ca;
	\xendz=\tone-\a+\cc;
	\qca=\qphstartx-\tbeg;
	\qtone=\qpsione-\qca;
	\qqcc=\qtone+\qca;
	\qxendz=\qtone-\qa+\qqcc;
	\jca=\jphstartx-\tbeg;
	\jtone=\jpsione-\jca;
	\jqcc=\jtone+\jca;
	\jxendz=\jtone-\ja+\jqcc;
	\wa=5.3;  
	\wpsione=-1;
	\wttestx=3.7;
	\wca=\wphstartx-\tbeg;
	\wtone=\wpsione-\wca;
	\wwcc=\wtone+\wca;
	\wxendz=\wtone-\wa+\wwcc;
	\wat=\wa-\wttestx;
	\wct=\wttestx;
	\wlam = 0.87;
		\we=0.1;
	\tthree=\cc+\tone-\psitwo;
	\circsize=0.08;
	\md = (\a+\h)/2;
	\tlen=0.75;
	\textscale = 0.7;
	\picscale = 0.78;
	\nudge=0.1;
	\tnudge=(\ttwo-\tone)/2+0.8;
	\ttesta=2*\tone-\ttwo-\tnudge;
	\sonea=\a+\psione-\ttesta;
	\sadd=0.5;
	\lrange = 8.7;
	\rrange =\lrange; 
	\timex=\rrange-1.5;
	\ttestx=2.4;
	\at=\a-\ttestx;
	\ct=\ttestx;
	\qttestx=1.2;
	\qat=\qa-\qttestx;
	\qct=\qttestx;
	\qlam = 0.87;
	\jttestx=1.2;
	\jat=\ja-\jttestx;
	\jct=\jttestx;
	\jlam = 0.87;
	\lam = 0.87;
	\e = 0.1;
	\qe=0.1;
	\je=0.1;
	\hae=(2*pow(\at,3)*\lam*(2+\lam)+\e*\e*(2*\e-sqrt(4*\e*\e+\at*\at*\lam*\lam))-8*\at*\e*(-2*\e+sqrt(4*\e*\e+\at*\at*\lam*\lam))-2*\at*\at*(2+\lam)*(-2*\e+sqrt(4*\e*\e+\at*\at*\lam*\lam)))/(4*\at*\lam*(2*\at+2*\e-sqrt(4*\e*\e+\at*\at*\lam*\lam)))-(\e*\e/4+\at*\at*(-1+\lam))/(\at*\lam);
	\qhae=(2*pow(\qat,3)*\qlam*(2+\qlam)+\qe*\qe*(2*\qe-sqrt(4*\qe*\qe+\qat*\qat*\qlam*\qlam))-8*\qat*\qe*(-2*\qe+sqrt(4*\qe*\qe+\qat*\qat*\qlam*\qlam))-2*\qat*\qat*(2+\qlam)*(-2*\qe+sqrt(4*\qe*\qe+\qat*\qat*\qlam*\qlam)))/(4*\qat*\qlam*(2*\qat+2*\qe-sqrt(4*\qe*\qe+\qat*\qat*\qlam*\qlam)))-(\qe*\qe/4+\qat*\qat*(-1+\qlam))/(\qat*\qlam);
		\whae=(2*pow(\wat,3)*\wlam*(2+\wlam)+\we*\we*(2*\we-sqrt(4*\we*\we+\wat*\wat*\wlam*\wlam))-8*\wat*\we*(-2*\we+sqrt(4*\we*\we+\wat*\wat*\wlam*\wlam))-2*\wat*\wat*(2+\wlam)*(-2*\we+sqrt(4*\we*\we+\wat*\wat*\wlam*\wlam)))/(4*\wat*\wlam*(2*\wat+2*\we-sqrt(4*\we*\we+\wat*\wat*\wlam*\wlam)))-(\we*\we/4+\wat*\wat*(-1+\wlam))/(\wat*\wlam);
			\jhae=(2*pow(\jat,3)*\jlam*(2+\jlam)+\je*\je*(2*\je-sqrt(4*\je*\je+\jat*\jat*\jlam*\jlam))-8*\jat*\je*(-2*\je+sqrt(4*\je*\je+\jat*\jat*\jlam*\jlam))-2*\jat*\jat*(2+\jlam)*(-2*\je+sqrt(4*\je*\je+\jat*\jat*\jlam*\jlam)))/(4*\jat*\jlam*(2*\jat+2*\je-sqrt(4*\je*\je+\jat*\jat*\jlam*\jlam)))-(\je*\je/4+\jat*\jat*(-1+\jlam))/(\jat*\jlam);
	} 
	\tikzstyle{scorestars}=[star, star points=5, star point ratio=2.25, draw, inner sep=1pt]%
	\tikzstyle{scoresquare}=[draw, rectangle, minimum size=1mm,inner sep=0pt,outer sep=0pt]%
	\begin{tikzpicture}[scale=\picscale,
	declare function={
		testxonep(\ps,\t)=\a+\ps-\t;
		testxonem(\ps,\t)=\ps-\a+\t; 
		bl(\x)=\ct+\at-\e/2-1/2*(sqrt(\lam*\lam*pow(\x+\hae+\at,2)+\e*\e)-\e+\lam*(\x+\hae+\at));
		br(\x)=\ct+\at-\e/2-1/2*(sqrt(\lam*\lam*pow(\x-\hae-\at,2)+\e*\e)-\e-\lam*(\x-\hae-\at));
		bc(\x)=\ct+sqrt(\lam*\lam*\x*\x+\e*\e)-\e;
		qbl(\qx)=\qct+\qat-\qe/2-1/2*(sqrt(\qlam*\qlam*pow(\qx+\qhae+\qat,2)+\qe*\qe)-\qe+\qlam*(\qx+\qhae+\qat));
		qbr(\qx)=\qct+\qat-\qe/2-1/2*(sqrt(\qlam*\qlam*pow(\qx-\qhae-\qat,2)+\qe*\qe)-\qe-\qlam*(\qx-\qhae-\qat));
		qbc(\qx)=\qct+sqrt(\qlam*\qlam*\qx*\qx+\qe*\qe)-\qe;
		wbl(\wx)=\wct+\wat-\we/2-1/2*(sqrt(\wlam*\wlam*pow(\wx+\whae+\wat,2)+\we*\we)-\we+\wlam*(\wx+\whae+\wat));
		wbr(\wx)=\wct+\wat-\we/2-1/2*(sqrt(\wlam*\wlam*pow(\wx-\whae-\wat,2)+\we*\we)-\we-\wlam*(\wx-\whae-\wat));
		wbc(\wx)=\wct+sqrt(\wlam*\wlam*\wx*\wx+\we*\we)-\we;
		jbl(\jx)=\jct+\jat-\je/2-1/2*(sqrt(\jlam*\jlam*pow(\jx+\jhae+\jat,2)+\je*\je)-\je+\jlam*(\jx+\jhae+\jat));
		jbr(\jx)=\jct+\jat-\je/2-1/2*(sqrt(\jlam*\jlam*pow(\jx-\jhae-\jat,2)+\je*\je)-\je-\jlam*(\jx-\jhae-\jat));
		jbc(\jx)=\jct+sqrt(\jlam*\jlam*\jx*\jx+\je*\je)-\je;
		rr(\x)=2*(pow(\x,4)/4-pow(\offset*\x,2)/2+pow(\offset,4)/4)+\qttestx;
	},
	 ] 
	\definecolor{tempcolor}{RGB}{250,190,0}
	\definecolor{darkgreen}{RGB}{40,190,40}
	

	
	 
	 \draw[->,gray, thick] [domain=-\at/2-\offset:-\lrange, samples=150]   plot (\x, {qbl(\x+\offset)})  ;
	\draw[gray, thick] [domain=-\at/2-\offset:0-\offset, samples=150] plot (\x, {qbc(\x+\offset)})   ;
%	\draw[->,gray, thick] [domain=\at/2-\offset:\rrange, samples=150]   plot (\x, {qbr(\x+\offset)})   ;
	\draw[gray, thick] [domain=-\offset:-\joffset, samples=150] plot (\x, {\qttestx})   ;
	\draw[gray, thick, densely dashed] [domain=-\offset:-\joffset, samples=150] plot (\x, {rr(\x)})   ;
	
	
	 \draw[->,blue, thick] [domain=-\at/2-\offset:-\lrange, samples=150]   plot (\x, {wbl(\x+\offset)})  ;
	 \draw[blue, thick] [domain=-\at/2-\offset:0-\offset, samples=150] plot (\x, {wbc(\x+\offset)})   ;
	 \draw[blue, thick] [domain=-\offset:-\joffset, samples=150] plot (\x, {\wttestx})   ;
	 \draw[blue, thick, densely dashed] [domain=-\offset:-\joffset, samples=150] plot (\x, {rr(\x)+\wttestx-\qttestx})   ;
	 
	 \draw[->,blue, thick] [domain=\at/2+\offset:\lrange, samples=150]   plot (\x, {wbl(-\x+\offset)})  ;
	\draw[blue, thick] [domain=\at/2+\offset:\offset, samples=150] plot (\x, {wbc(-\x+\offset)})   ;

 %   \draw[->,gray, thick] [domain=-\at/2-\joffset:-\lrange, samples=150]   plot (\x, {jbl(\x+\joffset)})  ;
	\draw[gray, thick] [domain=-\joffset:\at/2-\joffset, samples=150] plot (\x, {jbc(\x+\joffset)})   ;
	\draw[->,gray, thick] [domain=\at/2-\joffset:\rrange, samples=150]   plot (\x, {jbr(\x+\joffset)})   ;

	  \node[scale=\textscale]  at (0,\qttestx-0.23) {$S_{n,i}$}; 
	  \node[scale=\textscale]  at (0,\wttestx-0.23) {$S_{n,m}$}; 

	
	\draw[<->] (-\lrange, \h) node[left, scale=\textscale] {$S_0$} -- (\rrange, \h) node[right, scale=\textscale] {$S_0$};
	\draw[<->] (-\lrange, \a) node[left, scale=\textscale] {$S$} -- (\rrange, \a) node[right, scale=\textscale] {$S$};
				  
	\draw[->, shorten <= 5pt,  shorten >= 1pt] (\psione,\h)  node[above right, scale=\textscale]{$z_L$}-- (\psione,\a) ;
	\draw[->, shorten <= 5pt,  shorten >= 1pt] (\jpsione,\h)  node[above right, scale=\textscale]{$z_R$}-- (\jpsione,\a) ;
	
	
	\draw[dashed, tempcolor,  ultra thick](\phstartx,\tbeg) node[below left=-2,scale=\textscale]{}--(\psione,\tone) node[above, pos=0.3, rotate=45,scale=0.7,black] {} node[below, pos=0.3, rotate=45,scale=0.7,black] {};
	\draw[dashed, tempcolor,  ultra thick](\psione,\tone)--(\xendz,\a);
	
		\draw[dashed, tempcolor,  ultra thick](\qphstartx,\tbeg) node[below left=-2,scale=\textscale]{}--(\psione,\qtone) node[above, pos=0.3, rotate=45,scale=0.7,black] {} node[below, pos=0.3, rotate=45,scale=0.7,black] {};
	\draw[dashed, tempcolor,  ultra thick](\psione,\qtone)--(\qxendz,\a);
	
			\draw[dashed, tempcolor,  ultra thick](-\qphstartx,\tbeg) node[below left=-2,scale=\textscale]{}--(-\psione,\qtone) node[above, pos=0.3, rotate=45,scale=0.7,black] {} node[below, pos=0.3, rotate=45,scale=0.7,black] {};
	\draw[dashed, tempcolor,  ultra thick](-\psione,\qtone)--(-\qxendz,\a);
	
		\draw[dashed, tempcolor,  ultra thick](-\phstartx,\tbeg) node[below left=-2,scale=\textscale]{}--(-\psione,\tone) node[above, pos=0.3, rotate=45,scale=0.7,black] {} node[below, pos=0.3, rotate=45,scale=0.7,black] {};
	\draw[dashed, tempcolor,  ultra thick](\jpsione,\tone)--(-\xendz,\a);



	
	\draw[->] (\timex,\md-\tlen/2) --  (\timex,\md+\tlen/2) node[midway,right, scale=\textscale]{time}; 
	 

	\draw[dotted](\psione,\qttestx)--({testxonep(\psione,\qttestx)},\a);
	\draw[dotted](\psione,\qttestx)--({testxonem(\psione,\qttestx)},\a);
	\draw[dotted](\jpsione,\jttestx)--({testxonep(\jpsione,\jttestx)},\a);
	\draw[dotted](\jpsione,\jttestx)--({testxonem(\jpsione,\jttestx)},\a);

	\draw (\psione,\qttestx) node[ scoresquare, fill=gray]  {} node [black,below=6,right,scale=\textscale] {$t_i$};
    \draw (\jpsione,\jttestx) node[ scoresquare, fill=gray]  {} node [black,below=6,right,scale=\textscale] {$t_i$};
\draw (\psione,\wttestx) node[ scoresquare, fill=gray]  {} node [black,below=6,right,scale=\textscale] {$t_m$};
    \draw (\jpsione,\wttestx) node[ scoresquare, fill=gray]  {} node [black,below=6,right,scale=\textscale] {$t_m$};



	
	\draw [black,fill](\xendz,\a) circle [radius=\circsize] node [black,above=9,right=-4,scale=\textscale] {$\gamma_i^{(\mathcal{A}_L)}$}; 
	\draw [black,fill](-\xendz,\a) circle [radius=\circsize] node [black,above=9,right=-4,scale=\textscale] {$\gamma_i^{(\mathcal{A}_R)}$}; 
	\draw [black,fill](\qxendz,\a) circle [radius=\circsize] node [black,above=9,right=-4,scale=\textscale] {$\gamma_{m,+}^{(\mathcal{A}_L)}$}; 
	\draw [black,fill](-\qxendz,\a) circle [radius=\circsize] node [black,above=9,right=-4,scale=\textscale] {$\gamma_{m,+}^{(\mathcal{A}_R)}$}; 
	
	%\draw [black,fill] (\xendz,\a) circle [radius=\circsize] node [black,above,scale=\textscale] {$\gamma_L$}; 
	\end{tikzpicture}% pic 1
	
	\vspace*{2px}
	\caption{Depicts a Bell-type experiment where the state of some photons $\gamma_i^{(\mathcal{A}_L)}$ and $\gamma_i^{(\mathcal{A}_R)}$ on the spacelike hypersurface $S$ determines the choice of measurement parameters of the left wing and right wing of the experiment respectively, and some photons $\gamma_{m,+}^{(\mathcal{A}_L)}$ and $\gamma_{m,+}^{(\mathcal{A}_R)}$ on the spacelike hypersurface $S$ determine the measurement outcome of the experiment on the left wing and the right wing respectively. The dashed lines on the spacelike hypersurfaces $S_{n,m}$ and $S_{n,i}$ indicate other choices for the spacelike hypersurfaces, but they still lead to the same probability being calculated.   }
	\label{bellsolution}
	\end{figure}
There is a left wing of the experiment located in the vicinity of $z_L$, and a right wing of the experiment located in the vicinity of $z_R$. Shortly before time $t_i$, photons interact with a Stern-Gerlach apparatus on the left wing and a Stern-Gerlach apparatus on the right wing, with the result that the photons being measured on a spacelike hypersurface $S_{n,i}\cap S$ to be in states $\ket*{\gamma_i^{(\mathcal{A}_L)}}$ and $\ket*{\gamma_i^{(\mathcal{A}_R)}}$ determine the measurement parameters of the apparatuses on the left wing and the right wing of the experiment respectively. 

We need to adapt Kent's sequences of spacelike hypersurfaces in order to proceed.  Sequences of spacelike hypersurfaces $S_{n,i}$ are chosen so that they all contain the spacetime locations $y_L=(t_i, z_L)$ and $y_R=(t_i,z_R)$, and that in the limit, $\lim_{n\rightarrow\infty}S_{n,i}$ contains as much of $S^1(y_L)$ and $S^1(y_R)$ as possible, where as usual, $S^1(y)$ denotes the subset of $S$ lying outside the light cone of $y$. Ultimately, this limit (unlike the limit of Kent's spacelike hypersurfaces) will not contain the whole of $S^1(y_L)$ or $S^1(y_R)$, but only serves to guarantee that we use as much of the information in $S$ as possible in calculating the expectation values of observables at $(t_i, z_L)$ and $(t_i,z_R)$. There will be some degree of freedom in what we choose for the spacelike hypersurface between $(t_i, z_L)$ and $(t_i,z_R)$ as depicted by the dashed line in the figure. However, such freedom will have no effect on the probabilities calculated, because under the assumption that the spacelike hypersurface is very far into the future, there will be no choice of spacelike hypersurface in this region that would give us more information in $S$ to condition on. Also, we recall that the stress-energy operators in the Tomonaga-Schwinger formulation of relativistic quantum physics are chosen so that they are invariant under perturbations of the spacelike hypersurface, so under the assumption that all physical observables will be ultimately expressible in terms of the stress-energy operators, the arbitrary choice of the spacelike hypersurfaces in regions that can't intersect with $S$ will have no effect of the probabilities calculated. 

On the spacelike hypersurface $S_{n,i}$, we assume that there are some photons ``measured'' on it to be in the states $\ket*{\gamma_i^{(\mathcal{A}_L)}}$ and $\ket*{\gamma_i^{(\mathcal{A}_R)}}$ that determine the choice of measurement axes for the left and right wings of the experiment respectively.  We assume that the axis of orientation of the right wing Stern-Gerlach apparatus makes an angle $\theta$ with the axis of the left wing apparatus.

We also assume that there are two particles that together form a Bell-state \begin{equation}\label{bellstatePI}
	\frac{1}{\sqrt{2}}(\ket*{\uvbp{s}}_L\ket*{\uvbm{s}}_R-\ket*{\uvbm{s}}_L\ket*{\uvbp{s}}_R).
\end{equation}
We saw  in footnote \ref{bellstate2pf} on page \pageref{bellstate2pf} that a Bell state does not depend on the orientation of $\uvb{s}$, so without loss of generality, we can suppose that the $\ket*{\uvbp{s}}_L$ and $\ket*{\uvbm{s}}_L$ are pointer states for the apparatus on the left-wing of the experiment. This means there will be a ready state $\ket{a}_L$  as well as two states $\ket{a+}_L$ and $\ket{a-}_L$ of the left wing apparatus such that 
$$\ket*{\uvbpm{s}}_L\ket{a}_L\rightarrow\ket*{\uvbpm{s}}_L\ket{a\pm}_L.$$

As for the right wing of the experiment, we let  $\ket*{\bm{\hat{s}_\theta+}}_R$ and $\ket*{\bm{\hat{s}_\theta-}}_R$ be pointer states for the apparatus so that there is a ready state $\ket{a}_R$  as well as two states $\ket{a_\theta+}_R$ and $\ket{a_\theta-}_R$ of the right wing apparatus such that 
$$\ket*{\bm{\hat{s}}_\theta\pm}_R\ket{a}_R\rightarrow\ket*{\bm{\hat{s}}_\theta\pm}_R\ket{a_\theta\pm}_R.$$
As in approximation (\ref{phii}), the detections of the photons on $S_{n,i}\cap S$ being in state $\ket*{\gamma_i^{(\mathcal{A}_L)}}$ and $\ket*{\gamma_i^{(\mathcal{A}_R)}}$ determine the two particles and the apparatuses on both wings of the experiment to be in the state 
\begin{equation}\label{bellstatePI2}
	\ket{\Phi_i}\approx \frac{1}{\sqrt{2}}(\ket*{\uvbp{s}}_L\ket*{\uvbm{s}}_R-\ket*{\uvbm{s}}_L\ket*{\uvbp{s}}_R)\ket{a}_L\ket{a}_R\ket*{\gamma_i^{(\mathcal{A}_L)}}\ket*{\gamma_i^{(\mathcal{A}_R)}}.
\end{equation}
As in equations (\ref{spintrans1}) and (\ref{spintrans2}), we have
\begin{align*}
\ket*{\uvbp{s}}_R&= \alpha_\theta\ket*{\bm{\hat{s}_\theta+}}_R+\beta_\theta \ket*{\bm{\hat{s}_\theta-}}_R,\\
\ket*{\uvbm{s}}_R&= \alpha_\theta\ket*{\bm{\hat{s}_\theta-}}_R-\beta_\theta \ket*{\bm{\hat{s}_\theta+}}_R,
\end{align*}
where $\alpha_\theta=\cos(\theta/2)$, and $\beta_\theta=\sin(\theta/2).$
Substituting this into (\ref{bellstatePI2}), we can express the state of the two particles as
\begin{equation}\label{bellstatePI3}
	\begin{split}
	\ket{\Phi_{n,i}}&\approx\frac{1}{\sqrt{2}}(\alpha_\theta\ket*{\uvbp{s}}_L\ket*{\bm{\hat{s}_\theta-}}_R
	-\beta_\theta\ket*{\uvbp{s}}_L\ket*{\bm{\hat{s}_\theta+}}_R\\
	&\quad-\alpha_\theta\ket*{\uvbm{s}}_L\ket*{\bm{\hat{s}_\theta+}}_R
	-\beta_\theta\ket*{\uvbm{s}}_L\ket*{\bm{\hat{s}_\theta-}}_R)\ket{a}_L\ket{a}_R\ket*{\gamma_i^{(\mathcal{A}_L)}}\ket*{\gamma_i^{(\mathcal{A}_R)}}.
	\end{split}
\end{equation}
If we apply the unitary operator $U_{S_{n,m},S_{n,i}}$ to each of the terms of (\ref{bellstatePI3}), we get 
\begin{equation}
	\begin{split}
U_{S_{n,m},S_{n,i}}\ket*{\uvbpm{s}}_L\ket*{\bm{\hat{s}_\theta\pm'}}_R\ket{a}_L&\ket{a}_R\ket*{\gamma_i^{(\mathcal{A}_L)}}\ket*{\gamma_i^{(\mathcal{A}_R)}}\\
&=\ket*{\uvbpm{s}}_L\ket*{\bm{\hat{s}_\theta\pm'}}_R\ket{a\pm}_L\ket{a_\theta\pm'}_R\ket*{\gamma_{m,\pm}^{(\mathcal{A}_L)}}\ket*{\gamma_{m,\pm'}^{(\mathcal{A}_R)}}
	\end{split}
\end{equation}
where $\ket*{\gamma_{m,\pm}^{(\mathcal{A}_L)}}$ are the states of possible detections of photons on $S_{n,m}$ that would determine the left wing apparatus to be in the state $\ket{a\pm}_L$, and where $\ket*{\gamma_{m,\pm}^{(\mathcal{A}_R)}}$ are the states of possible detections of photons on $S_{n,m}$ that would determine the right wing apparatus to be in the state $\ket{a_\theta\pm'}_R$.
Using the Born Rule, we therefore see that given a measurement of $T_S(x)$ determines the state of the spacelike hypersurface $S_{n,i}$ to be in state $\ket{\Phi_{n,i}}$,  the probability that the spacelike hypersurface $S_{n,m}$ will be found to be in the state 
$$\ket*{\uvbp{s}}_L\ket*{\bm{\hat{s}_\theta+}}_R\ket{a+}_L\ket{a_\theta+}_R\ket*{\gamma_{m,+}^{(\mathcal{A}_L)}}\ket*{\gamma_{m,+}^{(\mathcal{A}_R)}}$$
will be 
$$\frac{1}{2}\abs{\beta_\theta}^2=\frac{1}{2}\sin^2(\theta/2) $$
From this it follows that the probability that the left wing particle will be in state $\ket*{\uvbp{s}}_L$ and that the right wing particle will be in state $\ket*{\bm{\hat{s}_\theta+}}_R$ given the initial conditions will also be $\frac{1}{2}\sin^2(\theta/2) $. This is the same probability as that given by standard quantum theory on page \pageref{bellsin}. 

Also note that if we define the observable $[\uvbp{s}]_L=\ket{\uvbp{s}}_L\prescript{}{L}{\bra{\uvbp{s}}}$ that depends on spacetime location $(t_1, z_L)$, and the observable $[\bm{\hat{s}_\theta+}]_R=\ket{\bm{\hat{s}_\theta+}}_R\prescript{}{R}{\bra{\bm{\hat{s}_\theta+}}}$ that depends on spacetime location $(t_1, z_R),$ then we can construct the observable
$[\uvbp{s}]_L [\bm{\hat{s}_\theta+}]_R $, and with the adapted sequence $S_{n,i}$ of spacelike hypersurfaces, we can calculate the conditional expectation
$$\ev*{[\bm{\hat{s}+}]_L[\bm{\hat{s}_\theta+}]_R}_{\tau_S}=\lim_{n\rightarrow\infty}\frac{\ev*{\pi_{n,i}[\bm{\hat{s}+}]_L[\bm{\hat{s}_\theta+}]_R}{\Psi_{n,i}}}{\ev*{\pi_{n,i}}{\Psi_{n,i}}}.$$
With this adaption and the notional measurement of $T_S(x)$ on $S$ described in this section, it is easy to see that 
$$\ev*{[\bm{\hat{s}+}]_L[\bm{\hat{s}_\theta+}]_R}_{\tau_S}=\frac{1}{2}\sin^2(\theta/2)$$ 
which is the joint probability for finding the left wing particle in state $\ket{\uvbp{s}}_L$ and the right wing particle in state $\ket{\bm{\hat{s}_\theta+}}_R$. Thus, we can adapt Kent's model so that criterion \ref{hidden4} of page \pageref{hidden4} is satisfied and such that Kent's model gives the same probabilities as standard quantum theory. Hence, PI holds in Kent's theory.\label{kentpiend}


\section{An Alternative to Kent's Beables}
A final question to consider in this chapter is the nature of Kent's beables. As discussed on page \pageref{beabledef}, Bell hoped for a more satisfactory theory than standard quantum theory. This more satisfactory theory was to be a theory of beables rather than a theory of observables, and these beables would form the underlying reality that gives rise to all the familiar things in the world around us. 

Now in the context of Kent's theory, the definite mass/energy density $\tau_S(x)$ at a location $x\in S$ seems like a good candidate to be a beable. But what is less obvious is what the beables should be for a spacetime locations prior to $S$. Kent seems to be saying that the beables at a spacetime location $y$ prior to $S$ would consist in the stress-energy tensor whose values were given by the conditional expectation value  $\ev*{\hat{T}^{\mu\nu}(y)}_{\tau_S}$ for all the different combinations of $\mu$ and $\nu$. 

In some situations, the definite values of $\tau_S(x)$ on $S$ will give rise to $\hat{T}^{\mu\nu}(y)$-eigenstates at $y$, in which case the conditional expectation $\ev*{\hat{T}^{\mu\nu}(y)}_{\tau_S}$ will be identical to a definite value for $T^{\mu\nu}(y)$ as normally understood in standard quantum theory. But there will inevitably be situations in which the definite values of $\tau_S(x)$ on $S$ will not give rise to  $\hat{T}^{\mu\nu}(y)$-eigenstates at $y$. In ohter words, in the notation defined in section \ref{LorentzInvarianceSection} on page \pageref{tauprojection}, we may find that the state $\pi_n\ket{\Psi_n}$ at $y$ is in a superposition of eigenstates of $\hat{T}^{\mu\nu}(y)$. But in that case, it seems very dubious to claim that the expectation value $\ev*{\hat{T}^{\mu\nu}(y)}_{\tau_S}$ is the true state of reality and hence the beable at $y$. As an analogy, it seems a bit like saying if the throw of a six-sided dice was described quantum mechanically, then the dice could yield a beable of 3.5 for its outcome since 3.5 is the expectation value for the throw of a six-sided dice. In Kent's theory, there would of course nearly always be sufficient information in $\tau_S$ on the spacelike hypersurface $S$  to determine that the dice had an integer outcome between 1 and 6. But this still allows for the remote possibility that this information might not be sufficient. It nevertheless seems unnecessary to insist that  physical reality is so definite that we have to allow for the possibility of a six-sided dice having 3.5 as an outcome. 

One way to avoid such a possibility would be to suppose that there are degrees of definiteness in physical reality. In the context of Kent's theory, one could still suppose that the mass/energy density $\tau_S(x)$ on $S$ was perfectly definite, and that some definite facts about physical reality would flow from the definite facts about $S$. But there would also be much indefiniteness about physical reality. This would occur whenever the information in $\tau_S(x)$ was insufficient to determine whether the state $\pi_n\ket{\Psi_n}$ was an eigenstate of some observable. If light were to interact with the location $y$ in a different way, then this might be able to settle the question of which eigenstate $\pi_n\ket{\Psi_n}$ was in, but this definite outcome would then result in $y$ being indefinite with respect to other observables. But even though there would inevitably be many physical quantities of a system that lacked definiteness, we would still be able to make claims about how mass/energy \emph{might} have been measured on $S$ and hence how likely a physical quantity might have had a particular value corresponding to this possible measurement on $S$.

Another issue that should concern us is the question of what should be the  subject of the beables. Any beable presupposes a subject as well as a property that belongs to the subject. So in the case of Kent's beables, the spacetime location $y$ would be the subject and $\ev*{\hat{T}^{\mu\nu}(y)}_{\tau_S}$ would be a property of the subject. However, in section \ref{kentpi} (pages \pageref{kentpi}--\pageref{kentpiend}), we found that in order to deal with joint probabilities in Bell-type experiments, it was necessary to calculate expectation values for observables that depended on two spacetime locations rather than just one spacetime location. 

Now perhaps some people would argue that joint probabilities are not beables and they might therefore conclude that the need to use multiple spacetime locations in determining joint probabilities is irrelevant when it comes to deciding what the subject of a beable should be. But in defense of their relevance, it does seem that joint probabilities are saying something about reality. Moreover, in the absence of hidden variables, these joint probabilities can't be reduced to some more basic reality whose components only depend on  single spacetime locations. It therefore doesn't seem unreasonable to attribute beability to probabilities, and if a probability could count as being a beable, then when that probability depended on more than one location, we would obviously have a beable whose subject consisted in more than one location. 

But there is another reason for allowing beables whose subject  consists in multiple locations, for by allowing this, we can avoid some of the strange consequences that occur if we only allow a  single location to be the subject of a beable. For consider figure \ref{strangesubject}.
\begin{figure}[ht!]
	\captionsetup{justification=justified}
	\centering
	\tikzmath{
	\a=5.3;  
	\qa=5.3;  
	\ja=5.3;  
	\psione=-1;
	\jpsione=0.4;
	\qpsione=-1;
	\psitwo=1;
	\h=-1;	
	\phstartx=-2.1;
	\offset=-\psione;
	\joffset=-\jpsione;
	\qphstartx=-3.8;
	\jphstartx=-3.8;
	\wphstartx=-1.8;
	\phstarty=\h;
	\tbeg=\phstarty;
	\ca=\phstartx-\tbeg;
	\ttwo=\psitwo-\ca;
	\cb=\ttwo+\ca;
	\xend=\ttwo-\a+\cb;
	\tone=\psione-\ca;
	\cc=\tone+\ca;
	\xendz=\tone-\a+\cc;
	\qca=\qphstartx-\tbeg;
	\qtone=\qpsione-\qca;
	\qqcc=\qtone+\qca;
	\qxendz=\qtone-\qa+\qqcc;
	\jca=\jphstartx-\tbeg;
	\jtone=\jpsione-\jca;
	\jqcc=\jtone+\jca;
	\jxendz=\jtone-\ja+\jqcc;
	\wa=5.3;  
	\wpsione=-1;
	\wttestx=3.7;
	\wca=\wphstartx-\tbeg;
	\wtone=\wpsione-\wca;
	\wwcc=\wtone+\wca;
	\wxendz=\wtone-\wa+\wwcc;
	\wat=\wa-\wttestx;
	\wct=\wttestx;
	\wlam = 0.87;
		\we=0.1;
	\tthree=\cc+\tone-\psitwo;
	\circsize=0.08;
	\md = (\a+\h)/2;
	\tlen=0.75;
	\textscale = 0.7;
	\picscale = 0.59;
	\nudge=0.1;
	\tnudge=(\ttwo-\tone)/2+0.8;
	\ttesta=2*\tone-\ttwo-\tnudge;
	\sonea=\a+\psione-\ttesta;
	\sadd=0.5;
	\lrange = 5;
	\rrange =6.5; 
	\timex=\rrange-1.5;
	\ttestx=2.4;
	\at=\a-\ttestx;
	\ct=\ttestx;
	\qttestx=1.2;
	\qat=\qa-\qttestx;
	\qct=\qttestx;
	\qlam = 0.87;
	\jttestx=1.2;
	\jat=\ja-\jttestx;
	\jct=\jttestx;
	\jlam = 0.87;
	\lam = 0.87;
	\e = 0.1;
	\qe=0.1;
	\je=0.1;
	\xs=2.9;
	\xm=\psione;
	\ys=\h;
	\ym=-(\xm-\xs)+\ys;
	\yf=\a;
	\xf=\yf-\ym+\xm;
	\labelx=(\rrange-\lrange)/2;
	\labely=\h-0.5;
	\hae=(2*pow(\at,3)*\lam*(2+\lam)+\e*\e*(2*\e-sqrt(4*\e*\e+\at*\at*\lam*\lam))-8*\at*\e*(-2*\e+sqrt(4*\e*\e+\at*\at*\lam*\lam))-2*\at*\at*(2+\lam)*(-2*\e+sqrt(4*\e*\e+\at*\at*\lam*\lam)))/(4*\at*\lam*(2*\at+2*\e-sqrt(4*\e*\e+\at*\at*\lam*\lam)))-(\e*\e/4+\at*\at*(-1+\lam))/(\at*\lam);
	\qhae=(2*pow(\qat,3)*\qlam*(2+\qlam)+\qe*\qe*(2*\qe-sqrt(4*\qe*\qe+\qat*\qat*\qlam*\qlam))-8*\qat*\qe*(-2*\qe+sqrt(4*\qe*\qe+\qat*\qat*\qlam*\qlam))-2*\qat*\qat*(2+\qlam)*(-2*\qe+sqrt(4*\qe*\qe+\qat*\qat*\qlam*\qlam)))/(4*\qat*\qlam*(2*\qat+2*\qe-sqrt(4*\qe*\qe+\qat*\qat*\qlam*\qlam)))-(\qe*\qe/4+\qat*\qat*(-1+\qlam))/(\qat*\qlam);
		\whae=(2*pow(\wat,3)*\wlam*(2+\wlam)+\we*\we*(2*\we-sqrt(4*\we*\we+\wat*\wat*\wlam*\wlam))-8*\wat*\we*(-2*\we+sqrt(4*\we*\we+\wat*\wat*\wlam*\wlam))-2*\wat*\wat*(2+\wlam)*(-2*\we+sqrt(4*\we*\we+\wat*\wat*\wlam*\wlam)))/(4*\wat*\wlam*(2*\wat+2*\we-sqrt(4*\we*\we+\wat*\wat*\wlam*\wlam)))-(\we*\we/4+\wat*\wat*(-1+\wlam))/(\wat*\wlam);
			\jhae=(2*pow(\jat,3)*\jlam*(2+\jlam)+\je*\je*(2*\je-sqrt(4*\je*\je+\jat*\jat*\jlam*\jlam))-8*\jat*\je*(-2*\je+sqrt(4*\je*\je+\jat*\jat*\jlam*\jlam))-2*\jat*\jat*(2+\jlam)*(-2*\je+sqrt(4*\je*\je+\jat*\jat*\jlam*\jlam)))/(4*\jat*\jlam*(2*\jat+2*\je-sqrt(4*\je*\je+\jat*\jat*\jlam*\jlam)))-(\je*\je/4+\jat*\jat*(-1+\jlam))/(\jat*\jlam);
	} 
	\tikzstyle{scorestars}=[star, star points=5, star point ratio=2.25, draw, inner sep=1pt]%
	\tikzstyle{scoresquare}=[draw, rectangle, minimum size=1mm,inner sep=0pt,outer sep=0pt]%
	\begin{tikzpicture}[scale=\picscale,
		declare function={
			testxonep(\ps,\t)=\a+\ps-\t;
			testxonem(\ps,\t)=\ps-\a+\t; 
			bl(\x)=\ct+\at-\e/2-1/2*(sqrt(\lam*\lam*pow(\x+\hae+\at,2)+\e*\e)-\e+\lam*(\x+\hae+\at));
			br(\x)=\ct+\at-\e/2-1/2*(sqrt(\lam*\lam*pow(\x-\hae-\at,2)+\e*\e)-\e-\lam*(\x-\hae-\at));
			bc(\x)=\ct+sqrt(\lam*\lam*\x*\x+\e*\e)-\e;
			qbl(\qx)=\qct+\qat-\qe/2-1/2*(sqrt(\qlam*\qlam*pow(\qx+\qhae+\qat,2)+\qe*\qe)-\qe+\qlam*(\qx+\qhae+\qat));
			qbr(\qx)=\qct+\qat-\qe/2-1/2*(sqrt(\qlam*\qlam*pow(\qx-\qhae-\qat,2)+\qe*\qe)-\qe-\qlam*(\qx-\qhae-\qat));
			qbc(\qx)=\qct+sqrt(\qlam*\qlam*\qx*\qx+\qe*\qe)-\qe;
			wbl(\wx)=\wct+\wat-\we/2-1/2*(sqrt(\wlam*\wlam*pow(\wx+\whae+\wat,2)+\we*\we)-\we+\wlam*(\wx+\whae+\wat));
			wbr(\wx)=\wct+\wat-\we/2-1/2*(sqrt(\wlam*\wlam*pow(\wx-\whae-\wat,2)+\we*\we)-\we-\wlam*(\wx-\whae-\wat));
			wbc(\wx)=\wct+sqrt(\wlam*\wlam*\wx*\wx+\we*\we)-\we;
			jbl(\jx)=\jct+\jat-\je/2-1/2*(sqrt(\jlam*\jlam*pow(\jx+\jhae+\jat,2)+\je*\je)-\je+\jlam*(\jx+\jhae+\jat));
			jbr(\jx)=\jct+\jat-\je/2-1/2*(sqrt(\jlam*\jlam*pow(\jx-\jhae-\jat,2)+\je*\je)-\je-\jlam*(\jx-\jhae-\jat));
			jbc(\jx)=\jct+sqrt(\jlam*\jlam*\jx*\jx+\je*\je)-\je;
			rr(\x)=2*(pow(\x,4)/4-pow(\offset*\x,2)/2+pow(\offset,4)/4)+\qttestx;
		},
		 ] 
		\definecolor{tempcolor}{RGB}{250,190,0}
		\definecolor{darkgreen}{RGB}{40,190,40}
			
		 \draw[->,blue, thick] [domain=-\at/2-\offset:-\lrange, samples=150]   plot (\x, {wbl(\x+\offset)})  ;
		 \draw[blue, thick] [domain=-\at/2-\offset:\at/2-\offset, samples=150] plot (\x, {wbc(\x+\offset)})   ;
	
			 
		 \draw[->,blue, thick] [domain=\at/2-\offset:\rrange, samples=150]   plot (\x, {wbl(-\x-\offset)})  ;
		
		  \node[scale=\textscale]  at (-2.5,\wttestx+0.7) {$S_{n}$}; 
	
		
		\draw[<->] (-\lrange, \h) node[left, scale=\textscale] {$S_0$} -- (\rrange, \h) node[right, scale=\textscale] {$S_0$};
		\draw[<->] (-\lrange, \a) node[left, scale=\textscale] {$S$} -- (\rrange, \a) node[right, scale=\textscale] {$S$};
					  
		\draw[->, shorten <= 5pt,  shorten >= 1pt] (\psione,\h)  node[above right, scale=\textscale]{$z_L$}-- (\psione,\a) ;
		\draw[->, shorten <= 5pt,  shorten >= 1pt] (\jpsione,\h)  node[above right, scale=\textscale]{$z_R$}-- (\jpsione,\a) ;
		
	
		
				\draw[dashed, tempcolor,  ultra thick](\xs,\ys)--(\xm,\ym);
		
				\draw[dashed, tempcolor,  ultra thick](\xf,\yf)--(\xm,\ym);
	
		
		\draw[->] (\timex,\md-\tlen/2) --  (\timex,\md+\tlen/2) node[midway,right, scale=\textscale]{time};  
	
		\draw[dotted](\psione,\wttestx)--({testxonep(\psione,\wttestx)},\a);
		\draw[dotted](\psione,\wttestx)--({testxonem(\psione,\wttestx)},\a);
		\draw[dotted](\jpsione,\wttestx)--({testxonep(\jpsione,\wttestx)},\a);
		\draw[dotted](\jpsione,\wttestx)--({testxonem(\jpsione,\wttestx)},\a);
	\draw (\psione,\wttestx) node[ scoresquare, fill=gray]  {} ;
		\draw (\jpsione,\wttestx) node[ scoresquare, fill=gray]  {};
	
		\draw [black,fill](\xf,\a) circle [radius=\circsize] node [black,above=9,right=-4,scale=\textscale] {$\gamma$}; 
	\coordinate[label = below: (a)]  (D) at (\labelx,\labely);
		\end{tikzpicture}% pic 1		
	\begin{tikzpicture}[scale=\picscale,
	declare function={
		testxonep(\ps,\t)=\a+\ps-\t;
		testxonem(\ps,\t)=\ps-\a+\t; 
		bl(\x)=\ct+\at-\e/2-1/2*(sqrt(\lam*\lam*pow(\x+\hae+\at,2)+\e*\e)-\e+\lam*(\x+\hae+\at));
		br(\x)=\ct+\at-\e/2-1/2*(sqrt(\lam*\lam*pow(\x-\hae-\at,2)+\e*\e)-\e-\lam*(\x-\hae-\at));
		bc(\x)=\ct+sqrt(\lam*\lam*\x*\x+\e*\e)-\e;
		qbl(\qx)=\qct+\qat-\qe/2-1/2*(sqrt(\qlam*\qlam*pow(\qx+\qhae+\qat,2)+\qe*\qe)-\qe+\qlam*(\qx+\qhae+\qat));
		qbr(\qx)=\qct+\qat-\qe/2-1/2*(sqrt(\qlam*\qlam*pow(\qx-\qhae-\qat,2)+\qe*\qe)-\qe-\qlam*(\qx-\qhae-\qat));
		qbc(\qx)=\qct+sqrt(\qlam*\qlam*\qx*\qx+\qe*\qe)-\qe;
		wbl(\wx)=\wct+\wat-\we/2-1/2*(sqrt(\wlam*\wlam*pow(\wx+\whae+\wat,2)+\we*\we)-\we+\wlam*(\wx+\whae+\wat));
		wbr(\wx)=\wct+\wat-\we/2-1/2*(sqrt(\wlam*\wlam*pow(\wx-\whae-\wat,2)+\we*\we)-\we-\wlam*(\wx-\whae-\wat));
		wbc(\wx)=\wct+sqrt(\wlam*\wlam*\wx*\wx+\we*\we)-\we;
		jbl(\jx)=\jct+\jat-\je/2-1/2*(sqrt(\jlam*\jlam*pow(\jx+\jhae+\jat,2)+\je*\je)-\je+\jlam*(\jx+\jhae+\jat));
		jbr(\jx)=\jct+\jat-\je/2-1/2*(sqrt(\jlam*\jlam*pow(\jx-\jhae-\jat,2)+\je*\je)-\je-\jlam*(\jx-\jhae-\jat));
		jbc(\jx)=\jct+sqrt(\jlam*\jlam*\jx*\jx+\je*\je)-\je;
		rr(\x)=2*(pow(\x,4)/4-pow(\offset*\x,2)/2+pow(\offset,4)/4)+\qttestx;
	},
	 ] 
	\definecolor{tempcolor}{RGB}{250,190,0}
	\definecolor{darkgreen}{RGB}{40,190,40}
		
	 \draw[->,blue, thick] [domain=-\at/2-\offset:-\lrange, samples=150]   plot (\x, {wbl(\x+\offset)})  ;
	 \draw[blue, thick] [domain=-\at/2-\offset:0-\offset, samples=150] plot (\x, {wbc(\x+\offset)})   ;
	 \draw[blue, thick] [domain=-\offset:-\joffset, samples=150] plot (\x, {\wttestx})   ;
		 
	 \draw[->,blue, thick] [domain=\at/2-\joffset:\rrange, samples=150]   plot (\x, {wbl(-\x-\joffset)})  ;
	\draw[blue, thick] [domain=\at/2-\joffset:-\joffset, samples=150] plot (\x, {wbc(-\x-\joffset)})   ;

	  \node[scale=\textscale]  at (-2.5,\wttestx+0.7) {$S_{n}$}; 

	
	\draw[<->] (-\lrange, \h) node[left, scale=\textscale] {$S_0$} -- (\rrange, \h) node[right, scale=\textscale] {$S_0$};
	\draw[<->] (-\lrange, \a) node[left, scale=\textscale] {$S$} -- (\rrange, \a) node[right, scale=\textscale] {$S$};
				  
	\draw[->, shorten <= 5pt,  shorten >= 1pt] (\psione,\h)  node[above right, scale=\textscale]{$z_L$}-- (\psione,\a) ;
	\draw[->, shorten <= 5pt,  shorten >= 1pt] (\jpsione,\h)  node[above right, scale=\textscale]{$z_R$}-- (\jpsione,\a) ;
	

	
			\draw[dashed, tempcolor,  ultra thick](\xs,\ys)--(\xm,\ym);
	
			\draw[dashed, tempcolor,  ultra thick](\xf,\yf)--(\xm,\ym);

	
	\draw[->] (\timex,\md-\tlen/2) --  (\timex,\md+\tlen/2) node[midway,right, scale=\textscale]{time};  

	\draw[dotted](\psione,\wttestx)--({testxonep(\psione,\wttestx)},\a);
	\draw[dotted](\psione,\wttestx)--({testxonem(\psione,\wttestx)},\a);
	\draw[dotted](\jpsione,\wttestx)--({testxonep(\jpsione,\wttestx)},\a);
	\draw[dotted](\jpsione,\wttestx)--({testxonem(\jpsione,\wttestx)},\a);
\draw (\psione,\wttestx) node[ scoresquare, fill=gray]  {} ;
    \draw (\jpsione,\wttestx) node[ scoresquare, fill=gray]  {};

	\draw [black,fill](\xf,\a) circle [radius=\circsize] node [black,above=9,right=-4,scale=\textscale] {$\gamma$}; 
\coordinate[label = below: (b)]  (D) at (\labelx,\labely);
	\end{tikzpicture}% pic 1

	
	\vspace*{2px}
	\caption{(a) and (b) show how conditional expectations are going to depend on the subject. In (a) the subject consists of spacetime locations $(t, z_L)$ and $(t, z_R)$, whereas in (b) the subject is just $(t, z_L)$.  }
	\label{strangesubject}
	\end{figure}
We suppose that initially the state of $S_0$ is 
$$\ket{\Psi_0}=a\ket{\psi}_L\ket{0}_R+b\ket{0}_L\ket{\psi}_R$$ 
with $\abs{a}^2+\abs{b}^2=1$ (and where we have omitted all other terms corresponding to other spacetime locations). A photon then interacts with the joint system and is ``measured'' on $S$ to be in state $\ket{\gamma}$ thus determining the mass of the composite system to be located at $z_L$. Now we consider the observable $[\psi]_L$ and suppose its corresponding beable has as its subject a single spacetime location $(t,z_L)$, so that the expectation of $[\psi]_L$ is conditioned on the information in $S_n\cap S$ (taking the limit as $n\rightarrow\infty$ as usual) as depicted in figure \ref{strangesubject} (a). Then as soon as the photon has interacted with the mass centered at $z_L$, we can say that the system has collapsed to this mass center (since the photon will be detected on $S_n\cap S$ when we take the limit as $n\rightarrow\infty$). But if the photon registers at a location on $S$ that is within the light cone of $(t, z_R)$, and we consider the observable $[\psi]_R$ and its corresponding $(t, z_R)$-beable, then there is nothing to condition on in $S$ when calculating the expectation value of $[\psi]_R$, and so we will obtain a non-zero answer for this $(t, z_R)$-beable suggesting that there is some matter/energy at $z_R$ at time $t$, even though according to the $(t, z_L)$-beable all the mass/energy is located at $z_L$ at time $t$. This seems very strange. 

However, if we consider the beable whose subject consists of both $(t, z_L)$ and $(t, z_R)$, then because we have to choose a spacelike hypersurface that contains both $(t, z_L)$ and $(t, z_R)$ as depicted in figure \ref{strangesubject} (b), the system won't collapse to the mass center at $z_L$ as soon as the photon interacts with it. Rather, it will only collapse to this mass center once the photon registered on $S$ is outside the light cones of  both $(t', z_L)$ and $(t', z_R)$ for some $t'>t$. Once this time $t'$ has occurred, $z_L$ and the $z_R$ will no longer be entangled, and so we would be able to consider beables whose subjects were the single spacetime locations at $(t', z_L)$ and $(t', z_R)$.

This suggests we should take all the spacetime locations in an entangled system to be the subject of a beable where entanglement is determined by the information in $\tau_S$ as well as the initial state $\ket{\Psi_0}$. As the initial state evolves, particle interactions and particle spreading will tend to induce entanglement between spacetime locations, but ``measurement'' on $S$ will tend to disentangle spacetime locations. For example, if we have an entangled state of the form
$$a\ket{\psi^a_1}_L\ket{\psi^a_2}_R\ket{\gamma_a}+b\ket{\psi^b_1}_L\ket{\psi^b_2}_R\ket{\gamma_b}$$
then once the $\ket{\gamma}$-state is measured on $S$ outside the light cone of these two locations, then the state will disentangle to either 
$$\ket{\psi^a_1}_L\ket{\psi^a_2}_R\ket{\gamma_a}$$
or 
$$\ket{\psi^b_1}_L\ket{\psi^b_2}_R\ket{\gamma_b}.$$
So with all this entanglement and disentanglement going on, we can expect there to be many different beable-subjects.

In dealing with subjects that contain multiple spacetime locations, the issue of simultaneity needs to be considered carefully, for if all the spacetime locations are simultaneous in one frame of reference, they are not going to be simultaneous in another frame of reference. Moreover, what we deem to be simultaneous in a subject is going to affect which spacelike hypersurfaces $S_n$ contain the subject, and these spacelike hypersurfaces will in turn determine the state $\pi_n\ket{\Psi_n}$ and hence affect what spacetime locations are entangled. Since entanglement is being proposed as the criterion for spacetime locations to belong to the same subject, we therefore need to consider carefully what is the most appropriate frame of reference for the subject such that its spacetime locations are simultaneous. 

A natural candidate for such a frame of reference would be one in which the expectation value of the center of mass of the subject had zero velocity. We still need to be careful, since for greater $n$, the intersection between $S$ and the spacelike hypersurface $S_n$ that contains the subject is going to increase, and so there is typically going to be more  information in $\tau_S$ available to condition on. This additional information could then possibly lead to disentanglement of regions that were previously entangled for  smaller $n$. If such disentanglement occurred, we would then need to choose one of these new (and smaller) entangled regions and choose a frame of reference in which the new entangled region was simultaneous and in which the center of mass of the entangled region had zero velocity. So we would have to proceed in an iterative manner, but one would hope that eventually we would obtain a set of entangled locations that could serve as the natural subject of a beable. Hopefully, the corresponding beables of this entangled system would be Lorentz invariant in a way analogous to the rest mass of a particle being Lorentz invariant.

In addition to specifying what the subject of a beable should be, we would also need to consider what the beable itself should be. A guiding principle in specifying what the beables should be is to think about what the beables would be in the many-worlds interpretation, and then consider in this light what the most natural beables would be when one introduces a spacelike hypersurface $S$ with its ``measured'' value of $T_S$. 

Now proponents of the many-worlds interpretation often speak as though there is nothing more to reality that the universal state $\ket{\Psi}$ of the universe. If this state were to factorize as 
\begin{equation}\label{PhiBeables}
	\ket{\Psi}=\prod_j\ket{\Psi_j}
\end{equation}
 where $\ket{\Psi_j}$ cannot be factorized further, one might be inclined to say that the beables would be the states $\ket{\Psi_j}$ since these states could be specified independently of each other. But as soon as the systems these states described interacted with each other, they would become entangled, and since there is no principle of disentanglement in the  many-worlds interpretation, eventually all systems would get entangled with one another. It would thus seem natural to say there was only one beable in the many-worlds interpretation, namely $\ket{\Psi}$, and that its subject was the whole of physical reality. But once we have a principle by which disentanglement can occur, then we can expect a factorization of $\ket{\Psi}$ as in equation (\ref{PhiBeables}), and we could then deem the $\ket{\Psi_j}$ to be the beables, and the subject of each $\ket{\Psi_j}$ would be the entangled region in spacetime that it described.  

In the context of Kent's theory, I  am thus proposing that the beables would be of two sorts. Firstly, there would be the beables at every spacetime location $x$ of $S$, and the value of each beable would be the  ``measured'' value $\tau_S(x)$ of $T_S(x)$. Here, I am in agreement with Kent. 

Secondly, there would be a beable for every region of spacetime $B$ that satisfied the following conditions:
\begin{enumerate}
\item  the spacetime locations of $B$ are all spacelike separated from one another,
\item there is a Lorentz transformation under which all the members of $B$ become simultaneous,
\item  if $S_B$ is any spacelike hypersurface that contains $B$, and $S_B$ is tiled with equally sized cells that can serve as a course-graining for $S_B$, then $B$ will consist of a finite number of cells,
\item given a course-graining of $S_B$ in which $B$ consists of $M$ cells, there will be a sequence of spacelike hypersurfaces $S_n$ containing $B$ and corresponding projections $\pi_n$ (as described on page \pageref{tauprojection})  such that for sufficiently large $n$,  $\pi_n\ket{\Psi_n}$ will have an irreducible factor of the form
$$
\ket{\Psi_B}=\sum_{n\in\mathbb{N}^M} c_{n}\prod_{l=1}^{M}\ket{\xi_{k_{l},n_l}}
$$
where the set $\{k_l:l\}$ indexes the cells of $B$ and where we use the notation of  page \pageref{Sistate},
\item when the Lorentz transformation is applied that makes all the members of $B$ simultaneous,  the center of mass of the region $B$ calculated using $\ket{\Psi_B}$ will have zero velocity.
\end{enumerate}
If these conditions hold, then I propose that $\ket{\Psi_B}$ is the beable with $B$ as its subject. This in contrast to Kent's beable $\ev*{\hat{T}^{\mu\nu}(y)}$ whose subject is the single location $y$.

Given the first kind of beable $\tau_S(x)$, the second kind of beable, $\ket{\Psi_B}$, will not be superfluous, since $\ket{\Psi_B}$ will say something about the manner in which  $\tau_S(x)$ could have been other than it is. For if other photons had interacted with $B$ at different times, the manner in which they interacted with $B$ would depend on $\ket{\Psi_B}$, and from $\ket{\Psi_B}$ we would be able to calculate other possible ``measurements'' of $T_S(x)$ (and their probability) that could have made on $S$.

Despite the possible shortcomings of Kent's account of beables that have been mentioned in this section, Kent nevertheless provides a very valuable interpretation of quantum physics: it is Lorentz invariant, it makes predictions consistent with standard quantum theory, it is an interpretation in which parameter independence holds, and it is a one world interpretation. Since parameter independence holds, Kent's theory is superior to the pilot wave interpretation, and since it is a one-world interpretation, it avoids the absurdities of the many-worlds interpretation. Kent's theory therefore deserves to be taken very seriously as a possible solution to the measurement problem.

\nocite{Shimony93}
\nocite{Bell}

\section{A description of Kent's Interpretation of Quantum Physics}
In this section I will provide an account of Kent's interpretation of quantum physics focusing on the ideas Kent presents in his 2014 paper.\footnote{\cite{Kent2014}.} This section is primarily descriptive. We'll wait until the next section to consider how Kent's interpretation addresses the issues Butterfield raises.

Kent's interpretation of quantum physics has some similarities in common with the pilot wave interpretation. Firstly,  there is no wave-function collapse in Kent's interpretation. Secondly, some additional values beyond standard quantum theory (i.e. in addition to the quantum wave function) are included in Kent's interpretation. And thirdly, Kent's interpretation is a one-world interpretation of quantum physics. I'll consider these three features of Kent's interpretation in some detail as I describe his theory. I'll then present an account of his toy model that provides a simple example of how the ideas of his theory fit together. 

\subsection{The No-collapse Feature of Kent's Interpretation}
We first consider the no-collapse feature of Kent's interpretation. This is a feature that belongs both to the Many-World's interpretation and to the pilot wave interpretation. In all three interpretations, the wave-function deterministically evolves according to the Schr\"{o}dinger equation. The Schr\"{o}dinger equation itself describes how a quantum state evolves over time. The precise formula for the Schr\"{o}dinger equation need not concern us here, but all we need to know is that the Schr\"{o}dinger equation determines a so-called \textbf{unitary operator} $U(t',t)$. What this means is that if a system is in a state $\ket{\psi}$ at time $t$, then it will be in the state $\ket{\psi'}=U(t',t)\ket{\psi}$
at time $t'$. A unitary operator $U$ has the property that if $\ket{\psi'}=U\ket{\psi}$ and $\ket{\chi'}=U\ket{\chi}$, then 
\begin{equation}\label{unitarycond}
\ip{\chi'}{\psi'}=\ip{\chi}{\psi}.\protect\footnotemark
\end{equation}
\footnotetext{A unitary operator also has the property that it is invertible: there is an operator $U^{-1}$ such that $U U^{-1}$ and $U^{-1} U$ are the identity operator $I$, i.e. $U^{-1} U\ket{\psi}=UU^{-1}\ket{\psi}=\ket{\psi}$ for any state $\ket{\psi}$.}Under the Copenhagen interpretation, the state will evolve unitarily for the most part, but there will typically be a non-unitary change in the state whenever there is a measurement. To see why this is, we recall the situation described on page \pageref{span} where a state $\ket{\psi}$ is the sum of orthonormal eigenstates $\ket{\psi_i}$ of some observable: 
\begin{equation}\tag{\ref{span} revisited}
\ket{\psi}=\sum_{i=1}^N\alpha_i\ket{\psi_i}.
\end{equation}
If $\ket{\psi_i'}=U(t,t_0)\ket{\psi_i}$ and $\ket{\psi'}=U\ket{\psi}$, then the unitary condition (\ref{unitarycond}) implies that 
$\ip{\psi'_i}{\psi'}=\alpha_i$. But if on measurement at time $t$, the system collapses to the $\ket{\psi'_j}$ state for $j\neq i$ so that $\ket{\psi'}=\ket{\psi_j'},$ we will have $\ip{\psi'_i}{\psi'}=0$. So in the Copenhagen interpretation, the unitary condition (\ref{unitarycond})  will fail if $\alpha_i\neq 0$. 

However, in non-collapse models  such as the pilot wave interpretation, the many-worlds interpretation, and Kent's interpretation, the wave function always evolves unitarily. 

\subsection{The Additional Values of Kent's Interpretation\label{additional}}
Secondly, like the pilot wave interpretation, some additional values beyond standard quantum theory (i.e. in addition to the quantum wave function) are included in Kent's interpretation. In the pilot wave interpretation, these additional values are the positions and momenta of all the particles, whereas in Kent's interpretation, the additional values specify the mass-energy density on a three-dimensional distant future hypersurface in spacetime. We refer to this hypersurface as $S$. 

To understand the nature of this three-dimensional hyperspace $S$, we recall that in special relativity, there is no such thing as absolute time. So for instance, two spacetime locations might seem to be simultaneous from one frame of reference, but another person travelling at a different velocity would judge the same two spacetime locations to be non-simultaneous. But it is not the case that for any two spacetime locations we can always find a frame of reference in which the two spacetime locations are simultaneous. Sometimes this is not possible. But we refer to spacetime locations that could be simultaneous in some frame of references as being \textbf{spacelike-separated}. For example, the two spacetime locations $O$ and $A$  in figure \ref{cone} are spacelike-separated. 

\begin{figure}[!h]
\centering

 \tikzset{surface/.style={draw=blue!70!black, fill=yellow!20!white, fill opacity=.6}}

 \newcommand{\coneback}[4][]{
 \draw[canvas is xy plane at z=#2, #1] (45-#4:#3) arc (45-#4:225+#4:#3) -- (O) --cycle;
 }
 \newcommand{\conefront}[4][]{
 \draw[canvas is xy plane at z=#2, #1] (45-#4:#3) arc (45-#4:-135+#4:#3) -- (O) --cycle;
 }
 \begin{tikzpicture}[tdplot_main_coords, grid/.style={help lines,violet!40!white,opacity=0.5},scale=1.25]
  \coordinate (O) at (0,0,0);
  
     \coneback[surface]{-3}{2}{-12}
   \conefront[surface]{-3}{2}{-12} 
  
   \fill[brown!40!white,opacity=0.5] (-4,-4,0) -- (-4,4,0) -- (4,4,0) -- (4,-4,0) -- cycle;
  
   \foreach \x in {-4,...,4}
     \foreach \y in {-4,...,4}
     {
         \draw[grid] (\x,-4) -- (\x,4);
         \draw[grid] (-4,\y) -- (4,\y);
         \draw[violet] (-4,4)--(-4,-4)--(4,-4)--(4,4)--cycle;
     }

   \draw[->] (-4,0,0) -- (4,0,0) {};
   \draw[->] (0,-4,0) -- (0,4,0) {};
   \coneback[surface]{3}{2}{12}
   \draw[-,dashed] (0,0,-2.65) -- (0,0,2.65) node[above] {};
   \draw[-,dashed] (0,0,-4) -- (0,0,-3.35) node[above] {};
   \draw[->,dashed] (0,0,3.35) -- (0,0,4) node[above] {time};
   \conefront[surface]{3}{2}{12}

   \draw[red, thick] (0,0,0) -- (4,0,2) node[below, pos=0.6, rotate=26.5651,scale=0.80,black] {spacelike-separated};
   \draw[red, thick] (0,0,0) -- (1.56,0.6,2.4) node[below, pos=0.62, rotate=55.1459,scale=0.80,black] {lightlike-separated};
   \draw[red, thick] (0,0,0) -- (-0.5,-0.85,2.2) node[above, pos=0.57, rotate=-65.8557,scale=0.80,black] {timelike-separated};
   \node[black] at (0,0,3) {Future Light Cone};
   \node[black] at (0,0,-3) {Past Light Cone};
   
   \fill (4,0,2) circle (2pt) node[above right] {$A$};
   \fill (0,0,0) circle (2pt) node[below ] {$O$};
   \fill (-0.5,-0.85,2.2) circle (2pt) node[above left] {$C$};
   \fill (1.56,0.6,2.4) circle (2pt) node[above left] {$B$};
  
   \node[black] at (0,4.7,0) {space};
   \node[black] at (5,-0.3,0) {space};
 \end{tikzpicture}
 \caption{The meaning of spacelike, timelike and lightlike-separation when there are two space dimensions and one time dimension.}
 \label{cone}
\end{figure}
There are also spacetime locations in spacetime that could be connected by a beam of light such as the two spacetime locations $O$ and $B$ in figure \ref{cone}. Such spacetime locations are referred to as being \textbf{lightlike-separated}. For any given spacetime location, the spacetime locations that are lightlike-separated from it form two cones called the future light cone and the past light cone as shown in figure \ref{cone}. Because light appears to travel at the same speed no matter what frame of reference one uses, the light cone of a spacetime location remains invariant when one change from one reference frame to another. In other words, if another spacetime location lies on the light cone of a spacetime location in one frame of reference, then it must lie on the light cone of this spacetime location in every frame of reference. 

Figure \ref{cone} also depicts two spacetime locations $O$ and $C$ that are \textbf{timelike-separated}. Such spacetime locations lie within the light cones of each other, and when two spacetime locations are timelike-separated, it is always possible to choose a frame of reference in which the two spacetime locations are located at the same point in space, but with one spacetime location occuring after the other  depending on which spacetime location is in  the future light cone of the other. 

Now a three-dimensional hypersurface $S$ in spacetime is a three-dimensional surface in which all the spacetime locations of $S$ are spacelike-separated. Kent assumes that this hypersurface $S$ is in the distant future of an expanding universe so that nearly all the particles that can decay have already done so, and that all the particles that are not bound together are very far from each other so that the probability of any particle collisions is very small. In other words, all the interesting physics in the universe has played its course before $S$.

At every point of $x\in S$, there is a quantity $T_S(x)$ called the \label{massenergydensity}\textbf{mass-energy density}.\footnote{The definition of $T_S(x)$ will be discussed in section \ref{massenergydensity}.} The important thing to note about $T_S(x)$ is that it does not depend on which frame of reference one is in.\footnote{The reason for why this is will be discussed in section \ref{massenergydensity}.}
  This property is in contrast to many physical properties that do depend on which frame of reference one is in. For example, the kinetic energy of an object will depend on the calculated velocity of the object, and this velocity will in turn depend on the frame of reference in which this calculation is done. 
  
Now according to the Tomonaga-Schwinger formulation of relativistic quantum physics,\footnote{See \cite{SchwingerJulianI}; \cite{TomonagaI}} for any hypersurface $S$, there is a Hilbert space $H_S$ of states describing $S$. One of the properties of $H_S$ is that for any spacetime location $x\in S$, there is an observable $\hat{T}_S(x)$ acting on $H_S$ such that if  $\ket{\Psi}\in H_S$ is an eigenstate of  $\hat{T}_S(x)$  with eigenvalue $\tau(x)$, then $\ket{\Psi}$ corresponds to a state of $S$ in which the energy-density at $x$ is $\tau(x)$. This can be done in such a way that $\hat{T}_S(x)$ only depends on $x$ rather than on the hypersurface $S$ that contains $x$. Furthermore, if $x$ and $y$ are any two spacetime locations of $S$, then the observables  $\hat{T}_S(x)$ and $\hat{T}_S(y)$ commute. In other words,
$$\hat{T}_S(x)\hat{T}_S(y)=\hat{T}_S(y)\hat{T}_S(x).$$
The commutativity of all the $\hat{T}_S(x)$ for $x\in S$ means that if $\ket{\Psi}$ is an eigenstate of $\hat{T}_S(x)$, then for any $y\in S$, $\hat{T}_S(y)\ket{\Psi}$ is also an eigenstate of  $\hat{T}_S(x)$ with the same eigenvalue as $\ket{\Psi}$. The invariance of the $\hat{T}_S(x)$-eigenspace under the action of   $\hat{T}_S(y)$ means that we can create simultaneous eigenstates for both  $\hat{T}_S(x)$ and  $\hat{T}_S(y)$, albeit with different eigenvalues. But because $x$ and $y$ are arbitrary points of $S$, this means that we can express any state $H_S$ as a superposition of simultaneous $\hat{T}_S$-eigenstates of the form $\ket{\Psi^{(i)}}$ where
  $\hat{T}_S(x)\ket{\Psi^{(i)}}=\tau^{(i)}_S(x)\ket{\Psi^{(i)}}$ for all $x\in S$, where $\tau^{(i)}_S(x)\geq 0$ is a possible energy-density measurement over the whole of $S$.\footnote{We will gloss over \label{glosssim}the details of how to make this rigorous for continuous variables $x$ and continuous indices $i$. It is sufficient to approximate the continuous variables and indices as discrete variables and indices when thinking about the simultaneous $\hat{T}_S(x)$-eigenspaces, and one can choose the granularity of this approximation to achieve whatever level of accuracy one desires.}\label{simultaneous}
  
The additional values beyond standard quantum theory that are included in Kent's interpretation are given by one of these possible outcomes for an energy-density measurement over the whole of $S$. We will denote this outcome as $\tau_S(x)$, and we will let $\ket{\Psi}$ be the state such that $\hat{T}_S(x)\ket{\Psi}=\tau_S(x)\ket{\Psi}$ for all $x\in S$. But although we speak of the measurement of $T_S(x)$ over $S$ as being $\tau_S(x)$, this is only a notional measurement. Thus, we speak of the measurement of  $T_S(x)$ on $S$ only to mean that $T_S(x)$ has a determinate value on $S$ despite the quantum state of $S$ given by Schr\"{o}dinger's equation in general being in a superposition of simultaneous $\hat{T}_S(x)$-eigenstate for every $x\in S$. How this determination of $T_S(x)$ comes about is up to one's philosophical preferences. For example, one could suppose that it was simply by divine fiat that this determination  of $T_S(x)$ came about.\footnote{I will discuss my philosophical preference in the final chapter.} 

Nevertheless, the particular density $\tau_S(x)$ which is found to describe $S$ can't be absolutely anything. Rather, we suppose there is a much earlier hypersurface $S_0$ which is described by a state $\ket{\Psi_0}$ belonging to a Hilbert space $H_{S_0}$ as shown in figure \ref{S1}.  It is assumed that all physics that we wish to describe takes place between these two hypersurfaces $S_0$ and $S$. In figure \ref{S1}, we therefore let $y$ depicts a generic spacetime location that we wish to describe. 
 \begin{figure}[ht!]
\captionsetup{justification=justified}
\centering

\tikzmath{
\a= 1;  
\h=-1;
\md = (\a+\h)/2;
\lrange = 4;
\rrange=2;
\fictlabel=(\rrange-\lrange)/2;
\tlen=0.75;
\labelpos=(-\lrange-\a)/2;
} 

\begin{tikzpicture}[thick, scale=2]
\def\dotsize{0.7}
\definecolor{tempcolor}{RGB}{0,151,76}
\draw[<->] (-\lrange, \h) node[left] {$S_0$} -- (\rrange, \h) node[right] {$S_0$};
\draw[<->](-\lrange, \a) node[left] {$S$} --  (\rrange, \a)  node[right] {$S$};                   
\draw[->] (\rrange,\md-\tlen/2) --  (\rrange,\md+\tlen/2) node[midway,right]{time}; 
\coordinate[label = above: Notional Measurement of $T_S(x)$ on $S$]  (D) at (\fictlabel,\a+0.2); 
\node (start) at (\labelpos,\h) [below] {Initial State $\ket{\Psi_0}$};
\node (final) at (\labelpos,\a) [below] {Unitary Evolution $U_{SS_0}\ket{\Psi_0}$};
\draw [->, shorten <= 5pt] (start) [above] -- (final); 
\filldraw (0,0) circle (\dotsize pt) node [below right] {$y$} ;
\end{tikzpicture}

\vspace*{2px}
\caption{A notional measurement of $T_S(x)$ is made for all $x\in S$. The simultaneous  $\hat{T}_S$-eigenstate $\ket{\Psi}$ with $\hat{T}_S(x)\ket{\Psi}=\tau_S(x)\ket{\Psi}$ is selected with probability $\abs{\mel{\Psi}{U_{SS_0}}{\Psi_0}}^2$. The values $\tau_S(x)$ obtained for $T_S(x)$ are then used to calculate the physical properties at the spacetime location $y$.  }
\label{S1}
\end{figure} 
\vspace*{-12px}

 According to Schwinger,\footnote{\cite[p.1459]{SchwingerJulianI}} there is a unitary operator\label{SchwingerUnitaryOP}  $U_{SS_0}$ that maps states in $H_{S_0}$ such as $\ket{\Psi_0}$ to states in $H_S$. Then the probability $P(\Psi|\Psi_0)$ that  $S$ will be found to be in the state $\ket{\Psi}$ with mass-energy density $\tau_S(x)$ given that $S_0$ was initially in the state $\ket{\Psi_0}$ will be given by the Born Rule (see page \pageref{bornrule}):
 \begin{equation}\label{bornpsi}
 P(\Psi|\Psi_0) = \abs{\mel{\Psi}{U_{SS_0}}{\Psi_0}}^2.
 \end{equation}
It's possible that there might be several different states of $H_S$ that have the same mass-energy density $\tau_S(x)$, but one of these states is still realized with probability given by equation (\ref{bornpsi}). But it is $\tau_S(x)$ rather than one of the eigenstates with mass-energy density $\tau_S(x)$ that constitute the additional values that Kent adds to standard quantum theory. 

\subsection{The One-World Feature of Kent's Interpretation}
The third similarity Kent's interpretation shares with the pilot wave interpretation is that it is a one-world interpretation of quantum physics. It will be helpful to contrast this with the many-worlds interpretation. 

Unlike the many-worlds interpretation, there are no superpositions of living and dead cats in Kent's interpretation. Recall that in the many-worlds interpretation, Schr\"{o}dinger will still only observes his cat to be either dead or alive, and not both dead and alive, but Schr\"{o}dinger himself goes into a superposition of observing his cat to be alive and his cat to be dead. In the many-worlds interpretation, there is thus a difference between observing something to be so, and something actually being so: the observation is of a particular outcome, but the reality is a superposition of different outcomes. 

To capture this distinction between observation and reality, Bell speaks of \textbf{beables}.\label{beabledef} Bell introduces the term beable when speculating on what would be a more satisfactory physical theory than quantum physics currently has to offer.\footnote{See \cite{Bell2}} Bell says that such a theory should be able to say of a system not only that such and such is observed to be so, but that such and such be so. In other words, a more satisfactory theory would be a theory of beables rather than a theory of observables. On the macroscopic level, these beables should be the underlying reality that gives rise to all the familiar things in the world around us, things like cats, laboratories, procedures, and so on. For example proponents of the pilot wave interpretation believe that the beables are all the particles with their precise position and momentum. But whatever these beables are, it is because of them that a scientist can observe a physical system to be in such and such a state. Thus, observables are ontologically dependent on beables.   

Now the beables in Kent's one world interpretation are expressed in terms of a physical quantity called the \textbf{stress-energy tensor}  $T^{\mu\nu}(y)$.\label{stressenergy}  For any spacetime location $y$, the stress-energy tensor $T^{\mu\nu}(y)$ is an array of 16 values corresponding to each combination of $\mu, \nu=0,1,2,$ or $3$. The value $T^{00}(y)$ is the energy density at $y$ divided by $c^2$,\footnote{This is not to be confused with the mass-energy density $T_S(x)$ defined for $x$ on a hypersurface $S$. As will be shown in section \ref{LorentzInvarianceSection},   all 16 elements of $T^{\mu\nu}(x)$ will typically be needed to calculate $T_S(x)$.} whereas the other values of $T^{\mu\nu}(y)$ indicate how much energy and momentum flow across different surfaces in the neighborhood of $y$. 

It was mentioned in the previous section that for any spacetime location $x\in S$, there is an observable $\hat{T}_S(x)$ acting on $H_S$. It turns out that for any $\mu, \nu=0,1,2,$ or $3$, there is also an observable  $\hat{T}^{\mu\nu}(x)$ acting on $H_S$, such that if $\ket{\Psi}\in H_S$ is a simultaneous eigenstate of $\hat{T}^{\mu\nu}(x)$ with eigenvalue $\tau^{\mu\nu}(x)$ for all $x\in S$, then $\ket{\Psi}$ corresponds to a state of $S$ in which $T^{\mu\nu}(x)$ is  $\tau^{\mu\nu}(x)$ for all $x\in S$.\footnote{Note however, that such a simultaneous eigenstate is only for a fixed choice of $\mu$ and $\nu$, since in general, $\hat{T}^{\mu\nu}(x)$ and $\hat{T}^{\mu'\nu'}(x)$ will not commute for $\mu\neq\mu'$ or $\nu\neq\nu'$. } Moreover, the observable $\hat{T}_S(x)$ is expressible in terms of the  $\hat{T}^{\mu\nu}(x)$-observables.\footnote{See section  \ref{LorentzInvarianceSection} for an explanation for why this is so.} 

Now the  beables in Kent's interpretation are defined at each spacetime location $y$ that occurs after $S_0$ and before $S$. For such a spacetime location $y$, the beables will be determinate values of the stress-energy tensor $T^{\mu\nu}(y)$, but calculated from the expectation of the observable $\hat{T}^{\mu\nu}(y)$ conditional on the energy-density on $S$ being given by $\tau_S(x)$ for all $x$ outside the light cone of $y$. 

To explain what this means in more detail, recall the definition of expectation in equation (\ref{expectation2}) and the expectation formula (\ref{evev}) for an observable. If the beable in question was simply the expectation of $\hat{T}^{\mu\nu}(y)$ without conditioning on the value of the energy-density on $S$, then the $T^{\mu\nu}(y)$-beable would just be $\ev*{\hat{T}^{\mu\nu}(y)}{\Psi'}$ where $\ket{\Psi'}=U_{S'S_0}\ket{\Psi_0}$ for any hypersurface $S'$ that goes through $y$.\footnote{This can be done such that $\ev*{\hat{T}^{\mu\nu}(y)}{\Psi'}$ does not depend on the hypersurface $S'$ other than the fact that it contains $y$. For more details see \cite{SchwingerJulianI}.} However, such a beable would give a description of reality that was very different from what we observe -- for instance, in a Schr\"{o}dinger cat-like experiment, there would be energy-densities corresponding to both the cat being alive and the cat being dead in the same world. To overcome this defect, information about the mass-energy density on $S$ is required, specifically the values of $\tau_S(x)$ for $x\in S^1(y)$ where  $S^1(y)$ is defined to consist of all the spacetime locations of $S$ outside the light cone of $y$ as depicted in figure \ref{S2}.  



 \begin{figure}[ht!]
\captionsetup{justification=justified}
\centering

\tikzmath{
\a= 1;  
\e = 0.1;
\h=-1;
\hae=(3*\a*\a+6*\a*\e+7*\e*\e-3*\a*sqrt(\a*\a+4*\e*\e)-4*\e*sqrt(\a*\a+4\e*\e))/(4*\a+4*\e-2*sqrt(\a*\a+4*\e*\e));
\hae=0.0463647;
\hae=0.0858615;
\circsize=1.2;
\md = (\a+\h)/2;
\lrange = 4;
\rrange=2;
\ss=(-\lrange-\a)/2;
\sss=\a+(\rrange-\a)/2;
\tlen=0.75;
\labelpos=(-\lrange-\a)/2;
} 

\begin{tikzpicture}[thick, scale=2]

\def\dotsize{0.7}

\definecolor{tempcolor}{RGB}{0,151,76}
\draw[<->] (-\lrange, \h) node[left] {$S_0$} -- (\rrange, \h) node[right] {$S_0$};
\filldraw (0,0) circle (\dotsize pt) node [below right] {$y$} ;
              


\draw[<-] (-\lrange, \a)  -- (-\a, \a)  {};
\draw[gray, dotted] (-\a, \a) -- (0,0) {};
\draw[gray, dotted](0,0) -- (\a, \a) {};
\draw[->](\a, \a) --  (\rrange, \a)  ;         
\coordinate (B) at (\a,\a);
\node at (B)[red,circle,fill,inner sep=\circsize pt]{};
\coordinate (A) at (-\a,\a);
\node at (A)[red,circle,fill,inner sep=\circsize pt]{};
\coordinate (C) at (0,0);
\node at (C)[black,circle,fill,inner sep=\circsize pt]{};



\coordinate[label = above:$S^1(y)$]  (D) at (\ss,\a);
\coordinate[label = above:$S^1(y)$]  (D) at (\sss,\a);

\draw[->] (\rrange,\md-\tlen/2) --  (\rrange,\md+\tlen/2) node[midway,right]{time}; 
 
\node (start) at (\labelpos,\h) [below] {Initial State $\ket{\Psi_0}$};
\node (evolution) at (\labelpos,\md+0.05) [below] {Unitary Evolution $U_{S'S_0}\ket{\Psi_0}$};
\node (final) at (\labelpos,\a) [below] {Unitary Evolution $U_{SS_0}\ket{\Psi_0}$};
%\node at (-\ss+0.17,\mn-0.18){$-a_0$};
\draw [->, shorten <= 5pt] (start) [above] -- (evolution); 
\draw [->] (evolution) -- (final); 
\end{tikzpicture}

\vspace*{2px}
\caption{The set $S^1(y)$ consists of all the spacetime locations of $S$ outside the light cone of $y$. The $T^{\mu\nu}(y)$-beables are calculated using the initial state $\ket{\Psi_0}$ together with the values of $\tau_S(x)$ for $x\in S^1(y)$. }
\label{S2}
\end{figure}


The conditional expectation that we need to calculate depends on the notion of \textbf{conditional probability}. In probability theory, the conditional probability $P(q|r)$ that a statement $q$ is true given that a statement $r$ is true is given by the formula
\begin{equation} \label{conditionalprobability}
  P(q|r)=\frac{P(q\& r)}{P(r)}. 
\end{equation}
If we now define $q(\kappa)$ to be the statement that some quantity $K$  takes the value $\kappa$, then the  \textbf{conditional expectation} of  $K$ given $r$ will be given by the formula
\begin{equation}\label{conditionalexpectation}
\ev*{K}_r\myeq\sum_\kappa  \frac{P(q(\kappa),r)\kappa}{P(r)}
\end{equation}
where the summation is over all the possible values $\kappa$ that $K$ can take.

If we let $\ev*{T^{\mu\nu}(y)}_{\tau_S}$ stand\label{Kentbeable} for Kent's $T^{\mu\nu}(y)$-beable, then this can be calculated from (\ref{conditionalexpectation}) by taking $r$ to be the statement that 
 $T_S(x)=\tau_S(x)$ for all $x\in S^1(y)$, and $q(\tau)$ to be the statement that the universe is found to be in a quantum eigenstate of the observable $\hat{T}^{\mu\nu}(y)$ with eigenvalue $\tau$. It is these $T^{\mu\nu}(y)$-beables that give a one-world picture of reality in Kent's interpretation.
 
\subsection{Kent's toy example}\label{toysection}
To get a feel for how all the elements of Kent's interpretation fit together, it is helpful to consider Kent's toy model example that he discusses in his 2014 paper.\footnote{See \cite[p.3--4]{Kent2014}.} In his toy model, Kent considers a system in one spatial dimension which is the superposition of two localized states $\psi_0^\text{sys}=c_1 \psi_1^\text{sys}+c_2\psi_2^\text{sys}$ where $\psi_1^\text{sys}$ is localized at spatial location $z_1$, $\psi_2^\text{sys}$ is localized at spatial location $z_2$, and $\abs{c_1}^2+\abs{c_2}^2=1$. According to the Copenhagen interpretation, a measurement on this system would collapse the wave function of $\psi_0^\text{sys}$ to the wave function of $\psi_1^\text{sys}$ with probability $\abs{c_1}^2$, and to the wave function of $\psi_2^\text{sys}$ with probability $\abs{c_2}^2$. The purpose of Kent's toy model is to show that within his interpretation, there is something analogous to wave function collapse.  In order for this ``collapse'' to happen, one needs to consider how the system interacts with light. Thus, Kent supposes that a photon (which is modelled as a point particle) comes in from the left, and as it interacts with the two states $\psi_1^\text{sys}$ and $\psi_2^\text{sys}$, the photon enters into a superposition of states, corresponding to whether the photon reflects off the localized $\psi_1^\text{sys}$-state at time $t_1$ or the localized $\psi_2^\text{sys}$-state at time $t_2$. The photon in superposition then travels to the left and eventually reaches the one dimensional hypersurface $S$ at locations $\gamma_1$ and $\gamma_2$ as shown in figure  \ref{TM1}.

 \begin{figure}[ht!]
\captionsetup{justification=justified}
\centering
\tikzmath{
\a=5.3;  
\psione=0;
\psitwo=1;
\e = 0.1;
\h=-0.8;
\phstartx=-2.9;
\phstarty=-.8;
\tbeg=\phstarty;
\ca=\phstartx-\tbeg;
\ttwo=\psitwo-\ca;
\cb=\ttwo+\ca;
\xend=\ttwo-\a+\cb;
\tone=\psione-\ca;
\cc=\tone+\ca;
\xendz=\tone-\a+\cc;
\tthree=\cc+\tone-\psitwo;
\circsize=0.05;
\md = (\a+\h)/2;
\tlen=0.75;
\textscale = 0.8;
\picscale = 0.95;
\nudge=0.1;
\tnudge=(\ttwo-\tone)/2;
\ttesta=2*\tone-\ttwo-\tnudge;
\ttestb=2*\tone-\ttwo+\tnudge;
\ttestc=\ttwo-\tnudge;
\ttestd=\ttwo+\tnudge;
\sonea=\a+\psione-\ttesta;
\sadd=0.5;
\lrange = -(\psione-\a+\ttesta)+\sadd;
\rrange =\a+\psitwo-\ttesta+\sadd; 
\timex=\rrange-0.7;
} 
\begin{tikzpicture}[scale=1.2,
declare function={
	testxonep(\ps,\t)=\a+\ps-\t;
	testxonem(\ps,\t)=\ps-\a+\t; 
},
 ] 

\definecolor{tempcolor}{RGB}{250,190,0}
\definecolor{darkgreen}{RGB}{40,190,40}
\draw[<->] (-\lrange, \h) node[left, scale=\textscale] {$S_0$} -- (\rrange, \h) node[right, scale=\textscale] {$S_0$};
\draw[<->] (-\lrange, \a) node[left, scale=\textscale] {$S$} -- (\rrange, \a) node[right, scale=\textscale] {$S$};

              
\draw[->, shorten <= 5pt,  shorten >= 1pt] (\psione,\h) node[below, scale=\textscale]{$\psi_1^{\text{sys}}$} node[above left, scale=\textscale]{$Z=z_1$}-- (\psione,\a);
\draw[->,  shorten <= 5pt,  shorten >= 1pt] (\psitwo,\h) node[below, scale=\textscale]{$\psi_2^{\text{sys}}$} node[above right, scale=\textscale]{$Z=z_2$}-- (\psitwo,\a);

\draw[dashed, tempcolor, ultra thick](\phstartx,\tbeg) node[below left=-2,scale=\textscale]{}--(\psitwo,\ttwo ) node[above, pos=0.3, rotate=45,scale=0.7,black] {$Z=c(t-t_1)+z_1$} node[below, pos=0.3, rotate=45,scale=0.7,black] {photon};
\draw[dashed, tempcolor, ultra thick](\psitwo,\ttwo)--(\xend,\a) node[above, pos=0.48, rotate=-45,scale=0.7,black] {$Z=c(t_2-t)+z_2$} ;
\draw[dashed, tempcolor, ultra thick](\psione,\tone)--(\xendz,\a)node[above, pos=0.45, rotate=-45,scale=0.7,black] {$Z=c(t_1-t)+z_1$} ;

\draw [black,fill] (\xendz,\a) circle [radius=\circsize] node [black,above,scale=\textscale] {$\gamma_1$}; 
\draw [black,fill] (\xend,\a) circle [radius=\circsize] node [black,above,scale=\textscale] {$\gamma_2$}; 
%\draw[dashed, darkgreen, ultra thick](\psione,\tone)--(\psitwo,\tthree);
%\draw [black,fill] (\psitwo,\tthree) circle [radius=\circsize] node [black,below=4,right,scale=\textscale] {$t=2t_1-t_2$}; 


\draw [black,fill] (\psione,\tone) circle [radius=\circsize] node [black,left=4,scale=\textscale] {$t_1$}; 
\draw [black,fill](\psitwo,\ttwo)circle [radius=\circsize] node [black,right,scale=\textscale, align=left] {$t_2=t_1+\frac{z_2-z_1}{c}$.}; 

%\draw [decorate, decoration = {calligraphic brace}] (\psitwo+\nudge,\ttwo) --  (\psitwo+\nudge,\tone+\nudge/4) node[midway,right, scale=\textscale]{$t_2-t_1$}; 
%\draw [decorate, decoration = {calligraphic brace}] (\psitwo+\nudge,\tone-\nudge/4) --  (\psitwo+\nudge,2*\tone-\ttwo) node[midway,right, scale=\textscale]{$t_2-t_1$}; 

\draw[->] (\timex,\md-\tlen/2) --  (\timex,\md+\tlen/2) node[midway,right, scale=\textscale]{time}; 
 
%\draw[dotted](\psione,\ttesta)--({testxonep(\psione,\ttesta)},\a);
%\draw[dotted](\psione,\ttesta)--({testxonem(\psione,\ttesta)},\a);
%\draw [black,fill](\psione,\ttesta) circle [radius=\circsize] node [black,below=5,right,scale=\textscale] {$y_a$}; 

%\draw[magenta,->,ultra thick] ({testxonem(\psione,\ttesta)},\a)--(-\lrange,\a);
%\draw[magenta,->,ultra thick] ({testxonep(\psione,\ttesta)},\a)--(\rrange,\a);

\end{tikzpicture}% pic 1

\vspace*{2px}
\caption{Kent's toy model}
\label{TM1}
\end{figure}
We now suppose that when the mass-energy density $S$ is ``measured'', the energy of the photon is found to be at $\gamma_1$ rather than at $\gamma_2$. We then consider the mass-density at early spacetime locations $y^a_1=(z_1,t_a)$ and $y^a_2=(z_2,t_a)$ as show in figure \ref{TM2} (a) and (b). 
 
 
  \begin{figure}[ht!]
\captionsetup{justification=justified}
\centering
\tikzmath{
\a=5.3;  
\psione=0;
\psitwo=1;
\e = 0.1;
\h=-0.8;
\labelx=(\psione+\psitwo)/2;
\labely=-1.5;
\phstartx=-2.9;
\phstarty=-.8;
\tbeg=\phstarty;
\ca=\phstartx-\tbeg;
\ttwo=\psitwo-\ca;
\cb=\ttwo+\ca;
\xend=\ttwo-\a+\cb;
\tone=\psione-\ca;
\cc=\tone+\ca;
\xendz=\tone-\a+\cc;
\tthree=\cc+\tone-\psitwo;
\circsize=0.08;
\md = (\a+\h)/2;
\tlen=0.75;
\textscale = 0.7;
\picscale = 0.58;
\nudge=0.1;
\tnudge=(\ttwo-\tone)/2;
\ttesta=2*\tone-\ttwo-\tnudge;
\ttestb=2*\tone-\ttwo+\tnudge;
\ttestc=\ttwo-\tnudge;
\ttestd=\ttwo+\tnudge;
\sonea=\a+\psione-\ttesta;
\sadd=0.5;
\lrange = -(\psione-\a+\ttesta)+\sadd;
\rrange =\a+\psitwo-\ttesta+\sadd; 
\timex=\rrange-1.5;
} 
\begin{tikzpicture}[scale=\picscale,
declare function={
	testxonep(\ps,\t)=\a+\ps-\t;
	testxonem(\ps,\t)=\ps-\a+\t; 
},
 ] 
\definecolor{tempcolor}{RGB}{250,190,0}
\definecolor{darkgreen}{RGB}{40,190,40}
\draw[<->] (-\lrange, \h) node[left, scale=\textscale] {$S_0$} -- (\rrange, \h) node[right, scale=\textscale] {$S_0$};
\draw[<->] (-\lrange, \a) node[left, scale=\textscale] {$S$} -- (\rrange, \a) node[right, scale=\textscale] {$S$};
              
\draw[->, shorten <= 5pt,  shorten >= 1pt] (\psione,\h) node[below, scale=\textscale]{$\psi_1^{\text{sys}}$} node[above left, scale=\textscale]{$z_1$}-- (\psione,\a);
\coordinate[label = below: (a)]  (D) at (\labelx,\labely); 
\draw[->,  shorten <= 5pt,  shorten >= 1pt] (\psitwo,\h) node[below, scale=\textscale]{$\psi_2^{\text{sys}}$} node[above right, scale=\textscale]{$z_2$}-- (\psitwo,\a);

\draw[dashed, tempcolor, ultra thick](\phstartx,\tbeg) node[below left=-2,scale=\textscale]{}--(\psione,\tone) node[above, pos=0.3, rotate=45,scale=0.7,black] {} node[below, pos=0.3, rotate=45,scale=0.7,black] {};
\draw[dashed, gray](\psione,\tone) node[below left=-2,scale=\textscale]{}--(\psitwo,\ttwo );
\draw[dashed, gray](\psitwo,\ttwo)--(\xend,\a); 
\draw[dashed, tempcolor, ultra thick](\psione,\tone)--(\xendz,\a);

\draw [black,fill] (\xendz,\a) circle [radius=\circsize] node [black,above,scale=\textscale] {$\gamma_1$}; 

\draw[dashed, darkgreen, ultra thick](\psione,\tone)--(\psitwo,\tthree);
\draw [black,fill] (\psitwo,\tthree) circle [radius=\circsize] node [black,below=4,right,scale=\textscale] {$2t_1-t_2$}; 


\draw [black,fill] (\psione,\tone) circle [radius=\circsize] node [black,left=4,scale=\textscale] {$t_1$}; 
\draw [black,fill](\psitwo,\ttwo)circle [radius=\circsize] node [black,above right,scale=\textscale, align=left] {$t_2$}; 

\draw [decorate, decoration = {calligraphic brace}] (\psitwo+\nudge,\ttwo) --  (\psitwo+\nudge,\tone+\nudge/4) node[midway,right, scale=\textscale]{$t_2-t_1$}; 
\draw [decorate, decoration = {calligraphic brace}] (\psitwo+\nudge,\tone-\nudge/4) --  (\psitwo+\nudge,2*\tone-\ttwo) node[midway,right, scale=\textscale]{$t_2-t_1$}; 

\draw[->] (\timex,\md-\tlen/2) --  (\timex,\md+\tlen/2) node[midway,right, scale=\textscale]{time}; 
 
\draw[dotted](\psione,\ttesta)--({testxonep(\psione,\ttesta)},\a);
\draw[dotted](\psione,\ttesta)--({testxonem(\psione,\ttesta)},\a);
\draw [black,fill](\psione,\ttesta) circle [radius=\circsize] node [black,below=5,right,scale=\textscale] {$y^a_1$}; 

\draw[magenta,->,ultra thick] ({testxonem(\psione,\ttesta)},\a)--(-\lrange,\a);
\draw[magenta,->,ultra thick] ({testxonep(\psione,\ttesta)},\a)--(\rrange,\a);

\end{tikzpicture}% pic 1
 % <----------------- SPACE BETWEEN PICTURES
\begin{tikzpicture}[scale=\picscale,  %[x={10.0pt},y={10.0pt}]
declare function={
	testxonep(\ps,\t)=\a+\ps-\t;
	testxonem(\ps,\t)=\ps-\a+\t; 
},
 ] 
\definecolor{tempcolor}{RGB}{250,190,0}
\definecolor{darkgreen}{RGB}{40,190,40}
\draw[<->] (-\lrange, \h) node[left, scale=\textscale] {$S_0$} -- (\rrange, \h) node[right, scale=\textscale] {$S_0$};
\draw[<->] (-\lrange, \a) node[left, scale=\textscale] {$S$} -- (\rrange, \a) node[right, scale=\textscale] {$S$};
              
\draw[->, shorten <= 5pt,  shorten >= 1pt] (\psione,\h) node[below, scale=\textscale]{$\psi_1^{\text{sys}}$} node[above left, scale=\textscale]{$z_1$}-- (\psione,\a);
\coordinate[label = below: (b)]  (D) at (\labelx,\labely); 
\draw[->,  shorten <= 5pt,  shorten >= 1pt] (\psitwo,\h) node[below, scale=\textscale]{$\psi_2^{\text{sys}}$} node[above right, scale=\textscale]{$z_2$}-- (\psitwo,\a);

\draw[dashed, tempcolor, ultra thick](\phstartx,\tbeg) node[below left=-2,scale=\textscale]{}--(\psione,\tone) node[above, pos=0.3, rotate=45,scale=0.7,black] {} node[below, pos=0.3, rotate=45,scale=0.7,black] {};
\draw[dashed, gray](\psione,\tone) node[below left=-2,scale=\textscale]{}--(\psitwo,\ttwo );
\draw[dashed, gray](\psitwo,\ttwo)--(\xend,\a); 
\draw[dashed, tempcolor, ultra thick](\psione,\tone)--(\xendz,\a);

\draw [black,fill] (\xendz,\a) circle [radius=\circsize] node [black,above,scale=\textscale] {$\gamma_1$}; 

\draw[dashed, darkgreen, ultra thick](\psione,\tone)--(\psitwo,\tthree);
\draw [black,fill] (\psitwo,\tthree) circle [radius=\circsize] node [black,below=4,right,scale=\textscale] {$2t_1-t_2$}; 


\draw [black,fill] (\psione,\tone) circle [radius=\circsize] node [black,left=4,scale=\textscale] {$t_1$}; 
\draw [black,fill](\psitwo,\ttwo)circle [radius=\circsize] node [black,above right,scale=\textscale, align=left] {$t_2$}; 

\draw [decorate, decoration = {calligraphic brace}] (\psitwo+\nudge,\ttwo) --  (\psitwo+\nudge,\tone+\nudge/4) node[midway,right, scale=\textscale]{$t_2-t_1$}; 
\draw [decorate, decoration = {calligraphic brace}] (\psitwo+\nudge,\tone-\nudge/4) --  (\psitwo+\nudge,2*\tone-\ttwo) node[midway,right, scale=\textscale]{$t_2-t_1$}; 

\draw[->] (\timex,\md-\tlen/2) --  (\timex,\md+\tlen/2) node[midway,right, scale=\textscale]{time}; 
 
\draw[dotted](\psitwo,\ttesta)--({testxonep(\psitwo,\ttesta)},\a);
\draw[dotted](\psitwo,\ttesta)--({testxonem(\psitwo,\ttesta)},\a);
\draw [black,fill](\psitwo,\ttesta) circle [radius=\circsize] node [black,below=5,right,scale=\textscale] {$y^a_2$}; 

\draw[magenta,->,ultra thick] ({testxonem(\psitwo,\ttesta)},\a)--(-\lrange,\a);
\draw[magenta,->,ultra thick] ({testxonep(\psitwo,\ttesta)},\a)--(\rrange,\a);


\end{tikzpicture}% pic 1


\vspace*{2px}
\caption{(a) highlights the part of $S$ used to calculate the energy density at $y^a_1$ whose time is less than $2t_1-t_2$. (b) highlights the part of $S$ used to calculate the energy density at $y^a_2$ whose time is less than $2t_1-t_2$.}
\label{TM2}
\end{figure}

By early, we mean that $t_a<2t_1-t_2$. This will mean that the possible detection locations $\gamma_1$ and $\gamma_2$ will be outside the forward light cones of $y^a_1$ and $y^a_2$. Hence, $S^1(y^a_1)\cap S$ and $S^1(y^a_2)\cap S$  contain no additional information beyond standard quantum theory by which we could calculate the conditional expectation values of the energy at $y^a_1$  and $y^a_2$. Hence, according to Kent's interpretation, the total energy at time $t_a$ will be divided between the two spatial locations with a proportion of $\abs{c_1}^2$ at $z_1$  and a proportion of $\abs{c_2}^2$ at $z_2$.

However, the situation is different for two spacetime locations $y^b_1=(z_1, t_b)$ and $y^b_2=(z_2, t_b)$ with $t_b$ slightly after $2t_1-t_2$ as depicted in figure \ref{TM3}.

\begin{figure}[ht!]
\captionsetup{justification=justified}
\centering
\tikzmath{
\a=5.3;  
\psione=0;
\psitwo=1;
\e = 0.1;
\h=-0.8;
\labelx=(\psione+\psitwo)/2;
\labely=-1.5;
\phstartx=-2.9;
\phstarty=-.8;
\tbeg=\phstarty;
\ca=\phstartx-\tbeg;
\ttwo=\psitwo-\ca;
\cb=\ttwo+\ca;
\xend=\ttwo-\a+\cb;
\tone=\psione-\ca;
\cc=\tone+\ca;
\xendz=\tone-\a+\cc;
\tthree=\cc+\tone-\psitwo;
\circsize=0.08;
\md = (\a+\h)/2;
\tlen=0.75;
\textscale = 0.7;
\picscale = 0.58;
\nudge=0.1;
\tnudge=(\ttwo-\tone)/2;
\ttesta=2*\tone-\ttwo-\tnudge;
\ttestb=2*\tone-\ttwo+\tnudge;
\ttestc=\ttwo-\tnudge;
\ttestd=\ttwo+\tnudge;
\sonea=\a+\psione-\ttesta;
\sadd=0.5;
\lrange = -(\psione-\a+\ttesta)+\sadd;
\rrange =\a+\psitwo-\ttesta+\sadd; 
\timex=\rrange-1.5;
} 
\begin{tikzpicture}[scale=\picscale,
declare function={
	testxonep(\ps,\t)=\a+\ps-\t;
	testxonem(\ps,\t)=\ps-\a+\t; 
},
 ] 
\definecolor{tempcolor}{RGB}{250,190,0}
\definecolor{darkgreen}{RGB}{40,190,40}
\draw[<->] (-\lrange, \h) node[left, scale=\textscale] {$S_0$} -- (\rrange, \h) node[right, scale=\textscale] {$S_0$};
\draw[<->] (-\lrange, \a) node[left, scale=\textscale] {$S$} -- (\rrange, \a) node[right, scale=\textscale] {$S$};
              
\draw[->, shorten <= 5pt,  shorten >= 1pt] (\psione,\h) node[below, scale=\textscale]{$\psi_1^{\text{sys}}$} node[above left, scale=\textscale]{$z_1$}-- (\psione,\a);
\coordinate[label = below: (a)]  (D) at (\labelx,\labely); 
\draw[->,  shorten <= 5pt,  shorten >= 1pt] (\psitwo,\h) node[below, scale=\textscale]{$\psi_2^{\text{sys}}$} node[above right, scale=\textscale]{$z_2$}-- (\psitwo,\a);

\draw[dashed, tempcolor, ultra thick](\phstartx,\tbeg) node[below left=-2,scale=\textscale]{}--(\psione,\tone) node[above, pos=0.3, rotate=45,scale=0.7,black] {} node[below, pos=0.3, rotate=45,scale=0.7,black] {};
\draw[dashed, gray](\psione,\tone) node[below left=-2,scale=\textscale]{}--(\psitwo,\ttwo );
\draw[dashed, gray](\psitwo,\ttwo)--(\xend,\a); 
\draw[dashed, tempcolor, ultra thick](\psione,\tone)--(\xendz,\a);

\draw [black,fill] (\xendz,\a) circle [radius=\circsize] node [black,above,scale=\textscale] {$\gamma_1$}; 

\draw[dashed, darkgreen, ultra thick](\psione,\tone)--(\psitwo,\tthree);
\draw [black,fill] (\psitwo,\tthree) circle [radius=\circsize] node [black,below=4,right,scale=\textscale] {$2t_1-t_2$}; 


\draw [black,fill] (\psione,\tone) circle [radius=\circsize] node [black,left=4,scale=\textscale] {$t_1$}; 
\draw [black,fill](\psitwo,\ttwo)circle [radius=\circsize] node [black,above right,scale=\textscale, align=left] {$t_2$}; 

%\draw [decorate, decoration = {calligraphic brace}] (\psitwo+\nudge,\ttwo) --  (\psitwo+\nudge,\tone+\nudge/4) node[midway,right, scale=\textscale]{$t_2-t_1$}; 
%\draw [decorate, decoration = {calligraphic brace}] (\psitwo+\nudge,\tone-\nudge/4) --  (\psitwo+\nudge,2*\tone-\ttwo) node[midway,right, scale=\textscale]{$t_2-t_1$}; 

\draw[->] (\timex,\md-\tlen/2) --  (\timex,\md+\tlen/2) node[midway,right, scale=\textscale]{time}; 
 
\draw[dotted](\psione,\ttestb)--({testxonep(\psione,\ttestb)},\a);
\draw[dotted](\psione,\ttestb)--({testxonem(\psione,\ttestb)},\a);
\draw [black,fill](\psione,\ttestb) circle [radius=\circsize] node [black,below=6,left,scale=\textscale] {$y^b_1$}; 

\draw[magenta,->,ultra thick] ({testxonem(\psione,\ttestb)},\a)--(-\lrange,\a);
\draw[magenta,->,ultra thick] ({testxonep(\psione,\ttestb)},\a)--(\rrange,\a);

\end{tikzpicture}% pic 1
 % <----------------- SPACE BETWEEN PICTURES
\begin{tikzpicture}[scale=\picscale,  %[x={10.0pt},y={10.0pt}]
declare function={
	testxonep(\ps,\t)=\a+\ps-\t;
	testxonem(\ps,\t)=\ps-\a+\t; 
},
 ] 
\definecolor{tempcolor}{RGB}{250,190,0}
\definecolor{darkgreen}{RGB}{40,190,40}
\draw[<->] (-\lrange, \h) node[left, scale=\textscale] {$S_0$} -- (\rrange, \h) node[right, scale=\textscale] {$S_0$};
\draw[<->] (-\lrange, \a) node[left, scale=\textscale] {$S$} -- (\rrange, \a) node[right, scale=\textscale] {$S$};
              
\draw[->, shorten <= 5pt,  shorten >= 1pt] (\psione,\h) node[below, scale=\textscale]{$\psi_1^{\text{sys}}$} node[above left, scale=\textscale]{$z_1$}-- (\psione,\a);
\coordinate[label = below: (b)]  (D) at (\labelx,\labely); 
\draw[->,  shorten <= 5pt,  shorten >= 1pt] (\psitwo,\h) node[below, scale=\textscale]{$\psi_2^{\text{sys}}$} node[above right, scale=\textscale]{$z_2$}-- (\psitwo,\a);

\draw[dashed, tempcolor, ultra thick](\phstartx,\tbeg) node[below left=-2,scale=\textscale]{}--(\psione,\tone) node[above, pos=0.3, rotate=45,scale=0.7,black] {} node[below, pos=0.3, rotate=45,scale=0.7,black] {};
\draw[dashed, gray](\psione,\tone) node[below left=-2,scale=\textscale]{}--(\psitwo,\ttwo );
\draw[dashed, gray](\psitwo,\ttwo)--(\xend,\a); 
\draw[dashed, tempcolor, ultra thick](\psione,\tone)--(\xendz,\a);



\draw[dashed, darkgreen, ultra thick](\psione,\tone)--(\psitwo,\tthree);
\draw [black,fill] (\psitwo,\tthree) circle [radius=\circsize] node [black,below=4,right,scale=\textscale] {$2t_1-t_2$}; 


\draw [black,fill] (\psione,\tone) circle [radius=\circsize] node [black,left=4,scale=\textscale] {$t_1$}; 
\draw [black,fill](\psitwo,\ttwo)circle [radius=\circsize] node [black,above right,scale=\textscale, align=left] {$t_2$}; 

%\draw [decorate, decoration = {calligraphic brace}] (\psitwo+\nudge,\ttwo) --  (\psitwo+\nudge,\tone+\nudge/4) node[midway,right, scale=\textscale]{$t_2-t_1$}; 
%\draw [decorate, decoration = {calligraphic brace}] (\psitwo+\nudge,\tone-\nudge/4) --  (\psitwo+\nudge,2*\tone-\ttwo) node[midway,right, scale=\textscale]{$t_2-t_1$}; 

\draw[->] (\timex,\md-\tlen/2) --  (\timex,\md+\tlen/2) node[midway,right, scale=\textscale]{time}; 
 
\draw[dotted](\psitwo,\ttestb)--({testxonep(\psitwo,\ttestb)},\a);
\draw[dotted](\psitwo,\ttestb)--({testxonem(\psitwo,\ttestb)},\a);
\draw [black,fill](\psitwo,\ttestb) circle [radius=\circsize] node [black,right,scale=\textscale] {$y^b_2$}; 

\draw[magenta,->,ultra thick] ({testxonem(\psitwo,\ttestb)},\a)--(-\lrange,\a);
\draw[magenta,->,ultra thick] ({testxonep(\psitwo,\ttestb)},\a)--(\rrange,\a);
\draw [black,fill] (\xendz,\a) circle [radius=\circsize] node [black,above,scale=\textscale] {$\gamma_1$}; 

\end{tikzpicture}% pic 1
\vspace*{2px}
\caption{(a) highlights the part of $S$ used to calculate the energy density at $y^b_1$ whose time is greater than $2t_1-t_2$. (b) highlights the part of $S$ used to calculate the energy density at $y^b_2$ whose time is greater than $2t_1-t_2$.}
\label{TM3}
\end{figure}
In this situation, when we consider the location $y^b_1$, there is no additional information in $S^1(y^b_1)\cap S$ beyond standard quantum theory, so there will be a proportion of $\abs{c_1}^2$ of the total initial energy of the system at $y^b_1$. But at location $y^b_2$, the information in $S^1(y^b_2)\cap S$ shows that the photon has reflected from the localized $\psi_1^\text{sys}$-state, and so this additional information tells us that after time $t_b$, there is no energy localized at $z_2$ since from the perspective of $y^b_2$, the energy is known to be localized at $z_1$. So it is as though the information of $S^1(y^b_2)\cap S$ has determined that we are in a world in which there is an energy density of zero at $y^b_2$, and there are no other worlds in which the energy density  at $y^b_2$ is non-zero since all worlds have to be consistent with the notional measurement made on $S$. So for a short time the total energy of the system is reduced by a factor of $\abs{c_1}^2$.  

However, as shown in figure \ref{TM4}, for times $t_c$ greater than $t_1$, the total energy of the system is once again restored to the initial energy the system had when in the state $\psi_{0}^\text{sys}$. 

  \begin{figure}[ht!]
\captionsetup{justification=justified}
\centering
\tikzmath{
\a=5.3;  
\psione=0;
\psitwo=1;
\e = 0.1;
\h=-0.8;
\labelx=(\psione+\psitwo)/2;
\labely=-1.5;
\phstartx=-2.9;
\phstarty=-.8;
\tbeg=\phstarty;
\ca=\phstartx-\tbeg;
\ttwo=\psitwo-\ca;
\cb=\ttwo+\ca;
\xend=\ttwo-\a+\cb;
\tone=\psione-\ca;
\cc=\tone+\ca;
\xendz=\tone-\a+\cc;
\tthree=\cc+\tone-\psitwo;
\circsize=0.08;
\md = (\a+\h)/2;
\tlen=0.75;
\textscale = 0.7;
\picscale = 0.58;
\nudge=0.1;
\tnudge=(\ttwo-\tone)/2;
\ttesta=2*\tone-\ttwo-\tnudge;
\ttestb=2*\tone-\ttwo+\tnudge;
\ttestc=\ttwo-\tnudge;
\ttestd=\ttwo+\tnudge;
\sonea=\a+\psione-\ttesta;
\sadd=0.5;
\lrange = -(\psione-\a+\ttesta)+\sadd;
\rrange =\a+\psitwo-\ttesta+\sadd; 
\timex=\rrange-1.5;
} 
\begin{tikzpicture}[scale=\picscale,
declare function={
	testxonep(\ps,\t)=\a+\ps-\t;
	testxonem(\ps,\t)=\ps-\a+\t; 
},
 ] 
\definecolor{tempcolor}{RGB}{250,190,0}
\definecolor{darkgreen}{RGB}{40,190,40}
\draw[<->] (-\lrange, \h) node[left, scale=\textscale] {$S_0$} -- (\rrange, \h) node[right, scale=\textscale] {$S_0$};
\draw[<->] (-\lrange, \a) node[left, scale=\textscale] {$S$} -- (\rrange, \a) node[right, scale=\textscale] {$S$};
              
\draw[->, shorten <= 5pt,  shorten >= 1pt] (\psione,\h) node[below, scale=\textscale]{$\psi_1^{\text{sys}}$} node[above left, scale=\textscale]{$z_1$}-- (\psione,\a);
\coordinate[label = below: (a)]  (D) at (\labelx,\labely); 
\draw[->,  shorten <= 5pt,  shorten >= 1pt] (\psitwo,\h) node[below, scale=\textscale]{$\psi_2^{\text{sys}}$} node[above right, scale=\textscale]{$z_2$}-- (\psitwo,\a);

\draw[dashed, tempcolor, ultra thick](\phstartx,\tbeg) node[below left=-2,scale=\textscale]{}--(\psione,\tone) node[above, pos=0.3, rotate=45,scale=0.7,black] {} node[below, pos=0.3, rotate=45,scale=0.7,black] {};
\draw[dashed, gray](\psione,\tone) node[below left=-2,scale=\textscale]{}--(\psitwo,\ttwo );
\draw[dashed, gray](\psitwo,\ttwo)--(\xend,\a); 
\draw[dashed, tempcolor, ultra thick](\psione,\tone)--(\xendz,\a);


\draw[dashed, darkgreen, ultra thick](\psione,\tone)--(\psitwo,\tthree);
\draw [black,fill] (\psitwo,\tthree) circle [radius=\circsize] node [black,below=4,right,scale=\textscale] {$2t_1-t_2$}; 


\draw [black,fill] (\psione,\tone) circle [radius=\circsize] node [black,left=4,scale=\textscale] {$t_1$}; 
\draw [black,fill](\psitwo,\ttwo)circle [radius=\circsize] node [black,above right,scale=\textscale, align=left] {$t_2$}; 

%\draw [decorate, decoration = {calligraphic brace}] (\psitwo+\nudge,\ttwo) --  (\psitwo+\nudge,\tone+\nudge/4) node[midway,right, scale=\textscale]{$t_2-t_1$}; 
%\draw [decorate, decoration = {calligraphic brace}] (\psitwo+\nudge,\tone-\nudge/4) --  (\psitwo+\nudge,2*\tone-\ttwo) node[midway,right, scale=\textscale]{$t_2-t_1$}; 

\draw[->] (\timex,\md-\tlen/2) --  (\timex,\md+\tlen/2) node[midway,right, scale=\textscale]{time}; 
 
\draw[dotted](\psione,\ttestc)--({testxonep(\psione,\ttestc)},\a);
\draw[dotted](\psione,\ttestc)--({testxonem(\psione,\ttestc)},\a);
\draw [black,fill](\psione,\ttestc) circle [radius=\circsize] node [black,right,scale=\textscale] {$y^c_1$}; 

\draw[magenta,->,ultra thick] ({testxonem(\psione,\ttestc)},\a)--(-\lrange,\a);
\draw[magenta,->,ultra thick] ({testxonep(\psione,\ttestc)},\a)--(\rrange,\a);

\draw [black,fill] (\xendz,\a) circle [radius=\circsize] node [black,above,scale=\textscale] {$\gamma_1$}; 

\end{tikzpicture}% pic 1
 % <----------------- SPACE BETWEEN PICTURES
\begin{tikzpicture}[scale=\picscale,  %[x={10.0pt},y={10.0pt}]
declare function={
	testxonep(\ps,\t)=\a+\ps-\t;
	testxonem(\ps,\t)=\ps-\a+\t; 
},
 ] 
\definecolor{tempcolor}{RGB}{250,190,0}
\definecolor{darkgreen}{RGB}{40,190,40}
\draw[<->] (-\lrange, \h) node[left, scale=\textscale] {$S_0$} -- (\rrange, \h) node[right, scale=\textscale] {$S_0$};
\draw[<->] (-\lrange, \a) node[left, scale=\textscale] {$S$} -- (\rrange, \a) node[right, scale=\textscale] {$S$};
              
\draw[->, shorten <= 5pt,  shorten >= 1pt] (\psione,\h) node[below, scale=\textscale]{$\psi_1^{\text{sys}}$} node[above left, scale=\textscale]{$z_1$}-- (\psione,\a);
\coordinate[label = below: (b)]  (D) at (\labelx,\labely); 
\draw[->,  shorten <= 5pt,  shorten >= 1pt] (\psitwo,\h) node[below, scale=\textscale]{$\psi_2^{\text{sys}}$} node[above right, scale=\textscale]{$z_2$}-- (\psitwo,\a);

\draw[dashed, tempcolor, ultra thick](\phstartx,\tbeg) node[below left=-2,scale=\textscale]{}--(\psione,\tone) node[above, pos=0.3, rotate=45,scale=0.7,black] {} node[below, pos=0.3, rotate=45,scale=0.7,black] {};
\draw[dashed, gray](\psione,\tone) node[below left=-2,scale=\textscale]{}--(\psitwo,\ttwo );
\draw[dashed, gray](\psitwo,\ttwo)--(\xend,\a); 
\draw[dashed, tempcolor, ultra thick](\psione,\tone)--(\xendz,\a);



\draw[dashed, darkgreen, ultra thick](\psione,\tone)--(\psitwo,\tthree);
\draw [black,fill] (\psitwo,\tthree) circle [radius=\circsize] node [black,below=4,right,scale=\textscale] {$2t_1-t_2$}; 


\draw [black,fill] (\psione,\tone) circle [radius=\circsize] node [black,left=4,scale=\textscale] {$t_1$}; 
\draw [black,fill](\psitwo,\ttwo)circle [radius=\circsize] node [black,above right,scale=\textscale, align=left] {$t_2$}; 

%\draw [decorate, decoration = {calligraphic brace}] (\psitwo+\nudge,\ttwo) --  (\psitwo+\nudge,\tone+\nudge/4) node[midway,right, scale=\textscale]{$t_2-t_1$}; 
%\draw [decorate, decoration = {calligraphic brace}] (\psitwo+\nudge,\tone-\nudge/4) --  (\psitwo+\nudge,2*\tone-\ttwo) node[midway,right, scale=\textscale]{$t_2-t_1$}; 

\draw[->] (\timex,\md-\tlen/2) --  (\timex,\md+\tlen/2) node[midway,right, scale=\textscale]{time}; 
 
\draw[dotted](\psitwo,\ttestc)--({testxonep(\psitwo,\ttestc)},\a);
\draw[dotted](\psitwo,\ttestc)--({testxonem(\psitwo,\ttestc)},\a);
\draw [black,fill](\psitwo,\ttestc) circle [radius=\circsize] node [black,below=5,right,scale=\textscale] {$y^c_2$}; 

\draw[magenta,->,ultra thick] ({testxonem(\psitwo,\ttestc)},\a)--(-\lrange,\a);
\draw[magenta,->,ultra thick] ({testxonep(\psitwo,\ttestc)},\a)--(\rrange,\a);
\draw [black,fill] (\xendz,\a) circle [radius=\circsize] node [black,above,scale=\textscale] {$\gamma_1$}; 

\end{tikzpicture}% pic 1


\vspace*{2px}
\caption{(a) highlights the part of $S$ used to calculate the energy density at $y^c_1$ whose time is greater than $t_1$. (b) highlights the part of $S$ used to calculate the energy density at $y^c_2$ whose time is greater than $t_1$.}
\label{TM4}
\end{figure}

In this situation, there is now information in $S^1(y^c_1)\cap S$ that determines that the photon reflected off the localized $\psi_1^\text{sys}$-state. This means that when the conditional expectation of the energy density of $y^c_1$ is calculated, the extra information in $S^1(y^c_1)\cap S$ determines that all the energy of the system is located at location $z_1$ for times $t_c$ greater than $t_1$, and the energy is equal to the initial energy of the system so that energy is conserved.

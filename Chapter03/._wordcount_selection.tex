\chapter{Evaluating Kent's Theory\label{KentEval}}
In order to evaluate Kent's theory of quantum physics, it will first be helpful to remind ourselves of the problem we are trying to solve. 

In chapter \ref{BellChapter}, we discussed the EPR-Bohm paradox and the problem of explaining  the mysterious correlations between measurement outcomes of spin singlets in a way consistent with special relativity and the predictions of standard quantum theory. We saw that the Copenhagen interpretation does not seem to be consistent with special relativity. We also discussed Shimony's distinction between Outcome Independence (OI) and Parameter Independence (PI) and Shimony's idea that we should only accept a theory in which OI is false and PI is true. Since PI is false in the pilot-wave theory, we should reject it according to Shimony's criterion. 

But although Shimony's criterion is a promising line of inquiry, by itself, it is not sufficient to resolve the  EPR-Bohm paradox. This is because Shimony's criterion doesn't address the controversial issue of what is meant by an outcome. In chapter \ref{measprobchap}, we discussed this controversy over outcomes and why the many-worlds interpretation that denies the reality of outcomes is unsatisfactory. This motivated the discussion of Kent's theory in chapter \ref{measprobchap} in the hope that it might provide a satisfactory solution to the EPR-Bohm paradox. In the previous chapter, we only got as far as describing the key features of Kent's theory such as it being a one-world theory which posits additional variables to standard quantum theory. 

So having now reminded ourselves of the problem at hand, we see that there are several issues that we need to consider in order to evaluate whether Kent's theory provides a satisfactory solution to this problem. Firstly, we should consider whether the predictions of Kent's theory are consistent with the  predictions of quantum theory that have been experimentally validated. Since standard quantum theory (that is a theory of states whose evolution is determined by the Schr\"{o}dinger equation) predicts the correlations observed in the EPR-Bohm paradox, then if Kent's theory is consistent with standard quantum theory, these EPR-Bohm correlations will also be exhibited in Kent's theory. 

Secondly, since a satisfactory solution to our problem must be consistent with special relativity, we need to consider whether such consistency holds in Kent's theory. Consistency with special relativity is guaranteed in a theory if and only if it is invariant under a group of symmetries called Lorentz transformations. We therefore need to consider whether Kent's theory satisfies Lorentz invariance. 

Thirdly, since a satisfactory theory must be one in which there are outcomes, we need to consider whether Kent succeeds in giving us a convincing account of what an outcome is. In doing this, we will examine how Kent's theory ties in with decoherence theory and d'Espagnat's objection about improper mixtures. 

Butterfield has also emphasized the need to understand Kent's theory in the light of an important theorem proved in recent years, the so-called Collbeck-Renner Theorem\footnote{See \cite{LeegwaterGijs2016Aitf}, \cite{ColbeckRoger2011Neoq}, \cite{ColbeckRoger2012Tcoq}, \cite{LandsmanK2015OtCt}, and \cite{Landsman}.} which says that if a theory satisfies PI together with what is called a `no conspiracy' criterion, then this theory is reducible to standard quantum theory without any hidden variables. On the assumption that a violation of `no conspiracy' would not be acceptable,  this then suggests that any theory that satisfactorily addresses the  EPR-Bohm paradox can't be a hidden-variables theory. We will therefore need to consider whether Kent's model can be an interpretation of quantum physics without it being a hidden-variables theory.

I will argue that in the light all these considerations, Kent's theory does provide a satisfactory solution to the EPR-Bohm paradox. There are still questions concerning the nature of Kent's beables, and Kent's theory may also strike us as rather counter-intuitive given that his theory posits that present events should be conditioned on far-distant future states of affairs. We will therefore conclude this chapter by discussing how Kent's theory could be made to appear less counter-intuitive.

We can also choose a set of $\lambda$ so that criterion \ref{hidden5} holds. To do this, we consider equation (\ref{gammadecomp}) which presupposes there is a subregion $S_{\mathcal{S}}$ of $S^1(y_i)$ which determines the state $\mathcal{S}$ and another non-overlapping subregion $S_{\mathcal{A}}$ that determines the state $\mathcal{A}$. In general, we wouldn't be able to make this association between subregions of $S^1(y_i)$ and states of  $\mathcal{S}$  and $\mathcal{A}$ -- after all, $\mathcal{A}$ might not even exist. But we should be able to make this association if an appropriate choice for the state of the remainder of $S^1(y_i)$ is made. In (\ref{gammadecomp}), $\ket*{\Xi_i^{(S^1(y_i))}}$ is able to serve this role. We can then suppose that $\ket*{\gamma_i^{(\mathcal{S})}}$ is from a basis $\Lambda_{\mathcal{S}}=\{\ket*{\gamma_{i,1}^{(\mathcal{S})}},\ket*{\gamma_{i,2}^{(\mathcal{S})}},\ldots\}$ of states that describe all the states of the subregion $S_{\mathcal{S}}$ of $S^1(y_i)$ corresponding to $\mathcal{S}$ that together with $\ket*{\Xi_i^{(S^1(y_i))}}$ and $\ket*{\Psi_0}$ determine $\mathcal{S}$ to be in the state $\ket{s}$. We could then take the $\lambda$ of the system $\mathcal{S}$ in criterion to be one of the basis states in $\Lambda_{\mathcal{S}}$. Given that the interpretation of the states in $\Lambda_{\mathcal{S}}$ presupposes $\ket*{\Xi_i^{(S^1(y_i))}}$, we would take the probability $p_\lambda$ for $\lambda = \ket*{\gamma_i^{(\mathcal{S})}}$ to be the probability that the notional measurement on $S$ given $\ket*{\Xi_i^{(S^1(y_i))}}$ and $\ket*{\Psi_0}$ agreed with the energy-density values specified by $ \ket*{\gamma_i^{(\mathcal{S})}}$ on the subregion $S_{\mathcal{S}}$ of $S^1(y_1)$ that $ \ket*{\gamma_i^{(\mathcal{S})}}$ describes. If we let $\{\ket*{Z_1}, \ket*{Z_2},\ldots  \}$ be a basis of simultaneous $\hat{T}_S$-eigenstates for the subregion $(S\setminus S^1(y_1))\cup S_{\mathcal{A}}$ so that states of the form $\ket*{\gamma_{i,l}^{(\mathcal{S})}}\ket*{\Xi_i^{(S^1(y_i))}}\ket*{Z_k}$ will be simultaneous $\hat{T}_S$ for the whole of $S$, then the formula for the probability $p_\lambda$ will be 
\begin{equation}\label{plambda}
p_{\lambda}=\frac{\sum_k |\ip*{\Psi_S}{\gamma_i^{(\mathcal{S})}}\ket*{\Xi_i^{(S^1(y_i))}}\ket*{Z_k}|^2}{\sum_{k,l} |\ip*{\Psi_S}{\gamma_i^{(\mathcal{S})}}\ket*{\Xi_i^{(S^1(y_i))}}\ket*{Z_k}|^2}
\end{equation}

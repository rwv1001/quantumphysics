So let $\tau_S$ be the notional mass-energy density measurement on $S$. In order to avoid undue complexity, we will assume that there is no simultaneous $\hat{T}_S$-eigenstate degeneracy so that if $\ket*{\Psi^{(0)}}$ and $\ket*{\Psi^{(0')}}$ are (normalized\footnote{In this section we will assume that all states are normalized.}) simultaneous $\hat{T}_S$-eigenstates with simultaneous eigenvalues $\tau_S^{(0)}$ and $\tau_S^{(0')}$ respectively, then
$$(\forall x\in S) \,(\tau_S^{(0)}(x)=\tau_S^{(0')}(x))\implies \ket*{\Psi^{(0)}}=\ket*{\Psi^{(0')}}.$$
We will adopt a labelling convention such that $\tau_S=\tau_S^{(0)}$ from which it will follow that according to the Born Rule, the notional mass-energy density measurement $\tau_S$ will be selected with a probability 
$$P(T_S = \tau_S)=|\mel{\Psi^{(0)}}{U_{SS_0}}{\Psi_0}|^2$$
where $\ket*{\Psi_0}$ is the state of the initial hypersurface $S_0$, and where $U_{SS_0}$ is the unitary operator defined by equation (\ref{SchwingerUnitaryOP}).

Now suppose this notional measurement $\tau_S$ indicates that there is some apparatus $\mathcal{A}$ that exists in the vicinity of a spatial location $z_0$ at time $t_0$. This means that prior to $t_0$, the apparatus will have interacted with many photons so that the photon detections on $S$ outside the light cone of $y_0=(t_0, z_0)$ will indicate via Kent's stress-energy beables that there is some apparatus in the vicinity of $z_0$ that we can identify as $\mathcal{A}$. We further assume that Kent's stress-energy beables determine that $\mathcal{A}$ is in a state $\ket{a}$ which encapsulates among other things the measurement parameters of the apparatus.  We also suppose that the information in $\tau_S$ indicates that there is a particle $\mathcal{S}$ at time $t_0$ in a state
$\ket{s}$ that is heading towards the apparatus so that it will interact with it. This means that all the 
possible\footnote{i.e. possible given $\ket{\Psi_0}$ and the Born rule selection criterion.} 
simultaneous $\hat{T}_S$-eigenstates on $S$ whose simultaneous $\hat{T}_S$-eigenvalues agree with $\tau_S$ for all $x\in S^1(y_0)$,\footnote{Recall from page \pageref{S2} that $S^1(y_0)$ is the subset of $S$ that is outside the light cone of $y_0$.} 
where  $y_0=(t_0, z_0)$, are such that their simultaneous $\hat{T}_S$-eigenvalues within the light cone of $y_0$ indicate that at some time after $t_0$ the particle $\mathcal{S}$ would have interacted
with the apparatus $\mathcal{A}$. 

At this point, it will be helpful to define a \textbf{simultaneous $\hat{T}_S$-eigenstate of a subregion}\index{simultaneous $\hat{T}_S$-eigenstate of a subregion} $U$ of $S$ to be a state $\ket{\Psi_U}\in H_U$  which is an eigenstate of $\hat{T}_S(u)$ for every $u\in U$, where $H_U$ is the Hilbert space of states describing $U$. We will assume that we can write any state of $H_S$ as a superposition of states of the form $\ket*{\Psi_U}\ket*{\Psi_{S\setminus U}}$, where $\ket{\Psi_U}\in H_U $ is  a simultaneous $\hat{T}_S$-eigenstate of the subregion $U$ of $S$, and $\ket*{\Psi_{S\setminus U}}\in H_{S\setminus U}$ is a simultaneous
$\hat{T}_S$-eigenstate of the subregion $S\setminus U$ of $S$.\footnote{We effectively made this assumption in equation (\ref{Sistate}).} We will accordingly let 
$\ket*{\gamma^{(S^1(y_0))}_0}$ denote the simultaneous $\hat{T}_S$-eigenstate of the subregion $S^1(y_0)$ of $S$ with simultaneous $\hat{T}_S$-eigenvalue $\tau_S(x)$ respectively for all $x\in S^1(y_0)$. Then any  simultaneous $\hat{T}_S$-eigenstate over the whole of $S$  whose simultaneous $\hat{T}_S$-eigenvalue agrees with 
$\tau_S$ on the subregion $S^1(y_0)$ will be expressible as
$\ket*{\gamma^{(S^1(y_0))}_0}\ket*{\Psi_{S\setminus S^1(y_0)}}$, 
and any such simultaneous $\hat{T}_S$-eigenstate will determine the apparatus $\mathcal{A}$ and the particle $\mathcal{S}$ to exist in states $\ket{a}$ and $\ket{s}$ respectively given the state $\ket*{\Psi_0}$ of the initial hypersurface $S_0$.

We also assume that there are no further interactions of $\mathcal{S}$ with photons that are registered on $S$ until the particle has finished interacting with the apparatus $\mathcal{A}$.  In making this assumption, we suppose that there is a time $t_f>t_0$ such that if the notional measurement on $S$ had resulted in the outcome $\tau_S^{(j)}$ rather than $\tau_S$ where $\tau_S^{(j)}(x)=\tau_S(x)$ for all $x\in S^1(y_0)$, then for all $x \in S^1(y_f,y_0)=S^1(y_f)\setminus S^1(y_0)$,\footnote{i.e $S^1(y_f,y_0)$ is the subset of $S$ that is within the light cone of $y_0=(t_0, z_0)$ but outside the light cone of $y_f=(t_f, z_0)$.} we would be able to say of the mass-energy density $\tau_S^{(j)}(x)$ at $x$ that it wasn't caused by a photon that had been reflected from $\mathcal{S}$.\footnote{This assumption of our toy model assumes that photons are sufficiently well-localized so that we can discern fairly precise trajectories. In physical reality, we can't expect photons to have this degree of localization so that there will be some ambiguity about where the photons detected on $S$ came from.  } 

We will assume that given that notional measurement restricted to $S^1(y_0)$ is $\tau_S$, there is sufficiently little interaction of the composite system $\mathcal{S}+\mathcal{A}$ with its environment so that we can assume it evolves unitarily in accordance with the Schr\"{o}dinger equation. Therefore, given that the apparatus $\mathcal{A}$ is in a state $\ket{a}$ that encapsulates its parameter settings, and the particle $\mathcal{S}$ is on course to interact with $\mathcal{A}$, we can express the state $\ket{s}$ of $\mathcal{S}$ as a superposition $\ket{s}=\sum_k c_k \ket{s_k}$ where $\{\ket{s_k}:k\}$ is the set of pointer states corresponding to the parameter settings of the apparatus $\mathcal{A}$ that are encapsulated in the state $\ket{a}$. As described on page \pageref{pointer}, this means there are future states $\ket{a_k}$ of the apparatus $\mathcal{A}$ such that $\ip{a_k}{a_{k'}}\approx 0$ for $k\neq k'$ and such that the composite system $\mathcal{S}+\mathcal{A}$ will evolve according to the Schr\"{o}dinger equation as
$$\ket{s_k}\ket{a}\rightarrow \ket{s_k}\ket{a_k} $$
for each pointer state $\ket{s_k}$, and hence
$$\ket{s}\ket{a}\rightarrow \sum_k c_k \ket{s_k}\ket{a_k}. $$

We will also assume that given that notional measurement restricted to $S^1(y_0)$ is $\tau_S$, there is a time interval between $t_f$ and $t_m > t_f$ during which photons will reflect off the apparatus $\mathcal{A}$ and ultimately be detected at space time locations on the hypersurface $S$ that correspond to one of the definite measurement states $\ket{a_k}$ of the apparatus. This means that if the notional measurement on $S$ had resulted in the outcome $\tau_S^{(j)}$ rather than $\tau_S$ where $\tau_S^{(j)}(x)=\tau_S(x)$ for all $x\in S^1(y_0)$, then for all $x \in S^1(y_m,y_f),$ where $y_m=(t_m, z_0)$ we would be able to say of the mass-energy density $\tau_S^{(j)}(x)$ at $x$ that it was caused by a photon that had been reflected from the apparatus $\mathcal{A}$ being in the $\ket{a_j}$-state rather than one of the other states. We will refer to the subregion of $S^1(y_m,y_f)$ where photons coming from the apparatus arrive and determines which state the apparatus  is in as $S_{\mathcal{A}}$, and we will let $\ket{\gamma^{(\mathcal{A})}_k}$ denote the simultaneous $\hat{T}_S$ eigenstate of the subregion $S_{\mathcal{A}}$ of $S$ that corresponds to the apparatus $\mathcal{A}$ being in state $\ket{a_k}$.  We will also label the states on the subregion $S\setminus (S^1(y_0)\cup S_{\mathcal{A}})$ as $\ket{\Xi_l}$ so that we can express the state $\ket*{\Psi_S}=U_{SS_0}\ket*{\Psi_0}$ as a superposition
$$\ket*{\Psi_S}=\sum_k c_k \ket*{\gamma_0^{(S^1(y_0)}}\ket*{\gamma_k^{(\mathcal{A})}}\ket*{\Xi_k} +\sum_{j\neq 0} \ket*{\gamma_j^{(S^1(y_0)}}\ket*{\Xi_j'}$$ 
where the $\ket*{\gamma_j^{(S^1(y_0)}}$ for $j\neq 0$ are simultaneous $\hat{T}_S$-eigenstates of the subregion $S^1(y_0)$ whose simultaneous $\hat{T}_S$-eigenvalues restricted to $S^1(y_0)$ are distinct from $\tau_S$, and where $\ket*{\Xi_j'}$ are states of the subregion $S\setminus S^1(y_0)$. 



 



We will assume that we can write any state of $H_S$ as a superposition of states of the form $\ket*{\Psi_U}\ket*{\Psi_{S\setminus U}}$, where $\ket{\Psi_U}\in H_U $ is  a simultaneous $\hat{T}_S$-eigenstate of the subregion $U$ of $S$, and $\ket*{\Psi_{S\setminus U}}\in H_{S\setminus U}$ is a simultaneous
$\hat{T}_S$-eigenstate of the subregion $S\setminus U$ of $S$.

cor  sufficiently little interaction of $\mathcal{S}+\mathcal{A}$ with its environment that we can assume this composite system evolves unita  With these assumptions, we can then write $\ket*{\Psi^{(j)}}$ as 
$$\ket*{\Psi^{(j)}}= \ket*{\gamma^{(S^1(y_0))}}\ket*{\gamma^{(\mathcal{S},j)}}\ket*{\gamma^{(\mathcal{A},j)}}\ket*{\Xi^{(j)}}$$
where 
$\ket*{\gamma^{(S^1(y_0))}}$ is the  state of $S^1(y_0)$ which is the simultaneous eigenstate of $\hat{T}_S(x)$ for all $x\in S^1(y_0)$ with corresponding eigenvalue $\tau_S^{(j)}(x)=\tau_S(x)$,\footnote{Note that by the non-degeneracy assumption, there will be only one such simultaneous eigenstate.} $\ket*{\gamma^{(\mathcal{S},j)}}$  is a state of the subregion $S_{\mathcal{S}} $ of $S^1(y_f,y_0)$ where the photons detected would have been reflected from $\mathcal{S}$,  $\ket*{\gamma^{(\mathcal{A},j)}}$  is a state of the subregion $S_{\mathcal{A}}$ of $S^1(y^{(j)},y_0)$ where the photons detected would have been reflected from $\mathcal{A}$, and where $\ket*{\Xi^{(j)}}$ corresponds to everything else found in the measurement $\tau_S^{(j)}$.

Now suppose that the spacelike hypersurface $S$ has energy density $\tau_S(x)$ indicating that some photons have been ``measured'' on $S$ to be in a (normalized\footnote{In this section we will assume that all states are normalized.}) state $\ket*{\gamma_i^{(\mathcal{A})}}$ which is correlated with the apparatus being in a state $\ket{a}$ shortly before time $t_i$ and in the vicinity of spatial location $z_0$ as depicted in figure \ref{pisolution}. The state $\ket{a}$ will encapsulate (among other things) the parameter settings of the apparatus $\mathcal{A}$. We also suppose that $\tau_S(x)$ indicates that some photons have been ``measured'' on $S$ to be in a state $\ket*{\gamma_i^{(\mathcal{S})}}$ which is correlated with the particle being in a state $\ket{s}$ also shortly before time $t_i$ and in the vicinity of spatial location $z_0$. We assume that the time $t_i$ is just before .lthe particle enters the apparatus (given the measurement of $\ket*{\gamma_i^{(\mathcal{S})}}$ and  $\ket*{\gamma_i^{(\mathcal{A})}}$ on $S$) and that no more photons are measured on $S$ that have become entangled with particle  or the apparatus until the particle emerges from the apparatus at time $t_f$. Between times $t_i$ and $t_f$, the particle interacts with the apparatus, and this will result in the particle and the apparatus becoming entangled if the particle is in a superpostion state of the possible measurement outcomes of the apparatus given the apparatus's parameter settings.\footnote{Since it is assumed that the measurement outcome states form a basis for the system $\mathcal{S}$, if the particle is not in a superposition state of measurement outcomes, it must already be in a definite measurement outcome state, and in this case, the appartus will evolve to a state that displays the state of the particle, but the particle and the apparatus won't become entangled.} Then some more photons interact with the apparatus $\mathcal{A}$ and get entangled with it shortly before time $t_m$, and these photons are detected on $S$ to be in a state  $\ket*{\gamma_f^{\prime\prime}}$, and this state is correlated with the apparatus now being in a state $\ket{a_{f}}$, and hence the particle being in state $\ket{s_f}$. 

\begin{figure}[ht!]
	\captionsetup{justification=justified}
	\centering
	\tikzmath{
	\a=5.3;  
	\qa=5.3;  
	\psione=0;
	\qpsione=0;
	\psitwo=1.5;
	\h=-1;
	\labelx=(\psione+\psitwo)/2;
	\labely=-1.5;
	\phstartx=-1.1;
	\qphstartx=-3.8;
	\wphstartx=-0.8;
	\phstarty=\h;
	\tbeg=\phstarty;
	\ca=\phstartx-\tbeg;
	\ttwo=\psitwo-\ca;
	\cb=\ttwo+\ca;
	\xend=\ttwo-\a+\cb;
	\tone=\psione-\ca;
	\cc=\tone+\ca;
	\xendz=\tone-\a+\cc;
	\qca=\qphstartx-\tbeg;
	\qtone=\qpsione-\qca;
	\qqcc=\qtone+\qca;
	\qxendz=\qtone-\qa+\qqcc;
	\wa=5.3;  
	\wpsione=0;
	\wttestx=3.7;
	\wca=\wphstartx-\tbeg;
	\wtone=\wpsione-\wca;
	\wwcc=\wtone+\wca;
	\wxendz=\wtone-\wa+\wwcc;
	\wat=\wa-\wttestx;
	\wct=\wttestx;
	\wlam = 0.87;
		\we=0.1;
	\tthree=\cc+\tone-\psitwo;
	\circsize=0.08;
	\md = (\a+\h)/2;
	\tlen=0.75;
	\textscale = 0.7;
	\picscale = 0.78;
	\nudge=0.1;
	\tnudge=(\ttwo-\tone)/2+0.8;
	\ttesta=2*\tone-\ttwo-\tnudge;
	\sonea=\a+\psione-\ttesta;
	\sadd=0.5;
	\lrange = -(\psione-\a+\ttesta)+\sadd;
	\rrange =\lrange; 
	\timex=\rrange-1.5;
	\ttestx=2.4;
	\at=\a-\ttestx;
	\ct=\ttestx;
	\qttestx=1.2;
	\qat=\qa-\qttestx;
	\qct=\qttestx;
	\qlam = 0.87;
	\lam = 0.87;
	\e = 0.1;
	\qe=0.1;
	\hae=(2*pow(\at,3)*\lam*(2+\lam)+\e*\e*(2*\e-sqrt(4*\e*\e+\at*\at*\lam*\lam))-8*\at*\e*(-2*\e+sqrt(4*\e*\e+\at*\at*\lam*\lam))-2*\at*\at*(2+\lam)*(-2*\e+sqrt(4*\e*\e+\at*\at*\lam*\lam)))/(4*\at*\lam*(2*\at+2*\e-sqrt(4*\e*\e+\at*\at*\lam*\lam)))-(\e*\e/4+\at*\at*(-1+\lam))/(\at*\lam);
	\qhae=(2*pow(\qat,3)*\qlam*(2+\qlam)+\qe*\qe*(2*\qe-sqrt(4*\qe*\qe+\qat*\qat*\qlam*\qlam))-8*\qat*\qe*(-2*\qe+sqrt(4*\qe*\qe+\qat*\qat*\qlam*\qlam))-2*\qat*\qat*(2+\qlam)*(-2*\qe+sqrt(4*\qe*\qe+\qat*\qat*\qlam*\qlam)))/(4*\qat*\qlam*(2*\qat+2*\qe-sqrt(4*\qe*\qe+\qat*\qat*\qlam*\qlam)))-(\qe*\qe/4+\qat*\qat*(-1+\qlam))/(\qat*\qlam);
		\whae=(2*pow(\wat,3)*\wlam*(2+\wlam)+\we*\we*(2*\we-sqrt(4*\we*\we+\wat*\wat*\wlam*\wlam))-8*\wat*\we*(-2*\we+sqrt(4*\we*\we+\wat*\wat*\wlam*\wlam))-2*\wat*\wat*(2+\wlam)*(-2*\we+sqrt(4*\we*\we+\wat*\wat*\wlam*\wlam)))/(4*\wat*\wlam*(2*\wat+2*\we-sqrt(4*\we*\we+\wat*\wat*\wlam*\wlam)))-(\we*\we/4+\wat*\wat*(-1+\wlam))/(\wat*\wlam);
	} 
	\tikzstyle{scorestars}=[star, star points=5, star point ratio=2.25, draw, inner sep=1pt]%
	\tikzstyle{scoresquare}=[draw, rectangle, minimum size=1mm,inner sep=0pt,outer sep=0pt]%
	\begin{tikzpicture}[scale=\picscale,
	declare function={
		testxonep(\ps,\t)=\a+\ps-\t;
		testxonem(\ps,\t)=\ps-\a+\t; 
		bl(\x)=\ct+\at-\e/2-1/2*(sqrt(\lam*\lam*pow(\x+\hae+\at,2)+\e*\e)-\e+\lam*(\x+\hae+\at));
		br(\x)=\ct+\at-\e/2-1/2*(sqrt(\lam*\lam*pow(\x-\hae-\at,2)+\e*\e)-\e-\lam*(\x-\hae-\at));
		bc(\x)=\ct+sqrt(\lam*\lam*\x*\x+\e*\e)-\e;
		qbl(\qx)=\qct+\qat-\qe/2-1/2*(sqrt(\qlam*\qlam*pow(\qx+\qhae+\qat,2)+\qe*\qe)-\qe+\qlam*(\qx+\qhae+\qat));
		qbr(\qx)=\qct+\qat-\qe/2-1/2*(sqrt(\qlam*\qlam*pow(\qx-\qhae-\qat,2)+\qe*\qe)-\qe-\qlam*(\qx-\qhae-\qat));
		qbc(\qx)=\qct+sqrt(\qlam*\qlam*\qx*\qx+\qe*\qe)-\qe;
		wbl(\wx)=\wct+\wat-\we/2-1/2*(sqrt(\wlam*\wlam*pow(\wx+\whae+\wat,2)+\we*\we)-\we+\wlam*(\wx+\whae+\wat));
		wbr(\wx)=\wct+\wat-\we/2-1/2*(sqrt(\wlam*\wlam*pow(\wx-\whae-\wat,2)+\we*\we)-\we-\wlam*(\wx-\whae-\wat));
		wbc(\wx)=\wct+sqrt(\wlam*\wlam*\wx*\wx+\we*\we)-\we;
	},
	 ] 
	\definecolor{tempcolor}{RGB}{250,190,0}
	\definecolor{darkgreen}{RGB}{40,190,40}
	
	\draw[->,blue, thick] [domain=-\at/2:-\lrange, samples=150]   plot (\x, {bl(\x)})  ;
	\draw[blue, thick] [domain=-\at/2:\at/2, samples=150] plot (\x, {bc(\x)})   ;
	\draw[->,blue, thick] [domain=\at/2:\rrange, samples=150]   plot (\x, {br(\x)})   ;
	
	 
	 \draw[->,gray, thick] [domain=-\at/2:-\lrange, samples=150]   plot (\x, {qbl(\x)})  ;
	\draw[gray, thick] [domain=-\at/2:\at/2, samples=150] plot (\x, {qbc(\x)})   ;
	\draw[->,gray, thick] [domain=\at/2:\rrange, samples=150]   plot (\x, {qbr(\x)})   ;
	
	\draw[->,black, thick] [domain=-\at/2:-\lrange, samples=150]   plot (\x, {wbl(\x)})  ;
	 \draw[black, thick] [domain=-\at/2:\at/2, samples=150] plot (\x, {wbc(\x)})   ;
	 \draw[->,black, thick] [domain=\at/2:\rrange, samples=150]   plot (\x, {wbr(\x)})   ;

	  
	  
	
	
	 \node[scale=\textscale]  at (2.5,4.1) {$S_{n,f}$}; 
	  \node[scale=\textscale]  at (3.9,4.1) {$S_{n,i}$}; 
	  	  \node[scale=\textscale]  at (1.1,4.1) {$S_{n,m}$}; 
	
	\draw[<->] (-\lrange, \h) node[left, scale=\textscale] {$S_0$} -- (\rrange, \h) node[right, scale=\textscale] {$S_0$};
	\draw[<->] (-\lrange, \a) node[left, scale=\textscale] {$S$} -- (\rrange, \a) node[right, scale=\textscale] {$S$};
				  
	\draw[->, shorten <= 5pt,  shorten >= 1pt] (\psione,\h)  node[above right, scale=\textscale]{$z_0$}-- (\psione,\a) ;
	
	
	\draw[dashed, tempcolor,  ultra thick](\phstartx,\tbeg) node[below left=-2,scale=\textscale]{}--(\psione,\tone) node[above, pos=0.3, rotate=45,scale=0.7,black] {} node[below, pos=0.3, rotate=45,scale=0.7,black] {};
	\draw[dashed, tempcolor,  ultra thick](\psione,\tone)--(\xendz,\a);
	
	
	\draw[dashed, tempcolor, ultra thick](\qphstartx,\tbeg) node[below left=-2,scale=\textscale]{}--(\psione,\qtone) node[above, pos=0.3, rotate=45,scale=0.7,black] {} node[below, pos=0.3, rotate=45,scale=0.7,black] {};
	\draw[dashed, tempcolor, ultra thick](\psione,\qtone)--(\qxendz,\a);

	\draw[dashed, orange](\wphstartx,\tbeg) node[below left=-2,scale=\textscale]{}--(\psione,\wtone) node[above, pos=0.3, rotate=45,scale=0.7,black] {} node[below, pos=0.3, rotate=45,scale=0.7,black] {};
	\draw[dashed, orange](\psione,\wtone)--(\wxendz,\a);
	
	\draw[->] (\timex,\md-\tlen/2) --  (\timex,\md+\tlen/2) node[midway,right, scale=\textscale]{time}; 
	 
	\draw[dotted](\psione,\ttestx)--({testxonep(\psione,\ttestx)},\a);
	\draw[dotted](\psione,\ttestx)--({testxonem(\psione,\ttestx)},\a);
	\draw[dotted](\psione,\qttestx)--({testxonep(\psione,\qttestx)},\a);
	\draw[dotted](\psione,\qttestx)--({testxonem(\psione,\qttestx)},\a);
	
	\draw (\psione,\ttestx) node[ scoresquare, fill=gray]  {} node [black,below=6,right,scale=\textscale] {$t_f$};
	\draw (\psione,\qttestx) node[ scoresquare, fill=gray]  {} node [black,below=6,right,scale=\textscale] {$t_i$};
	\draw (\psione,\wttestx) node[ scoresquare, fill=gray]  {} node [black,below=6,right,scale=\textscale] {$t_m$};
	
%	\draw[magenta,->,ultra thick] ({testxonem(\psione,\qttestx)},\a)--(-\lrange,\a);
%	\draw[magenta,ultra thick] ({testxonem(\psione,\ttestx)},\a)-- ({testxonem(\psione,\qttestx)},\a);
%	\draw[magenta,ultra thick] ({testxonep(\psione,\ttestx)},\a)-- ({testxonep(\psione,\qttestx)},\a);
%	\draw[magenta,->,ultra thick] ({testxonep(\psione,\qttestx)},\a)--(\rrange,\a);
	
	\draw [black,fill](\xendz,\a) circle [radius=\circsize] node [black,above=9,right=-4,scale=\textscale] {$\gamma_i^{(\mathcal{A})}$}; 
	\draw [black,fill](\wxendz,\a) circle [radius=\circsize] node [black,above=9,left=-7,scale=\textscale] {$\gamma_i^{(\mathcal{S})}$}; 
	\draw [black,fill](\qxendz,\a) circle [radius=\circsize] node [black,above,scale=\textscale] {$\gamma_f^{\prime\prime}$}; 
	\draw [black,fill](-3.7,\a) circle [radius=\circsize] node [black,above=9,left=-7,scale=\textscale] {$\gamma_0'$}; 
	%\draw [black,fill] (\xendz,\a) circle [radius=\circsize] node [black,above,scale=\textscale] {$\gamma_1$}; 
	\end{tikzpicture}% pic 1
	
	\vspace*{2px}
	\caption{Depicts an experiment where the state of some photons $\gamma_i^{(\mathcal{S})}$ and $\gamma_i^{(\mathcal{A})}$ on the spacelike hypersurface $S$ determines the initial conditions of an experimental setup of a particle $\mathcal{S}$ and apparatus $\mathcal{A}$ in the vicinity of the spacetime location $(z_0, t_i)$. The state of the photons $\gamma_f^{\prime\prime}$ on the spacelike hypersurface $S$ determines the final state of the apparatus after the particle has left it at time $t_f$ so that the apparatus at time $t_m$ displays a definite measurement outcome. It is assumed that no incoming photons have become entangled with the experiment after the $\gamma_i^{(\mathcal{S})}$ and $\gamma_i^{(\mathcal{A})}$ photons and before the $\gamma_f^{\prime\prime}$ photons have become entangled with the experiment.  }
	\label{pisolution}
	\end{figure}

 
We aim to show that within Kent's theory, we can calculate the probability the particle emerges from the measuring apparatus $\mathcal{A}$ in state $\ket{s_{f}}$ given that it enters $\mathcal{A}$ in state $\ket{s}$, and that this probability is the same as if one ignored $S$ and just applied the Born Rule to $\ket{s}$ and $\ket{s_f}$. 

In order to show this, let us choose a sequence of spacelike hypersurfaces $S_{n,i}$ which go through the spacetime location $y_i=(t_i, z_0)$ such that $\lim_{n\rightarrow\infty} S_{n,i}\cap S=S^1(y_i),$\footnote{We understand this limit by giving $S^1(y_i)$ and $S_{n,i}$  topologies that makes them locally homeomorphic to Euclidean space, and so by saying there is this limit of hypersurfaces, we mean that every point $u\in S^1(y_i)$ has a neighborhood $U\subset S^1(y_i)$ for which there is an integer $N$ such that $U\subset S_{n,i}$ for all $n\geq N$.} where as usual,  $S^1(y_i)$ consists of all the spacetime locations of $S$ outside the light cone of $y_i$. Let us assume that $n$ is sufficiently large so that the photons described by $\ket*{\gamma_i^{(\mathcal{S})}}$ and $\ket*{\gamma_i^{(\mathcal{A})}}$ belong to $S_{n,i}$. The spacelike hypersurface $S_{n,i}$ and the photons being reflected from the vicinity of $z_0$ just before time $t_i$ are also depicted in figure \ref{pisolution}.

Typically, the quantum state $\ket{\Psi_{n,i}}=U_{S_{n,i},S_0}\ket{\Psi_0}$ of the spacelike hypersurface $S_{n,i}$ (where  $U_{S_{n,i},S_0}$ is the unitary operator relating the states of two spacelike hypersurfaces as discussed on page \pageref{SchwingerUnitaryOP}) will also include photon correlations with $\mathcal{S}$ and $\mathcal{A}$ corresponding to other possible ``measurements'' of $T_S(x)$ besides $\tau_S(x)$. So in general, we would expect the state of $S_{n,i}$ to be of the form
$$ \ket{\Psi_{n,i}}=\sum_{j,k}c_{j,k}\ket{\sigma_j}\ket{\alpha_k}\ket*{\gamma_j^{(\mathcal{S})}}\ket*{\gamma_k^{(\mathcal{A})}},$$
where $\{\ket{\sigma_j}:j\}$ 
is an orthonormal basis of states for the particle $\mathcal{S}$ with $\ket{s}\in\{\ket{\sigma_j}:j\}$,   $\{\ket{\alpha_k}:k\}$ 
is an orthonormal basis of states describing the apparatus $\mathcal{A}$ with $\ket{a}\in\{\ket{\alpha_k}:k\}$,  $\{\ket*{\gamma_j^{(\mathcal{S})}}:j\}$ 
are normalized states of photons in $S_{n,i}\cap S$ that are entangled with the particle $\mathcal{S}$ such that $\ip*{\gamma_j^{(\mathcal{S})}}{\gamma_{j'}^{(\mathcal{S})}}\approx 0$ for $j\neq j'$,   $\{\ket*{\gamma_k^{(\mathcal{A})}}:k\}$  are normalized states of photons in $S_{n,i}\cap S$ that are entangled with the apparatus $\mathcal{A}$ such that $\ip*{\gamma_k^{(\mathcal{A})}}{\gamma_{k'}^{(\mathcal{A})}}\approx 0$ for $k\neq k'$, and for clarity, we have absorbed any other environmental information into the states $\ket{\alpha_k}$. 

If we now define the projection $\pi_{n,i}$ corresponding to the ``measurement outcome''  $\tau_S(x)$ on $S_{n,i}\cap S$ as in equation (\ref{tauprojection}), and if we also assume that the bases are indexed so that $\ket{s}=\ket{\sigma_i}$
 and $\ket{a}=\ket{\alpha_i}$ then
\begin{equation}\label{piphi}
	\pi_{n,i}\ket{\Psi_{n,i}}\approx c\ket{s}\ket{a}\ket*{\gamma_i^{(\mathcal{S})}}\ket*{\gamma_i^{(\mathcal{A})}}
\end{equation}
 where $c=c_{i,i}$. For convenience, we will omit the reference to $n$ and write $S_i$ for $S_{n,i}$ and $\pi_i$ for $\pi_{n,i}$. We will also write $\ket{\Phi_i}$ for the normalized state of $\pi_i\ket{\Psi_i}$ so that 
 \begin{equation}\label{phii}
	\ket{\Phi_i}\approx \ket{s}\ket{a}\ket*{\gamma_i^{(\mathcal{S})}}\ket*{\gamma_i^{(\mathcal{A})}}.
\end{equation}

 We now suppose that at time $t_i$, $\ket{a}$ is the ready state of the apparatus with pointer states $\{\ket{a_j}:j\}$ so that if $\ket{s}=\sum_i c_j\ket{s_j}$, then under Schr\"{o}dinger evolution from time $t_i$ to $t_f$,
 $$\ket{s}\ket{a}\rightarrow\sum_j c_j\ket{s_j}\ket{a_j}.$$
 We assume that before time $t_f$, no photons have had a chance to get entangled with $\mathcal{S}+\mathcal{A}$. It is only by time $t_m$ that we assume a measurement of photons in state $\ket{\gamma_f^{\prime\prime}}$ on $S$ outside the light cone of $(t_m,z_0)$ is able to determine that the apparatus is in state $\ket{a_f}$ and hence that the particle is in state $\ket{s_f}$. Although the measurement outcome $\tau_S(x)$ on the whole of $S$ determines with probability $1$ that the apparatus and the particle will be in the states $\ket{a_f}$ and $\ket{s_f}$ respectively at time $t_m$, if we consider the probability $P(f|\ket{\Phi_i})$ that this outcome occurs based just on the state $\ket{\Phi_i}$, then typically this probability is going to be less than $1$.
 
 To calculate this probability, we first consider the evolution of $\ket{\Phi_i}$ to $ U_{S_f,S_i}\ket{\Phi_{i}}$. It's possible that there may be photons ``measured'' on $S$ between the spacelike hypersurfaces $S_i$ and $S_f$ as indicated in figure \ref{pisolution} (i.e. on $(S\cap S_f)\setminus(S\cap S_i)$), but we are assuming that they do not get entangled with the different pointer states of the apparatus. In other words, $\ket{\Phi_i}$ will evolve to a state of the form 
\begin{equation}\label{USfievolve1}
	U_{S_f,S_i}\ket{\Phi_{i}}\approx \sum_j c_j\ket{s_j}\ket{a_j}\ket*{\gamma_i^{(\mathcal{S})}}\ket*{\gamma_i^{(\mathcal{A})}}\sum_kg_k\ket{\gamma'_k},
\end{equation}
where $\ket{\gamma_k'}$ correspond to the possible measurements of $T_S(x)$  on $(S\cap S_f)\setminus(S\cap S_i)$, and $\sum_k\abs{g_k}^2=1$.

But from time $t_f$ to $t_m$, we assume that the apparatus does get entangled with photons which are measured on $S\cap S_m$. Thus, if $\{\ket{\gamma_j^{\prime\prime}}:j\}$ are the normalized states representing the possible measurements outcomes of these photons such that $\ip{\gamma_j^{\prime\prime}}{\gamma_k^{\prime\prime}}\approx 0$ for $j\neq k$, then 
$$U_{S_m,S_f} U_{S_f,S_i}\ket{\Phi_{i}}\approx \sum_j c_j\ket{s_j}\ket{a_j}\ket*{\gamma_i^{(\mathcal{S})}}\ket*{\gamma_i^{(\mathcal{A})}}\sum_kg_k\ket{\gamma'_k}\ket{\gamma_j^{\prime\prime}}.$$
Since we are assuming that at time $t_m$ a measurement of photons on $S\cap S_m$ is able to determine that the apparatus is in state $\ket{a_f}$, this can only happen if $U_{S_m,S_f} U_{S_f,S_i}\ket{\Phi_{i}}$ is found to be in one of the states $\ket{\Phi_{k,f}}$ for some $k$ where
$$\ket{\Phi_{k,j}}=\ket{s_j}\ket{a_j}\ket*{\gamma_i^{(\mathcal{S})}}\ket*{\gamma_i^{(\mathcal{A})}}\ket{\gamma'_k}\ket{\gamma_j^{\prime\prime}}.$$
By the Born Rule, the probability $\ket{\Phi_i}$ will be found to be in state $\ket{\Phi_{k,j}}$ will be
$$\abs{\ip{\Phi_{k,j}}{\Phi_i}}=\abs{c_j}^2\abs{g_k}^2.$$
Therefore,
$$P(f|\ket{\Phi_i})=\sum_k \abs{\ip{\Phi_{k,f}}{\Phi_i}}=\abs{c_f}^2=\abs{\ip{s_f}{s}}^2.$$
Hence, the probability that a complete measurement of $T_S(x)$ on $S$ will give a measurement outcome of the particle being in state $\ket{s_f}$ given the partial measurement of $T_S(x)$ on $S_i\cap S$ determines the particle to be initially in the state $\ket{s}$ will be the same as the standard Born Rule probability $\abs{\ip{s_f}{s}}^2$ of $\ket{s}$ being found to be in state $\ket{s_f}$.

We can also recover this probability using Kent's conditional expectation. To do this, we recall that in standard quantum theory, if for some state $\ket{\psi}$ of a system we define the operator $[\psi]=\dyad{\psi}$, then when the system is in some initial state $\ket{\chi}$, the Born Rule implies that $\ev{[\psi]}{\chi}=P(\psi|\chi)$, where $P(\psi|\chi)$ is the probability that the system will be found to be in state $\ket{\psi}$ given that it was initially in state $\ket{\chi}$. But by (\ref{evev}),  $\ev{[\psi]}{\chi}$ is just the expectation $\ev*{\psi}_\chi$ of $[\psi]$ when $[\psi]$ is treated as an observable. 

Now in equation (\ref{kentconsistency0}), we saw how to calculate the expectation value $\ev*{T^{\mu\nu}(y)}_{\tau_S}$ of the observable $\hat{T}^{\mu\nu}(y)$ given the notional measurement $\tau_S$ on $S$ outside the light cone of $y$. This suggests that the expectation value of any observable $\hat{O}$ defined at spacetime location $(t_i,z_0)$ given the notional measurement $\tau_S$ on $S$ outside the light cone of $(t_i,z_0)$ is going to be
$$\ev*{\hat{O}}_{\tau_S}=\frac{\ev*{\pi_i\hat{O}}{\Psi_i}}{\ev*{\pi_i}{\Psi_i}}.$$ 
By (\ref{piphi}), $\ev*{\pi_i}{\Psi_i}=\abs{c}^2$, and so taking $\hat{O}$ to be $[s_f]$ we have 
$$ \ev*{[s_f]}_{\tau_S}=\frac{\abs{c}^2\abs{\ip{s_f}{s}}^2}{\abs{c}^2}=\abs{\ip{s_f}{s}}^2.$$
Thus, Kent's conditional expectation $\ev*{[s_f]}_{\tau_S}$ gives us the same probability $\abs{\ip{s_f}{s}}^2$ for a particle transitioning from state $\ket{s}$ to state $\ket{s_f}$ as in standard quantum theory.

Also note that we can typically expect the $\ket*{\gamma_i^{(\mathcal{S})}}$-state to be independent of the $\ket*{\gamma_i^{(\mathcal{A})}}$-state. Therefore, since $\ket*{\gamma_i^{(\mathcal{A})}}$ will determine the measurement choice, and since $\ket*{\gamma_i^{(\mathcal{S})}}$ determines the initial state of the particle, we can expect the state of the particle to be independent of the measurement choice in Kent's theory. Thus, we can fulfil one of the necessary criteria (i.e. criterion \ref{hidden2}) for PI to be a well-defined notion.
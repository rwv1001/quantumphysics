
\section{The No-collapse Feature of Kent's Theory}
We first consider the no-collapse feature of Kent's theory. This is a feature that belongs both to the many-worlds interpretation and to the Bohmian interpretation. In all three interpretations, the quantum state deterministically evolves according to the Schr\"{o}dinger equation. The Schr\"{o}dinger equation itself describes how a quantum state evolves over time when there are no outside influences. The precise formula for the Schr\"{o}dinger equation need not concern us here, but all we need to know is that the Schr\"{o}dinger equation determines a so-called \textbf{unitary operator}\index{unitary operator} $U(t',t)$. %
\nomenclature{$U(t',t)$}{A unitary operator that determines the evolution of states from time $t$ to time $t'$,  \nomrefpage}%
What this means is that if a system is in a state $\ket*{\psi}$ at time $t$, then it will be in the state $\ket*{\psi'}=U(t',t)\ket*{\psi}$
at time $t'$. A unitary operator $U$ has the property that if $\ket*{\psi'}=U\ket*{\psi}$ and $\ket*{\chi'}=U\ket*{\chi}$, then 
\begin{equation}\label{unitarycond}
\ip{\chi'}{\psi'}=\ip{\chi}{\psi}.\protect\footnotemark
\end{equation} 
\footnotetext{A unitary operator $U$ must also be linear so that for any two states $\ket*{\psi}$ and $\ket*{\phi}$ and complex numbers $\alpha$ and $\beta$, we have 
\begin{equation*}
U(\alpha\ket*{\psi}+\beta\ket*{\phi})=\alpha U\ket*{\psi}+\beta U\ket*{\phi},
\end{equation*}
 and furthermore, a unitary operator must have the property that it is invertible: there is a linear operator $U^{-1}$ such that $U U^{-1}$ and $U^{-1} U$ are the identity operator $I$, i.e. $U^{-1} U\ket*{\psi}=UU^{-1}\ket*{\psi}=\ket*{\psi}$ for any state $\ket*{\psi}$.}Under the Copenhagen interpretation, a system will evolve unitarily for the most part, but there will typically be a non-unitary change in the state describing the system whenever there is a measurement.\footnote{Note that to say that the change in a state is non-unitary when a measurement is made is not to say that there is a non-unitary collapse operator that maps the quantum state to an eigenstate of some observable. Such a mapping would not make sense, since the collapse is not deterministic given the initial state. However, one could have a well-defined mapping from a time value $t$ to the quantum state of the system $\ket*{\psi(t)}$ %
\nomenclature{$\ket*{\psi(t)}$}{The state of a system at time $t$,  \nomrefpage}%
 at time $t$. We then say that a system changes unitarily if and only if there is a unitary operator $U(t_1,t_0)$ for any two times $t_0$ and $t_1$ such that whenever the state of the system at time $t_0$ is given by $\ket*{\psi(t_0)}$, then the state of the system at time $t_1$ is given by $\ket*{\psi(t_1)}=U(t_1, t_0)\ket*{\psi(t_0)}$, and that for an intermediate time $t$, $U(t_1,t_0)=U(t_1, t)U(t, t_0).$ So to say that the change in a state is non-unitary when a measurement is made is to say that the state $\ket*{\psi(t)}$ describing the system does not change unitarily in the process of making a measurement. Now to see why this is the case under the Copenhagen interpretation, we suppose that at time $t_0$ a system is in the state $\ket*{\psi(t_0)}$ and that as long as no measurements are made up until a time $t\geq t_0$, the state evolves to a state $\ket*{\psi^{(U)}(t)}=U(t,t_0)\ket*{\psi(t_0)}$  %
\nomenclature{$\ket*{\psi^{(U)}(t)}$}{$\ket*{\psi^{(U)}(t)}=U(t,t_0)\ket*{\psi(t_0)}$,  \nomrefpage}%
 where $U(t,t_0)$ is a unitary operator determined by Schr\"{o}dinger's equation. Furthermore, we suppose that there is a measurable quantity with which we associate an observable $\hat{\bm{O}}$ so that whenever the state of the system is an eigenstate of $\hat{\bm{O}}$, the value of the measurable quantity for the system will be a determinate value and equal to the corresponding eigenvalue of $\hat{\bm{O}}$. At time $t_0$, we can express $\ket*{\psi(t_0)}$ as a linear combination
  \begin{equation*}
\ket*{\psi(t_0)}=\sum_i c_i\ket*{s_i(t_0)}
\end{equation*}
 where the $\ket*{s_i(t_0)}$ are eigenstates of $\hat{\bm{O}}$ with distinct eigenvalues. As long as no measurement is made, this will evolve as 
 \begin{equation*}
\ket*{\psi^{(U)}(t)}=\sum_i c_iU(t,t_0)\ket*{s_i(t_0)}.
\end{equation*}  
 We assume that as the state $\ket*{s_i(t_0)}$ evolves to the state $\ket*{s_i(t_1)}$ from time $t_0$ to $t_1$, it remains an eigenstate of $\hat{\bm{O}}$ with approximately the same eigenvalue. This assumption is based on the principle that in practice, performing a measurement is not instantaneous, but rather must take place over a time interval, and so the eigenstate and eigenvalue must be stable enough over this time interval so as to specify a definite outcome. We also assume that when the system is already in an eigenstate $\ket*{s_i(t_0)}$ of the observable $\hat{\bm{O}}$, it will evolve unitarily as $\ket*{s_i(t)}=U(t,t_0)\ket*{s_i(t_0)}$ for $t$ between $t_0$ and $t_1$, and that performing the measurement corresponding to $\hat{\bm{O}}$ will have no effect on the system when it is an eigenstate  $\ket*{s_i(t)}$ of $\hat{\bm{O}}$ -- otherwise we couldn't be sure that whenever we looked at the measurement readout that we weren't changing the value of the quantity we were trying to measure. \strut \\[\baselineskip]
Now according to the Copenhagen interpretation, when the measurement corresponding to $\hat{\bm{O}}$ is made, the system must enter into one of the eigenstates of the observable $\hat{\bm{O}}$, and at time $t_1$ shortly after the measurement has been made, the probability the system will be in the $\ket*{s_i(t_1)}$-state given that it was in the $\ket*{\psi(t_0)}$-state at time $t_0$ will be $|\ip{s_i(t_1)}{\psi^{(U)}(t_1)}|^2$ in accordance with the Born rule. So taking $\ket*{\psi(t_1)}$ to be proportional to $\ket*{s_i(t_1)}$ for some $i$, we see that for $j\neq i$, $\ip*{s_j(t_1)}{\psi(t_1)}=0$. This is because eigenstates of a Hermitian operator that have different eigenvalues must be orthogonal. However, since $U(t_1,t_0)$ is unitary,
 \begin{equation*}
\ip*{s_j(t_1)}{\psi^{(U)}(t_1)}=\ip*{s_j(t_0)}{\psi^{(U)}(t_0)}=c_j.
\end{equation*}
 So we see that $\ket*{\psi^{(U)}(t_1)}\neq\ket*{\psi(t_1)}$ if $\psi(t_0)$ is not initially in an eigenstate of $\hat{\bm{O}}$, and hence $\ket*{\psi(t)}$ doesn't evolve unitarily up to time $t_1$ as $\ket*{\psi^{(U)}(t)}$ does.} However, in non-collapse models  such as the Bohmian interpretation, the many-worlds interpretation, and Kent's theory, the quantum state always evolves unitarily. 

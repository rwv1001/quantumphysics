\chapter*{Introduction}
\addcontentsline{toc}{chapter}{Introduction}
In recent years, it has become increasingly common for popularizers of quantum physics to tell us that we need to let go of our naïve common sense understanding of reality. We're told we must replace this common sense understanding with something that at first seems very bizarre and counter-intuitive: a many-worlds interpretation of reality. This is the idea that whenever there is quantum indeterminacy among several possibilities, then all these possibilities are realized, and the actualization of these possibilities can be extrapolated up to the macroscopic level. Thus, many-worlds advocates, when reflecting on the famous Schr\"{o}dinger's Cat thought experiment do not question the foundations of quantum mechanics on which the thought experiment is based, but rather they embrace the seemingly absurd conclusion of Schr\"{o}dinger that the foundations of quantum mechanics imply that a cat could be both dead and alive. Many-worlds advocates thus speak of the cat being dead in one world and the cat being alive in another world that is just as real as the first. 

In this dissertation, I will consider in detail the EPR-Bohm paradox and the mysterious correlations that it exhibits, and I will explain why the EPR-Bohm paradox can lead one to think that our best physics suggests a many-worlds picture of reality. I will thus consider the many-worlds interpretation in some detail and explain why it is so attractive. But ultimately, my goal is not to endorse the many-worlds interpretation of quantum physics, but rather it is to argue that the reasons for endorsing the many-worlds interpretation are not compelling. My argument relies on a recent interpretation of quantum physics by the physicist Adrian Kent. Kent's interpretation shares several features with the many-worlds interpretation that make it seem attractive, yet in Kent's interpretation, there is only one world. 

Like many interpretations of quantum physics, Kent's interpretation is highly speculative, so I am not making any claims about whether Kent's interpretation is actually true. Rather, I only aim to show that if one fully accepts the predictive power of quantum physics and accepts the truth of special relativity, then one is not obliged to accept a many-worlds interpretation of quantum physics, since Kent's interpretation offers us a viable alternative.

In this dissertation, I will do my best to avoid unnecessary mathematical jargon, but since so many of the ideas in quantum physics are expressed in mathematical terms, a certain amount of mathematics is unavoidable. I will be assuming a basic understanding of trigonometry and probability, but beyond that, I will endeavor to explain all the mathematical terminology I use as I go along. There will, however, be some sections which may be very challenging to readers who do not have much of a background in mathematics or physics. These sections will be marked with an asterisk *.\label{asteriskmeaning} There is also a lot of terminology from theoretical physics that I will need to invoke, so to aid the reader, I will use the convention of putting terminology in italics whenever the terminology is first defined, and there is also a list of notation at the end of this dissertation.
\section{Isolating the Culprit}
Given the experimental violation of Bell's inequality, a strategy some philosophers of physics take is to reexamine the assumptions that lead to Bell's Inequality. Because of the violation of Bell's inequality, one of these assumptions will have to be discarded. The false assumption that is used to prove Bell's Inequality is sometimes referred to as \textbf{the culprit}\index{culprit, the}.\footnote{This is the terminology Butterfield uses following Abner Shimony. e.g. see \cite[1]{Butterfield}.} We therefore need to isolate the culprit, that is, we need to decide which assumption we should discard while keeping in mind that we wish to maintain a theory that is compatible with the experimental findings of quantum physics and special relativity.

Shimony noticed that there are two key assumptions in the proof of Bell's Inequality that might be identified as the culprit. He refers to one assumption as {Outcome Independence} (OI), and to the other assumption as {Parameter Independence} (PI).\footnote{See \cite[146-147]{Shimony86}.} Shimony argued that if we only denied OI, then the proof of Bell's Inequality would fail to go through. Yet by continuing to assume PI, there is a sense in which special relativity is not obviously violated. Shimony therefore thought that denying OI and assuming PI was sufficient to ensure peaceful coexistence between quantum theory and special relativity. In other words, Shimony thought OI was the culprit. 

\section{Parameter Independence\label{PISec}}
To explain Shimony's\footnote{See \cite[146-147]{Shimony86} and \cite[7-9]{Butterfield}.}  notion of  Parameter Independence, we suppose we have an experimental setup similar to the experimental setup described in the previous section. Thus, we suppose there are two particles labeled $q_A$, and $q_B$, and that a measurement can be made on particle  $q_A$ at one location (e.g. Alice's laboratory), and a measurement can be made on particle $q_B$ at some other location (e.g. Bob's laboratory). We will assume that Alice can make a choice of one of $n$ measurements to be made. These are labeled $a_1,\ldots, a_n$. For example, $a_1$ might be a measurement of $q_A$'s spin along the $z$-axis, whereas $a_2$ might be the measurement of $q_A$'s spin along an axis that is at  a $45^\circ$ angle to the $z$-axis etc. We use the variable $x$ to denote Alice's choice so that $x=a_i$ for some $i\in\{1,\ldots,n\}$. If Alice chooses to make measurement $a_i$ (i.e. $x=a_i$), the measurement outcome is labeled $A_i$, and this outcome can take values $+1$ or $-1$. For example, Alice could use the convention in which $+1$ corresponds to a spin up outcome, and $-1$ corresponds to a spin down outcome. We will use the variable $X$ to denote the measurement outcome Alice obtains, so for example, if Alice chooses to make the $a_1$ measurement so that $x=a_1$ and obtains the outcome $A_1=1$, then $X=1$. Similarly, we use the notation $b_i,\, y$, and $B_i,\, Y$ to correspond to the measurement choices and measurement outcomes for Bob.

We now suppose that there is a complete state $\lambda\in\Lambda$ describing both $q_A$ and $q_B$ that is independent of Alice and Bob's measurement choices, but that encodes all other features that would influence the corresponding measurement outcomes. Here, the domain $\Lambda$ of all such complete states will depend on how the two particles are prepared and the model we are assuming. We also assume that  $q_A$ and $q_B$ are initially coupled together in such a way that Alice and Bob would always get opposite results when they made their measurements in the same direction. For instance, for $n=3$, we might assume a model in which 
\begin{equation}\label{bellLambda}
\Lambda=\big\{(A_1, A_2, A_3 ,B_1, B_2, B_3):A_1,\, A_2,\, A_3=\pm1,\, B_i=-A_i\big\}.
\end{equation}
 In this case, $\lambda\in\Lambda$ would fully determine Alice and Bob's measurement outcomes along the three axes. This would be like the model described in the proof of Bell's Inequality with all the states of $\Lambda$ being described in table \ref{hiddentable} of section \ref{eprsec}. 
 
However, in general, we don't insist on such determinism. Rather, we suppose that given a complete state $\lambda\in\Lambda$, and given that Alice makes a measurement choice $x$ and Bob makes a measurement choice $y$, then there will be a probability $P_{\lambda,x,y}(X , Y)$ representing the probability Alice gets outcome $X$ and Bob gets outcome $Y$. In deterministic models, $P_{\lambda,x,y}(X , Y)$ will have values restricted to either $0$ or $1$. In non-deterministic models, there will have to be some situations when $P_{\lambda,x,y}(X, Y)$ has a value strictly between $0$ and $1$. For example, if we assume the Copenhagen interpretation model, we could take $\lambda$ to be the Bell state (\ref{bell}). Then it follows from equation (\ref{bellstate2}) that as long as Alice's and Bob's measurement choices $x$ and $y$ are in the same direction, then $P_{\lambda,x,y}(1,-1)=1/2$. Incidentally, we also note that equation (\ref{bellstate2}) implies the domain $\Lambda$ consists of a single state:
\begin{equation}\label{quantumLambda}
\Lambda=\Bigg\{\frac{1}{\sqrt{2}}\big(\ket*{\uvbp{a}}_A\ket*{\uvbm{a}}_B-\ket*{\uvbm{a}}_A\ket*{\uvbp{a}}_B\big)\Bigg\}.
\end{equation}

In both models (\ref{bellLambda}) and (\ref{quantumLambda}), we see that if we define
\begin{align}
P_{A, \lambda,x,y}(X)&=P_{\lambda,x,y}(X, 1)+P_{\lambda,x,y}(X, -1),\label{PIone}\\
P_{B, \lambda,x,y}(Y)&=P_{\lambda,x,y}(1, Y)+P_{\lambda,x,y}(-1, Y),\label{PItwo}
\end{align}
then $P_{A,\lambda,x,y}(X)$ is independent of Bob's choice of measurement $y$, and $P_{B,\lambda,x,y}(Y)$ is independent of Alice's choice of measurement $x$.\footnote{To see that this is true for model (\ref{bellLambda}), it is obvious that $P_{A, \lambda,x,y}(X)=1$ or 0 regardless of what $y$ is. As for model (\ref{quantumLambda}), it is straightforward to show\label{onehalf} that $P_{A, \lambda,x,y}(X)=1/2$ and $P_{B,\lambda,x,y}(Y)=1/2$ for any $X,\, Y$. E.g. for $x=\vb{\hat{a}}$ and $y=\vb{\hat{b}}$, by (\ref{bellstate2}), we can assume the two particles are in the state 
$$\ket{\zeta}=\frac{1}{\sqrt{2}}\big(\ket*{\uvbp{b}}_A\ket*{\uvbm{b}}_B-\ket*{\uvbm{b}}_A\ket*{\uvbp{b}}_B\big).$$
Since the inner product on the composite system is given by  $\ip{\xi'}{\xi}=\ip{\psi'}{\psi}_A\ip{\chi'}{\chi}_B$ for $\ket{\xi}=\ket{\psi}_A\ket{\chi}_B$ and $\ket{\xi'}=\ket{\psi'}_A\ket{\chi'}_B$, it follows that 
$$\prescript{}{A}{ \bra*{\vb{\hat{a}+}}}\prescript{}{B}{\ip*{\vb{\hat{b}\pm}}{\zeta}}=\mp\frac{1}{\sqrt{2}}\ip*{\vb{\hat{a}+}}{\vb{\hat{b}\mp}}_A.$$ 
Therefore, by the Born Rule (see page \pageref{bornrule})
$$P_{\lambda,\vb{\hat{a}},\vb{\hat{b}}}(\vb{\hat{a}+}, \vb{\hat{b}+})+P_{\lambda,\vb{\hat{a}},\vb{\hat{b}}}(\vb{\hat{a}+}, \vb{\hat{b}-})=\frac{1}{2}\abs*{\ip*{\vb{\hat{a}+}}{\vb{\hat{b}-}}_A}^2+\frac{1}{2}\abs*{\ip*{\vb{\hat{a}+}}{\vb{\hat{b}+}}_A}^2.$$
But since
$$\ket*{\vb{\hat{a}+}}_A=\ip*{\vb{\hat{b}+}}{\vb{\hat{a}+}}_A\ket*{\vb{\hat{b}+}}_A+\ip*{\vb{\hat{b}-}}{\vb{\hat{a}+}}_A\ket*{\vb{\hat{b}-}}_A$$
it follows that 
$$\abs*{\ip*{\vb{\hat{a}+}}{\vb{\hat{b}+}}_A}^2+\abs*{\ip*{\vb{\hat{a}+}}{\vb{\hat{b}-}}_A}^2 =1. $$
Therefore, 
$$P_{\lambda,\vb{\hat{a}},\vb{\hat{b}}}(\vb{\hat{a}+}, \vb{\hat{b}+})+P_{\lambda,x,y}(\vb{\hat{a}+}, \vb{\hat{b}-})=\frac{1}{2}.$$
 } In other models, however, it's possible that such independence does not hold. So to distinguish between such possibilities, we say a model satisfies \textbf{Parameter Independence}\index{Parameter Independence} (PI) \label{PIdef} if and only if $P_{A,\lambda,x,y}(X)$ is independent of $y$, and $P_{B,\lambda,x,y}(Y)$ is independent of $x$. In particular, PI holds in the model (\ref{bellLambda}) in the Copenhagen interpretation model (\ref{quantumLambda}). In other words, PI holds if and only if (\ref{PIone}) and (\ref{PItwo}) hold for all $\lambda,$ $x,$ $y,$ $X,$ and $Y$. If PI fails to hold in a model, we say that the model satisfies \textbf{Parameter Dependence}\index{Parameter Dependence} (PD).

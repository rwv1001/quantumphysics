\section{Isolating the Culprit}
Given the experimental violation of Bell's inequality, a strategy some philosophers of physics take is to reexamine the assumptions that lead to Bell's Inequality. Because of the observed violation of Bell's inequality, one of these assumptions will have to be discarded. The false assumption that is used to prove Bell's Inequality is sometimes referred to as \textbf{the culprit}\index{culprit, the}.\footnote{This is the terminology Butterfield uses following Abner Shimony. e.g. see \cite[1]{Butterfield}.} We therefore need to isolate the culprit, that is, we need to decide which assumption we should discard while keeping in mind that we wish to maintain a theory that is compatible with the experimental findings of quantum physics and special relativity.

Shimony noticed that there are two key assumptions in the proof of Bell's Inequality that might be identified as the culprit. He refers to one assumption as {Outcome Independence} (OI), %
\nomenclature{OI}{Outcome Independence, \nomrefpage}%
and to the other assumption as {Parameter Independence} (PI).\footnote{See \cite[146-147]{Shimony86}.}  %
\nomenclature{PI}{Parameter Independence, \nomrefpage} % 
Shimony argued that if we only denied OI, then the proof of Bell's Inequality would fail to go through. Yet by continuing to assume PI, there is a sense in which special relativity is not obviously violated. Shimony therefore thought that denying OI and assuming PI was sufficient to ensure peaceful coexistence between quantum theory and special relativity. In other words, Shimony thought OI was the culprit. 

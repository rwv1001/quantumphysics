\section{The Copenhagen Interpretation and the EPR-Bohm Paradox}
Given the equations (\ref{spintrans1}) and (\ref{spintrans2}) that  relate the states $\ket*{\uvbpm{a}}$ and $\ket*{\uvbpm{b}}$ to each other, we can calculate probabilities such as the probability a particle will be measured to be in the $\ket*{\uvbp{a}}$-state given that it is in the $\ket*{\uvbp{b}}$-state. There however arises the question of what the physical meaning of one of these states is. Clearly, the $\ket*{\uvbp{b}}$-state says something about the spin of a particle; but is this a complete description of the particle's spin state? For the $\ket*{\uvbp{b}}$-state only tells us what the outcome of a spin measurement would be along one particular axis $\uvb{b}$. For a spin measurement along another axis $\uvb{a}\neq\pm\uvb{b}$, $\ket*{\uvbp{b}}$ only tells us the probabilities (via equation (\ref{spintrans1}) and the Born rule) that the measurement outcome would be $\ket*{\uvbp{a}}$ or $\ket*{\uvbp{a}}$ -- the state doesn't determine either of these outcomes. So we want to know whether this indetermination of the measurement outcome along the $\uvb{a}$-axis is merely a reflection of our lack of knowledge of a more complete specification of the particle's spin state, or alternatively, whether the  $\ket*{\uvbp{b}}$-state is a complete description of the spin state of the particle so that there is no fact of the matter about what spin state the particle would be found to be in along the $\uvb{a}$-axis until a measurement of spin along the $\uvb{a}$-axis is made. 

Now Bohr and Heisenberg believed the latter to be the case. This was because their mathematical formalism of quantum physics implied that there are physical quantities of particles that couldn't be simultaneously determined. For example, their mathematical formalism is incapable of representing a particle which has a definite spin in both the $\uvb{a}$-direction and the $\uvb{b}$-direction when $\uvb{a}\neq\pm\uvb{b}.$ So when a particle that is in the $\ket*{\uvbp{b}}$-state is measured along the $\uvb{a}$-axis and is found to be in the $\ket*{\uvbp{a}}$-state, there is a so-called collapse of the $\ket*{\uvbp{b}}$-state:
$$ \ket*{\uvbp{b}}= \cos(\theta/2)\ket*{\uvbp{a}}+\sin(\theta/2) \ket*{\uvbm{a}}\xrightarrow{\text{Collapse!!}}\ket*{\uvbp{a}}$$
so that after the measurement, the particle is no longer in the $\ket*{\uvbp{b}}$-state. This interpretation of the quantum state as a complete physical description in which the state collapses to another state upon measurement is known as the \textbf{Copenhagen Interpretation}.

Einstein, Podolsky, and Rosen, however, strongly objected to the Copenhagen Interpretation, and they introduced their EPR paradox to explain what troubled them.\footnote{See \cite{EinsteinPodolskyRosen}.} The EPR paradox was originally expressed in terms of the position and momentum of a particle, but Bohm translated the EPR paradox to the context of spin,\footnote{e.g. see \cite[p. 29, Ch. 5 sec. 3, and Ch. 22 sec. 19]{BohmQuantumTheory}. } and this is the version we will consider here.   

The EPR-Bohm paradox arises in the context of particle pairs known as spin-singlets. A \textbf{spin singlet} describes the state of two particles which a single particle of zero spin has decayed into. For example, a high energy \textbf{photon}, that is, a particle of light, can decay into a negatively charged electron, and a positively charged position (where a \textbf{positron} is a fundamental particle like an electron but of opposite charge). Since spin is a conserved physical quantity, the spin of the two particles $q_A$ and $q_B$ of a spin singlet state must be equal and opposite when measured along the same axis, no matter what direction this axis happens to point in. The existence of spin singlet states thus raises the question of what the physical mechanism or principle is that ensures two experimenters, Alice and Bob say, will always obtain opposite spin measurement results if Alice measures the spin of particle $q_A$, and Bob measures the spin of particle $q_B$ along the same axis. 

Naively, one would expect that if the spin of $q_A$ were to be measured, then this would have no effect on any spin-measurement of $q_B$. This assumption is a special case of \textbf{Einstein's locality principle}:\label{EinsteinLocalityPrinciple} For two spatially separated systems $S_1$ and $S_2$,  the real factual situation of the system $S_2$ should be independent of what is done to the system $S_1$.\footnote{Einstein expressed this locality principle in his autobiographical notes: ``But on one supposition we should, in my opinion, absolutely hold fast: the real factual situation of the system $S_2$ is independent of what is done with the system $S_1$, which is spatially separated from the former.'' \cite[p. 85]{EinsteinLocality}.} If Einstein's locality principle holds, we would be able to attribute a state $\ket{\alpha}_A$ to particle $q_A$, and a state $\ket{\beta}_B$ to particle $q_B$, so that if Alice were to perform a Stern-Gerlach experiment on particle $q_A$ in which one of the possible outcomes was a spin state $\ket*{\alpha'}_A$, then by the Born rule, the probability Alice would find $q_A$ to be in state $\ket{\alpha'}_A$ would be $|\ip{\alpha'}{\alpha}_A|^2$. Likewise, if Bob were to perform a Stern-Gerlach experiment on particle $q_B$ in which one of the possible outcomes was a spin state $\ket*{\beta'}$, then the probability Bob would find $q_B$ to be in state $\ket{\beta'}_B$ would be $|\ip{\beta'}{\beta}_B|^2$. 

Now in order to decide how to represent the joint state of the particles $q_A$ and $q_B$, we recall that  in probability theory, we say that two events $X$ and $Y$ are \textbf{independent} if and only if 
\begin{equation}\label{indep}
    P(X,Y)=P(X)P(Y)
\end{equation}
where $P(X)$ is the probability that $X$ occurs, $P(Y)$ is the probability that $Y$ occurs, and $P(X,Y)$ is the probability that both $X$ and $Y$ both occur. We also say that two events $X$ and $Y$ are \textbf{conditionally independent} given some third event $Z$ if and only if 
\begin{equation}\label{indepcond}
P(X,Y|Z)=P(X|Z)P(Y|Z)
\end{equation} where $P(X|Z)$ is the conditional probability that $X$ occurs given $Z$, $P(Y|Z)$ is the conditional probability that $Y$ occurs given $Z$, and $P(X,Y|Z)$ is the conditional probability that both $X$ and $Y$  occur given $Z$.   

Now given that $q_A$ is in the $\ket*{\alpha}_A$-state and $q_B$ is in the $\ket*{\beta}_B$-state, the Born rule implies that the conditional probability that Alice would measure her particle to be in the state $\ket*{\alpha'}_A$ is not going to depend on $\ket*{\beta}_B$,  and so  $P(\alpha'|\alpha,\beta)=P(\alpha'|\alpha)$. Likewise, $P(\beta'|\alpha,\beta)=P(\beta'|\beta)$. 
If Alice and Bob's measurement outcomes are conditionally independent, then according to (\ref{indepcond}), we would obtain the conditional probability
\begin{equation}\label{indepprob}
    P_{A,B}(\alpha',\beta'|\alpha,\beta)=|\ip{\alpha'}{\alpha}_A|^2\times|\ip{\beta'}{\beta}_B|^2=|\ip{\alpha'}{\alpha}_A\ip{\beta'}{\beta}_B|^2.
\end{equation}
This suggests that if we write $\ket{\alpha}_A\ket{\beta}_B$ for the state of the composite system of both particles, then the bra-ket of $\ket{\alpha'}_A\ket{\beta'}_B$ and $\ket{\alpha}_A\ket{\beta}_B$ 
would be
\begin{equation}\label{innerprod}
    \prescript{}{B}{ \bra*{\beta'}}\prescript{}{A}{\ip*{\alpha'}{\alpha}}_A\ket*{\beta}_B=\ip{\alpha'}{\alpha}_A\ip{\beta'}{\beta}_B.
\end{equation}
However, when the particles $q_A$ and $q_B$ form a spin singlet, it will not be possible to express their joint state as $\ket{\alpha}_A\ket{\beta}_B$ because according to (\ref{indepprob}), we will always be able to find a direction $\uvb{a}$ such that $P_{A,B}(\uvbp{a},\uvbp{a}|\alpha,\beta)\neq 0$, whereas in reality, the state $Z$ describing the singlet has to satisfy $P_{A,B}(\uvbp{a},\uvbp{a}|Z)=0$ for all  $\uvb{a}$.
But it turns out that the summation of states:
\begin{equation}\label{bell}
    \ket*{\Psi_{\text{Bell}}}=\frac{1}{\sqrt{2}}(\ket*{\uvbp{a}}_A\ket*{\uvbm{a}}_B-\ket*{\uvbm{a}}_A\ket*{\uvbp{a}}_B).
\end{equation}
can describe the singlet state of the two particles. We refer to the state (\ref{bell}) as a \textbf{Bell state}.\footnote{By convention, the states $ \frac{1}{\sqrt{2}}(\ket*{\uvbp{a}}_A\ket*{\uvbm{a}}_B+\ket*{\uvbm{a}}_A\ket*{\uvbp{a}}_B)$, $ \frac{1}{\sqrt{2}}(\ket*{\uvbp{a}}_A\ket*{\uvbp{a}}_B-\ket*{\uvbm{a}}_A\ket*{\uvbm{a}}_B)$, and $ \frac{1}{\sqrt{2}}(\ket*{\uvbp{a}}_A\ket*{\uvbp{a}}_B+\ket*{\uvbm{a}}_A\ket*{\uvbm{a}}_B)$ are also referred to as Bell states.} If the composite system is in the Bell state $\ket*{\Psi_{\text{Bell}}}$, then according to the Born Rule, the probability that Alice measures her particle to be in state $\ket*{\alpha}$ and Bob measures his particle to be in the state $\ket*{\beta}$ will be:
\begin{equation}
\begin{split}
    P_{A,B}(\alpha,\beta|\Psi_{\text{Bell}})&=|\prescript{}{B}{ \bra*{\beta}}\prescript{}{A}{ \ip{\alpha}{\Psi_{\text{Bell}}}}|^2\\
        &=\frac{1}{2}|\ip*{\alpha}{\uvbp{a}}_A\ip*{\beta}{\uvbm{a}}_B-\ip*{\alpha}{\uvbm{a}}_A\ip*{\beta}{\uvbp{a}}_B|^2
\end{split}
\end{equation}
This means that whatever axis Bob decides to measure along, if Alice measures her particle along the $\uvb{a}$-axis, then the Born rule predicts that she will measure the particle to be in either the $\ket*{\uvbp{a}}_A$-state or the $\ket*{\uvbm{a}}_A$-state, each with probability of $\frac{1}{2}$.\footnote{To see this, suppose Bob performs his measurement along an arbitrary axis $\uvb{b}$. Then the probability Alice measures her particle to be in the $\ket*{\uvbp{a}}$-state will be 
\begin{equation}
    \begin{split}
        P_{A,B}(\uvbp{a},\uvbp{b}|\Psi_{\text{Bell}})&+P_{A,B}(\uvbp{a},\uvbm{b}|\Psi_{\text{Bell}})= \\
        =&\frac{1}{2}\big| \ip*{\uvbp{a}}{\uvbp{a}}_A\ip*{\uvbp{b}}{\uvbm{a}}_B-\ip*{\uvbp{a}}{\uvbm{a}}_A\ip*{\uvbp{b}}{\uvbp{a}}_B\big|^2\\
        &+\frac{1}{2}\big|\ip*{\uvbp{a}}{\uvbp{a}}_A\ip*{\uvbm{b}}{\uvbm{a}}_B-\ip*{\uvbp{a}}{\uvbm{a}}_A\ip*{\uvbm{b}}{\uvbp{a}}_B\big|^2\\
        =&\frac{1}{2}\big|\ip*{\uvbp{b}}{\uvbm{a}}_B|^2+\frac{1}{2}|\ip*{\uvbm{b}}{\uvbm{a}}_B\big|^2=\frac{1}{2}
    \end{split}
\end{equation} 
where on the last line we have used the fact that 
$$\ket*{\uvbm{a}}_B=\ket*{\uvbp{b}}_B\ip*{\uvbp{b}}{\uvbm{a}}_B+\ket*{\uvbm{b}}\ip*{\uvbm{b}}{\uvbm{a}}_B$$
and 
$$\ip*{\uvbm{a}}{\uvbm{a}}_B=1,\quad\ip*{\uvbpm{b}}{\uvbpm{b}}_B=1,\quad\text{and}\quad \ip*{\uvbpm{b}}{\uvbmp{b}}_B=0$$
so that 
$$|\ip*{\uvbp{b}}{\uvbm{a}}_B|^2+\ip*{\uvbm{b}}{\uvbm{a}}_B|^2=1.$$
} But also, the Born rule implies that if both Alice and Bob measure their respective particles along the same $\uvb{a}$-axis, then the probability Bob will measure his particle to have the same spin as Alice's particle will be zero, and the probability that Bob measures his particle to have the opposite spin from Alice's particle will be one.\footnote{This is because the probability Alice and Bob will measure their particles to have the same spin will be 
\begin{equation}
\begin{split}
    P_{A,B}(\uvbp{a},\uvbp{a}|\Psi_{\text{Bell}})&+P_{A,B}(\uvbm{a},\uvbm{a}|\Psi_{\text{Bell}})= \\
    =&\frac{1}{2}\big| \ip*{\uvbp{a}}{\uvbp{a}}_A\ip*{\uvbp{a}}{\uvbm{a}}_B-\ip*{\uvbp{a}}{\uvbm{a}}_A\ip*{\uvbp{a}}{\uvbp{a}}_B\big|^2\\
    &+\frac{1}{2}\big|\ip*{\uvbm{a}}{\uvbp{a}}_A\ip*{\uvbm{a}}{\uvbm{a}}_B-\ip*{\uvbm{a}}{\uvbm{a}}_A\ip*{\uvbm{a}}{\uvbp{a}}_B\big|^2=0,
\end{split}
\end{equation}
and the probability Alice and Bob will measure their particles to have different spins will be
\begin{equation}
\begin{split}
    P_{A,B}(\uvbp{a},\uvbm{a}|\Psi_{\text{Bell}})&+P_{A,B}(\uvbm{a},\uvbp{a}|\Psi_{\text{Bell}})= \\
    =&\frac{1}{2}\big| \ip*{\uvbp{a}}{\uvbp{a}}_A\ip*{\uvbm{a}}{\uvbm{a}}_B-\ip*{\uvbp{a}}{\uvbm{a}}_A\ip*{\uvbm{a}}{\uvbp{a}}_B\big|^2\\
    &+\frac{1}{2}\big|\ip*{\uvbm{a}}{\uvbp{a}}_A\ip*{\uvbp{a}}{\uvbm{a}}_B-\ip*{\uvbm{a}}{\uvbm{a}}_A\ip*{\uvbp{a}}{\uvbp{a}}_B\big|^2=1.
\end{split}
\end{equation}} 
These probabilities predicted by the Born rule using the Bell state $\ket*{\Psi_{\text{Bell}}}$ correspond to the probabilities observed experimentally. Also note that despite the of appearance the formula (\ref{bell}), $\ket*{\Psi_{\text{Bell}}}$ is independent of the axis $\uvb{a}$. That is, for any other direction $\uvb{b}$, it can be shown that,
\begin{equation}\label{bellstate2}\frac{1}{\sqrt{2}}(\ket*{\uvbp{a}}_A\ket*{\uvbm{a}}_B-\ket*{\uvbm{a}}_A\ket*{\uvbp{a}}_B)=\frac{1}{\sqrt{2}}(\ket*{\uvbp{b}}_A\ket*{\uvbm{b}}_B-\ket*{\uvbm{b}}_A\ket*{\uvbp{b}}_B).\protect\footnotemark\end{equation}
\footnotetext{\label{bellstate2pf}To see this, using the transformation rules given in equation (\ref{spintrans}) we have 
\begin{align*}\frac{1}{\sqrt{2}}(\ket*{\uvbp{b}}\ket*{\uvbm{b}}&-\ket*{\uvbm{b}}\ket*{\uvbp{b}})\\
      &=\frac{1}{\sqrt{2}}((\cos(\theta/2)\ket*{\uvbp{a}}+\sin(\theta/2)\ket*{\uvbm{a}})(\cos(\theta/2)\ket*{\uvbm{a}}-\sin(\theta/2)\ket*{\uvbp{a}})\\
&\quad-(\cos(\theta/2)\ket*{\uvbm{a}}-\sin(\theta/2)\ket*{\uvbp{a}})(\cos(\theta/2)\ket*{\uvbp{a}}+\sin(\theta/2)\ket*{\uvbm{a}}))\\ 
&=\frac{1}{\sqrt{2}}(\cos(\theta/2)\ket*{\uvbp{a}}\cos(\theta/2)\ket*{\uvbm{a}}-\cos(\theta/2)\ket*{\uvbp{a}}\sin(\theta/2)\ket*{\uvbp{a}}\\
&\quad+\sin(\theta/2)\ket*{\uvbm{a}}\cos(\theta/2)\ket*{\uvbm{a}}-\sin(\theta/2)\ket*{\uvbm{a}}\sin(\theta/2)\ket*{\uvbp{a}}\\
&\quad-\cos(\theta/2)\ket*{\uvbm{a}}\cos(\theta/2)\ket*{\uvbp{a}}-\cos(\theta/2)\ket*{\uvbm{a}}\sin(\theta/2)\ket*{\uvbm{a}}\\
&\quad+\sin(\theta/2)\ket*{\uvbp{a}}\cos(\theta/2)\ket*{\uvbp{a}}+
\sin(\theta/2)\ket*{\uvbp{a}}\sin(\theta/2)\ket*{\uvbm{a}})\\
&=\frac{1}{\sqrt{2}}((\cos[2](\theta/2)+\sin[2](\theta/2))\ket*{\uvbp{a}}\ket*{\uvbm{a}}-(\cos[2](\theta/2)+\sin[2](\theta/2))\ket*{\uvbm{a}}\ket*{\uvbp{a}})\\
&=\frac{1}{\sqrt{2}}(\ket*{\uvbp{a}}\ket*{\uvbm{a}}-\ket*{\uvbm{a}}\ket*{\uvbp{a}}).
\end{align*}}
Therefore, if Alice had chosen to measure her particle along the $\uvb{b}$-axis rather than the $\uvb{a}$-axis, she would still obtain equal probabilities for finding her particle to be in either the state $\ket*{\uvbp{b}}_A$ or $\ket*{\uvbm{b}}_A$, and the same equal probabilities for Bob's measurement outcomes hold as well.

Now to explain why this spells trouble for the Copenhagen interpretation, we suppose that Alice has a measurement device which we will denote by $O_{\uvbp{a}}$\label{Lambdaa} and which outputs the number 1 when her particle is detected at location $\uvbp{a}$, and outputs 0 when her particle is detected at location $\uvbm{a}$. Given that Alice knows that the state of both particles together is given by equation (\ref{bell}), she can work out the expectation value of her measurement $\ev*{O_{\uvbp{a}}}$ by summing up the product of each probability measurement outcome with the value of each measurement. This will give an expectation value of $\ev*{O_{\uvbp{a}}}=\frac{1}{2}\times 1 + \frac{1}{2}\times 0 = \frac{1}{2}.$ More generally, if Alice had a measuring device $O$ with $N$ measurement outcome values $o_1,\ldots,o_N$ and with respective probabilities $p_1,\ldots,p_N$ so that $\sum_{i=1}^N p_i=1,$ then the expectation $\ev*{O}$ would be given by the formula
\begin{equation}\label{expectation}\ev*{O} =\sum_{i=1}^N p_io_i.
\end{equation}
Now under the Copenhagen interpretation, the state $\ket*{\Psi_{\text{Bell}}}$ given by equation (\ref{bell}) encodes everything that can be said about the spins of the two particle system forming a spin singlet, it is tempting to suppose that the expectation value $\ev*{O_{\uvbp{a}}}$ tells us something objective about the system rather than just something about Alice's state of knowledge about the system.  Given this assumption, there then arises the question of what happens when a measurement is made. According to the Copenhagen interpretation, when Bob makes his measurement, the quantum state collapses to a component of the quantum state corresponding to the measurement Bob makes. Thus, if Bob's measurement device $O_{\uvbm{a}}$ outputs the number $1$, (i.e. Bob's particle is detected at location $\uvbm{a}$), then the state of the combined system would change accordingly as:
$$\frac{1}{\sqrt{2}}(\ket*{\uvbp{a}}_A\ket*{\uvbm{a}}_B-\ket*{\uvbm{a}}_A\ket*{\uvbp{a}}_B)\xrightarrow{\text{Collapse!!}}\ket*{\uvbp{a}}_A\ket*{\uvbm{a}}_B $$
If Bob makes his measurement first with his measurement device $O_{\uvbm{a}}$  outputting $1$, then with probability $1$ Alice's measurement device $O_{\uvbp{a}}$ will output $1$. Hence, once Bob has made this measurement, then the expectation value for Alice's measurement will be $\ev*{O_{\uvbp{a}}}=1.$ Thus, the expectation value for Alice's measurement device changes from $\frac{1}{2}$ to $1$ when Bob makes his measurement.
 
Now the problem\label{Copenhagenproblem} with Bob's ability to change the expectation value for Alice's measurement is not just that it violates Einstein's locality principle (specified on page \pageref{EinsteinLocalityPrinciple}), but also that Bob's ability to do this presupposes that he performs his measurement before Alice performs her measurement. But according to Einstein's theory of relativity, whoever performs their measurement first is going to depend on which inertial frame of reference one is in.\footnote{In special relativity, an inertial frame of reference is a spacetime coordinate system $(t, x, y, z)$ in which all objects which have no forces acting on them have trajectories that are straight lines. Thus, we can move to another inertial frame by moving to a reference frame with constant velocity $\vb{v}$ with respect to the first reference frame. In the case when $\vb{v}=(v, 0, 0)$, Einstein's theory of special relativity tells us that under such a “boost”, spacetime coordinates will transform as $(t,\vb{x})\rightarrow(t',x',y',z')=(\gamma\big(t - \frac{vx}{c^2}\big),\gamma(x-vt),y,z)$ where $c$ is the speed of light and $\gamma=\Big(\sqrt{1-\frac{v^2}{c^2}}\Big)^{-1}.$ } Thus, if we are moving at one velocity, it may appear that Alice makes her measurement first and so Alice causes the expectation value of Bob's measurement to be changed, whereas if we are moving at another velocity, it may appear that Bob makes his measurement first and causes the expectation of Alice's measurement to be changed. ***************************************** This suggests the expectation values for Alice and Bob's measuring devices will depend on which frame of reference we are in. However, Einstein's theory of relativity tells us that scalar quantities such as the expectation values for Alice and Bob's measuring devices should be independent of which inertial frame we are in. 

Now many physicists would be loath to reject Einstein's theory of relativity. At the same time, many physicists are also convinced by the violation of Bell's inequality that there are no hidden variables for the spin states of particles, and hence they are convinced that the Bell state is not just a description of someone's epistemic state: rather it is a complete physical description of two coupled fermionic particles with regard to their spins. The way many physicists seek to resolve this tension between Einstein's theory of relativity and the violation of Bell's inequality is to deny the Copenhagen interpretation of quantum physics so that there is no quantum state collapse. But if one denies that there is any quantum state collapse and denies that there are any hidden variables, the question then arises of how is one  meant to interpret the quantum state? 




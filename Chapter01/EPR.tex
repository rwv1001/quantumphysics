\section{The EPR Paradox}
Things get even more interesting when we perform Stern-Gerlach measurements on pairs of particles. We thus suppose there are two particles $q_A$ and $q_B$. Naively, one would expect that if the spin of $q_A$ were to be measured, then this would have no effect on any measurement outcome for $q_B$. This assumption is a special case of \textbf{Einstein's locality principle}: For two spatially separated systems $S_1$ and $S_2$,  the real factual situation of the system $S_2$ should be independent of what is done to the system $S_1$.\footnote{Einstein expressed this locality principle in his autobiographical notes: ``But on one supposition we should, in my opinion, absolutely hold fast: the real factual situation of the system $S_2$ is independent of what is done with the system $S_1$, which is spatially separated from the former.'' \cite[p. 85]{EinsteinLocality}.} If Einstein's locality principle holds we would be able to attribute a state $\ket{\psi}_A$ to particle $q_A$, and a state $\ket{\chi}_B$ to particle $q_B$. If an experimenter, Alice say, were to perform a Stern-Gerlach experiment on particle $q_A$ then by the Born rule, the probability Alice would find $q_A$ to be in state $\ket{\psi'}_A$ would be $|\ip{\psi'}{\psi}_A|^2$. Likewise, if another  experimenter, Bob say, measured particle $q_B$, then the probability Bob would find $q_B$ to be in state $\ket{\chi'}_B$ would be $|\ip{\chi'}{\chi}_B|^2$. 

An important principle in probability theory is that two events $X$ and $Y$ are \textbf{independent} if and only if 
$P(X,Y)=P(X)P(Y)$
where $P(X)$ is the probability that $X$ occurs, $P(Y)$ is the probability that $Y$ occurs, and $P(X,Y)$ is the probability that both $X$ and $Y$ both occur. Then the independence of Alice and Bob's measurement outcomes would imply that the joint probability of Alice finding $q_A$ to be in state $
\ket{\psi'}_A$, and Bob finding $q_B$ to be in state $\ket{\chi'}_B$, would be 
$$P_{A,B}(\psi',\chi'|\psi,\chi)=|\ip{\psi'}{\psi}_A|^2\times|\ip{\chi'}{\chi}_B|^2=|\ip{\psi'}{\psi}_A\ip{\chi'}{\chi}_B|^2.$$
This suggests that if we write $\ket{\psi}_A\ket{\chi}_B$ for the state of the composite system of both particles, then the bra-ket of $\ket{\psi'}_A\ket{\chi'}_B$ and $\ket{\psi}_A\ket{\chi}_B$ 
would be
$$\prescript{}{B}{ \bra*{\chi'}}\prescript{}{A}{\ip*{\psi'}{\psi}}_A\ket*{\chi}_B=\ip{\psi'}{\psi}_A\ip{\chi'}{\chi}_B.
$$
However, it turns out that there are physical situations in which Alice and Bob's measurements will not be independent. An important example of this phenomenon occurs when a particle known as a meson decays into two particles called muons.\footnote{See \cite[p. 242]{Sakurai}.} \textbf{Mesons} are particles composed of two quarks, \textbf{quarks} are the fundamental particles out of which protons, neutrons, and mesons are made. Both protons and neutrons are made up of three quarks. \textbf{Muons} are fundamental particles similar to electrons having negative charge, but muons have a much greater mass. Both muons and electrons have spin so like silver atoms, they will be deflected in one of two directions when travelling through a Stern-Gerlach apparatus. The spin of one of these muons can then be measured by Alice, and the spin of the other muon can be measured by Bob. But remarkably, when Alice and Bob measure the spin of their respective muons along the same axis, they will always get the opposite results from one another. 

Muons, electrons, and protons, are all examples of particles know as fermions. \textbf{Fermions} are particles that have the property that no two fermions in the same system can be in exactly the same state. Thus, when a meson decays into two muons, the two muons form a single system, and so since mesons are fermions, they cannot have the same spin.

Fermions are contrasted with \textbf{bosons} which are particles which can coexist in the same system and be in exactly the same state. \textbf{Photons} which are particles of light are examples of bosons, as are mesons.

Since the outcomes of Alice and Bob's measurements of their respective muons would be independent if the state of the composite system was $\ket{\psi}_A\ket{\chi}_B$, it follows that it is not possible to represent the composite state of the two muons as $\ket{\psi}_A\ket{\chi}_B$. However, if we know that if Alice finds her particle to be in spin state $\ket*{\uvbpm{a}}_A$, then Bob will find his particle to be in the state $\ket*{\uvbmp{a}}_B$. If we now consider the summation of states:
\begin{equation}\label{bell}
    \frac{1}{\sqrt{2}}(\ket*{\uvbp{a}}_A\ket*{\uvbm{a}}_B-\ket*{\uvbm{a}}_A\ket*{\uvbp{a}}_B).
\end{equation}
then it turns out that (\ref{bell}) is independent of the axis $\uvb{a}$. That is, for any other direction $\uvb{b}$ it can be shown that,
\begin{equation}\label{bellstate2}\frac{1}{\sqrt{2}}(\ket*{\uvbp{a}}_A\ket*{\uvbm{a}}_B-\ket*{\uvbm{a}}_A\ket*{\uvbp{a}}_B)=\frac{1}{\sqrt{2}}(\ket*{\uvbp{b}}_A\ket*{\uvbm{b}}_B-\ket*{\uvbm{b}}_A\ket*{\uvbp{b}}_B).\protect\footnotemark\end{equation}
\footnotetext{\label{bellstate2pf}To see this, using the transformation rules given in equation (\ref{spintrans}) we have 
\begin{align*}\frac{1}{\sqrt{2}}(\ket*{\uvbp{b}}\ket*{\uvbm{b}}&-\ket*{\uvbm{b}}\ket*{\uvbp{b}})\\
      &=\frac{1}{\sqrt{2}}((\cos(\theta/2)\ket*{\uvbp{a}}+\sin(\theta/2)\ket*{\uvbm{a}})(\cos(\theta/2)\ket*{\uvbm{a}}-\sin(\theta/2)\ket*{\uvbp{a}})\\
&\quad-(\cos(\theta/2)\ket*{\uvbm{a}}-\sin(\theta/2)\ket*{\uvbp{a}})(\cos(\theta/2)\ket*{\uvbp{a}}+\sin(\theta/2)\ket*{\uvbm{a}}))\\ 
&=\frac{1}{\sqrt{2}}(\cos(\theta/2)\ket*{\uvbp{a}}\cos(\theta/2)\ket*{\uvbm{a}}-\cos(\theta/2)\ket*{\uvbp{a}}\sin(\theta/2)\ket*{\uvbp{a}}\\
&\quad+\sin(\theta/2)\ket*{\uvbm{a}}\cos(\theta/2)\ket*{\uvbm{a}}-\sin(\theta/2)\ket*{\uvbm{a}}\sin(\theta/2)\ket*{\uvbp{a}}\\
&\quad-\cos(\theta/2)\ket*{\uvbm{a}}\cos(\theta/2)\ket*{\uvbp{a}}-\cos(\theta/2)\ket*{\uvbm{a}}\sin(\theta/2)\ket*{\uvbm{a}}\\
&\quad+\sin(\theta/2)\ket*{\uvbp{a}}\cos(\theta/2)\ket*{\uvbp{a}}+
\sin(\theta/2)\ket*{\uvbp{a}}\sin(\theta/2)\ket*{\uvbm{a}})\\
&=\frac{1}{\sqrt{2}}((\cos[2](\theta/2)+\sin[2](\theta/2))\ket*{\uvbp{a}}\ket*{\uvbm{a}}-(\cos[2](\theta/2)+\sin[2](\theta/2))\ket*{\uvbm{a}}\ket*{\uvbp{a}})\\
&=\frac{1}{\sqrt{2}}(\ket*{\uvbp{a}}\ket*{\uvbm{a}}-\ket*{\uvbm{a}}\ket*{\uvbp{a}}).
\end{align*}}
We refer to the state (\ref{bell}) as a \textbf{Bell state}.\footnote{By convention, the states $ \frac{1}{\sqrt{2}}(\ket*{\uvbp{a}}_A\ket*{\uvbm{a}}_B+\ket*{\uvbm{a}}_A\ket*{\uvbp{a}}_B)$, $ \frac{1}{\sqrt{2}}(\ket*{\uvbp{a}}_A\ket*{\uvbp{a}}_B-\ket*{\uvbm{a}}_A\ket*{\uvbm{a}}_B)$, and $ \frac{1}{\sqrt{2}}(\ket*{\uvbp{a}}_A\ket*{\uvbp{a}}_B+\ket*{\uvbm{a}}_A\ket*{\uvbm{a}}_B)$ are also referred to as Bell states.}

Now it turns out that states for the composite system of both particles can be added together in an analogous way to how single particle states are added as in equation (\ref{vectoradd1}) for example. However, when the composite system of both particles is described by such an addition, then in general we find that the measurement outcomes that Alice and Bob make will no longer be independent. 
 
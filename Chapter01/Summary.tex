
\section{Summary}
In this chapter, we have considered the EPR-Bohm paradox and the problem it raises of how to account for the mysterious correlations between the measurement outcomes of two observers measuring the spin properties of spin singlets. The question the EPR-Bohm paradox raises is how one can  account for these correlations in a way that is consistent with special relativity and the predictions of standard quantum theory.

The Copenhagen interpretation in which the act of observation causes the quantum state to collapse does not seem to be consistent with special relativity. Theories such as the GRW interpretation which posit the spontaneous collapse of quantum states make predictions that violate the predictions of standard quantum theory, and to date, there is no experimental evidence for such violations. 

If the proposed theory posits the existence of hidden variables in addition to the traditional quantum state of standard quantum theory, then it is not possible for the theory to satisfy both Parameter Independence (PI) and Outcome Independence (OI) because otherwise, Bell's inequality would hold in this theory, and this is not consistent with the violation of Bell's inequality in physical reality. A denial of PI would allow for the superluminal propagation of signals if one knew the hidden variables, and so without a compelling reason for why these hidden variables must be  unknown, denying PI does not sit easily with special relativity in which superluminal signalling is impossible. The denial of OI seems more promising, but this approach by itself is not sufficient to provide an adequate account of the mysterious correlations of the EPR-Bohm paradox since it does not address the thorny issue of what we mean by outcome or whether there is in fact any physical reality to outcomes at all. 

One theory that denies the reality of experimental outcomes is the many-worlds interpretation. In the next chapter, we will discuss the many-worlds interpretation in some detail, and we will see why it does not provide a satisfactory account of the correlations of the EPR-Bohm paradox. 
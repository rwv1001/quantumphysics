
\section{Summary}
In this chapter, we have considered the EPR-Bohm paradox and the problem it raises of how to account for the mysterious correlations between the measurement outcomes of two observers measuring the spin properties of spin singlets. The question the EPR-Bohm paradox raises is how one can  account for these correlations in a way that is consistent with special relativity and the predictions of standard quantum theory. 

{\interfootnotelinepenalty=10000 The Copenhagen interpretation in which the act of observation causes the quantum state to collapse does not seem to be consistent with special relativity. Theories such as the GRW interpretation which posit the spontaneous collapse of quantum states make predictions that violate the predictions of standard quantum theory, and to date, there is no experimental evidence for such  violations.\footnote{To see why the GRW interpretation implies a violation of standard quantum theory, we recall that the GRW interpretation supposes that there is a random spontaneous collapse of entangled states of particles which is frequent enough so that we don't have superpositions of macroscopic objects such as live and dead cats, but such that this collapse is infrequent enough to account for the correlations of that have been experimentally observed in EPR-type experiments. Nevertheless, GRW still violates the predictions of standard quantum theory, for according to standard quantum theory, if one particle of a spin singlet is sent to Alice and the other particle of a spin singlet is sent to Bob, then so long as there are no other particle interactions with the two particles of the spin singlet prior to Alice and Bob's measurement, then they will always obtain opposite results. This follows from how the state evolves according to the Schr\"{o}dinger equation. But now suppose GRW is true so that after a very long time, the spin singlet collapses to a state in which Alice's particle points up in the spin $z$-direction and Bob's particle points down in the spin $z$-direction, and furthermore suppose Alice and Bob perform their measurements along the $x$-axis. Then since the two particles are no longer entangled,  there will be a non-zero probability that Alice and Bob will measure their particles to have the same spin along the $x$-axis. This is not a possibility that standard quantum theory predicts. Granted, the probability of a collapse for any given entangled state is going to be extremely small, so it is very unlikely that Alice and Bob will obtain the same spin measurements, but if such collapses do occur as GRW predicts, then Alice and Bob can get the same spin measurements, and this will constitute a violation of the predictions of standard quantum theory which predicts Alice and Bob should always obtain different spin measurements. Perhaps GRW is right which would entail that standard quantum theory is slightly wrong, but in this dissertation, I will not be considering this possibility, and will instead assume that the predictions that standard quantum theory makes are always right.} 

If the proposed theory posits the existence of hidden variables in addition to the traditional quantum state of standard quantum theory, then it is not possible for the theory to satisfy both Parameter Independence (PI) and Outcome Independence (OI) because otherwise, Bell's inequality would hold in this theory, and this is not consistent with the violation of Bell's inequality in physical reality. A denial of PI would allow for the superluminal propagation of signals if one knew the hidden variables, and so without a compelling reason for why these hidden variables must be  unknown, denying PI does not sit easily with special relativity in which superluminal signalling is impossible. The denial of OI seems more promising, but this approach by itself is not sufficient to provide an adequate account of the mysterious correlations of the EPR-Bohm paradox since it does not address the thorny issue of what we mean by outcome or whether there is in fact any physical reality to outcomes at all. 

One theory that denies the reality of experimental outcomes is the many-worlds interpretation. In the next chapter, we will discuss the many-worlds interpretation in some detail, and we will see why it does not provide a satisfactory account of the correlations of the EPR-Bohm paradox. }
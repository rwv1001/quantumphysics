\section{The Copenhagen Interpretation}
       The experiment described in the previous section implies that the behavior of Alice's and Bob's particles can't be explained in terms of local hidden variables. But this experiment also calls into question the Copenhagen interpretation of quantum physics. To explain the Copenhagen interpretation and what is problematic about it, suppose Alice has a measurement device which we will denote by $\Lambda_{\uvbp{a}}$\label{Lambdaa} and which outputs the number 1 when her particle is detected at location $\uvbp{a}$, and outputs 0 when her particle is detected at location $\uvbm{a}$. Given that Alice knows that the state of both particles together is given by equation (\ref{bell}), she can work out the expectation value of her measurement $\ev*{\Lambda_{\uvbp{a}}}$ by summing up the product of each probability measurement outcome with the value of each measurement. This will give an expectation value of $\ev*{\Lambda_{\uvbp{a}}}=\frac{1}{2}\times 1 + \frac{1}{2}\times 0 = \frac{1}{2}.$ More generally, if Alice had a measuring device $\Lambda$ with $N$ measurement outcome values $o_1,\ldots,o_N$ and with respective probabilities $p_1,\ldots,p_N$ so that $\sum_{i=1}^N p_i=1,$ then the expectation $\ev*{\Lambda}$ would be given by the formula
      \begin{equation}\label{expectation}\ev*{\Lambda} =\sum_{i=1}^N p_io_i.
      \end{equation}
      Now given that there are no hidden variables and that equation (\ref{bell}) encodes everything that can be said about the spins of the two particle system, it is tempting to suppose that the expectation value $\ev*{\Lambda_{\uvbp{a}}}$ tells us something objective about the system rather than just something about Alice's state of knowledge about the system.  Given this assumption, there then arises the question of what happens when a measurement is made. According to the Copenhagen interpretation, when Bob makes his measurement, the quantum state collapses to a component of the quantum state corresponding to the measurement Bob makes. Thus, if Bob's measurement device $\Lambda_{\uvbm{a}}$ outputs the number $1$, (i.e. Bob's particle is detected at location $\uvbm{a}$), then the state of the combined system would change accordingly as:
      $$\frac{1}{\sqrt{2}}(\ket*{\uvbp{a}}_A\ket*{\uvbm{a}}_B-\ket*{\uvbm{a}}_A\ket*{\uvbp{a}}_B)\xrightarrow{\text{Collapse!!}}\ket*{\uvbp{a}}_A\ket*{\uvbm{a}}_B $$
      If Bob makes his measurement first with his measurement device $\Lambda_{\uvbm{a}}$  outputting $1$, then with probability $1$ Alice's measurement device $\Lambda_{\uvbp{a}}$ will output $1$. Hence, once Bob has made this measurement, then the expectation value for Alice's measurement will be $\ev*{\Lambda_{\uvbp{a}}}=1.$ Thus, the expectation value for Alice's measurement device changes from $\frac{1}{2}$ to $1$ when Bob makes his measurement.
       
      Now the problem\label{Copenhagenproblem} with this change in expectation value for Alice's measurement is that it will depend on whether Bob performs his measurement first or whether Alice performs her measurement first. But according to Einstein's theory of relativity, who performs their measurement first will depend on which inertial frame of reference one is in.\footnote{In special relativity, an inertial frame of reference is a spacetime coordinate system $(t, x, y, z)$ in which all objects which have no forces acting on them have trajectories that are straight lines. Thus, we can move to another inertial frame by moving to a reference frame with constant velocity $\vb{v}$ with respect to the first reference frame. In the case when $\vb{v}=(v, 0, 0)$, Einstein's theory of special relativity tells us that under such a “boost”, spacetime coordinates will transform as $(t,\vb{x})\rightarrow(t',x',y',z')=(\gamma\big(t - \frac{vx}{c^2}\big),\gamma(x-vt),y,z)$ where $c$ is the speed of light and $\gamma=\Big(\sqrt{1-\frac{v^2}{c^2}}\Big)^{-1}.$ } Thus, if we are moving at one velocity, it may appear that Alice makes her measurement first, whereas if we are moving at another velocity, it may appear that Bob makes his measurement first. This suggests the expectation values for Alice and Bob's measuring devices will depend on which frame of reference we are in. However, Einstein's theory of relativity tells us that scalar quantities such as the expectation values for Alice and Bob's measuring devices should be independent of which inertial frame we are in. 
      
      Now many physicists would be loath to reject Einstein's theory of relativity. At the same time, many physicists are also convinced by the violation of Bell's inequality that there are no hidden variables for the spin states of particles, and hence they are convinced that the Bell state is not just a description of someone's epistemic state: rather it is a complete physical description of two coupled fermionic particles with regard to their spins. The way many physicists seek to resolve this tension between Einstein's theory of relativity and the violation of Bell's inequality is to deny the Copenhagen interpretation of quantum physics so that there is no quantum state collapse. But if one denies that there is any quantum state collapse and denies that there are any hidden variables, the question then arises of how is one  meant to interpret the quantum state? 
     
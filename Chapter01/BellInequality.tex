 
      \section{Bell's Inequality}\label{BellSection}
      Now although a hidden variables interpretation seems rather intuitive, a problem arises when two spin particles are coupled together with each other. This problem is known as the \textbf{EPR paradox.}\footnote{For more details of this problem, see \cite[241-249]{Sakurai}.} To explain how this problem arises, we need to consider two identical fermionic particles that are coupled together. Fermions have the property that no two particles that are coupled together can be in exactly the same spin state. Thus, if we call our fermionic particles $q_A$ and $q_B$, and suppose particle $q_A$ was in the state $\ket*{\uvbp{a},\uvbp{b},\uvbm{c},\ldots}$, then on the assumption that making a measurement on particle $q_A$ has no effect on the state of particle $q_B$, it would follow that particle $q_B$ would be in the  state  $\ket*{\uvbm{a},\uvbm{b},\uvbp{c},\ldots}$. It is thus appropriate to refer to the spin directions describing $q_A$ and $q_B$ as \textbf{local hidden variables}. It seems very natural to assume these  hidden variables are local. This assumption is a special case of \textbf{Einstein's locality principle}: For two spatially separated systems $S_1$ and $S_2$,  the real factual situation of the system $S_2$ should be independent of what is done to the system $S_1$.\footnote{Einstein expressed this locality principle in his autobiographical notes: ``But on one supposition we should, in my opinion, absolutely hold fast: the real factual situation of the system $S_2$ is independent of what is done with the system $S_1$, which is spatially separated from the former.''\cite[p. 85]{EinsteinLocality}.} 
       
      We now suppose we have an experimental setup so that in each run of the experiment, we have two fermionic particles $q_A$ and $q_B$,  and particle $q_A$ is sent to Alice who measures $q_A$'s spin in a direction of her choosing, and particle $q_B$ is sent to Bob who measures $q_B$'s spin in a direction of his choosing. We assume that in each run of the experiment, Alice and Bob independently measure the spin of their particles along one of three possible directions $\uvb{a}$, $\uvb{b}$, and $\uvb{c}$, and that Einstein's locality principle holds. Furthermore, we assume that in each run of the experiment, the outcome of Alice's measurement will be statistically independent of any of the other measurement outcomes for different runs of the experiment, and for any of the three axes she measures along, she will get a spin up outcome or a spin down outcome with equal probability of $\frac{1}{2}.$ Likewise, we assume Bob's measurement outcomes are also similarly independent between different runs of the experiment. We also assume that the $8=2^3$ states $\ket*{\uvbpm{a},\uvbpm{b},\uvbpm{c}}_A$ exhaust all the possible states for Alice's particles that can be distinguished from one another by one of the three possible measurements she can make. Thus, Alice can distinguish between the $\ket*{\uvbp{a},\uvbp{b},\uvbp{c}}_A$-state and the $\ket*{\uvbp{a},\uvbp{b},\uvbm{c}}_A$-state by making a measurement along the $\uvb{c}$-axis, though if she happened to make her measurement along the $\uvb{a}$ or $\uvb{b}$-axis, she wouldn't be able to distinguish between these two states. But in principle, she can distinguish between these two states if she happens to make her measurement along the right axis, in this case the $\uvb{c}$-axis. We similarly assume the states $\ket*{\uvbpm{a},\uvbpm{b},\uvbpm{c}}_B$  exhaust all the possible states for Bob's particles that he can distinguish between, and we assume that if Alice and Bob measure the particle along the same axis, they will always obtain opposite results from one another. For instance, if Alice's particle is in state $\ket*{\uvbp{a},\uvbp{b},\uvbp{c}}_A$, then Bob's particle must be in state $\ket*{\uvbm{a},\uvbm{b},\uvbm{c}}_B$. Now suppose the experiment is run $N$ times for large $N$,\footnote{$N$ has to be large since a frequentist definition of probability is being assumed.} and let $N_i$ be the number of times particle $q_A$ is in the $i$th state so that\footnote{The notation $\sum_{i=1}^8 N_i$ is shorthand for $N_1+N_2+N_3+N_4+N_5+N_6+N_7+N_8$.} $N=\sum_{i=1}^8 N_i$ as shown in table \ref{hiddentable}.
      
      \begin{table}[ht]
      \caption{Spin-components of particles $q_A$ and $q_B$ in the hidden variable theory}
      \centering
      \begin{tabular}{c c c} 
      \\ 
      \hline
      \textbf{Population}& \textbf{Particle} $\bm{q_A}$ & \textbf{Particle} $\bm{q_B}$ \\ [0.5ex] 
      \hline
      $N_1$ & $\ket*{\uvbp{a},\uvbp{b},\uvbp{c}}_A$ & $\ket*{\uvbm{a},\uvbm{b},\uvbm{c}}_B$ \\ 
      
      $N_2$ & $\ket*{\uvbp{a},\uvbp{b},\uvbm{c}}_A$ & $\ket*{\uvbm{a},\uvbm{b},\uvbp{c}}_B $\\ 
      
      $N_3$ & $\ket*{\uvbp{a},\uvbm{b},\uvbp{c}}_A$ & $\ket*{\uvbm{a},\uvbp{b},\uvbm{c}}_B$ \\ 
      
      $N_4$ & $\ket*{\uvbp{a},\uvbm{b},\uvbm{c}}_A$ & $\ket*{\uvbm{a},\uvbp{b},\uvbp{c}}_B $\\ 
      
      $N_5$ & $\ket*{\uvbm{a},\uvbp{b},\uvbp{c}}_A$ & $\ket*{\uvbp{a},\uvbm{b},\uvbm{c}}_B$ \\ 
      
      $N_6$ & $\ket*{\uvbm{a},\uvbp{b},\uvbm{c}}_A$ & $\ket*{\uvbp{a},\uvbm{b},\uvbp{c}}_B$ \\ 
      
      $N_7$ & $\ket*{\uvbm{a},\uvbm{b},\uvbp{c}}_A$ & $\ket*{\uvbp{a},\uvbp{b},\uvbm{c} }_B$\\ 
      
      $N_8$ & $\ket*{\uvbm{a},\uvbm{b},\uvbm{c}}_A$ & $\ket*{\uvbp{a},\uvbp{b},\uvbp{c}}_B$ \\ 
      \hline
      \end{tabular}
      \label{hiddentable}
      \end{table}
       We define $P_{AB}(\uvbp{a};\uvbp{b})$ to be the probability that Alice measures particle $q_A$ to be at location $\uvbp{a}$ on her detection screen and Bob measures particle $q_B$ to be at location $\uvbp{b}$ on his detection screen. We similarly define the probabilities for all other combinations of detection locations. It is relatively easy to calculate all these probabilities in terms of the values $N_i$ from table \ref{hiddentable},\footnote{e.g.  $P_{AB}(\uvbp{a};\uvbp{b})=\frac{N_3+N_4}{N},\, P_{AB}(\uvbp{a};\uvbp{c})=\frac{N_2+N_4}{N},\,P_{AB}(\uvbp{c};\uvbp{b})=\frac{N_3+N_7}{N}$  } or alternatively by simply measuring the frequency of these different outcomes for where Alice and Bob detect their particles. Although the values of $N_i$ are unknown, on the assumption that there is a fact of the matter of which states in table \ref{hiddentable} obtain, and on the assumption that the states to which the $N_i$ correspond exhaust all the possible states for Alice's and Bob's particles, we can show that\footnote{This inequality follows since $P_{AB}(\uvbp{a};\uvbp{b})=\frac{N_3+N_4}{N}\leq \frac{N_2+N_4+N_3+N_7}{N}=P_{AB}(\uvbp{a};\uvbp{c})+P_{AB}(\uvbp{c};\uvbp{b})$.}
       \begin{equation}\label{bellinequality}
      P_{AB}(\uvbp{a};\uvbp{b})\leq P_{AB}(\uvbp{a};\uvbp{c})+P_{AB}(\uvbp{c};\uvbp{b}).
      \end{equation}
      This inequality is known as \textbf{Bell's inequality}, and it follows from Einstein's locality principle.  However, it turns out that when this experiment is actually performed, Bell's inequality is violated. Because of this violation, the most natural conclusion to draw is that it is wrong to suppose that there any hidden variables describing possible spin measurement outcomes. But it also turns out that this violation of Bell's inequality is entirely predictable if we assume that when the two particles $q_A$ and $q_B$ are coupled together, everything that can be said about their spins is encoded in the so called \textbf{Bell state}: 
      \begin{equation}\label{bell}
      \frac{1}{\sqrt{2}}(\ket*{\uvbp{a}}_A\ket*{\uvbm{a}}_B-\ket*{\uvbm{a}}_A\ket*{\uvbp{a}}_B).\end{equation}
      This state means that if both Alice and Bob measure their particles along the measurement axis $\uvb{a}$, then with probability $\frac{1}{2}$, Alice will detect her particle at location $\uvbp{a}$ and Bob will detect his particle at location $\uvbm{a}$. Likewise, with probability $\frac{1}{2}$, Alice will detect her particle at location $\uvbm{a}$ and Bob will detect his particle at location $\uvbp{a}$. Prima facie, it looks like the Bell state depends on the direction of $\uvb{a}$. However it can be shown that for any other direction $\uvb{b}$,
      \begin{equation}\label{bellstate2}\frac{1}{\sqrt{2}}(\ket*{\uvbp{a}}_A\ket*{\uvbm{a}}_B-\ket*{\uvbm{a}}_A\ket*{\uvbp{a}}_B)=\frac{1}{\sqrt{2}}(\ket*{\uvbp{b}}_A\ket*{\uvbm{b}}_B-\ket*{\uvbm{b}}_A\ket*{\uvbp{b}}_B).\protect\footnotemark\end{equation}
      \footnotetext{To see this, using the transformation rules given in equation (\ref{spintrans}) we have 
      \begin{align*}\frac{1}{\sqrt{2}}(\ket*{\uvbp{b}}\ket*{\uvbm{b}}-\ket*{\uvbm{b}}\ket*{\uvbp{b}})&=\frac{1}{\sqrt{2}}((\cos(\theta/2)\ket*{\uvbp{a}}+\sin(\theta/2)\ket*{\uvbm{a}})(\cos(\theta/2)\ket*{\uvbm{a}}-\sin(\theta/2)\ket*{\uvbp{a}})\\
      &\quad-(\cos(\theta/2)\ket*{\uvbm{a}}-\sin(\theta/2)\ket*{\uvbp{a}})(\cos(\theta/2)\ket*{\uvbp{a}}+\sin(\theta/2)\ket*{\uvbm{a}}))\\ 
      &=\frac{1}{\sqrt{2}}(\cos(\theta/2)\ket*{\uvbp{a}}\cos(\theta/2)\ket*{\uvbm{a}}-\cos(\theta/2)\ket*{\uvbp{a}}\sin(\theta/2)\ket*{\uvbp{a}}\\
      &\quad+\sin(\theta/2)\ket*{\uvbm{a}}\cos(\theta/2)\ket*{\uvbm{a}}-\sin(\theta/2)\ket*{\uvbm{a}}\sin(\theta/2)\ket*{\uvbp{a}}\\
      &\quad-\cos(\theta/2)\ket*{\uvbm{a}}\cos(\theta/2)\ket*{\uvbp{a}}-\cos(\theta/2)\ket*{\uvbm{a}}\sin(\theta/2)\ket*{\uvbm{a}}\\
      &\quad+\sin(\theta/2)\ket*{\uvbp{a}}\cos(\theta/2)\ket*{\uvbp{a}}+
      \sin(\theta/2)\ket*{\uvbp{a}}\sin(\theta/2)\ket*{\uvbm{a}})\\
      &=\frac{1}{\sqrt{2}}((\cos[2](\theta/2)+\sin[2](\theta/2))\ket*{\uvbp{a}}\ket*{\uvbm{a}}\\
      &\quad-(\cos[2](\theta/2)+\sin[2](\theta/2))\ket*{\uvbm{a}}\ket*{\uvbp{a}})\\
      &=\frac{1}{\sqrt{2}}(\ket*{\uvbp{a}}\ket*{\uvbm{a}}-\ket*{\uvbm{a}}\ket*{\uvbp{a}}).
      \end{align*}}Thus, without loss of generality we can write the joint state of both Alice and Bob's particles as in equation (\ref{bell}), from which it would follow that 
      \begin{equation*}
      P_{AB}(\uvbp{a};\uvbp{b})=\frac{1}{2}\sin^2(\theta/2)
      \end{equation*}
      where $\theta$ is the angle between the $\uvb{a}$-axis and $\uvb{b}$-axis.\footnote{To see why this is, let $P_A(\uvbp{a})$ be the probability that Alice would detect her particle at location $\uvbp{a}$ given that she is making a measurement along the $\uvb{a}$-axis, and let $P_{BA}(\uvbp{b}|\uvbp{a})$ be the probability that Bob will detect his particle at location $\uvbp{b}$ given that he is making a measurement along the $\uvb{b}$-axis and Alice has detected her particle at location $\uvbp{a}$. Given that the joint state of the particles is given by equation (\ref{bell}), $P_A(\uvbp{a})=\frac{1}{2}.$ But also note that if  Alice has detected her particle at location  
      $\uvbp{a}$, then Bob's particle must be in state $\ket*{\uvbm{a}}$. From the Born Rule (see page \pageref{bornrule}) and equation (\ref{spintrans1}) it follows that 
      $$P_{BA}(\uvbp{b}|\uvbp{a})= |\ip*{\uvbp{b}}{\uvbm{a}}|^2=\sin^2(\theta/2).$$ 
      Therefore, $$P_{AB}(\uvbp{a};\uvbp{b})= P_A(\uvbp{a})P_{BA}(\uvbp{b}|\uvbp{a})=\frac{1}{2}\sin^2(\theta/2).$$} Then taking the angle between the $\uvb{a}$ and $\uvb{b}$-axes to be $90^\circ$, and the $\uvb{c}$-axis to be at $45^\circ$ to both the $\uvb{a}$ and $\uvb{b}$-axes, we would find that $P_{AB}(\uvbp{a};\uvbp{b})=\frac{1}{4}$ and $P_{AB}(\uvbp{a};\uvbp{c})+P_{AB}(\uvbp{c};\uvbp{b})=0.1464\ldots$, and so Bell's inequality would be violated if we assumed that the probability of each outcome is determined by the Bell state  (\ref{bell}). 
      \section{The Copenhagen Interpretation}
       The experiment described in the previous section implies that the behavior of Alice's and Bob's particles can't be explained in terms of local hidden variables. But this experiment also calls into question the Copenhagen interpretation of quantum physics. To explain the Copenhagen interpretation and what is problematic about it, suppose Alice has a measurement device which we will denote by $\Lambda_{\uvbp{a}}$\label{Lambdaa} and which outputs the number 1 when her particle is detected at location $\uvbp{a}$, and outputs 0 when her particle is detected at location $\uvbm{a}$. Given that Alice knows that the state of both particles together is given by equation (\ref{bell}), she can work out the expectation value of her measurement $\ev*{\Lambda_{\uvbp{a}}}$ by summing up the product of each probability measurement outcome with the value of each measurement. This will give an expectation value of $\ev*{\Lambda_{\uvbp{a}}}=\frac{1}{2}\times 1 + \frac{1}{2}\times 0 = \frac{1}{2}.$ More generally, if Alice had a measuring device $\Lambda$ with $N$ measurement outcome values $\lambda_1,\ldots,\lambda_N$ and with respective probabilities $p_1,\ldots,p_N$ so that $\sum_{i=1}^N p_i=1,$ then the expectation $\ev*{\Lambda}$ would be given by the formula
      \begin{equation}\label{expectation}\ev*{\Lambda} =\sum_{i=1}^N p_i\lambda_i.
      \end{equation}
      Now given that there are no hidden variables and that equation (\ref{bell}) encodes everything that can be said about the spins of the two particle system, it is tempting to suppose that the expectation value $\ev*{\Lambda_{\uvbp{a}}}$ tells us something objective about the system rather than just something about Alice's state of knowledge about the system.  Given this assumption, there then arises the question of what happens when a measurement is made. According to the Copenhagen interpretation, when Bob makes his measurement, the quantum state collapses to a component of the quantum state corresponding to the measurement Bob makes. Thus, if Bob's measurement device $\Lambda_{\uvbm{a}}$ outputs the number $1$, (i.e. Bob's particle is detected at location $\uvbm{a}$), then the state of the combined system would change accordingly as:
      $$\frac{1}{\sqrt{2}}(\ket*{\uvbp{a}}_A\ket*{\uvbm{a}}_B-\ket*{\uvbm{a}}_A\ket*{\uvbp{a}}_B)\xrightarrow{\text{Collapse!!}}\ket*{\uvbp{a}}_A\ket*{\uvbm{a}}_B $$
      If Bob makes his measurement first with his measurement device $\Lambda_{\uvbm{a}}$  outputting $1$, then with probability $1$ Alice's measurement device $\Lambda_{\uvbp{a}}$ will output $1$. Hence, once Bob has made this measurement, then the expectation value for Alice's measurement will be $\ev*{\Lambda_{\uvbp{a}}}=1.$ Thus, the expectation value for Alice's measurement device changes from $\frac{1}{2}$ to $1$ when Bob makes his measurement.
       
      Now the problem\label{Copenhagenproblem} with this change in expectation value for Alice's measurement is that it will depend on whether Bob performs his measurement first or whether Alice performs her measurement first. But according to Einstein's theory of relativity, who performs their measurement first will depend on which inertial frame of reference one is in.\footnote{In Special Relativity, an inertial frame of reference is a space-time coordinate system $(t, x, y, z)$ in which all objects which have no forces acting on them have trajectories that are straight lines. Thus, we can move to another inertial frame by moving to a reference frame with constant velocity $\vb{v}$ with respect to the first reference frame. In the case when $\vb{v}=(v, 0, 0)$, Einstein's theory of Special Relativity tells us that under such a “boost”, space-time coordinates will transform as $(t,\vb{x})\rightarrow(t',x',y',z')=(\gamma\big(t - \frac{vx}{c^2}\big),\gamma(x-vt),y,z)$ where $c$ is the speed of light and $\gamma=\Big(\sqrt{1-\frac{v^2}{c^2}}\Big)^{-1}.$ } Thus, if we are moving at one velocity, it may appear that Alice makes her measurement first, whereas if we are moving at another velocity, it may appear that Bob makes his measurement first. This suggests the expectation values for Alice and Bob's measuring devices will depend on which frame of reference we are in. However, Einstein's theory of relativity tells us that scalar quantities such as the expectation values for Alice and Bob's measuring devices should be independent of which inertial frame we are in. 
      
      Now many physicists would be loath to reject Einstein's theory of relativity. At the same time, many physicists are also convinced by the violation of Bell's inequality that there are no hidden variables for the spin states of particles, and hence they are convinced that the Bell state is not just a description of someone's epistemic state: rather it is a complete physical description of two coupled fermionic particles with regard to their spins. The way many physicists seek to resolve this tension between Einstein's theory of relativity and the violation of Bell's inequality is to deny the Copenhagen interpretation of quantum physics so that there is no quantum state collapse. But if one denies that there is any quantum state collapse and denies that there are any hidden variables, the question then arises of how is one  meant to interpret the quantum state? 
     
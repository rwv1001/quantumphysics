 
    
    \section{Evaluating the Many-Worlds Interpretation}\label{manyworldsinterpretation2}
    Given the above account of the many-worlds hypothesis of quantum physics, it does seem understandable why physicists would find it so attractive. Although we can't specify an exact moment at which branching occurs, the idea of branching and of there being many worlds itself is not particularly mysterious. This can all be explained in terms of the dynamics of the system and the environment, and decoherence theory allows us to understand why the interference effects that are the hallmark of quantum physics generally disappear on the macroscopic level. 
    
    There are other advantages  of the many-worlds hypothesis besides these which we need not discuss here.\footnote{More details can be found in \cite{Schlosshauer} and \cite{joos2013decoherence}.} But for all the advantages of the many-worlds hypothesis, there is one fundamental problem, and that is its patent absurdity. It seems that we should be able to say whether a cat is alive or dead without having to say what state the rest of the universe is in. However, the many-worlds hypothesis suggests that for any subsystem of the universe, we will in general only be able to say what state it is in with respect to the state of the rest of the universe. For example, if the state $\mathcal{S}$ is the system constituting a cat-wise configuration of particles and $\mathcal{E}$ is the rest of the universe, then given that the composite system $\mathcal{U}=\mathcal{S}+\mathcal{E}$ is described by the state $$\ket{\Psi(t)}=\frac{1}{\sqrt{2}}\big(\ket{\text{Cat Alive}}_\mathcal{S}\ket{E_\text{Cat Alive}}_\mathcal{E}+\ket{\text{Cat Dead}}_\mathcal{S}\ket{E_\text{Cat Dead}}_\mathcal{E}\big),$$ then we are in no position to make an absolute matter of fact claim about the system $\mathcal{S}$ and say the cat is dead or the cat is alive. Rather we have to say with respect to the environment described by $\ket{E_\text{Cat Alive}}_\mathcal{E}$, the cat is alive, and with respect to the environment $\ket{E_\text{Cat Dead}}_\mathcal{E}$, the cat is dead. According to the many-worlds hypothesis, the branching into multiple worlds doesn't just occur in rare instances, such as in Schr\"{o}dinger's cat type experiments. On the contrary, branching is supposed to be happening all the time.  Now anyone who has a common sense understanding of science would say that science enables us to understand absolute matters of fact about subsystems of the universe and the principles that govern them. But if there really are no such matters of fact, then we have to abandon this common sense understanding of science. Intuitively, it also seems obvious that I can know I am alive without needing to know the state of the rest of the universe, but the many-worlds hypothesis does not allow me to make this absolute matter of fact claim. So from a common sense point of view, the many-worlds hypothesis really is absurd.
    
     Of course some hypotheses may initially seem absurd, but once the hypothesis has been fully explained, it can appear far more plausible. For instance, time dilation in Special Relativity might initially sound absurd to some people, but once one has a better grasp of Special Relativity and is open to the possibility that  systems moving close to the speed of light with respect to ourselves might have properties rather different to systems that move with much slower speeds, then Special Relativity doesn't seem absurd at all.  
     
     However, the many-worlds hypothesis as presented here is different in this regard since it is not hypothesizing about some extreme situation. It is hypothesizing about ordinary situations. And one can have a fairly good understanding of the many-worlds hypothesis and still find it absurd. Some people choose to embrace the absurdity and reject common sense. But throughout much of human history, when a hypothesis has entailed an absurd conclusion, reasonable people have usually thought it better to reject the hypothesis rather than embrace the absurdity. 
      
     But in rejecting a hypothesis as absurd, it doesn't mean that absolutely everything in the hypothesis needs to be rejected, for  a hypothesis might be formulated in terms of sub-hypotheses, some of which might be very plausible, in which case something of the original hypothesis might be salvageable. In the case of the many-worlds hypothesis, I believe it does have something that is salvageable, namely decoherence theory. In the next chapter I will consider Adrian Kent's one-world interpretation of quantum physics in which the basic ideas of decoherence theory remain intact.
    
    
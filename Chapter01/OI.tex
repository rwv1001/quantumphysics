\section{Outcome Independence\label{OISec}}

Although a PI violation can account for the violation of Bell's Inequality, this is not the only possible culprit to consider. Another assumption of Bell's Inequality that might be violated is \textbf{Outcome Independence}\index{Outcome Independence} (OI). Outcome independence is the assumption
\begin{equation}\label{OI}
P_{\lambda,x,y}(X,Y)=P_{A,\lambda,x,y}(X)\cdot P_{B,\lambda,x,y}(Y),
\end{equation}
where $P_{A,\lambda,x,y}(X)$ and $P_{B,\lambda,x,y}(Y)$ are defined in equations (\ref{PIone}) and (\ref{PItwo}) respectively.
So the difference between OI and PI is the following: with OI, given Alice and Bob's choice of measurements $x$ and $y$, and the hidden variable $\lambda$, Alice and Bob's measurement outcomes will be statistically independent from one another, whereas with PI, given Alice's choice of measurement $x$ and the hidden variable $\lambda$, whatever measurement choice Bob makes, this will have absolutely no effect on the probabilities of Alice's measurement outcomes, and similarly, Alice's choice of measurement will have absolutely no effect on the probabilities of Bob's measurement outcomes. 

Now we can see that if OI holds in any model which gives the same predictions as standard quantum theory when averaged over the hidden variables, then PI must be violated in such a model. For if both PI and OI hold, then for any measurement choices $\vb{\hat{a}},$ $\vb{\hat{b}},$ and $\vb{\hat{c}}$, and hidden variable $\lambda$, we have
\begin{equation}\label{OIPI1}
\begin{split}
 P_{\lambda,\bm{\hat{a}},\bm{\hat{c}}}&(\uvbp{a},\uvbp{c})=P_{A,\lambda,\bm{\hat{a}},\bm{\hat{c}}}(\uvbp{a})\cdot P_{B,\lambda,\bm{\hat{a}},\bm{\hat{c}}}(\uvbp{c})\\
 &= \Big( P_{\lambda,\bm{\hat{a}},\bm{\hat{c}}}(\uvbp{a},\uvbp{c})+ P_{\lambda,\bm{\hat{a}},\bm{\hat{c}}}(\uvbp{a},\uvbm{c})\Big)\cdot\Big( P_{\lambda,\bm{\hat{a}},\bm{\hat{c}}}(\uvbp{a},\uvbp{c})+ P_{\lambda,\bm{\hat{a}},\bm{\hat{c}}}(\uvbm{a},\uvbp{c})\Big)\\
 &= \Big( P_{\lambda,\bm{\hat{a}},\bm{\hat{c}}}(\uvbp{a},\uvbp{c})+ P_{\lambda,\bm{\hat{a}},\bm{\hat{c}}}(\uvbp{a},\uvbm{c})\Big)\cdot\Big( \cancelto{0}{P_{\lambda,\bm{\hat{c}},\bm{\hat{c}}}(\uvbp{c},\uvbp{c})}+ P_{\lambda,\bm{\hat{c}},\bm{\hat{c}}}(\uvbm{c},\uvbp{c})\Big)\\
 &=\Big( P_{\lambda,\bm{\hat{a}},\bm{\hat{b}}}(\uvbp{a},\uvbp{b})+ P_{\lambda,\bm{\hat{a}},\bm{\hat{b}}}(\uvbp{a},\uvbm{b})\Big)\cdot P_{\lambda,\bm{\hat{c}},\bm{\hat{c}}}(\uvbm{c},\uvbp{c})\\
 &\geq P_{\lambda,\bm{\hat{a}},\bm{\hat{b}}}(\uvbp{a},\uvbp{b})\cdot P_{\lambda,\bm{\hat{c}},\bm{\hat{c}}}(\uvbm{c},\uvbp{c}) 
\end{split}
\end{equation}
Similarly, we have
\begin{equation}\label{OIPI2}
\begin{split}
 P_{\lambda,\bm{\hat{c}},\bm{\hat{b}}}&(\uvbp{c},\uvbp{b})=P_{A,\lambda,\bm{\hat{c}},\bm{\hat{b}}}(\uvbp{c})\cdot P_{B,\lambda,\bm{\hat{c}},\bm{\hat{b}}}(\uvbp{b})\\
 &= \Big( P_{\lambda,\bm{\hat{c}},\bm{\hat{b}}}(\uvbp{c},\uvbp{b})+ P_{\lambda,\bm{\hat{c}},\bm{\hat{b}}}(\uvbp{c},\uvbm{b})\Big)\cdot\Big( P_{\lambda,\bm{\hat{c}},\bm{\hat{b}}}(\uvbp{c},\uvbp{b})+ P_{\lambda,\bm{\hat{c}},\bm{\hat{b}}}(\uvbm{c},\uvbp{b})\Big)\\
 &= \Big(  \cancelto{0}{P_{\lambda,\bm{\hat{c}},\bm{\hat{c}}}(\uvbp{c},\uvbp{c})}+ P_{\lambda,\bm{\hat{c}},\bm{\hat{c}}}(\uvbp{c},\uvbm{c})\Big)\cdot\Big(P_{\lambda,\bm{\hat{c}},\bm{\hat{b}}}(\uvbp{c},\uvbp{b})+ P_{\lambda,\bm{\hat{c}},\bm{\hat{b}}}(\uvbm{c},\uvbp{b})\Big)\\
 &=P_{\lambda,\bm{\hat{c}},\bm{\hat{c}}}(\uvbp{c},\uvbm{c})\cdot\Big( P_{\lambda,\bm{\hat{a}},\bm{\hat{b}}}(\uvbp{a},\uvbp{b})+ P_{\lambda,\bm{\hat{a}},\bm{\hat{b}}}(\uvbm{a},\uvbp{b})\Big)\\
 &\geq P_{\lambda,\bm{\hat{c}},\bm{\hat{c}}}(\uvbp{c},\uvbm{c})\cdot P_{\lambda,\bm{\hat{a}},\bm{\hat{b}}}(\uvbp{a},\uvbp{b}) .
\end{split}
\end{equation}
But since the hidden variable $\lambda$ is assumed to be independent of Alice and Bob's measurement, and since Alice and Bob will always get opposite results when they make the same choice of measurement, it follows that 
\begin{equation}\label{OIPI3}
P_{\lambda,\bm{\hat{c}},\bm{\hat{c}}}(\uvbp{c},\uvbm{c})+P_{\lambda,\bm{\hat{c}},\bm{\hat{c}}}(\uvbm{c},\uvbp{c})=1.
\end{equation}
Therefore, putting (\ref{OIPI1}), (\ref{OIPI2}), and (\ref{OIPI3}) together, we have
\begin{equation}
P_{\lambda,\bm{\hat{a}},\bm{\hat{c}}}(\uvbp{a},\uvbp{c})+P_{\lambda,\bm{\hat{c}},\bm{\hat{b}}}(\uvbp{c},\uvbp{b})\geq P_{\lambda,\bm{\hat{a}},\bm{\hat{b}}}(\uvbp{a},\uvbp{b}).
\end{equation}
We have thus proved that OI and PI together imply Bell's Inequality (\ref{bellinequality2}). But since Bell's Inequality does not hold in reality, it follows that if OI is always true, then PI must be violated.\label{OIPIproofend}


\label{OIdet}In the case of deterministic models in which there are outcomes,\footnote{For deterministic models in which there are outcomes, the probability of a particular outcome given a complete description of a system will be either $0$ or $1$. For models in which there are no outcomes, such as the many-worlds interpretation which will be discussed in the next chapter, we can't speak of outcome independence.} OI necessarily holds. To see why, we first note that for deterministic models, either $P_{\lambda,x,y}(X,Y)=1$ or $P_{\lambda,x,y}(X,Y)=0$. When $P_{\lambda,x,y}(X,Y)=1$, then by (\ref{PIone}), $P_{A, \lambda,x,y}(X)=1$, and by (\ref{PItwo}), $P_{B, \lambda,x,y}(Y)=1$, so (\ref{OI}) is seen to hold in this case. On the other hand, when $P_{\lambda,x,y}(X,Y)=0$,  if $P_{A, \lambda,x,y}(X)=1$, then by (\ref{PIone}), $P_{\lambda,x,y} (X,-Y)=1$ so that by (\ref{PItwo}), $P_{B, \lambda,x,y}(-Y)=1$, and hence $P_{B, \lambda,x,y}(Y)=0$ in which case (\ref{OI}) holds. And similarly, if $P_{B, \lambda,x,y}(Y)=1$, by (\ref{PItwo}), $P_{\lambda,x,y} (-X,Y)=1$ so that by (\ref{PIone}), $P_{A, \lambda,x,y}(-X)=1$, and hence $P_{A, \lambda,x,y}(X)=0$, so again (\ref{OI}) holds. And (\ref{OI}) obviously holds when $P_{A, \lambda,x,y}(X)=P_{B, \lambda,x,y}(Y)=0$. It therefore follows that OI holds in any deterministic model. In particular, OI holds under the pilot wave interpretation. 

When it comes to the Copenhagen interpretation of quantum physics, however, OI fails to hold. For instance, if $x=y=\vb{\hat{a}}$, then $P_{\lambda,\vb{\hat{a}},\vb{\hat{a}}}(\vb{\hat{a}+},\vb{\hat{a}+})=0,$ but $P_{A, \lambda,\vb{\hat{a}},\vb{\hat{a}}}(\vb{\hat{a}+})=P_{B, \lambda,\vb{\hat{a}},\vb{\hat{a}}}(\vb{\hat{a}+})=1/2.$\footnote{See footnote \ref{onehalf}. } Hence, OI fails. 


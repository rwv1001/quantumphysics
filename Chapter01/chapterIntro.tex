\begin{comment}
    \section{Introduction}
    \emph{}
    
    Common sense is very often underrated. This is especially so in the light of modern physics. Not infrequently, one hears people claim that modern physics shows us that reality is fundamentally weird and that we must discard our naïve common sense intuitions. Now perhaps these people are right, but if we are to accept their claims, they ought to have really compelling reasons. The usual response to an argument that results in weird or seemingly absurd conclusions is to question the argument's premises or examine whether the argument is logically valid. 
\end{comment}
    \chapter{The Appeal of the Many-World's Interpretation of Quantum Physics}
    In recent times, it has become increasingly common for popularizers of quantum physics to tell us that we need to let go of our naïve common sense understanding of reality. We're told we must replace this common sense understanding with something that at first seems very bizarre and counter-intuitive: a many-worlds account of reality. In this chapter, I will try to explain what is meant by this many-worlds account of reality and why some theoretical physicists find it so appealing.
    
    To begin with, I will give an overview of the Copenhagen interpretation and the hidden variables interpretation of quantum physics. This overview will be helpful since it will not only serve to introduce to the reader the notation of quantum theory, but it will also give us an opportunity to consider the problem with the Copenhagen interpretation and the hidden variables interpretation that the many-worlds interpretation seeks to resolve. 

    Many ideas in quantum physics are expressed in mathematical terms. I will do my best to avoid unnecessary mathematical jargon, but in order to explain the ideas of this thesis, a certain amount of mathematics is unavoidable. I will endeavor to explain all the mathematical terminology as I go along. However, there will be some sections which may be very challenging to readers who do not have a mathematics or physics background. These sections will be marked with and asterisk *.\label{asteriskmeaning}
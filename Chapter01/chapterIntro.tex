\begin{comment}
    \section{Introduction}
    \emph{}
    
    Common sense is very often underrated. This is especially so in the light of modern physics. Not infrequently, one hears people claim that modern physics shows us that reality is fundamentally weird and that we must discard our naïve common sense intuitions. Now perhaps these people are right, but if we are to accept their claims, they ought to have really compelling reasons. The usual response to an argument that results in weird or seemingly absurd conclusions is to question the argument's premises or examine whether the argument is logically valid. 
\end{comment}
    \chapter{Confronting the EPR-Bohm Paradox\label{BellChapter}}
   
    
    The central challenge that one-world interpretations of quantum physics must face is how to account for the mysterious correlations of the so-called EPR-Bohm paradox in a way that is consistent with Einstein's theory of special relativity.  In this chapter, no  prior knowledge of quantum theory will be assumed. I will therefore need to describe some key ideas of quantum theory, and I will do this in the context of the  Stern-Gerlach experiment. 
    I will then describe the EPR-Bohm paradox and the difficulty the traditional Copenhagen interpretation of quantum physics has in dealing with this paradox. A seemingly natural way to overcome this paradox is to supplement standard quantum theory with hidden variables. I will thus describe one way in which this can be done, and I will show how this leads to the remarkable inequality first derived by Bell. However, Bell's inequality is known to be experimentally violated. This means there must be something wrong with Bell's assumptions. I will therefore consider Shimony's analysis of the proof of Bell's inequality in which Shimony draws a distinction between parameter independence and outcome independence. Finally, following Butterfield, I will briefly explain why Shimony's analysis does not adequately resolve the EPR-Bohm Paradox. 


    
\begin{comment}
    \section{Introduction}
    \emph{}
    
    Common sense is very often underrated. This is especially so in the light of modern physics. Not infrequently, one hears people claim that modern physics shows us that reality is fundamentally weird and that we must discard our naïve common sense intuitions. Now perhaps these people are right, but if we are to accept their claims, they ought to have really compelling reasons. The usual response to an argument that results in weird or seemingly absurd conclusions is to question the argument's premises or examine whether the argument is logically valid. 
\end{comment}
    \chapter{Confronting the EPR-Bohm Paradox\label{BellChapter}}
    In recent times, it has become increasingly common for popularizers of quantum physics to tell us that we need to let go of our naïve common sense understanding of reality. We're told we must replace this common sense understanding with something that at first seems very bizarre and counter-intuitive: a many-worlds interpretation of reality. This is the idea that whenever there is quantum indeterminacy among several possibilities, then all these possibilities are realized, and the actualization of these possibilities can be extrapolated up to the macroscopic level. Thus, many-worlds advocates, when reflecting on the famous Schr\"{o}dinger's Cat thought experiment do not question the foundations of quantum mechanics on which the thought experiment is based, but rather they embrace the seemingly absurd conclusion of Schr\"{o}dinger that a cat could be both dead and alive. They thus speak of the cat being dead in one world and the cat being alive in another world that is just as real as the first. We will be examining the many-worlds interpretation of quantum physics in chapter \ref{kentchapter}, but in this chapter, we will consider why some people are so keen to reject a one-world interpretation of quantum physics.
    
    The central challenge that one-world interpretations of quantum physics must deal with is how to make sense of the experimental violation of Bell's inequality in a way that is consistent with Einstein's theory of special relativity.  In this chapter, no  prior knowledge of quantum theory will be assumed. We will therefore need to describe some key ideas of quantum theory, and this we will do in the context of the  Stern-Gerlach experiment. 
    We will then describe the EPR-Bohm paradox and the difficulty the traditional Copenhagen interpretation of quantum physics has in dealing with this paradox. A seemingly natural way to overcome this paradox is to supplement standard quantum theory with hidden variables. We will thus describe one way in which this can be done, and we will show how this leads to the remarkable inequality first derived by Bell. However, Bell's inequality is known to be experimentally violated. This means there must be something wrong with Bell's assumptions. We will therefore consider Shimony's analysis of the proof of Bell's inequality in which Shimony draws a distinction between parameter independence and outcome independence. Finally, following Butterfield, we will briefly explain why Shimony's analysis does not adequately resolve the EPR-Bohm Paradox. 


    
\section{Peaceful Coexistence of Special Relativity and Quantum Physics}
So far we've seen that PI holds in the Copenhagen interpretation (section \ref{PISec}), that OI holds in the pilot wave interpretation (section \ref{OISec}), and that due to the violation of Bell's inequality, it is not possible for both OI and PI to hold in a model that is consistent with physical reality (section \ref{OISec}). Nevertheless, as long as PI holds, the failure of OI does not enable Bob to send messages to Alice faster than light because Bob only has control over the measurement choice he makes rather than the outcome he observes. Assuming Bob's mental states have no effect on the measurement outcome, there is nothing he can do to influence his outcome, so although Alice will be able to work out Bob's measurement outcome if she already happens to know which choice of measurement he has made, she will not be able to work out which measurement Bob makes (or even whether Bob has made a measurement at all) by measuring the outcome of her particle. For Shimony,\footnote{See \cite[146-147]{Shimony86}.} this inability to send super-luminal messages between Alice and Bob when PI holds and OI is violated was deemed sufficient for the theories of standard quantum physics and special relativity to peacefully coexist. 

However, Butterfield is not satisfied with Shimony's solution to peaceful coexistence.\footnote{See \cite[p. 12]{Butterfield}.} Firstly, he notes that proofs of non-super-luminal signaling\footnote{e.g. see \cite[p. 113--116]{Redhead}; \cite[p. 139--140]{Hiley}} make no assumptions about spacetime locations. One would have thought that any proof that super-luminal signalling between two points is impossible would have to show that a signal cannot be transmitted from one point to the other in less time than the time it takes light to travel between the two points. But if nothing is said about the location of these two points or what is so special about the speed of light compared to the speed of any other particle, then there does not seem to be enough information in the premises to draw the desired conclusion that super-luminal signaling is impossible in quantum physics.

Secondly,  Butterfield notes that Shimony thinks peaceful coexistence of quantum physics and special relativity is guaranteed by the denial of OI and the acceptance of PI, but OI itself depends on the (often) rather vague notion of what an outcome really is. For instance, in the many-worlds interpretation that we will be describing in the next chapter, it is not clear that there are any outcomes at all. Rather, there is just a universal quantum state that tells us the probability of certain outcomes, if there were such things as outcomes -- it doesn't tell us that there really are any outcomes. But the notion of what an outcome is doesn't have to be vague. In the pilot wave interpretation of quantum physics, it is very clear what an outcome of an experiment is since all the particles have definite positions and momenta. Because of this, the pointers and displays of measuring devices which are made up of particles will have definite readouts which will correspond to the definite positions of particles being measured (assuming the measurement device is working properly). So unlike the many-worlds interpretation, measurements in the pilot wave interpretation have definite outcomes, and hence there is only a single world in the pilot wave interpretation of quantum physics. But as we've just seen, the problem with the pilot wave interpretation is the violation of PI.

Thus, a satisfactory account of the peaceful coexistence of quantum physics and special relativity requires an interpretation of quantum physics in which not only PI holds, but also an interpretation of quantum physics that has special relativity built into it (thus satisfying Butterfield's first objection), and in which we can make sense of what it means to be an outcome (thus satisfying Butterfield's second objection). 

To fully address Butterfield's first objection would require quantum field theory, and this would be beyond the scope of this dissertation. But a more modest aspiration that would go some way to address Butterfield's first objection would be to insist on an interpretation of quantum physics which has a clear notion of outcome and which also has a property known as Lorentz invariance. This provides a motivation for the consideration of Kent's theory of quantum physics that has this property of Lorentz invariance. But before we consider Kent's theory, we will examine the many-worlds interpretation of quantum physics discussing both its appeal and its drawbacks.
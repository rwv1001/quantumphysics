
\section{Parameter Dependence in the Bohmian Interpretation}\label{PI}
One model which exhibits PD is the \textbf{Bohmian interpretation}\index{Bohmian interpretation} of standard quantum theory. In this interpretation, it is assumed that at any instant of time $t$, the particles $q_A$ and $q_B$ will have definite positions $\vb{x}_A$ and $\vb{x}_B$ %
\nomenclature{$\vb{x}_A, \vb{x}_B$}{Exact locations of particles $q_A$ and $q_B$ in the Bohmian interpretation, \nomrefpage}%
and definite momenta $\vb{p}_A$ and $\vb{p}_B$ %
\nomenclature{$\vb{p}_A, \vb{p}_B$}{Exact momenta of particles $q_A$ and $q_B$ in the Bohmian interpretation, \nomrefpage}%
respectively. But in addition to the positions and momenta of the particles, it is also assumed that there is a so-called \textbf{pilot wave}\index{pilot wave} 
\begin{equation}\label{pilotwave}
\psi(\vb{x}_A, \vb{x}_B, t)=r(\vb{x}_A, \vb{x}_B, t)e^{i S(\vb{x}_A, \vb{x}_B, t)} 
\end{equation}where $r(\vb{x}_A, \vb{x}_B, t)>0$ %
\nomenclature{$r(\vb{x}_A, \vb{x}_B, t)$}{Modulus of the pilot wave $\psi(\vb{x}_A, \vb{x}_B, t)$, \nomrefpage}% 
is the modulus of $\psi(\vb{x}_A, \vb{x}_B, t)$, %
\nomenclature{$\psi(\vb{x}_A, \vb{x}_B, t)$}{Pilot wave, \nomrefpage}%
and the real-valued function\footnote{A \emph{function}\index{function} is a mathematical object that takes an element from a set called the function's \emph{domain}\index{domain} as its input, and returns an output from a set called the function's \emph{range}\index{range}. A \emph{real-valued function}\index{function!real-valued} is a function whose range is contained within the set of real numbers. If $x$ is in the domain of a function $f$, we denote the corresponding output of the function $f$ as $f(x)$. The $\sin$ function is an example of a real-valued function whose domain is the set of real numbers and whose range is the set of real numbers between $-1$ and $1$. Some functions need to  take several mathematical objects as input to produce an output. For example, the function $S$ in equation (\ref{pilotwave}) takes two vectors $\vb{x}_A$ and $\vb{x}_B$, and a real number $t$ as input from which it produces an output denoted by $S(\vb{x}_A, \vb{x}_B, t)$.} $S(\vb{x}_A, \vb{x}_B,t )$ is %
\nomenclature{$S(\vb{x}_A, \vb{x}_B,t )$}{Phase of the pilot wave $\psi(\vb{x}_A, \vb{x}_B, t)$, \nomrefpage}%
the complex phase\footnote{The \textbf{phase}\index{phase of a complex number} of a complex number $z$ is the angle $\theta$ (in radians) such that $z=r(\cos \theta + i \sin \theta)$ where $r>0$ is a real number called the \textbf{modulus}\index{modulus of a complex number}. Since $\cos \theta + i\sin \theta=e^{i \theta}$, %
\nomenclature{$e^{i \theta}$}{Defined to be equal to $\cos \theta + i\sin \theta$, \nomrefpage}%
it follows from (\ref{pilotwave}) that $r(\vb{x}_A, \vb{x}_B, t)>0$ is the modulus of $\psi(\vb{x}_A, \vb{x}_B, t)$, and $S(\vb{x}_A, \vb{x}_B,t )$ is the phase of $\psi(\vb{x}_A, \vb{x}_B, t)$.} of $\psi(\vb{x}_A, \vb{x}_B, t).$
The time evolution of the pilot wave is deterministically governed by the Schr\"{o}dinger equation, and the phase $S(\vb{x}_A, \vb{x}_B, t )$ relates the positions $\vb{x}_A$ and $\vb{x}_B$ to the momenta $\vb{p}_A$ and $\vb{p}_B$ via the gradient\footnote{Given a smooth function $F$ that maps a vector $\vb{x}$ to a real number $F(\vb{x})$, the \emph{gradient}\index{gradient} of $F$ at $\vb{x}$, written $\grad F(\vb{x})$ is the vector which points in the direction of the steepest ascent of $F$ at $\vb{x}$ and whose magnitude is determined by the rate of the steepest ascent at $\vb{x}$. When $F$ is a smooth function mapping two vectors $\vb{x}_A$ and $\vb{x}_B$ to a real number $F(\vb{x}_A, \vb{x}_B)$, we write $\grad_A F(\vb{x}_A, \vb{x}_B)$ for the gradient of $F$ at $\vb{x}_A$ when $F$ is just considered as a function of $\vb{x}_A$ with $\vb{x}_B$ fixed, and we write $\grad_B F(\vb{x}_A, \vb{x}_B)$ for the gradient of $F$ at $\vb{x}_B$ when $F$ is just considered as a function of $\vb{x}_B$ with $\vb{x}_A$ fixed.} of $S$:
\begin{equation}
\vb{p}_A=\grad_A{S}(\vb{x}_A, \vb{x}_B),\qquad
\vb{p}_B=\grad_B{S}(\vb{x}_A, \vb{x}_B).
\end{equation}
In other words, %
\nomenclature{$\grad_A{S}(\vb{x}_A, \vb{x}_B)$}{Gradient of the function ${S}(\vb{x}_A, \vb{x}_B)$ with respect to the variable $\vb{x}_A$, \nomrefpage}%
if we fix $\vb{x}_B$ %
\nomenclature{$\grad_B{S}(\vb{x}_A, \vb{x}_B)$}{Gradient of the function ${S}(\vb{x}_A, \vb{x}_B)$ with respect to the variable $\vb{x}_B$, \nomrefpage}%
and consider $S$ to be just a function of $\vb{x}_A$, then the momentum $\vb{p}_A$ is in the direction and has the magnitude of the steepest ascent of $S$ considered as a function of $\vb{x}_A$. The momentum $\vb{p}_B$ is determined similarly. 

In reality, we don't know the exact positions of all the particles, but based on what we know about an experimental setup, we can average over our uncertainty and recover exactly the same predictions that standard quantum theory would make.\footnote{See \cite{BohmDavid1952A} and \cite{BohmDavid1952B}.} So for instance, our knowledge of the experimental setup above should enable us  to know that $q_A$ and $q_B$ are contained within a bounded region $V,$ %
\nomenclature{$V$}{The region particles $q_A$ and $q_B$ are confined to in the Bohmian interpretation, \nomrefpage}%
and the experimental setup should also enable us to work out the probability  $p(V_i, V_j)$ %
\nomenclature{$p(V_i, V_j)$}{The probability given our knowledge that particle  $q_A$ belongs to the region $V_i$ and particle $q_B$ belongs to the region $V_j$, \nomrefpage}%
that particle $q_A$  will be in a region $V_i$, and particle $q_B$ will be in a region $V_j,$ where the $V_i$ are small non-overlapping regions such that $V=\bigcup_iV_i$. %
\nomenclature{$V_i$}{Small non-overlapping regions whose union is $V=\bigcup_iV_i$, \nomrefpage}%
If we are interested in some physical quantity $O(\vb{x}_A, \vb{x}_B)$ %
\nomenclature{$O(\vb{x}_A, \vb{x}_B)$}{A physical quantity dependent on the locations of the particles $q_A$ and $q_B$ in the Bohmian interpretation, \nomrefpage}%
that depends on the positions $\vb{x}_A$ and $\vb{x}_B$ of the two particles, then when the regions $V_i$ are  sufficiently small so that almost everywhere,\footnote{I am using the expression ``almost everywhere'' in a measure theoretic sense to mean that the set of all points where discontinuities occur has measure $0$. The motivation for this qualification is that whilst it might seem reasonable to suppose that there could be physical quantities that are discontinuous in some places, it doesn't seem reasonable to suppose that there are physical quantities that could be pathologically discontinuous, e.g. we wouldn't expect there to be a physical quantity that took the value of $1$ unit when all the coordinates of a location were rational and a value of $2$ units when one of the coordinates was irrational. Given this qualification about $O(\vb{x}_A,\vb{x}_B)$ being continuous almost everywhere, then even though $O(\vb{x}_A,\vb{x}_B)$ could be discontinuous at some points so that there could be some  $V_i\times V_j$ cells in which $O(\vb{x}_A,\vb{x}_B)$ varied non-negligibly for $(\vb{x}_A,\vb{x}_B)\in V_i\times V_j$, it would still be the case that the sum of the $p(V_i, V_j)\max_{(\vb{x}_i,\vb{x}_j)\in V_i\times V_j}|O(\vb{x}_i,\vb{x}_j)|$ terms for such cells would be negligible. This follows from the Lebesgue-Vitali theorem that states that any bounded function on a compact interval is Riemann integrable if and only if it is continuous almost everywhere. See \cite{wiki:Riemann}.} $O(\vb{x}_i, \vb{x}_j)$ varies negligibly for any $\vb{x}_i\in V_i$ and $\vb{x}_j\in V_j$,  the average value (also known as the \textbf{expectation value}\index{expectation value!pilot wave})
\begin{equation}\label{bohmconsistency}
\ev{O}=\sum_{i,j}p(V_i, V_j)O(\vb{x}_i, \vb{x}_j)
\end{equation}
calculated %
\nomenclature{$\ev{O}$}{Expectation value of the physical quantity $O(\vb{x}_A, \vb{x}_B)$ in the Bohmian interpretation \nomrefpage}%
in the Bohmian interpretation turns out to be the same as the expectation value\footnote{See (\ref{expectation2}) for the definition of the expectation value of an observable in standard quantum theory.} for $O$ predicted by standard quantum theory.\footnote{In this explanation, I've refrained from using measure theory, but basically this explanation is saying that when we construct a measure $\mu$ on $V\times V$ based on our knowledge of the experimental setup,  $\int_{V\times V}O(\vb{x}_i, \vb{x}_j)\dd\mu$ will be the same as the expectation value for $O$ predicted by standard quantum theory. }

To see why PI fails to hold in the Bohmian interpretation, we first note that since the Bohmian interpretation makes the same predictions as standard quantum theory when averaged over all the hidden variables, the violation of Bell's inequality (\ref{bellinequality}) implies there must be some hidden variable $\lambda$ and choices of measurement directions $\bm{\hat{a}}$, $\bm{\hat{b}}$, and $\bm{\hat{c}}$ such that  %
\nomenclature{$P_{\lambda,\bm{\hat{a}},\bm{\hat{b}}}(\uvbp{a},\uvbp{b})$}{The probability particle $q_A$ is measured to be in the $\ket*{\uvbp{a}}$-state and particle $q_B$ is measured to be in the $\ket*{\uvbp{b}}$-state given the complete state $\lambda$ of both particles, \nomrefpage}%
 \begin{equation}\label{bellinequality2}
P_{\lambda,\bm{\hat{a}},\bm{\hat{b}}}(\uvbp{a},\uvbp{b})> P_{\lambda,\bm{\hat{a}},\bm{\hat{c}}}(\uvbp{a},\uvbp{c})+P_{\lambda,\bm{\hat{c}},\bm{\hat{b}}}(\uvbp{c},\uvbp{b}).\protect\footnotemark
\end{equation}\footnotetext{To see why this is, we consider the observable $O_{\uvbp{a},\uvbp{b}}(\vb{x}_A,\vb{x}_B)$ which returns $1$ if and only if the particle $q_A$ will end up at location $\uvbp{a}$ and particle $q_B$ will end up at location $\uvbp{b}$ as determined by the pilot wave, and otherwise  $O_{\uvbp{a},\uvbp{b}}(\vb{x}_A,\vb{x}_B)$ returns $0$. Then by (\ref{bohmconsistency}), 
\begin{equation}\label{bohmconsistency2}
P_{AB}(\uvbp{a},\uvbp{b})=\sum_{\lambda}p_{\lambda}P_{\lambda,\bm{\hat{a}},\bm{\hat{b}}}(\uvbp{a},\uvbp{b})
\end{equation} 
where $P_{AB}(\uvbp{a},\uvbp{b})$ is the probability described in section \ref{hiddenbellsection}, $\lambda$ ranges over the pairs of indices $(i,j)$ corresponding to the cell $V_i\times V_j$, $p_{\lambda}=p(V_i,V_j)$ for $\lambda=(i,j)$, and we have used the fact that  $P_{\lambda,\bm{\hat{a}},\bm{\hat{b}}}(\uvbp{a},\uvbp{b})=O_{\uvbp{a},\uvbp{b}}(\vb{x}_i,\vb{x}_j)$ for almost all $(\vb{x}_i,\vb{x}_j)\in V_i\times V_j$. Now if (\ref{bellinequality2}) is false, then for all $\lambda$, $\bm{\hat{a}}$, and $\bm{\hat{b}}$,  
\begin{equation}\label{falsebellinequality2}
P_{\lambda,\bm{\hat{a}},\bm{\hat{b}}}(\uvbp{a},\uvbp{b})\leq P_{\lambda,\bm{\hat{a}},\bm{\hat{c}}}(\uvbp{a},\uvbp{c})+P_{\lambda,\bm{\hat{c}},\bm{\hat{b}}}(\uvbp{c},\uvbp{b})
\end{equation} Therefore, since $p_{\lambda}\geq 0$, it follows that 
\begin{equation}\label{falsebellinequality3}
\sum_\lambda p_{\lambda}P_{\lambda,\bm{\hat{a}},\bm{\hat{b}}}(\uvbp{a},\uvbp{b})\leq \sum_\lambda p_{\lambda}P_{\lambda,\bm{\hat{a}},\bm{\hat{c}}}(\uvbp{a},\uvbp{c})+\sum_\lambda p_{\lambda} P_{\lambda,\bm{\hat{c}},\bm{\hat{b}}}(\uvbp{c},\uvbp{b}).
\end{equation} 
It would therefore follow from (\ref{falsebellinequality3}) and  (\ref{bohmconsistency2}) that Bell's inequality (\ref{bellinequality}) would hold. But since (\ref{bellinequality}) is experimentally violated, it is not the case that (\ref{falsebellinequality2}) holds for all $\lambda$, $\bm{\hat{a}}$, and $\bm{\hat{b}}$. Hence, there must be some $\lambda$, $\bm{\hat{a}}$, and $\bm{\hat{b}}$,  for which (\ref{bellinequality2}) holds.}Since physics in the Bohmian interpretation is deterministic, probabilities must be either $0$ or $1$. Therefore, the only way (\ref{bellinequality2}) can be satisfied is for
 \begin{align}
P_{\lambda,\bm{\hat{a}},\bm{\hat{b}}}(\uvbp{a},\uvbp{b})&=1, \label{PDproof1}\\
P_{\lambda,\bm{\hat{a}},\bm{\hat{c}}}(\uvbp{a},\uvbp{c})&=0,\label{PDproof2}\\
P_{\lambda,\bm{\hat{c}},\bm{\hat{b}}}(\uvbp{c},\uvbp{b})&=0. \label{PDproof3}
 \end{align}
 \vspace{\alignspace pt}\newline
 We suppose that PI holds, and we will try to arrive at a contradiction.\footnote{For another proof of this result, see \cite{BellJ.S.1964OtEP}.} If both Alice and Bob make their measurement in the $\bm{\hat{c}}$-direction, there are two possibilities: either Alice measures $q_A$ to be in the state $\uvbm{c}$ and Bob measures $q_B$ to be in the state $\uvbp{c}$, or Alice measures $q_A$ to be in the state $\uvbp{c}$ and Bob measures $q_B$ to be in the state $\uvbm{c}.$ So expressed in terms of probabilities, these two possibilities are equivalent to either 
 \begin{equation}\label{PDproofcase1}
 P_{\lambda,\bm{\hat{c}},\bm{\hat{c}}}(\uvbm{c},\uvbp{c})=1 \qquad\text{and}\qquad  P_{\lambda,\bm{\hat{c}},\bm{\hat{c}}}(\uvbp{c},\uvbm{c})=0,
 \end{equation}
 or 
  \begin{equation}\label{PDproofcase2}
 P_{\lambda,\bm{\hat{c}},\bm{\hat{c}}}(\uvbp{c},\uvbm{c})=1 \qquad\text{and}\qquad P_{\lambda,\bm{\hat{c}},\bm{\hat{c}}}(\uvbm{c},\uvbp{c})=0.
 \end{equation}
 It will be sufficient to show that (\ref{PDproofcase1}) leads to a contradiction, since then we can just swap Alice and Bob to show that (\ref{PDproofcase2}) also leads to a contradiction. So note that
\begin{equation}
 P_{\lambda,\bm{\hat{c}},\bm{\hat{c}}}(\uvbp{c},\uvbm{c})+P_{\lambda,\bm{\hat{c}},\bm{\hat{c}}}(\uvbm{c},\uvbm{c})=0.
\end{equation}
 Therefore, since we are assuming PI, 
 \begin{equation}
 P_{\lambda,\bm{\hat{a}},\bm{\hat{c}}}(\uvbp{a},\uvbm{c})+P_{\lambda,\bm{\hat{a}},\bm{\hat{c}}}(\uvbm{a},\uvbm{c})=0.
\end{equation}
 In particular, 
  \begin{equation}\label{PDproof4}
  P_{\lambda,\bm{\hat{a}},\bm{\hat{c}}}(\uvbp{a},\uvbm{c})=0.
  \end{equation}
  But by (\ref{PDproof1}), we know that 
  \begin{equation}
  P_{\lambda,\bm{\hat{a}},\bm{\hat{b}}}(\uvbp{a},\uvbp{b})+P_{\lambda,\bm{\hat{a}},\bm{\hat{b}}}(\uvbp{a},\uvbm{b})=1,
  \end{equation}
  so using this together with PI, we must have
    \begin{equation}\label{PDproof5}
  P_{\lambda,\bm{\hat{a}},\bm{\hat{c}}}(\uvbp{a},\uvbp{c})+P_{\lambda,\bm{\hat{a}},\bm{\hat{c}}}(\uvbp{a},\uvbm{c})=1.
  \end{equation}
 But by (\ref{PDproof2}) and (\ref{PDproof4})
\begin{equation}\label{PDproof6}
  P_{\lambda,\bm{\hat{a}},\bm{\hat{c}}}(\uvbp{a},\uvbp{c})+P_{\lambda,\bm{\hat{a}},\bm{\hat{c}}}(\uvbp{a},\uvbm{c})=0.
\end{equation}
Since (\ref{PDproof5}) contradicts (\ref{PDproof6}), the assumption (\ref{PDproofcase1}) must be false if PI is to hold. Similarly, we will find that (\ref{PDproofcase2}) must be false if PI is to hold. So we can only conclude that PI fails to hold in the Bohmian interpretation.   But we can conclude even more than that: any deterministic hidden variable model that gives the same predictions as standard quantum theory when averaged over the hidden variables must violate PI.\label{PIdeterminism} 

  





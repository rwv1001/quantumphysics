\section{Hidden Variables and Bell's Inequality\label{hiddenbellsection}}
Given the problem with the Copenhagen interpretation that the EPR-Bohm paradox and Schr\"{o}dinger's Cat highlight, it is tempting to suppose that states such as $\ket*{\uvbp{a}}$ and $\ket*{\uvbm{a}}$ merely represents our limited knowledge of a more complete physical state that would also include a specification of the particle's spin state along other axes besides the $\uvb{a}$-axis. If we were to make this supposition, there would be a fact of the matter, albeit unknown to us, concerning what spin state the particle would be found to be in  were we to measure its spin along some $\uvb{b}$-axis for $\uvb{b}\neq\uvb{a}$. And  even though we might decide not to measure the spin of the particle along the $\uvb{b}$-axis, there would still be this hidden fact about the particle's spin in this direction. Furthermore, since there would be no reason to suppose there was anything special about the $\uvb{a}$ or $\uvb{b}$-axes, it would then be reasonable to suppose that there were hidden facts about what spin direction the particle would be found to be in for every possible axis orientation. This would mean that a complete description of the particle's spin state would require an infinite list of outcomes for all the possible orientations we could configure the magnetic field of our Stern-Gerlach apparatus to be in. For example, the complete spin state of a particle which according to our limited knowledge was in the $\ket*{\uvbp{b}}$-state could be depicted as $\ket*{\uvbp{a},\uvbp{b},\ldots}$ or  $\ket*{\uvbm{a},\uvbp{b},\ldots}$, etc. where the ellipses would range over one of the two possible measurement outcomes for every other magnetic field orientation. However, because we would never in practice be able to perform all these experiments, and since only one such experiment would be needed to alter this infinite list,\footnote{In other words, it is assumed that directly measuring the particle will involve perturbing it so that its state will change.} nearly all of the entries in this infinite list would remain forever hidden. Hence, this would be an example of a \textbf{hidden variables}\index{hidden variables} interpretation of quantum theory.  Moreover, if we're assuming Einstein's locality principle, it follows that any changes in these hidden spin outcomes for possible measurements of a specific particle can't affect the hidden variables of any other spatially separated localized particles. Therefore, given Einstein's locality principle, it is appropriate to refer to these hidden variables as \textbf{local}\index{hidden variables!local} hidden variables.

Now although a local hidden variables theory seems rather intuitive, in 1964, John Bell derived an inequality based on the local hidden variables theory just described.\footnote{In providing a derivation of Bell's inequality, we will follow \cite[241-249]{Sakurai}.} Moreover, it is now known that Bell's inequality can be violated experimentally.\footnote{Aspect, Clauser and Zeilinger received the 2022 physics Nobel Prize for establishing the experimental violation of Bell's inequality. See \cite{Nobel2022}.} This doesn't mean we must abandon all hidden variables theories. Rather it just means we need to abandon local hidden variables theories like the one Bell used in deriving his inequality.  It is nevertheless somewhat paradoxical that local hidden variables cannot account for the correlations between the measurement outcomes of spin singlets in the EPR-Bohm paradox. Local hidden variables were meant to resolve the EPR-Bohm paradox, and so the violation of Bell's inequality really heightens this paradox. Therefore, any satisfactory interpretation of quantum physics has to face up to and resolve this paradox by offering a suitable alternative to a local hidden variables theory. 

In order to describe Bell's inequality, we again consider two experimenters Alice and Bob making spin measurements on a spin singlet as described in section \ref{eprsec}. So in each run of the experiment, a spin singlet consisting of two particles $q_A$ and $q_B$ will be generated with particle $q_A$ being sent to Alice who measures $q_A$'s spin in a direction of her choosing, and with particle $q_B$ being sent to Bob who measures $q_B$'s spin in a direction of his choosing.  But now, instead of describing the two particles by the state $\ket*{\Psi_{\text{Bell}}}$, we describe particles $q_A$ and $q_B$ in terms of all the spin outcomes one would obtain for every possible measurement axis in such a way that if $q_A$ is spin up with respect to an axis $\uvb{a}$, then $q_B$ would be spin down with respect to this axis. For example, if the complete spin state for $q_A$ was given by $\ket*{\Psi_A}=\ket*{\uvbp{a},\uvbp{b},\uvbm{c},\ldots}_A$, then the complete spin state for $q_B$ would be given by $\ket*{\Psi_B}=\ket*{\uvbm{a},\uvbm{b},\uvbp{c},\ldots}_B$.

We also assume that in each run of the experiment, Alice and Bob independently measure the spin of their particles along one of three possible directions $\uvb{a}$, $\uvb{b}$, and $\uvb{c}$, and that Einstein's locality principle holds. Furthermore, we assume that in each run of the experiment, the outcome of Alice's measurement will be statistically independent of any of the other measurement outcomes for different runs of the experiment, and for any of the three axes she measures along, she will get a spin up outcome or a spin down outcome with equal probability of $\frac{1}{2}.$ Likewise, we assume Bob's measurement outcomes are also similarly independent between different runs of the experiment. We also assume that the $8=2^3$ states $\ket*{\uvbpm{a},\uvbpm{b},\uvbpm{c}}_A$ exhaust all the possible states for Alice's particles that can be distinguished from one another by making one of the three possible measurement choices available. Thus, Alice can distinguish between the $\ket*{\uvbp{a},\uvbp{b},\uvbp{c}}_A$-state and the $\ket*{\uvbp{a},\uvbp{b},\uvbm{c}}_A$-state by making a measurement along the $\uvb{c}$-axis, though if she happened to make her measurement along the $\uvb{a}$ or $\uvb{b}$-axis, she wouldn't be able to distinguish between these two states. But in principle, she can distinguish between these two states if she happens to make her measurement along the right axis, in this case the $\uvb{c}$-axis. We similarly assume the states $\ket*{\uvbpm{a},\uvbpm{b},\uvbpm{c}}_B$  exhaust all the possible states for Bob's particles that he can distinguish between, and we assume that if Alice and Bob measure the particle along the same axis, they will always obtain opposite results from one another. For instance, if Alice's particle is in state $\ket*{\uvbp{a},\uvbp{b},\uvbp{c}}_A$, then Bob's particle must be in state $\ket*{\uvbm{a},\uvbm{b},\uvbm{c}}_B$. Now suppose the experiment is run $N$ times for large $N$,\footnote{$N$ has to be large since a frequentist definition of probability is being assumed.} and let $N_i$ be the number of times particle $q_A$ is in the $i$th state so that\footnote{The notation $\sum_{i=1}^8 N_i$ is shorthand for $N_1+N_2+N_3+N_4+N_5+N_6+N_7+N_8$.} $N=\sum_{i=1}^8 N_i$ as shown in table \ref{hiddentable}.
      
      \begin{table}[ht]
      \caption{Spin-components of particles $q_A$ and $q_B$ in the hidden-variable theory}
      \centering
      \begin{tabular}{c c c} 
      \\ 
      \hline
      \textbf{Population}& \textbf{Particle} $\bm{q_A}$ & \textbf{Particle} $\bm{q_B}$ \\ [0.5ex] 
      \hline
      $N_1$ & $\ket*{\uvbp{a},\uvbp{b},\uvbp{c}}_A$ & $\ket*{\uvbm{a},\uvbm{b},\uvbm{c}}_B$ \\ 
      
      $N_2$ & $\ket*{\uvbp{a},\uvbp{b},\uvbm{c}}_A$ & $\ket*{\uvbm{a},\uvbm{b},\uvbp{c}}_B $\\ 
      
      $N_3$ & $\ket*{\uvbp{a},\uvbm{b},\uvbp{c}}_A$ & $\ket*{\uvbm{a},\uvbp{b},\uvbm{c}}_B$ \\ 
      
      $N_4$ & $\ket*{\uvbp{a},\uvbm{b},\uvbm{c}}_A$ & $\ket*{\uvbm{a},\uvbp{b},\uvbp{c}}_B $\\ 
      
      $N_5$ & $\ket*{\uvbm{a},\uvbp{b},\uvbp{c}}_A$ & $\ket*{\uvbp{a},\uvbm{b},\uvbm{c}}_B$ \\ 
      
      $N_6$ & $\ket*{\uvbm{a},\uvbp{b},\uvbm{c}}_A$ & $\ket*{\uvbp{a},\uvbm{b},\uvbp{c}}_B$ \\ 
      
      $N_7$ & $\ket*{\uvbm{a},\uvbm{b},\uvbp{c}}_A$ & $\ket*{\uvbp{a},\uvbp{b},\uvbm{c} }_B$\\ 
      
      $N_8$ & $\ket*{\uvbm{a},\uvbm{b},\uvbm{c}}_A$ & $\ket*{\uvbp{a},\uvbp{b},\uvbp{c}}_B$ \\ 
      \hline
      \end{tabular}
      \label{hiddentable}
      \end{table}
As in section \ref{eprsec}, we define $P_{AB}(\uvbp{a},\uvbp{b})$ to be the probability that Alice measures particle $q_A$ to be at location $\uvbp{a}$ on her detection screen and Bob measures particle $q_B$ to be at location $\uvbp{b}$ on his detection screen. We similarly define the probabilities for all other combinations of detection locations. It is relatively easy to calculate all these probabilities in terms of the values $N_i$ from table \ref{hiddentable},\footnote{e.g.  $P_{AB}(\uvbp{a},\uvbp{b})=\frac{N_3+N_4}{N},\, P_{AB}(\uvbp{a},\uvbp{c})=\frac{N_2+N_4}{N},\,P_{AB}(\uvbp{c},\uvbp{b})=\frac{N_3+N_7}{N}$  } or alternatively by simply measuring the frequency of these different outcomes for where Alice and Bob detect their particles. Note that the values of $N_i$ will be unknown, but on the assumption that there is a fact of the matter of which states in table \ref{hiddentable} obtain, and on the assumption that the states to which the $N_i$ correspond exhaust all the possible states for Alice's and Bob's particles, we can show that\footnote{This inequality follows since 
$$P_{AB}(\uvbp{a},\uvbp{b})=\frac{N_3+N_4}{N}\leq \frac{N_2+N_4+N_3+N_7}{N}=P_{AB}(\uvbp{a},\uvbp{c})+P_{AB}(\uvbp{c},\uvbp{b}).$$}
       \begin{equation}\label{bellinequality}
      P_{AB}(\uvbp{a},\uvbp{b})\leq P_{AB}(\uvbp{a},\uvbp{c})+P_{AB}(\uvbp{c},\uvbp{b}).
      \end{equation}
This inequality is known as \textbf{Bell's inequality}\index{Bell's inequality}, and it follows from Einstein's locality principle.  However, as already mentioned, when this experiment is actually performed, we can choose the three axes so that Bell's inequality is violated. Nevertheless,  it also turns out that this violation of Bell's inequality is entirely predictable if we assume that the state of the spin singlet consisting of the two particles $q_A$ and $q_B$ is given by the Bell state: 
    \begin{equation}
      \ket*{\Psi_{\text{Bell}}}=\frac{1}{\sqrt{2}}(\ket*{\uvbp{a}}_A\ket*{\uvbm{a}}_B-\ket*{\uvbm{a}}_A\ket*{\uvbp{a}}_B).
    \end{equation}
      
When the spin singlet is in the $\ket*{\Psi_{\text{Bell}}}$-state, it can be shown that 
      \begin{equation*}\label{bellsin}
      P_{AB}(\uvbp{a},\uvbp{b})=\frac{1}{2}\sin^2(\theta/2)
      \end{equation*}
where $\theta$ is the angle between the $\uvb{a}$-axis and $\uvb{b}$-axis.\footnote{To see why this is, let $P_A(\uvbp{a})$ be the probability that Alice would detect her particle at location $\uvbp{a}$ given that she is making a measurement along the $\uvb{a}$-axis, and let $P_{BA}(\uvbp{b}|\uvbp{a})$ be the probability that Bob will detect his particle at location $\uvbp{b}$ given that he is making a measurement along the $\uvb{b}$-axis and Alice has detected her particle at location $\uvbp{a}$. Given that the joint state of the particles is given by equation (\ref{bell}), $P_A(\uvbp{a})=\frac{1}{2}.$ But also note that if  Alice has detected her particle at location $\uvbp{a}$, then Bob's particle must be in state $\ket*{\uvbm{a}}$. From the Born Rule (see page \pageref{bornrule}) and equation (\ref{spintrans1}) it follows that 
      $$P_{BA}(\uvbp{b}|\uvbp{a})= |\ip*{\uvbp{b}}{\uvbm{a}}|^2=\sin^2(\theta/2).$$ 
Therefore, 
$$P_{AB}(\uvbp{a},\uvbp{b})= P_A(\uvbp{a})P_{BA}(\uvbp{b}|\uvbp{a})=\frac{1}{2}\sin^2(\theta/2).$$} 
Then taking the angle between the $\uvb{a}$ and $\uvb{b}$-axes to be $90^\circ$, and the $\uvb{c}$-axis to be at $45^\circ$ to both the $\uvb{a}$ and $\uvb{b}$-axes, we would find that $P_{AB}(\uvbp{a},\uvbp{b})=\frac{1}{4}$ and $P_{AB}(\uvbp{a},\uvbp{c})+P_{AB}(\uvbp{c},\uvbp{b})=0.1464\ldots$, and so Bell's inequality would be violated if we assumed that the probability of each outcome is determined by the Bell state  (\ref{bell}). 
      
\section{Hidden Variables }
Now it is tempting to suppose that the expression of $\ket*{\uvbp{b}}$ in terms of $\ket*{\uvbp{a}}$ and $\ket*{\uvbm{a}}$ merely represents our knowledge of the true spin state of a particle along the $\uvb{a}$ axis given our knowledge that it would be detected at location $\uvbp{b}$ with probability $1$ should we decide to measure the particle's state along the $\uvb{b}$-axis. If we were to make this supposition, there would be a fact of the matter, albeit unknown to us, concerning what spin state the particle would be found to be in  were we to measure its spin along the $\uvb{a}$-axis. And  even though we might decide not to measure the spin of the particle along the $\uvb{a}$-axis, there would still be this hidden fact about the particle's spin in this direction. And given this supposition, since there would be no reason to suppose there was anything special about the $\uvb{a}$-axis, it would then be reasonable to suppose that there were hidden facts about what spin direction the particle would be found to be in for every possible axis orientation. This would mean that a complete description of the particle's spin state would require an infinite list of outcomes for all the possible orientations we could configure the magnetic field of our Stern-Gerlach apparatus. Given this assumption, as well as the assumption that it is already known that the particle would be detected at $\uvbp{b}$, a complete description of the particle's  state could be depicted as $\ket*{\uvbp{a},\uvbp{b},\ldots}$ or  $\ket*{\uvbm{a},\uvbp{b},\ldots}$, etc. where the ellipses would range over one of the two possible measurement outcomes for every other magnetic field orientation. However, because we would never in practice be able to perform all these experiments, and since only one such experiment would be needed to alter this infinite list,\footnote{In other words, it is assumed that directly measuring the particle will involve perturbing it so that its state will change.} nearly all of the entries in this infinite list would remain forever hidden. Hence, this would be an example of a \textbf{hidden variables} interpretation of quantum theory.  

\section{Hidden variables and the Colbeck-Renner theorem}
Butterfield assumes that the hidden variables in Kent's interpretation consist in the outcome $\tau_S(x)$ of $T_S(x)$ over the whole of $S$, and so far, I haven't questioned this assumption. However, this assumption is going to cause difficulties in the context of Shimony's analysis. This is because in Kent's interpretation, the information in $\tau_S(x)$ over the whole of $S$ clearly would determine which parameters are chosen in a Bell experiment, for this information would determine where a silver atom coming out of a Stern-Gerlach apparatus would be detected on a detection screen (as depicted in figure \ref{rotate}), and from the position of this detection, one could determine the orientation of the magnetic field used in the Stern-Gerlach experiment. So if we stipulated that $\lambda=\tau_S$ is the hidden variable of every system in Kent's interpretation, then Kent's interpretation wouldn't satisfy the preconditions necessary for defining OI and PI. This would make Kent's Interpretation radically different from the pilot wave interpretation where one can define OI and PI because the hidden variables, being the positions and momenta of the particles, are independent of the measurement choices. An unfortunate consequence of not being able to define OI and PI is that we wouldn't be able to evaluate Kent's interpretation in the light of Shimony's analysis of why Bell's inequality fails to hold. 

But it is not obvious that we should stipulate that $\lambda=\tau_S$ is the hidden variable of every system in Kent's interpretation. Just because we give $\tau_S$ a single label $\lambda$, it doesn't follow that $\tau_S$ is a single piece of information. There is typically going to be a huge amount of information in $\tau_S$, and for a given system $\mathcal{S}$, we should  discern carefully what collection of information in $\tau_S$ should be stipulated as being the hidden variable $\lambda$ of $\mathcal{S}$. The criteria on which we should make such a decision should at least include the following:
\begin{enumerate}
	\item all the information of $\lambda$ is about $\mathcal{S}$ so that a change in $\lambda$ corresponds to a change in the system $\mathcal{S}$.\label{hidden1}
\end{enumerate} 
In the pilot wave interpretation, the positions and momenta of the particles that constitute a system would fulfil this criterion. On the other hand, all the information in $\tau_S$ of Kent's interpretation would not fulfil this criterion unless of course $\mathcal{S}$ was the whole universe. 

Note, however, that we don't insist that a difference in $\mathcal{S}$ entails a difference in $\lambda$. This is because a hidden-variables theory is envisaged as augmenting standard quantum theory. So in the case when $\mathcal{S}$ is not entangled with any other system, there will be a quantum state describing $\mathcal{S}$, and this quantum state can be other than it is (indicating that $\mathcal{S}$ can be in a different physical state)  whilst the hidden variable remains the same. We thus impose a second criterion for a hidden-variables theory:
\begin{enumerate}
	\setcounter{enumi}{1}
	\item \label{hidden3} If $\lambda$ is the hidden variable of a system $\mathcal{S}$ and if $\ket{\phi}$ is the quantum state of $\mathcal{S}$ or of some composite system $\mathcal{U}$ that contains $\mathcal{S}$ as a subsystem, then it is possible for there to be a different quantum state $\ket{\phi'}$ of $\mathcal{S}$ (or $\mathcal{U}$) while the hidden variable $\lambda$ remains unchanged, and it is possible for there to be a different hidden variable $\lambda$ while $\ket{\phi}$ remains unchanged.
\end{enumerate}
This criterion is satisfied in the pilot wave interpretation, since the quantum state is the pilot wave itself. The pilot could be other than it is without any of the positions and momenta of the particles changing, but changing the pilot wave would result in a physical change of the system since the pilot wave governs how  the positions and the momenta of the particles subsequently evolve over time.

Another criterion for a collection of information $\lambda$ to constitute the hidden variable of a system $\mathcal{S}$ is the following: 
\begin{enumerate}
	\setcounter{enumi}{2}
\item \label{hidden2} it should be possible to change the measurement parameters when measuring $\mathcal{S}$ without this having any affect on $\lambda$. 
\end{enumerate} 
If this criterion doesn't hold, we cannot even begin to consider whether PI holds in a given theory. In the pilot wave interpretation, the positions and momenta of the particles that constitute a system would fulfil this criterion, whereas all the information in $\tau_S$ of Kent's interpretation would not. We used this criterion when showing that OI implies the negation of PI. 

Closely related to criterion \ref{hidden2} is the following criterion: 
\begin{enumerate}
	\setcounter{enumi}{3}
	\item \label{hidden5} If $p_\lambda$ is the probability that a system $\mathcal{S}$ has hidden variable $\lambda$, then $p_\lambda$ must be independent of any choice of measurement made on $\mathcal{S}$.
\end{enumerate} 
We are thus assuming there is a whole range of possibilities for the hidden variable $\lambda$, but because we don't know what the hidden variable $\lambda$ is, we can only assign it a probability. Knowledge of the quantum state of the system may help us assign such a probability, but this probability cannot depend on the choice of any future measurement we might make on the system. Butterfield refers to criterion \ref{hidden5} as the 'no-conspiracy' assumption, though he adds that this is a rather unfair label since there wouldn't necessarily be anything conspiratorial if this assumption was violated.\footnote{See \cite[34]{Butterfield}.}

We should also state explicitly a fifth criterion:
\begin{enumerate}
	\setcounter{enumi}{4}
	\item \label{hidden4} Suppose  $\mathcal{A}$ is any system that is entangled with $\mathcal{S}$, and that the quantum state of the composite system $\mathcal{S}+\mathcal{A}$  is $\ket{\phi}_{\mathcal{S}+\mathcal{A}}$. Then for any measurement $O_\mathcal{S}$ on $\mathcal{S}$ and $O_\mathcal{A}$ on $\mathcal{A}$, there is a probability $P_\lambda^{\ket{\phi}_{\mathcal{S}+\mathcal{A}}}(O_\mathcal{S}=o_\mathcal{S},O_\mathcal{A}=o_\mathcal{A})$ for the joint measurement of $O_\mathcal{S}$ and $O_\mathcal{A}$ on $\mathcal{S}+\mathcal{A}$ that is a function of $\lambda$  despite $\lambda$ only referring to the system $\mathcal{S}$.	
\end{enumerate}

In addition to these five criteria for a hidden variable $\lambda$ of a system $\mathcal{S}$, it is also desirable for a hidden-variables theory to satisfy PI and empirical adequacy. We defined PI for a two-outcome measurement on page \pageref{PIdef}, but it is easy to generalize the definition of PI for measurements with more than two outcomes. Thus, using the notation of criterion \ref{hidden4} and letting $O_\mathcal{A}'$ denote a second choice of measurement  on $\mathcal{A}$,  PI states that
\begin{equation}\tag{PI}
\sum_{o_\mathcal{A}}P_\lambda^{\ket{\phi}_{\mathcal{S}+\mathcal{A}}}(O_\mathcal{S}=o_\mathcal{S},O_\mathcal{A}=o_\mathcal{A}) = \sum_{o'_\mathcal{A}}P_\lambda^{\ket{\phi}_{\mathcal{S}+\mathcal{A}}}(O_\mathcal{S}=o_\mathcal{S},O'_\mathcal{A}=o'_\mathcal{A}) 
\end{equation}
where the summations on both sides are over all the possible measurement outcomes of $O_\mathcal{A}$ and $O_\mathcal{A}'$ respectively.

As for the definition of \textbf{empirical adequacy} (EA), using the notation of criteria \ref{hidden5} and \ref{hidden4}, this states that
\begin{equation}\tag{EA}\label{adeq}
	\sum_{\lambda\in\Lambda}p_\lambda P_\lambda^{\ket{\phi}_{\mathcal{S}+\mathcal{A}}}(O_\mathcal{S}=o_\mathcal{S},O_\mathcal{A}=o_\mathcal{A})=P^{\ket{\phi}_{\mathcal{S}+\mathcal{A}}}(O_\mathcal{S}=o_\mathcal{S},O_\mathcal{A}=o_\mathcal{A})
\end{equation}
where $\Lambda$ is the set of all hidden variables so that $\sum_{\lambda\in\Lambda} p_\lambda = 1$, and where 
$$P^{\ket{\phi}_{\mathcal{S}+\mathcal{A}}}(O_\mathcal{S}=o_\mathcal{S},O_\mathcal{A}=o_\mathcal{A})$$
 is the standard probability calculated using the Born rule with the eigenstates of the observables $\hat{O}_\mathcal{S}$ and $\hat{O}_\mathcal{A}$ and the quantum state $\ket{\phi}_{\mathcal{S}+\mathcal{A}}$. EA is essentially the same as equation (\ref{bohmconsistency}). It also has some similarities with (\ref{kentconsistency}), though the main difference is the range of the summation -- the index of the summands of (\ref{kentconsistency}) does not parametrize hidden variables that satisfy criteria \ref{hidden1} to \ref{hidden4} above.

Now it turns out that criteria \ref{hidden1} to \ref{hidden4} together with the conditions of PI and EA are very restrictive. In his 2016 paper, Leegwater proves a version of the Colbeck-Renner theorem.\footnote{See \cite{LeegwaterGijs2016Aitf}.} Leegwater's version takes the following form: if one defines hidden variables according to criteria \ref{hidden1} to \ref{hidden4}, then in any hidden-variables theory for which PI and EA hold, the hidden variables are redundant. In other words, in the notation of criterion \ref{hidden4},
\begin{equation}\label{colbeckrenner}
P_\lambda^{\ket{\phi}_{\mathcal{S}+\mathcal{A}}}(O_\mathcal{S}=o_\mathcal{S},O_\mathcal{A}=o_\mathcal{A})=P^{\ket{\phi}_{\mathcal{S}+\mathcal{A}}}(O_\mathcal{S}=o_\mathcal{S},O_\mathcal{A}=o_\mathcal{A})
\end{equation}
for any measurement $O_\mathcal{S}$ on $\mathcal{S}$ and $O_\mathcal{A}$ on $\mathcal{A}$.\footnote{Strictly speaking, we should say that equation (\ref{colbeckrenner}) holds for almost all $\lambda$, but we need not concern ourselves here with the details of measure theory that would be needed to make sense of this qualification.} 

Thus, the Colbeck-Renner theorem means that we cannot hope to make Kent's interpretation into a hidden-variables theory that satisfies PI and AE by simply defining more carefully what the hidden variables should be, for the information in Kent's interpretation is clearly non-redundant.

But nevertheless, it still seems that we should be able to make some kind of sense of PI and AE in Kent's interpretation and that we should be able to evaluate Kent's interpretation on the basis of whether these notions of PI and EA are true in this context. To achieve this aim, one strategy would be to relax one of the five criteria for a hidden variable. Since we still want to be able to make sense of PI and AE,  a process of elimination suggests that the most obvious hidden variable criterion to drop would be criterion \ref{hidden3}. In other words, instead of thinking of $\tau_S$ as an augmentation of standard quantum theory, we could instead think of $\tau_S$ as a rather elaborate way of stipulating the initial quantum states of experiments as well as the quantum states of measurement outcomes. The information of $\tau_S$ would then be non-redundant. Moreover, if we could appropriately partition the information in $\tau_S(x)$ on the basis of whether it determined the quantum state of the particle, or the quantum state of the apparatus, or the quantum state of the rest of the universe, we could then consider whether Kent's interpretation gave the same predictions as standard quantum theory. If it did, then PI and AE would hold in Kent's interpretation, since these both hold in standard quantum theory. And since Kent's interpretation is formulated in the Lorentz invariant setting of Schwinger and Tomonaga, this would mean that Kent's interpretation is a solution to the measurement problem!

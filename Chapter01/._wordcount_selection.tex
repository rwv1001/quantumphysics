Now naively, one would expect that if the spin of $q_A$ were to be measured, then this would have no effect on any spin-measurement of $q_B$. This assumption is a special case of \textbf{Einstein's locality principle}\index{Einstein's locality principle}:\label{EinsteinLocalityPrinciple} For two spatially separated systems $\mathcal{S}_1$ and $\mathcal{S}_2$, %
\nomenclature{$\mathcal{S}_1, \mathcal{S}_2$}{Two spatially separated systems, \nomrefpage}%
the real factual situation of the system $\mathcal{S}_2$ should be independent of what is done to the system $\mathcal{S}_1$.\footnote{Einstein expressed this locality principle in his autobiographical notes: ``But on one supposition we should, in my opinion, absolutely hold fast: the real factual situation of the system $\mathcal{S}_2$ is independent of what is done with the system $\mathcal{S}_1$, which is spatially separated from the former.'' \cite[p. 85]{EinsteinLocality}.} This principle was so intuitive to Einstein, he was convinced it was true.\footnote{\cite[449]{IsaacsonWalter2008E:hl}.} Writing to Max Born, Einstein famously said ``Physics should represent a reality in time and space, free from spooky  action at a distance.''\footnote{From a letter written March 3, 1947, \cite[155]{BornMax2005TBl:}. Quoted from \cite[450]{IsaacsonWalter2008E:hl}.} Although Einstein offered no argument in support of his principle of locality, several arguments have been proposed in its favor. For instance,  Van Laer makes the following argument: 
\begin{adjustwidth}{1cm}{}
	\begin{displayquote}
Suppose that body $A$ is located in void space, and that body $P$ is now brought into its vicinity. If $P$ is acted on by $A$, then either $A$'s activity ``across'' the intervening space pre-existed $P$'s arrival, in which case an action existed without an object and without a correlative passion, both of which run counter to the ratio action; or $A$'s activity only occurred when $P$ became present to it, in which case $A$ would somehow have to have ``known'' $P$'s becoming present, which is not consonant with the nature of mere body as such.\footnote{See \cite[59-144]{LaerP.H.van1953Pp}. The quote here is from \cite[246]{Grove} who in turn quotes a slightly modified version of Van Laer's argument found in \cite[257]{TallaricoJames1962AaaD}. Grove also appeals to the Aristotelian doctrine of hylomorphism in order to defend the principle of locality: 
\begin{adjustwidth}{1cm}{}
	\begin{displayquote}
        What does it mean for one hylomorphic substance to act on another? It is for the one to effect accidental or (by way of the accidental) substantial change in the other. As the accidents involved are inseparable from spatiality or extension, so must be the operations which occur through them. That is to say, that the interactions of cosmic (nonintellectual) beings are in essence spatial (and therefore spatiotemporal): they involve that which characterizes the spatial as such, i.e., "parts outside parts," extendedness and adjacency. For one hylomorphic being to act on another (and for that other to "receive" the action of the one), there must be an ontological ground or commonality which is located precisely in their spatiality; they must "share space," not in violation of the principle of unilocality but along a common boundary, thereby forming a unity through relation, i.e., "existing with respect to one another."
    \end{displayquote}  }
\end{displayquote}

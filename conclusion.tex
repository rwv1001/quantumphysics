\chapter*{Conclusion}
\addcontentsline{toc}{chapter}{Conclusion}
The purpose of this dissertation has been to evaluate the merits of Kent's one-world interpretation of quantum physics. I began this dissertation by discussing the mysterious correlations of spin measurements that are observed in the EPR-Bohm paradox: how is it that two distant observers Alice and Bob always get opposite results when they perform a spin measurement along parallel axes of two particles belonging to a spin singlet? Special relativity should rule out any faster than light communication between the two particles, and the violation of Bell's inequality rules out the possibility that the results of the spin measurements are already encoded in the states of the two particles.

Shimony attempted to resolve this paradox by making a distinction between Outcome Independence (OI) and Parameter Independence (PI). Shimony argued that if PI held, then the correlations observed in the EPR-Bohm paradox would present no threat to special relativity since it would then be impossible for Alice and Bob to send messages to each other faster than the speed of light by choosing the axis on which they were to perform their spin measurements. On the other hand, if OI was false, the assumption on which Bell's inequality depends would not be valid. Thus, by accepting PI and denying OI, Shimony thought that quantum physics could be reconciled with special relativity without Bell's inequality needing to be satisfied.

However, Shimony's solution is not entirely satisfactory since it relies on the rather vague notion of what an outcome is. This problem of what an outcome is goes back to the Copenhagen interpretation in which a physical system is supposedly described by a quantum state which determines with what probability a measurement outcome will occur. According to the Copenhagen interpretation, on making a measurement, the quantum state collapses to the state of one of the measurement outcomes. However, the Copenhagen interpretation doesn't specify the criteria which would determine when such a collapse occurs. 

Another popular interpretation, the Bohmian interpretation doesn't have this problem, since there is no state collapse in this interpretation. Instead, the fact of there being a measurement outcome is grounded in the fact of where all the particles of a measurement apparatus are located which in turn would result in the apparatus displaying a definite reading that was correlated with the state of the system that was being measured.  However, the Bohmian interpretation suffers from the problem of violating PI, and so for this reason, many people find the Bohmian interpretation unattractive. 

In chapter \ref{measprobchap}, I considered the many-worlds interpretation which attempts to overcome the problems with the Copenhagen interpretation and the Bohmian interpretation. Like the Bohmian interpretation, the many-worlds interpretation denies there is any state collapse. But on the other hand, the many-worlds interpretation differs from both the Copenhagen interpretation and the many-worlds interpretation by denying that there are any measurement outcomes by which one could make objective claims about the results of experiments. Now although there are some appealing features of the many-worlds interpretation, I argued that we shouldn't accept this interpretation since otherwise we would be accepting an understanding of reality that undermines the belief that science is a discipline concerned with objective facts. 

In chapter \ref{kentchapterdesc} I then described Kent's theory of quantum physics. Kent supposes that there is an initial quantum state that specifies the conditions of the universe across a three-dimensional hypersurface $S_0$ in spacetime, but the information in this quantum state is supplemented with information given by a notional mass-energy density measurement across a distant future spacelike hypersurface $S$. To determine how history unfolds, Kent proposes that at any spacetime location $y$ between the initial hypersurface $S_0$ and the final hypersurface $S$, the stress-energy tensor at $y$ is given by the expectation value predicted by the initial quantum state subject to the condition that the mass-energy density across $S$ is in agreement with the mass-energy density selected in the notional mass-energy density measurement. Since there is no state collapse in Kent's theory, he doesn't need to address the question that plagues the Copenhagen interpretation of what the criteria are for a quantum state collapse to occur. But the information contained in the notional mass-energy density measurement is also sufficient to make Kent's theory into a one-world interpretation of quantum physics, so it doesn't suffer from the problems present in the many-worlds interpretation. 

In the final chapter, I offered an evaluation of Kent's theory. I argued that Kent's theory would not contradict the empirical observations that standard quantum theory would predict. I also argued that the predictions Kent's theory makes satisfy Lorentz invariance, where Lorentz invariance is the fundamental property a theory must satisfy if special relativity is to hold within it. 

I also considered Kent's theory in the light of the Colbeck-Renner theorem. The assumptions of the Colbeck-Renner theorem lead to the conclusion that there are no non-trivial extensions of standard quantum theory in which PI holds. However, I argued that the assumptions that the Colbeck-Renner theorem makes about the nature of hidden variables don't all apply to the information contained within the notional mass-energy density measurement on $S$. This means that Kent's theory can be a non-trivial extension of quantum physics in which PI holds, without contradicting the Colbeck-Renner theorem. 

I also argued that the information in the notional mass-energy density measurement on $S$ allows us to extract from the universal quantum state on $S_0$ a local conditioned quantum state in the vicinity of a spacetime location $y$ between $S_0$ and $S$, and that this local conditioned quantum state behaves in much the same way as quantum states of standard quantum theory. By adapting Kent's theory so that it can make probabilistic predictions about multiple measurements, I argued that it will make predictions that are consistent with the validity of PI. 

Finally, I discussed some features of Kent's theory such as its inherent determinism as well as the nature of the beables which specify the underlying physical facts that govern how the world appears. I argued that it is not necessary to endorse the beables that Kent proposes which he takes to be the expectation values of the stress-energy tensor conditioned on the notional mass-energy density measurement outcome on $S$. Instead, I suggested we could take the beables to be the entangled components of the state of a hypersurface whose mass-energy density on its intersection with $S$ is given by the notional mass-energy density measurement. I also argued that by re-imagining what we meant by time in a more Aristotelian manner, indeterminism can be reintroduced into Kent's theory if one so desires. 

Inevitably, Kent's theory and my adaption to it are highly speculative. In this dissertation, I am not making claims about how physical reality must be given one's acceptance of standard quantum theory and special relativity. Rather, in this dissertation, I have been trying to argue that Kent's theory is able to resolve the EPR-Bohm paradox and can address the measurement problem in a way that is much more appealing than the many-worlds interpretation and the Bohmian interpretation of quantum physics.
 In this dissertation, I will consider in detail the EPR-Bohm paradox and the mysterious correlations that it exhibits, and I will explain why the EPR-Bohm paradox can lead one to think that our best physics suggests a many-worlds picture of reality. I will thus consider the many-worlds interpretation in some detail and explain why it is so attractive. But ultimately, my goal is not to endorse the many-worlds interpretation of quantum physics, but rather it is to argue that the reasons for endorsing the many- worlds interpretation are not compelling. My argument relies on a recent interpretation of quantum physics by the physicist Adrian Kent. Kent's interpretation shares several features with the many-worlds interpretation that make it seem attractive, yet in Kent's interpretation, there is only one world.
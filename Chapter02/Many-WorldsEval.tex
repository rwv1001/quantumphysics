 
    
    \section{Evaluating the Many Worlds Interpretation}\label{manyworldsinterpretation2}
    Given the above account of the many-worlds hypothesis of quantum physics, it does seem understandable why physicists would find it so attractive. Although we can't specify an exact moment at which branching occurs, the idea of branching and of there being many worlds itself is not particularly mysterious. This can all be explained in terms of the dynamics of the system and the environment, and decoherence theory allows us to understand why the interference effects that are the hallmark of quantum physics generally disappear on the macroscopic level. 
    
    There are other advantages  of the many-worlds hypothesis besides these which we need not discuss here.\footnote{More details can be found in \cite{Schlosshauer} and \cite{joos2013decoherence}.} But for all the advantages of the many-worlds hypothesis, there is one fundamental problem, and that is its patent absurdity. It seems that we should be able to say whether a cat is alive or dead without having to say what state the rest of the universe is in. However, the many-worlds hypothesis suggests that for any subsystem of the universe, we will in general only be able to say what state it is in with respect to the state of the rest of the universe. For example, if the state $\mathcal{S}$ is the system constituting a cat-wise configuration of particles and $\mathcal{E}$ is the rest of the universe, then given that the composite system $\mathcal{U}=\mathcal{S}+\mathcal{E}$ is described by the state $$\ket{\Psi(t)}=\frac{1}{\sqrt{2}}\big(\ket{\text{Cat Alive}}_\mathcal{S}\ket{E_\text{Cat Alive}}_\mathcal{E}+\ket{\text{Cat Dead}}_\mathcal{S}\ket{E_\text{Cat Dead}}_\mathcal{E}\big),$$ then we are in no position to make an absolute matter of fact claim about the system $\mathcal{S}$ and say the cat is dead or the cat is alive. Rather we have to say with respect to the environment described by $\ket{E_\text{Cat Alive}}_\mathcal{E}$, the cat is alive, and with respect to the environment $\ket{E_\text{Cat Dead}}_\mathcal{E}$, the cat is dead. Moreover, according to the many-worlds hypothesis, the branching into multiple worlds doesn't just occur in rare instances, such as in Schr\"{o}dinger's cat-type experiments. On the contrary, branching is supposed to be happening all the time.  
    
    In order to convey how ubiquitous branching is in the many-worlds interpretation, one just has to consider the behavior of an electron. If we suppose that a free electron is initially described by a wave packet whose width is around $\du{e-10}{\m}$  (which is of the order of the width of and atom), then according to the Schr\"{o}dinger equation which governs how wave packets evolve over time, after one second the width of the wave packet will have spread to a width of around $\du{1000}{\km}$.\footnote{See \cite[117]{Schlosshauer}.} The only reason electrons remain localized rather than spreading out to such vast distances is because the electron is continually interacting with its environment.  But according to the many-worlds interpretation, these continual interactions of the electron with its environment will result in a continual branching out of worlds corresponding to the possible locations the environment localizes the particle to. Therefore, because the electron rapidly gets entangled with its environment, we cannot establish matter of fact claims about where the electron is localized to -- we can only establish matter of fact claims about the composite system of the electron and its environment which expands with astonishing rapidity.     
    
   Because of the ubiquity of branching in the many-worlds interpretation, this interpretation appears to undermine the conditions for the possibility of doing science, for surely one of the conditions for the possibility of doing science is that measuring devices exist, but it doesn't look like there are such things as measuring devices in the many-worlds interpretation. To see why, consider the properties a measuring device should possess. Firstly, it must be capable of interacting either directly or indirectly\footnote{It is acceptable for a measuring device to interact with an environment that has interacted with the thing that is being measured.} with another entity which is the thing to be measured. Secondly, there must be some kind of correspondence between the range of states the measuring device can be in and the range of states the thing being measured can be in. Thirdly, when the measuring device interacts with the thing being measured, the measuring device should enter into the state that corresponds to the state of the thing being measured.\footnote{Although the act of measuring may change the state of the thing being measured, a measuring device should still be able to tell us what the state of the thing being measured is in immediately after the measuring device has interacted with it.} But according to the many-worlds interpretation, what is taken to be a measuring device will  in general become entangled with the thing  that is being measured,  and so there will be no fact of the matter regarding what state the measuring device is in. Rather there will at the very most only be a fact of the matter regarding the state of the composite system that includes both the measuring device and the thing being measured.\footnote{In fact, it is questionable whether we can even make a matter of fact claim about there being a measuring device at all -- rather, in the many-worlds interpretation, we can only say there is a superposition of states in which there is a measuring device in existence with respect to some environments, and of states in which there is not a measuring device with respect to other environments.  } Thus, in the many-worlds interpretation, there are no measuring devices satisfying the criteria one would expect a measuring device to satisfy. And so without such measuring devices, it does not appear to be possible to do science according to what we normally mean by science.  
   
   Another reason for rejecting the many-worlds interpretation is that intuitively, it seems obvious that I can know I am alive without needing to know the state of the rest of the universe, but the many-worlds hypothesis does not allow me to make this absolute matter of fact claim. 
    So from both a common sense point of view and a scientific point of view, the many-worlds hypothesis really is absurd.
    
     Of course some hypotheses may initially seem absurd, but once the hypothesis has been fully explained, it can appear far more plausible. For instance, time dilation in special relativity might initially sound absurd to some people, but once one has a better grasp of special relativity and is open to the possibility that  systems moving close to the speed of light with respect to ourselves might have properties rather different to systems that move with much slower speeds, then special relativity doesn't seem absurd at all. However, the many-worlds hypothesis as presented here is different in this regard since it is not hypothesizing about some extreme situation. It is hypothesizing about ordinary situations. In order to accept the many-worlds interpretation, the arguments in its favor would have to be at least as convincing as the common sense beliefs it is calling into question such as the belief that we can do science and the belief that I am alive. But arguments for the many-worlds interpretation clearly fail to meet this criterion. Some people may choose to embrace the absurdity of the many-worlds interpretation and reject the most basic notions of common sense. But throughout much of human history, when a hypothesis has entailed an absurd conclusion, reasonable people have usually thought it better to reject the hypothesis rather than embrace the absurdity. 
      
     But in rejecting a hypothesis because of its absurd consequences, it doesn't mean that absolutely everything in the hypothesis needs to be rejected, for  a hypothesis might be formulated in terms of sub-hypotheses, some of which might be very plausible and which don't of themselves entail absurdities, in which case something of the original hypothesis might be salvageable. In the case of the many-worlds hypothesis, I believe it does have something that is salvageable, namely decoherence theory. In the next chapter I will consider Adrian Kent's one-world interpretation of quantum physics in which the basic ideas of decoherence theory remain intact.
    
    
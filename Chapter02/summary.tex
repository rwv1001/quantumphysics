\section{Summary}\label{manyworldssummary}
In this chapter I have given an account of the many-worlds interpretation and discussed why anyone would find this interpretation attractive. In order to provide this account, it was necessary to explain some of the mathematical formalism of standard quantum theory. In this formalism, states of a system are thought of as belonging to a space of states called a Hilbert space. A Hilbert space has an inner product that enables us to calculate how likely one state is to be found in another state via the Born Rule. 

By associating experiments with operators (which are referred to as observables) that act on the Hilbert space, we can calculate the average outcome of an experiment if the experiment is performed many times. 

However, this formalism raises some problems. For the space of observables in the mathematical formalism of quantum theory is much larger than the kinds of measurements we can perform. For instance, we can perform measurements by which we can distinguish between a cat being alive and a cat being dead, but we can't perform measurements that distinguish between different superposition states of cats. This is the preferred basis problem. Also, there is the question of why very small objects exhibit interference, whereas larger objects like cats don't exhibit interference. This is the problem of the nonobservability of interference. 

Decoherence theory enables us to answer these two problems by making a distinction between a system and its environment. When this distinction is made, we can work out how the system's interaction with its environment affects the observations made on the system. By averaging over the unknown states of the environment as the system interacts with it, we can form a generalized notion of a state for the system via an operation called the partial trace. This enables us to obtain a reduced density matrix describing the system. The reduced density matrix contains all the information necessary to calculate expectation values of observables. In many situations, via a process called decoherence, the reduced density matrix essentially becomes diagonal, so that it is as though the system is in an unknown state corresponding to one of the diagonal entries of the  reduced density matrix. The criterion for decoherence to occur enables us to solve the preferred basis problem and the problem of the nonobservability of interference.

However, decoherence theory doesn't enable us to solve what is known as the problem of outcomes. For decoherence theory relies on a subjective distinction between a system and its environment, but when one considers the composite of the system and its environment together, the composite system is in a superposition of states rather than a definite state. Many-worlds adherents conclude that the different states of this superposition are just as real as each other. 

Despite some attractive features of the many-worlds interpretation, it calls for such a radical departure from common sense that it undermines our most basic belief that there are objective facts about reality. The proposed solution to the EPR-Bohm paradox offered by the many-world interpretation is therefore deeply unsatisfactory. Thus, we have good reason  to look elsewhere in our search for a solution to the EPR-Bohm paradox. We will therefore continue our search in the next chapter by examining Kent's interpretation of quantum physics. 

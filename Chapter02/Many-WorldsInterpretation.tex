 
    \section{The Many-Worlds Interpretation}\label{manyworldsinterpretation1}
    Not everyone is convinced that the problem of outcomes is a genuine problem. In particular, people who endorse the many-worlds interpretation of quantum physics effectively argue that there are no outcomes in the traditional sense. In this section, we give an account of the many-worlds interpretation of quantum physics and why physicists find it attractive. To this end, let us consider a physical universe $\mathcal{U}=\mathcal{S}+\mathcal{A}+\mathcal{P}_A+\mathcal{P}_B+\mathcal{E}$ consisting of subsystems $\mathcal{S}, \mathcal{A}, \mathcal{P}_A,\mathcal{P}_B$  %
\nomenclature{$\mathcal{P}_A,\mathcal{P}_B$}{The physical systems corresponding to two scientists, Alice and Bob respectively, \nomrefpage}%
    and $\mathcal{E}$. $\mathcal{S}$ is the physical system under investigation; $\mathcal{A}$ is some measuring apparatus that interacts with $\mathcal{S}$; $\mathcal{P}_A$ and $\mathcal{P}_B$ are the physical systems corresponding to two scientists, Alice and Bob who observed the apparatus $\mathcal{A}$; and $\mathcal{E}$ is the remainder of the physical universe $\mathcal{U}$.  For convenience, we define the composite subsystem $\mathcal{V}=\mathcal{S}+\mathcal{A}+\mathcal{P}_A+\mathcal{P}_B$ %
\nomenclature{$\mathcal{V}$}{The composite subsystem $\mathcal{S}+\mathcal{A}+\mathcal{P}_A+\mathcal{P}_B$ so that $\mathcal{U}=\mathcal{V}+\mathcal{E}$, \nomrefpage}%
    so that $\mathcal{U}=\mathcal{V}+\mathcal{E}$. As above on page \pageref{pointer}, we  assume that there is an orthonormal basis $\{\ket{s_i}:i\}$ of $H_\mathcal{S}$ which we again refer to as pointer states, but now we assume that there are ready states $\ket{a_r(t)}\in H_\mathcal{A},\,\ket{p_{r, A}(t)}\in H_{\mathcal{P}_A},\,\ket{p_{r, B}(t)}\in H_{\mathcal{P}_B},$ and  %
\nomenclature{$\ket{p_{r, A}(t)}, \ket{p_{r, B}(t)}$}{Ready states of Alice and Bob before they observe a measurement outcome, \nomrefpage}%
    $\ket{E_r(t)}\in H_\mathcal{E}$ and %
    \nomenclature{$\ket{E_r(t)}$}{Ready states of the environment $\mathcal{E}$, \nomrefpage}%
    that for each $i$, there are normalized states $\ket{a_i(t)}\in H_\mathcal{A},\,\ket{p_{i, A}(t)}\in H_{\mathcal{P}_A},\,\ket{p_{i, B}(t)}\in H_{\mathcal{P}_B},$ and $\ket{E_i(t)}\in H_\mathcal{E}$ such 
     %
    \nomenclature{$\ket{E_i(t)}$}{State of the environment $\mathcal{E}$ corresponding to the pointer state $\ket{s_i}$, \nomrefpage}%
    that 
      %
\nomenclature{$\ket{p_{i, A}(t)}, \ket{p_{i, B}(t)}$}{The states of Alice and Bob after they have observed the apparatus to be in the state $\ket{a_i(t)}$, \nomrefpage}%
    \begin{enumerate}[noitemsep, nosep, topsep=0pt]
    \item for any $t\geq 0$ we have the evolution of the states 
    \begin{align*}\ket{s_i}\ket{a_r(t)}\ket{p_{r, A}(t)}&\ket{p_{r, B}(t)}\ket{E_r(t)}\\ &\xrightarrow{\text{time evolution}}\ket{s_i}\ket{a_i(t)}\ket{p_{i, A}(t)}\ket{p_{i, B}(t)}\ket{E_i(t)},\end{align*}
    \item there exists $\delta>0$ such that if $t>t_0+\delta$, then for $i\neq j$, $\ip{a_i(t)}{a_j(t)}\approx 0$, $\ip{p_{i, A}(t)}{p_{j, A}(t)}\approx 0$, $\ip{p_{i, B}(t)}{p_{j, B}(t)}\approx 0$ and $\ip{E_i(t)}{E_j(t)}\approx 0$.\footnote{Again, recall footnote \ref{approx}.}
    \end{enumerate}
    We also suppose that the $\ket{p_{i,A}(t)}$-state would describe actions of Alice consistent with her observing the apparatus being in the $\ket{a_i(t)}$-state, for example, her writing down in her log book that the apparatus is in the $\ket{a_i(t)}$-state or telling her colleague that this is the case. Likewise, we assume the $\ket{p_{i,B}(t)}$-state is consistent with Bob also observing the apparatus to be in the $\ket{a_i(t)}$-state.
    
    Now suppose the initial (normalized) state of $\mathcal{S}$ is $\ket{\psi}=\sum_i c_i\ket{s_i}$, so that the state for the composite system $\mathcal{U}$ is $\ket{\Psi(t)}=\sum_i c_i \ket{\xi_i(t)}\ket{E_i(t)}$ where $\ket{\xi_i(t)}=\ket{s_i}\ket{a_i(t)}\ket{p_{i,A}(t)}\ket{p_{i,B}(t)}$. We also define the corresponding density matrix for the composite system $\hat{\rho}=\dyad{\Psi}$. If we were unable to make any observations on $\mathcal{E}$, then 
     the partial trace
    $\hat{\rho}_{\mathcal{V}}(t)=\Tr_\mathcal{E}(\hat{\rho}(t))$ will contain all the information we need to work out the expectation values for any observables of  $\mathcal{V}.$ So just as with equation (\ref{reduced}), we will have
    \begin{equation} \label{reducedv}
    \begin{split}
      \hat{\rho}_{\mathcal{V}}(t)&=\sum_{i}\abs{c_i}^2\dyad{\xi_i(t)}+\sum_{i\neq j}c_i\overline{c_j}\ip{E_j(t)}{E_i(t)}\dyad{\xi_i(t)}{\xi_j(t)}\\
      &\approx \sum_{i}\abs{c_i}^2\dyad{\xi_i(t)}
      \end{split}\end{equation}
      for $t>t_0+\delta.$ Then the expectation values of any observables on $\mathcal{V}$ will be indistinguishable from the scenario in which $\mathcal{V}$ is actually in one of the $\ket{\xi_i(t)}$-states with probability $\abs{c_i}^2$.\footnote{Again recall the discussion following equation (\ref{rhodiag}) on page \pageref{rhodiag}. There is the question of uniqueness of $\hat{\rho}_{\mathcal{V}}=\sum_{i}\abs{c_i}^2\dyad{\xi_i(t)}$. If all the $\abs{c_i}^2$ are unique, then if we have another decomposition $\hat{\rho}_{\mathcal{S}+\mathcal{A}+\mathcal{P}_A+\mathcal{P}_B}=\sum_{i}\abs{c_i}^2\dyad{\xi_i'(t)}$ it follows that $\ket{\xi_i(t)}\propto\ket{\xi_i'(t)}$. But even if some of the $\abs{c_i}^2$ are the same, criteria 1 and 2 above will ensure that states with the same  value of $\abs{c_i}^2$ will be determined up to permutation.} It would nevertheless be incorrect for us to conclude on the basis of decoherence theory alone that $\mathcal{V}$ actually was in one of those $\ket{\xi_i(t)}$-states, since equation (\ref{reducedv}) is based on a subjective distinction between $\mathcal{V}$ and $\mathcal{E}$ in the decomposition $\mathcal{U}=\mathcal{V}+\mathcal{E}.$ Human scientists make this distinction to reflect the fact that they can only perform measurements on $\mathcal{V}$ and can't measure $\mathcal{E}$. But if a super-intelligent being could measure everything in  $\mathcal{U}$, then such a being would not say that $\mathcal{V}$ was in one of the  $\ket{\xi_i(t)}$-states, but rather that $\mathcal{U}$ was in the state $\ket{\Psi(t)}$. As we have already discussed on pages \pageref{subtle}--\pageref{subtleend}, the density matrix $ \hat{\rho}_{\mathcal{V}}(t)$ is not a mixed state, but is an improper mixture. 
    
    Now if we define the observables $\hat{O}_{i,\mathcal{A}}(t)=\dyad{p_{i,A}(t)}$ that %
\nomenclature{$\hat{O}_{i,\mathcal{A}}(t), \hat{O}_{i,\mathcal{B}}(t)$}{The observables $\dyad{p_{i,A}(t)}$ and $\dyad{p_{i,B}(t)}$ respectively,  \nomrefpage}%
    would measure the behavior of Alice, and the observables $\hat{O}_{i,\mathcal{B}}(t)=\dyad{p_{i,B}(t)}$ that would measure the behavior of Bob, then for $t>t_0+\delta$, we see that $\hat{O}_{i,\mathcal{A}}(t)\hat{O}_{j,\mathcal{B}}(t)\ket{\Psi(t)}\approx 0,$ when $i\neq j$. This means that when we consider $\hat{O}_{i,\mathcal{A}}(t)\hat{O}_{j,\mathcal{B}}(t)$ as an observable acting on $\mathcal{V}$,  the expectation value $\ev*{\hat{O}_{i,\mathcal{A}}(t)\hat{O}_{j,\mathcal{B}}(t)}_{\rho_\mathcal{V}(t)}$ will be approximately zero for $i\neq j$. What this means is that if we consider ourselves as observing Alice and Bob observing the apparatus, then after time $t_0+\delta$, the probability we would see Alice and Bob disagreeing with each other concerning their observations of the apparatus would be approximately 0. 
     On the other hand, since $\hat{O}_{i,\mathcal{A}}(t)\hat{O}_{i,\mathcal{B}}(t)\ket{\Psi(t)} \approx c_i \ket{\xi_i(t)}\ket{E_i(t)}$ for $t>t_0+\delta$, it follows that $\ev*{\hat{O}_{i,\mathcal{A}}(t)\hat{O}_{i,\mathcal{B}}(t)}_{\rho_\mathcal{V}(t)}=\abs{c_i}^2.$ 
     We would thus observe Alice and Bob observing the apparatus to be in the $\ket{a_i(t)}$-state with probability $\abs{c_i}^2.$ 
     
     But note that on the assumption that there are no hidden variables, if we did actually make such an observation and this observation corresponded to reality, then the quantum state $\ket{\Psi(t)}$ would have had to have changed to $\ket{\xi_i(t)}\ket{E_i(t)}$, since before our observation when $\ket{\Psi(t)}$ was a complete description of $\mathcal{U}$, we would say Alice and Bob will measure the $\ket{a_i(t)}$-state with probability $\abs{c_i}^2$, but when we are actually seeing them measuring the $\ket{a_i(t)}$-state, we would have to say that now the probability they are measuring the $\ket{a_i(t)}$-state is $1$, and hence we would say that the system was in the $\ket{\xi_i(t)}\ket{E_i(t)}$-state. Whether or not the process of the state going from $\ket{\Psi(t)}$ to $\ket{\xi_i(t)}\ket{E_i(t)}$ was instantaneous or took a non-infinitesimal amount of time, this interpretation would be susceptible to the problems already discussed with the Copenhagen interpretation in section \ref{eprsec}.
    
    But in the \textbf{many-worlds interpretation}\index{many-worlds interpretation}, rather than assuming that $\ket{\Psi(t)}=\sum_i c_i \ket{\xi_i(t)}\ket{E_i(t)}$ is the complete description of $\mathcal{U}$ that enables us to work the probability of certain outcomes, we simply say that $\ket{\Psi(t)}$ is a complete description of the state of $\mathcal{U}$, and we drop the assumption that we need to interpret this state as describing probabilities of outcomes. Thus a many-worlds adherent would say we can understand what the state of $\mathcal{U}$ is on its own terms without the need to appeal to any other extrinsic principle such as measurement. Just as we don't puzzle over how to interpret what a sphere is in terms of an extrinsic principle, we don't need to puzzle over how to interpret the space of states of $\mathcal{U}$. We can think of the mathematical formalism $\ket{\Psi(t)}=\sum_i c_i \ket{\xi_i(t)}\ket{E_i(t)}$ describing the state of $\mathcal{U}$ as being somewhat akin to a point lying on a sphere given by the equation $x^2+y^2+z^2=1$.\footnote{On the assumption that $\ket*{\Psi(t)}$ is normalized, we could think of $\ket*{\Psi(t)}$ as specifying a point on the (possibly infinite-dimensional) hypersphere $\{(c_1,c_2\ldots):\sum_i|c_i|^2=1\}.$} Although we might be tempted to interpret $\ket{\Psi(t)}$ as describing the probability of outcomes, we are not obliged to do so,  since these probabilities can instead be understood to be grounded in the symmetries the system possesses rather than in terms of the frequency of how many measurement outcomes are likely to occur. For instance, when we see a coin and judge that it will come up heads with probability $\frac{1}{2}$ and tails with probability $\frac{1}{2}$, we intuit this by looking at the symmetry of the coin rather than tossing the coin millions of times and counting how often it comes up heads and how often it comes up tails. Thus, we might suppose $\ket*{\Psi(t)}$ has analogous symmetries that allow us to determine its $c_i$ coefficients without the need to posit any of the $\ket*{\xi_i(t)}\ket*{E_i(t)}$ measurement outcomes being realized.
     
    As for the decomposition $\ket{\Psi(t)}=\sum_i c_i \ket{\xi_i(t)}\ket{E_i(t)}$  in terms of the $\ket{\xi_i(t)}\ket{E_i(t)}$ basis states, decoherence theory gives us a natural account of why we should choose this basis rather than any other. When $\mathcal{U}$ is in the state $\ket{\Psi(t_0)}=\Big(\sum_i c_i \ket{\xi_i(t_0)}\Big)\ket{E_r(t)}$, we can think of this state as describing one world, $W$ say.  %
\nomenclature{$W$}{A world described by the state $\ket{\Psi(t_0)}=\Big(\sum_i c_i \ket{\xi_i(t_0)}\Big)\ket{E_r(t)}$,  \nomrefpage}%
    But once $t>t_0+\delta$ so that $\ip{E_i(t)}{E_j(t)}\approx 0$ for $i\neq j$, we can think of each $\ket{\xi_i(t)}\ket{E_i(t)}$-component as a different world $W_i$.   %
\nomenclature{$W_i$}{A world described by the state $\ket{\xi_i(t)}\ket{E_i(t)}$,  \nomrefpage}%
    Thus, for $t>t_0+\delta$, we say the world $W$ has \textbf{branched}\index{branching} into as many-worlds $W_i$ for which the $c_i$ are non-zero. 
    
    But why should we think that there are literally many worlds? Well, from an ontological point of view, one might very well think that there is really only one world and that this world is described by $\ket{\Psi(t)};$ it would be a rather weird world since the entanglement between $\mathcal{V}$ and $\mathcal{E}$ would mean there wouldn't be any absolute matters of fact describing $\mathcal{V}$. But it might not be a bad thing to say that the ``many'' in the many-worlds interpretation is really just a figure of speech that we shouldn't take too literally. After all, a common objection to the many-worlds interpretation is that it is ontologically extravagant and that we should appeal to Occam's Razor. But if we just say that there is actually only one world described by $\ket{\Psi(t)}$ then this ``many''-worlds interpretation is actually rather parsimonious from an ontological point of view. 
    
    But if by literal, we mean descriptive rather than ontological, it does seem rather natural to say that there are literally many worlds. For even though we might initially suspect that the worlds $W_i$ and $W_j$ are not clearly delineated given the fact that $\ip{E_i(t)}{E_j(t)}$ is very small but not zero for $i\neq j$, we can nevertheless expect $\ip{\xi_i(t)}{\xi_j(t)}$ to be identically zero for $i\neq j$, just as we can expect $\ip*{\uvbp{a}}{\uvbm{a}}$ to be identically zero.\footnote{It is also reasonable to suppose that in situations such as the double-slit experiment described on page \pageref{psi_slit} that $\ip{\psi_1(t)}{\psi_2(t)}$ is identically zero. This is because $\ip{\psi_1(t)}{\psi_2(t)}=0$ when $t$ is the time at which the particle is going through the slit, and this will remain zero because of a property of the time evolution operator known as unitarity.} Thus, if we define $\ket{W_i(t)}=\ket{\xi_i(t)}\ket{E_i(t)},$ then the $\ip{W_i(t)}{W_j(t)}$ will be identically zero for $i\neq j$, and so we would in fact be able to clearly delineate these worlds.
    
    Still, the supposition that $\ket{\Psi(t)}= \sum_i c_i\ket{W_i(t)}$  with $\ip{W_i}{W_j}=0$ for $i\neq j$ is not of itself sufficient justification for describing the state $\ket{\Psi(t)}$ as a composition of mutually exclusive world descriptions given by the $\ket{W_i(t)}$. After all, the fact that
    \begin{align*}\ket{\text{Cat Alive}}= &\frac{1}{\sqrt{2}}\Big(\frac{1}{\sqrt{2}}(\ket{\text{Cat Alive}}+\ket{\text{Cat Dead}})\Big)\\&+\frac{1}{\sqrt{2}}\Big( \frac{1}{\sqrt{2}}(\ket{\text{Cat Alive}}-\ket{\text{Cat Dead}})\Big). \end{align*} does not incline us to think of the state $\ket{\text{Cat Alive}}$ as being composed of the mutually exclusive cat states  $\frac{1}{\sqrt{2}}(\ket{\text{Cat Alive}}+\ket{\text{Cat Dead}})$ and $\frac{1}{\sqrt{2}}(\ket{\text{Cat Alive}}-\ket{\text{Cat Dead}})$. 
    
    But the key justification for describing the state $\ket{\Psi(t)}$ as a composition of the mutually exclusive $\ket{W_i(t)}$-states is the fact that the states $\ket{\xi_i(t)}$ and $\ket{\xi_j(t)}$ decohere for $i\neq j$, that is, the off-diagonal entries $\dyad{\xi_i(t)}{\xi_j(t)}$ of the reduced density matrix $\hat{\rho}_\mathcal{V}(t)$ will tend to zero, and as we saw in section \ref{Nonobservability}, it will follow that quantum interference effects between $\ket{\xi_i(t)}$ and $\ket{\xi_j(t)}$ will then tend to zero. Thus, when it comes to observables defined on $\mathcal{V}$, using equations (\ref{reducedev}) and (\ref{reducedv}), we can calculate the expectation value of an observable $\hat{O}_\mathcal{V}$  as a weighted sum of expectation values for each of the states $\ket{\xi_i(t)}$:
    \begin{equation}\label{manyapprox}\ev*{\hat{O}_\mathcal{U}}_{\Psi(t)}\approx\sum_i |c_i|^2 \ev*{\hat{O}_\mathcal{V}}_{\xi_i(t)}.\end{equation}
    The fact that (\ref{manyapprox}) is only an approximation suggests that the time at which branching occurs is not well-defined. All that we can do is choose a  sufficiently large time interval $\delta$ such that for  $t>t_0+\delta$, the approximation (\ref{manyapprox}) meets our desired level of accuracy. 
    
    Despite this vagueness on when branching occurs, we can still form a natural and well-defined notion of worlds according to the following definition: \label{rigorousworld} a set $\{W_i: i\}$ is the set of worlds for a universe $\mathcal{U}=\mathcal{V}+\mathcal{E}$ when 
     \begin{enumerate}[noitemsep, nosep, topsep=0pt]
     \item $W_i$ is a description of $\mathcal{U}$ given by $\ket{W_i(t)}=\ket{\xi_i(t)}\ket{E_i(t)}$ where $\ket{\xi_i(t)}$ is a state of $\mathcal{V}$ and $\ket{E_i(t)}$ is a state of $\mathcal{E}$,
     \item $\mathcal{U}$ is in the state $\ket{\Psi(t)}=\sum_i c_i \ket{W_i(t)}$ with all $c_i\neq 0$.
     \item $\ip{\xi_i(t)}{\xi_j(t)}=0$ for $i\neq j$,
     \item $\ip{E_i(t)}{E_j(t)}\rightarrow 0$ as $t\rightarrow\infty$ for $i\neq j$, and the convergence is such that for any observable $\hat{O}_\mathcal{V}$ defined on $\mathcal{V}$, $\ev*{\hat{O}_\mathcal{U}}_{\Psi(t)}\rightarrow\sum_i |c_i|^2 \ev*{\hat{O}_\mathcal{V}}_{\xi_i(t)}$. 
     \end{enumerate}
    Note that according to this definition, the description $\ket{\Psi(t)}$ is rather trivially a world -- we just take the environment $\mathcal{E}$ to be empty so that there would be only one $\ket{E_i(t)}$ which would be the vacuum state. So there is at least one world according to this definition. There is a question of whether there could be more than one world, and this would depend on whether we could really have a non-trivial decomposition $\mathcal{U}=\mathcal{V}+\mathcal{E}$, 
    for the supposition that there is such a decomposition requires that it is possible to distinguish $\mathcal{V}$ and $\mathcal{E}$. But this might not in fact be possible. For instance, if the ultimate fate of the universe was that it would collapse into a singularity, then there would come a point at which it wouldn't be possible to make a distinction between $\mathcal{V}$ and $\mathcal{E}$. But despite this possible concern, the above definition makes it seem plausible that there could be many well-defined worlds $W_i$.\footnote{For the purposes of this chapter, plausibility is enough. I am only trying to show why physicists might find the many-worlds interpretation of quantum physics attractive. I am certainly not trying to show that the many-worlds interpretation is the most convincing and satisfactory interpretation of quantum physics. }
    
    When we look at a particular $\ket{\xi_i(t)}$, it will look like it is describing a fairly classical world with scientists performing their measurements and agreeing about what they measure. And as long as the $\ket{\xi_i(t)}$-states remain pointer states with respect to $\ket{E_i(t)}$, no branching will occur. But typically, a $\ket{\xi_i(t)}$-state will not indefinitely remain a pointer state with respect to $\ket{E_i(t)}$. We can think of how this happens with the Stern-Gerlach experiment. For if one Stern-Gerlach apparatus has its magnetic field orientated in the $\uvb{a}$-direction, then $\ket{\uvbp{a}}$ and $\ket{\uvbm{a}}$ will be pointer states for a silver atom in the vicinity of this apparatus. But if the same silver atom then travels onward to another Stern-Gerlach apparatus with its magnetic field now orientated in the $\uvb{b}$-direction with $\uvb{b}\neq\uvb{a}$, $\ket{\uvbp{a}}$ and $\ket{\uvbm{a}}$ will no longer be pointer states with respect to their environment, and so branching will occur. But this is not necessarily a problem for the definition of many-worlds given above on page \pageref{rigorousworld}, for when a $\ket{\xi_i(t)}$-state does not indefinitely remain a pointer state with respect to $\ket{E_i(t)}$, we can just rewrite $\ket{\xi_i(t)}$ as a sum of pointer states $\ket{\xi_{ij}(t)}$ and $\ket{E_i(t)}$  as a sum of their respective environments $\ket{E_{ij}(t)}$, and then $\ket{E_i(t)}$ will be like a ready state for the $\ket{E_{ij}(t)}$. Assuming we can do this so that the $\ket{\xi_{ij}(t)}$ are orthogonal to the $\ket{\xi_{i'j'}(t)}$ when $i'\neq i$ or $j'\neq j$, then we would still be able to have well-defined worlds according to the definition given above. 




   
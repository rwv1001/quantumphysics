
\section{Outcome Independence versus Parameter Independence}\label{OIVI}
To explain Shimony's\footnote{See \cite[146-147]{Shimony86} and \cite[7-9]{Butterfield}.}  notion of  Outcome Independence and Parameter Independence, we suppose we have an experimental setup similar to the experimental setup described in section \ref{BellSection} on Bell's Inequality. Thus, we suppose there are two particles labeled $q_A$, and $q_B$, and that a measurement can be made on particle  $q_A$ at one location (e.g. Alice's laboratory), and a measurement can be made on particle $q_B$ at some other location (e.g. Bob's laboratory). Alice can make a choice of one of $n$ measurements to be made. These are labeled $a_1,\ldots, a_n$. For example, $a_1$ might be a measurement of $q_A$'s spin along the $z$-axis, whereas $a_2$ might be the measurement of $q_A$'s spin along an axis that is at  a $45^\circ$ angle to the $z$-axis etc. We use the variable $x$ to denote Alice's choice so that $x=a_i$ for some $i\in\{1,\ldots,n\}$. If Alice chooses to make measurement $a_i$ (i.e. $x=a_i$), the measurement outcome is labeled $A_i$, and this outcome can take values $+1$ or $-1$. For example, Alice could use the convention in which $+1$ corresponds to a spin up outcome, and $-1$ corresponds to a spin down outcome. We will use the variable $X$ to denote the measurement outcome Alice obtains, so for example, if Alice chooses to make the $a_1$ measurement so that $x=a_1$ and obtains the outcome $A_1=1$, then $X=1$. Similarly, we use the notation $b_i,\, y$, and $B_i,\, Y$ to correspond to the measurement choices and measurement outcomes for Bob.

We now suppose that there is a complete state $\lambda\in\Lambda$ describing both $q_A$ and $q_B$ that is independent of Alice and Bob's measurement choices, but that encodes all other features that would influence the corresponding measurement outcomes. Here, the domain $\Lambda$ of all such complete states will depend on how the two particles are prepared and the model we are assuming. We also assume that  $q_A$ and $q_B$ are initially coupled together in such a way that Alice and Bob would always get opposite results when they made their measurements in the same direction. For instance, for $n=3$, we might assume a model in which 
\begin{equation}\label{bellLambda}
\Lambda=\big\{(A_1, A_2, A_3 ,B_1, B_2, B_3):A_1,\, A_2,\, A_3=\pm1,\, B_i=-A_i\big\}.
\end{equation}
 In this case, $\lambda\in\Lambda$ would fully determine Alice and Bob's measurement outcomes. This would be like the model described in the proof of Bell's Inequality with all the states of $\Lambda$ being described in table \ref{hiddentable} of section \ref{BellSection}. 
 
However, in general, we don't insist on such determinism. Rather, we suppose that given a complete state $\lambda\in\Lambda$, and given that Alice makes a measurement choice $x$ and Bob makes a measurement choice $y$, then there will be a probability $P_{\lambda,x,y}(X , Y)$ representing the probability Alice gets outcome $X$ and Bob gets outcome $Y$. It is only in deterministic models that $P_{\lambda,x,y}(X , Y)$ will only have values restricted to either $0$ or $1$. In non-deterministic models, there will have to be some situations when $P_{\lambda,x,y}(X, Y)$ has a value strictly between $0$ and $1$. For example, if our model was standard quantum mechanics, we could take $\lambda$ to be the Bell state (\ref{bell}). Then it follows from equation (\ref{bellstate2}) that as long as Alice's and Bob's measurement choices $x$ and $y$ are in the same direction, then $P_{\lambda,x,y}(1,-1)=1/2$. Incidentally, we also note that equation (\ref{bellstate2}) implies the domain $\Lambda$ consists of a single state:
\begin{equation}\label{quantumLambda}
\Lambda=\Bigg\{\frac{1}{\sqrt{2}}\big(\ket*{\uvbp{a}}_A\ket*{\uvbm{a}}_B-\ket*{\uvbm{a}}_A\ket*{\uvbp{a}}_B\big)\Bigg\}.
\end{equation}

In both models (\ref{bellLambda}) and (\ref{quantumLambda}), we see that if we define
\begin{align}
P_{A, \lambda,x,y}(X)&=P_{\lambda,x,y}(X, 1)+P_{\lambda,x,y}(X, -1),\label{PIone}\\
P_{B, \lambda,x,y}(Y)&=P_{\lambda,x,y}(1, Y)+P_{\lambda,x,y}(-1, Y),\label{PItwo}
\end{align}
then $P_{A,\lambda,x,y}(X)$ is independent of Bob's choice of measurement $y$, and $P_{B,\lambda,x,y}(Y)$ is independent of Alice's choice of measurement $x$.\footnote{To see that this is true for model (\ref{bellLambda}), it is obvious that $P_{A, \lambda,x,y}(X)=1$ or 0 regardless of what $y$ is. As for model (\ref{quantumLambda}), it is straightforward to show\label{onehalf} that $P_{A, \lambda,x,y}(X)=1/2$ and $P_{B,\lambda,x,y}(Y)=1/2$ for any $X,\, Y$. E.g. for $x=\vb{\hat{a}}$ and $y=\vb{\hat{b}}$, by (\ref{bellstate2}), we can assume the two particles are in the state 
$$\ket{\zeta}=\frac{1}{\sqrt{2}}\big(\ket*{\uvbp{b}}_A\ket*{\uvbm{b}}_B-\ket*{\uvbm{b}}_A\ket*{\uvbp{b}}_B\big).$$
Since the inner product on the composite system is given by  $\ip{\xi'}{\xi}=\ip{\psi'}{\psi}_A\ip{\chi'}{\chi}_B$ for $\ket{\xi}=\ket{\psi}_A\ket{\chi}_B$ and $\ket{\xi'}=\ket{\psi'}_A\ket{\chi'}_B$, it follows that 
$$\prescript{}{A}{ \bra*{\vb{\hat{a}+}}}\prescript{}{B}{\ip*{\vb{\hat{b}\pm}}{\zeta}}=\mp\frac{1}{\sqrt{2}}\ip*{\vb{\hat{a}+}}{\vb{\hat{b}\mp}}_A.$$ 
Therefore, by the Born Rule (see page \pageref{bornrule})
$$P_{\lambda,\vb{\hat{a}},\vb{\hat{b}}}(\vb{\hat{a}+}, \vb{\hat{b}+})+P_{\lambda,\vb{\hat{a}},\vb{\hat{b}}}(\vb{\hat{a}+}, \vb{\hat{b}-})=\frac{1}{2}\abs*{\ip*{\vb{\hat{a}+}}{\vb{\hat{b}-}}_A}^2+\frac{1}{2}\abs*{\ip*{\vb{\hat{a}+}}{\vb{\hat{b}+}}_A}^2.$$
But since
$$\ket*{\vb{\hat{a}+}}_A=\ip*{\vb{\hat{b}+}}{\vb{\hat{a}+}}_A\ket*{\vb{\hat{b}+}}_A+\ip*{\vb{\hat{b}-}}{\vb{\hat{a}+}}_A\ket*{\vb{\hat{b}-}}_A$$
it follows that 
$$\abs*{\ip*{\vb{\hat{a}+}}{\vb{\hat{b}+}}_A}^2+\abs*{\ip*{\vb{\hat{a}+}}{\vb{\hat{b}-}}_A}^2 =1. $$
Therefore, 
$$P_{\lambda,\vb{\hat{a}},\vb{\hat{b}}}(\vb{\hat{a}+}, \vb{\hat{b}+})+P_{\lambda,x,y}(\vb{\hat{a}+}, \vb{\hat{b}-})=\frac{1}{2}.$$
 } In other models, however, it's possible that such independence does not hold. So to distinguish between such possibilities, we say a model satisfies \textbf{Parameter Independence} (PI) \label{PIdef} if and only if $P_{A,\lambda,x,y}(X)$ is independent of $y$, and $P_{B,\lambda,x,y}(Y)$ is independent of $x$. In other words, PI holds if and only if (\ref{PIone}) and (\ref{PItwo}) hold for all $\lambda,$ $x,$ $y,$ $X,$ and $Y$.

One model in which PI fails to hold is the \textbf{pilot wave interpretation} of quantum mechanics. In this interpretation, it is assumed that at any instant of time $t$, the particles $q_A$ and $q_B$ will have definite positions $\vb{x}_A$ and $\vb{x}_B$ and definite momenta $\vb{p}_A$ and $\vb{p}_B$ respectively. But in addition to the positions and momenta of the particles, it is also assumed that there is a so-called \textbf{pilot wave} 
\begin{equation}
\psi(\vb{x}_A, \vb{x}_B, t)=r(\vb{x}_A, \vb{x}_B, t)e^{i S(\vb{x}_A, \vb{x}_B, t)} 
\end{equation}
where $r(\vb{x}_A, \vb{x}_B, t)>0$ is the magnitude of $\psi(\vb{x}_A, \vb{x}_B, t)$, and the real-valued function $S(\vb{x}_A, \vb{x}_B,t )$ is the complex phase of $\psi(\vb{x}_A, \vb{x}_B, t).$
The time evolution of the pilot wave is deterministically governed by the Schr\"{o}dinger equation, and the phase $S(\vb{x}_A, \vb{x}_B, t )$ relates the positions $\vb{x}_A$ and $\vb{x}_B$ to the momenta $\vb{p}_A$ and $\vb{p}_B$ via the gradient of $S$:
\begin{equation}
\vb{p}_A=\grad_A{S}(\vb{x}_A, \vb{x}_B),\qquad
\vb{p}_B=\grad_B{S}(\vb{x}_A, \vb{x}_B).
\end{equation}
In other words, if we fix $\vb{x}_B$ and consider $S$ to be just a function of $\vb{x}_A$, then the momentum $\vb{p}_A$ is in the direction and has the magnitude of the steepest ascent of $S$ considered as a function of $\vb{x}_A$. The momentum $\vb{p}_B$ is determined similarly. 

In reality, we don't know the exact positions of all the particles, but based on what we know about an experimental setup, we can average over our uncertainty and recover exactly the same predictions that quantum mechanics would make.\footnote{See \cite{BohmDavid1952A} and \cite{BohmDavid1952B}.} So for instance, our knowledge of the experimental setup above should enable us  to know both that $q_A$ and $q_B$ are contained within a region $V,$ and also for us to work out the probability  $p(V_i, V_j)$ that particle $q_A$  will be in a region $V_i$, and $q_B$ will be in a region $V_j,$ where the $V_i$ are small non-overlapping regions such that $V=\bigcup_iV_i$. If we are interested in some physical quantity $O(\vb{x}_A, \vb{x}_B)$ that depends on the positions $\vb{x}_A$ and $\vb{x}_B$ of the two particles, then when the regions $V_i$ are  sufficiently small so that $O(\vb{x}_i, \vb{x}_j)$ varies negligibly for any $\vb{x}_i\in V_i$ and $\vb{x}_j\in V_j$,  the average value
\begin{equation}\label{bohmconsistency}
\ev{O}=\sum_{i,j}p(V_i, V_j)O(\vb{x}_i, \vb{x}_j)
\end{equation}
calculated in the pilot wave model turns out to be the same as the expectation value for $O$ predicted by standard quantum mechanics.\footnote{In this explanation, I've refrained from using measure theory, but basically this explanation is saying that we when we construct a measure $\mu$ on $V\times V$ based on our knowledge of the experimental setup,  $\int_{V\times V}O(\vb{x}_i, \vb{x}_j)\dd\mu$ will be the same as the expectation value for $O$ predicted by standard quantum physics. }

To see why PI fails to hold in the pilot wave model, we first note that since the pilot wave model makes the same predictions as quantum mechanics when averaged over all the hidden variables, the violation of Bell's inequality (\ref{bellinequality}) implies there must be some hidden variable $\lambda$ and choices of measurement directions $\bm{\hat{a}}$, $\bm{\hat{b}}$, and $\bm{\hat{c}}$ such that 
 \begin{equation}\label{bellinequality2}
P_{\lambda,\bm{\hat{a}},\bm{\hat{b}}}(\uvbp{a};\uvbp{b})> P_{\lambda,\bm{\hat{a}},\bm{\hat{c}}}(\uvbp{a};\uvbp{c})+P_{\lambda,\bm{\hat{c}},\bm{\hat{b}}}(\uvbp{c};\uvbp{b}).
\end{equation}
Since physics in the pilot wave model is deterministic, probabilities must be either $0$ or $1$. Therefore, the only way (\ref{bellinequality2}) can be satisfied is for
 \begin{align}
P_{\lambda,\bm{\hat{a}},\bm{\hat{b}}}(\uvbp{a};\uvbp{b})&=1, \label{PDproof1}\\
P_{\lambda,\bm{\hat{a}},\bm{\hat{c}}}(\uvbp{a};\uvbp{c})&=0,\label{PDproof2}\\
P_{\lambda,\bm{\hat{c}},\bm{\hat{b}}}(\uvbp{c};\uvbp{b})&=0. \label{PDproof3}
 \end{align}
 We suppose that PI holds, and we will try to arrive at a contradiction. If both Alice and Bob make their measurement in the $\bm{\hat{c}}$-direction, there are two possibilities: either Alice measures $q_A$ to be in the state $\uvbm{c}$ and Bob measures $q_B$ to be in the state $\uvbp{c}$, or Alice measures $q_A$ to be in the state $\uvbp{c}$ and Bob measures $q_B$ to be in the state $\uvbm{c}.$ So expressed in terms of probabilities, these two possibilities are equivalent to either 
 \begin{equation}\label{PDproofcase1}
 P_{\lambda,\bm{\hat{c}},\bm{\hat{c}}}(\uvbm{c};\uvbp{c})=1 \qquad\text{and}\qquad  P_{\lambda,\bm{\hat{c}},\bm{\hat{c}}}(\uvbp{c};\uvbm{c})=0.
 \end{equation}
 or 
  \begin{equation}\label{PDproofcase2}
 P_{\lambda,\bm{\hat{c}},\bm{\hat{c}}}(\uvbp{c};\uvbm{c})=1 \qquad\text{and}\qquad P_{\lambda,\bm{\hat{c}},\bm{\hat{c}}}(\uvbm{c};\uvbp{c})=0
 \end{equation}
 Let's first consider case (\ref{PDproofcase1}). Note that
\begin{equation}
 P_{\lambda,\bm{\hat{c}},\bm{\hat{c}}}(\uvbp{c};\uvbm{c})+P_{\lambda,\bm{\hat{c}},\bm{\hat{c}}}(\uvbm{c};\uvbm{c})=0.
\end{equation}
 Therefore, since we are assuming PI, 
 \begin{equation}
 P_{\lambda,\bm{\hat{a}},\bm{\hat{c}}}(\uvbp{a};\uvbm{c})+P_{\lambda,\bm{\hat{a}},\bm{\hat{c}}}(\uvbm{a};\uvbm{c})=0.
\end{equation}
 In particular, 
  \begin{equation}\label{PDproof4}
  P_{\lambda,\bm{\hat{a}},\bm{\hat{c}}}(\uvbp{a};\uvbm{c})=0.
  \end{equation}
  But by (\ref{PDproof1}), we know that 
  \begin{equation}
  P_{\lambda,\bm{\hat{a}},\bm{\hat{b}}}(\uvbp{a};\uvbp{b})+P_{\lambda,\bm{\hat{a}},\bm{\hat{b}}}(\uvbp{a};\uvbm{b})=1,
  \end{equation}
  so using this together with PI, we must have
    \begin{equation}\label{PDproof5}
  P_{\lambda,\bm{\hat{a}},\bm{\hat{c}}}(\uvbp{a};\uvbp{c})+P_{\lambda,\bm{\hat{a}},\bm{\hat{c}}}(\uvbp{a};\uvbm{c})=1.
  \end{equation}
 But by (\ref{PDproof2}) and (\ref{PDproof4})
\begin{equation}\label{PDproof6}
  P_{\lambda,\bm{\hat{a}},\bm{\hat{c}}}(\uvbp{a};\uvbp{c})+P_{\lambda,\bm{\hat{a}},\bm{\hat{c}}}(\uvbp{a};\uvbm{c})=0.
\end{equation}
Since (\ref{PDproof5}) contradicts (\ref{PDproof6}), the assumption (\ref{PDproofcase1}) must be false if PI is to hold.

So we now consider the alternative case when (\ref{PDproofcase2}) holds. We will again see that this assumption leads to a contradiction. First note that
\begin{equation}
 P_{\lambda,\bm{\hat{c}},\bm{\hat{c}}}(\uvbm{c};\uvbp{c})+P_{\lambda,\bm{\hat{c}},\bm{\hat{c}}}(\uvbm{c};\uvbm{c})=0.
\end{equation}
 By PI
\begin{equation}
P_{\lambda,\bm{\hat{c}},\bm{\hat{b}}}(\uvbm{c};\uvbp{b})+P_{\lambda,\bm{\hat{c}},\bm{\hat{b}}}(\uvbm{c};\uvbm{b})=0.
\end{equation}
In particular, 
\begin{equation}\label{PDproof8}
P_{\lambda,\bm{\hat{c}},\bm{\hat{b}}}(\uvbm{c};\uvbp{b})=0.
\end{equation}
   But by (\ref{PDproof1}), we know that 
  \begin{equation}
  P_{\lambda,\bm{\hat{a}},\bm{\hat{b}}}(\uvbp{a};\uvbp{b})+P_{\lambda,\bm{\hat{a}},\bm{\hat{b}}}(\uvbm{a};\uvbp{b})=1,
  \end{equation}   
 so using this together with PI, we must have
    \begin{equation}\label{PDproof7}
  P_{\lambda,\bm{\hat{c}},\bm{\hat{b}}}(\uvbp{c};\uvbp{b})+P_{\lambda,\bm{\hat{c}},\bm{\hat{b}}}(\uvbm{c};\uvbp{b})=1.
  \end{equation} 
   But by (\ref{PDproof3}) and (\ref{PDproof8})
\begin{equation}\label{PDproof9}
 P_{\lambda,\bm{\hat{c}},\bm{\hat{b}}}(\uvbp{c};\uvbp{b})+P_{\lambda,\bm{\hat{c}},\bm{\hat{b}}}(\uvbm{c};\uvbp{b})=0.
\end{equation}
Since (\ref{PDproof7}) contradicts (\ref{PDproof9}), the assumption (\ref{PDproofcase2}) must also be false if PI is to hold. So we can only conclude that PI fails to hold in the pilot wave model.   But we can conclude even more than that: any deterministic hidden variable model that gives the same predictions as quantum mechanics when averaged over the hidden variables must violate PI.\label{PIdeterminism} 

Now the violation of PI in the pilot wave model does not sit easily with Einstein's theory of relativity, for according to Einstein's theory, it should be impossible to send signals faster than the speed of light. However, if PI is violated, then if Alice happened to know what $\lambda$ was for each run of the experiment, and if Bob made the same measurement, then because the distribution of Alice's outcomes will depend on Bob's choice of measurement, with enough runs of the experiment, Alice should be able to work out what measurement Bob is making. And this should be possible even if Alice and Bob are separated by many light years. So it seems faster than light communication would be possible. The only thing preventing such communication would be Alice's lack of knowledge of $\lambda$. 

But although a PI violation can account for the violation of Bell's Inequality, this is not the only possible culprit to consider. Another assumption of Bell's Inequality that might be violated is \textbf{Outcome Independence} (OI). Outcome independence is the assumption
\begin{equation}\label{OI}
P_{\lambda,x,y}(X,Y)=P_{A,\lambda,x,y}(X)\cdot P_{B,\lambda,x,y}(Y),
\end{equation}
We can see that if OI holds in any model which gives the same predictions as standard quantum theory when averaged over the hidden variables, then PI must be violated in such a model. For if both PI and OI hold, then for any measurement choices $\vb{\hat{a}},$ $\vb{\hat{b}},$ and $\vb{\hat{c}}$, and hidden variable $\lambda$, we have
\begin{equation}\label{OIPI1}
\begin{split}
 P_{\lambda,\bm{\hat{a}},\bm{\hat{c}}}&(\uvbp{a};\uvbp{c})=P_{A,\lambda,\bm{\hat{a}},\bm{\hat{c}}}(\uvbp{a})\cdot P_{B,\lambda,\bm{\hat{c}},\bm{\hat{a}}}(\uvbp{c})\\
 &= \Big( P_{\lambda,\bm{\hat{a}},\bm{\hat{c}}}(\uvbp{a};\uvbp{c})+ P_{\lambda,\bm{\hat{a}},\bm{\hat{c}}}(\uvbp{a};\uvbm{c})\Big)\cdot\Big( P_{\lambda,\bm{\hat{a}},\bm{\hat{c}}}(\uvbp{a};\uvbp{c})+ P_{\lambda,\bm{\hat{a}},\bm{\hat{c}}}(\uvbm{a};\uvbp{c})\Big)\\
 &= \Big( P_{\lambda,\bm{\hat{a}},\bm{\hat{c}}}(\uvbp{a};\uvbp{c})+ P_{\lambda,\bm{\hat{a}},\bm{\hat{c}}}(\uvbp{a};\uvbm{c})\Big)\cdot\Big( \cancelto{0}{P_{\lambda,\bm{\hat{c}},\bm{\hat{c}}}(\uvbp{c};\uvbp{c})}+ P_{\lambda,\bm{\hat{c}},\bm{\hat{c}}}(\uvbm{c};\uvbp{c})\Big)\\
 &=\Big( P_{\lambda,\bm{\hat{a}},\bm{\hat{b}}}(\uvbp{a};\uvbp{b})+ P_{\lambda,\bm{\hat{a}},\bm{\hat{b}}}(\uvbp{a};\uvbm{b})\Big)\cdot P_{\lambda,\bm{\hat{c}},\bm{\hat{c}}}(\uvbm{c};\uvbp{c})\\
 &\geq P_{\lambda,\bm{\hat{a}},\bm{\hat{b}}}(\uvbp{a};\uvbp{b})\cdot P_{\lambda,\bm{\hat{c}},\bm{\hat{c}}}(\uvbm{c};\uvbp{c}) 
\end{split}
\end{equation}
Similarly, we have
\begin{equation}\label{OIPI2}
\begin{split}
 P_{\lambda,\bm{\hat{c}},\bm{\hat{b}}}&(\uvbp{c};\uvbp{b})=P_{A,\lambda,\bm{\hat{c}},\bm{\hat{b}}}(\uvbp{c})\cdot P_{B,\lambda,\bm{\hat{c}},\bm{\hat{b}}}(\uvbp{b})\\
 &= \Big( P_{\lambda,\bm{\hat{c}},\bm{\hat{b}}}(\uvbp{c};\uvbp{b})+ P_{\lambda,\bm{\hat{c}},\bm{\hat{b}}}(\uvbp{c};\uvbm{b})\Big)\cdot\Big( P_{\lambda,\bm{\hat{c}},\bm{\hat{b}}}(\uvbp{c};\uvbp{b})+ P_{\lambda,\bm{\hat{c}},\bm{\hat{b}}}(\uvbm{c};\uvbp{b})\Big)\\
 &= \Big(  \cancelto{0}{P_{\lambda,\bm{\hat{c}},\bm{\hat{c}}}(\uvbp{c};\uvbp{c})}+ P_{\lambda,\bm{\hat{c}},\bm{\hat{c}}}(\uvbp{c};\uvbm{c})\Big)\cdot\Big(P_{\lambda,\bm{\hat{c}},\bm{\hat{b}}}(\uvbp{c};\uvbp{b})+ P_{\lambda,\bm{\hat{c}},\bm{\hat{b}}}(\uvbm{c};\uvbp{b})\Big)\\
 &=P_{\lambda,\bm{\hat{c}},\bm{\hat{c}}}(\uvbp{c};\uvbm{c})\cdot\Big( P_{\lambda,\bm{\hat{a}},\bm{\hat{b}}}(\uvbp{a};\uvbp{b})+ P_{\lambda,\bm{\hat{a}},\bm{\hat{b}}}(\uvbm{a};\uvbp{b})\Big)\\
 &\geq P_{\lambda,\bm{\hat{c}},\bm{\hat{c}}}(\uvbp{c};\uvbm{c})\cdot P_{\lambda,\bm{\hat{a}},\bm{\hat{b}}}(\uvbp{a};\uvbp{b}) .
\end{split}
\end{equation}
But since the hidden variable $\lambda$ is assumed to be independent of Alice and Bob's measurement, and since Alice and Bob will always get opposite results when they make the same choice of measurement, it follows that 
\begin{equation}\label{OIPI3}
P_{\lambda,\bm{\hat{c}},\bm{\hat{c}}}(\uvbp{c};\uvbm{c})+P_{\lambda,\bm{\hat{c}},\bm{\hat{c}}}(\uvbm{c};\uvbp{c})=1
\end{equation}
Therefore, putting (\ref{OIPI1}), (\ref{OIPI2}), and (\ref{OIPI3}) together, we have
\begin{equation}
P_{\lambda,\bm{\hat{a}},\bm{\hat{c}}}(\uvbp{a};\uvbp{c})+P_{\lambda,\bm{\hat{c}},\bm{\hat{b}}}(\uvbp{c};\uvbp{b})\geq P_{\lambda,\bm{\hat{a}},\bm{\hat{b}}}(\uvbp{a};\uvbp{b}).
\end{equation}
We have thus proved that OI and PI implies Bell's Inequality (\ref{bellinequality2}). But since Bell's Inequality does not hold in reality, it follows that if OI is always true, then PI must be violated.\label{OIPIproofend}


\label{OIdet}In the case of deterministic models, OI necessarily holds. To see why, we first note that for deterministic models, either $P_{\lambda,x,y}(X,Y)=1$ or $P_{\lambda,x,y}(X,Y)=0$. When $P_{\lambda,x,y}(X,Y)=1$, then by (\ref{PIone}), $P_{A, \lambda,x,y}(X)=1$, and by (\ref{PItwo}), $P_{B, \lambda,x,y}(Y)=1$, so (\ref{OI}) is seen to hold in this case. On the other hand, when $P_{\lambda,x,y}(X,Y)=0$,  if $P_{A, \lambda,x,y}(X)=1$, then by (\ref{PIone}), $P_{\lambda,x,y} (X,-Y)=1$ so that by (\ref{PItwo}), $P_{B, \lambda,x,y}(Y)=0$ in which case (\ref{OI}) holds. And similarly, if $P_{B, \lambda,x,y}(Y)=1$, by (\ref{PItwo}), $P_{\lambda,x,y} (-X,Y)=1$ so that by (\ref{PIone}), $P_{A, \lambda,x,y}(X)=0$, so again (\ref{OI}) holds. And (\ref{OI}) obviously holds when $P_{A, \lambda,x,y}(X)=P_{B, \lambda,x,y}(Y)=0$. It therefore follows that OI holds in any deterministic model.

When it comes to standard quantum mechanics, however, OI fails to hold. For instance, if $x=y=\vb{\hat{a}}$, then $P_{\lambda,\vb{\hat{a}},\vb{\hat{a}}}(\vb{\hat{a}+},\vb{\hat{a}+})=0,$ but $P_{A, \lambda,\vb{\hat{a}},\vb{\hat{a}}}(\vb{\hat{a}+})=P_{B, \lambda,\vb{\hat{a}},\vb{\hat{a}}}(\vb{\hat{a}+})=1/2.$\footnote{See footnote \ref{onehalf}. } Hence, OI fails. Nevertheless, as long as PI holds, the failure of OI does not enable Bob to send messages to Alice faster than light because Bob only has control over the measurement he makes. Assuming Bob's mental states have no effect on the measurement outcome, there is nothing he can do to influence his outcome, so although Alice will be able to work out Bob's measurement outcome if she already happens to know which choice of measurement he has made, she will not be able to work out which measurement Bob makes (or even whether Bob has made a measurement at all) by measuring the outcome of her particle. For Shimony\footnote{See \cite[146-147]{Shimony86}.} this inability to send super-luminal messages between Alice and Bob when PI holds and OI is violated was deemed sufficient for the theories of standard quantum physics and special relativity to peacefully coexist. 

However, Butterfield is not satisfied with Shimony's solution to peaceful coexistence.\footnote{See \cite[p. 12]{Butterfield}.} Firstly, he notes that proofs of non-super-luminal signaling\footnote{e.g. see \cite[p. 113--116]{Redhead}; \cite[p. 139--140]{Hiley}} make no assumptions about spacetime locations. One would have thought that any proof that super-luminal signalling between two points is impossible would have to show that a signal cannot be transmitted from one point to the other in less time than the time it takes light to travel between the two points. But if nothing is said about the location of these two points or what is so special about the speed of light compared to the speed of any other particle, then there does not seem to be enough information in the premises to draw the desired conclusion that super-luminal signaling is impossible in quantum physics.

Secondly,  Butterfield notes that Shimony thinks peaceful coexistence of quantum physics and special relativity is guaranteed by the denial of OI and the acceptance of PI, but OI itself depends on the (often) rather vague notion of what an outcome really is. For instance, in the Many-Worlds interpretation described in sections \ref{manyworldsinterpretation1} and \ref{manyworldsinterpretation2}, it is not clear that there are any outcomes at all. Rather, there is just a universal wave function that tells us the probability of certain outcomes, if there were such things as outcomes -- it doesn't tell us that there really are any outcomes. But as we saw in the previous chapter, the Many-Worlds interpretation does flow rather naturally from the postulates of standard quantum theory. 

Still, the notion of what an outcome is doesn't have to be vague. In the pilot wave interpretation of quantum physics, it is very clear what an outcome of an experiment is since all the particles have definite positions and momenta. Because of this, the pointers and displays of measuring devices which are made up of particles will have definite readouts which will correspond to the definite positions of particles being measured (assuming the measurement device is working properly). So unlike the Many-Worlds interpretation, measurements in the pilot wave interpretation have definite outcomes, and hence there is only a single world in the pilot wave interpretation of quantum physics. But as we've just seen, the problem with the pilot wave interpretation is the violation of PI.  

Thus, a satisfactory account of the peaceful coexistence of quantum physics and special relativity requires an interpretation of quantum physics in which not only PI holds, but also an interpretation of quantum physics that has special relativity built into it (thus satisfying Butterfield's first objection), and in which we can make sense of what it means to be an outcome (thus satisfying Butterfield's second objection). To fully address Butterfield's first objection would require quantum field theory, and this would be beyond the scope of this dissertation. But a more modest aspiration that would go some way to address Butterfield's first objection would be to insist on an interpretation of quantum physics that has a property known as Lorentz invariance. This provides a motivation for the consideration of Kent's interpretation of quantum physics that has this property of Lorentz invariance.
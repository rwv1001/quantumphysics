\section{The Preferred Basis Problem\protect\footnotemark}
\footnotetext{See \cite[53-55]{Schlosshauer}.}Now just because we can have an observable $\hat{O}$, there is no guarantee that there is a measuring device that could determine whether the system was in one of the eigenstates of $\hat{O}$. For instance, if $\ket{\text{Cat Alive}}$ is the physical state in which a cat is alive, and $\ket{\text{Cat Dead}}$ is the physical state in which the same cat is dead, then although there are measuring devices that can distinguish between the $\ket{\text{Cat Alive}}$-state and the $\ket{\text{Cat Dead}}$-state,\footnote{For example, we assume that human beings can be thought of as such measuring devices.} there are no known measuring devices that can distinguish between the $\frac{1}{\sqrt{2}}(\ket{\text{Cat Alive}}+\ket{\text{Cat Dead}})$-state and the $\frac{1}{\sqrt{2}}(\ket{\text{Cat Alive}}-\ket{\text{Cat Dead}})$-state. On the other hand, there are measuring devices that can distinguish between the $\frac{1}{\sqrt{2}}(\ket{\uvbp{a}}+\ket{\uvbm{a}})$-state and the $\frac{1}{\sqrt{2}}(\ket{\uvbp{a}}-\ket{\uvbm{a}})$-state in a Stern-Gerlach experiment. Why the difference?

This question is at the heart of the preferred basis problem. As mentioned already, a basis is just a set of states via which all other states of the system can be uniquely expressed. For instance, we can express the state $\frac{1}{\sqrt{2}}(\ket{\text{Cat Alive}}+\ket{\text{Cat Dead}})$ uniquely as a sum of elements from the basis $\{\ket{\text{Cat Alive}}, \ket{\text{Cat Dead}}\}$, and thus we think of $\frac{1}{\sqrt{2}}(\ket{\text{Cat Alive}}+\ket{\text{Cat Dead}})$ as being a superposition of the $\ket{\text{Cat Alive}}$ and $\ket{\text{Cat Dead}}$ basis states. However, we can also uniquely express  $\ket{\text{Cat Alive}}$ in terms of the basis $\{\frac{1}{\sqrt{2}}(\ket{\text{Cat Alive}}+\ket{\text{Cat Dead}}), \frac{1}{\sqrt{2}}(\ket{\text{Cat Alive}}-\ket{\text{Cat Dead}})\}.$\footnote{i.e. $\ket{\text{Cat Alive}}= \frac{1}{\sqrt{2}}\Big(\frac{1}{\sqrt{2}}(\ket{\text{Cat Alive}}+\ket{\text{Cat Dead}})\Big)+\frac{1}{\sqrt{2}}\Big( \frac{1}{\sqrt{2}}(\ket{\text{Cat Alive}}-\ket{\text{Cat Dead}})\Big). $} Nevertheless, we would not tend to think of $\ket{\text{Cat Alive}}$ as being in a superposition of the $\frac{1}{\sqrt{2}}(\ket{\text{Cat Alive}}+\ket{\text{Cat Dead}}) $ and $\frac{1}{\sqrt{2}}(\ket{\text{Cat Alive}}-\ket{\text{Cat Dead}})$ basis states. That is, we have a preference for the basis $\{\ket{\text{Cat Alive}}, \ket{\text{Cat Dead}}\}$ over the basis $\{\frac{1}{\sqrt{2}}(\ket{\text{Cat Alive}}+\ket{\text{Cat Dead}}), \frac{1}{\sqrt{2}}(\ket{\text{Cat Alive}}-\ket{\text{Cat Dead}})\}.$ As will be shown in section \ref{sectionPreferredBasis}, decoherence theory offers a very elegant solution to the preferred basis problem.

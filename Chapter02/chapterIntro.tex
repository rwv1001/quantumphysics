\chapter{The Measurement Problem and the Many-Worlds Interpretation\label{measprobchap}}
In the previous chapter we discussed the EPR-Bohm paradox which exhibits the mysterious correlations of measurement outcomes made on spin singlets by two different observers. We saw that the standard Copenhagen interpretation of quantum physics (which posits that upon measurement  an instantaneous state collapse to a definite measurement outcome occurs) must be rejected if one is to accept Einstein's theory of special relativity. Furthermore, we saw that the experimental violation of Bell's inequality implies that a local hidden variables theory must also be rejected. We also discussed Shimony's  suggestion that it would be acceptable to have non-local hidden variables so long as they satisfied Parameter Independence (PI) but failed to satisfy Outcome Independence (OI), since Bell's inequality wouldn't then follow from these two assumptions, and the denial of OI is a sufficiently weak form of non-locality that it doesn't imply superluminal singaling. Thus, Shimony argued that accepting PI and denying OI was sufficient for the peaceful coexistence of quantum physics and special relativity. However, Shimony's solution rests on there being a clear notion of what an outcome is. But the question of what an outcome is and even whether there are such things as measurement outcomes is very controversial and forms an important part of what is known as the measurement problem.  

The measurement problem actually consists of three parts: (1) the preferred basis problem, (2) the nonobservability of interference at the macroscopic level, and (3) the problem of outcomes.\footnote{See \cite[50]{Schlosshauer}} 
Decoherence theory is able to resolve parts (1) and (2) of the measurement problem, and following Schlosshauer, we will explain how this is done, but as we will see (also following Schlosshauer), decoherence theory is not able to resolve part (3) of the measurement problem, the problem of outcomes.

The problem of outcomes arises when we suppose that the quantum state gives a complete description of the system it is describing.\footnote{We need to make a small caveat here -- the example of the quantum states we have described so far such as $\ket*{\uvb{a}}$ and $\ket*{\Psi_{\text{Bell}}}$ only give a complete description of the spin state of a system. These states don't say anything about the position of the system's particles. Nevertheless, we could in principle write down an expression for the state that did include this positional information. We will therefore think of the expressions such as $\ket*{\uvb{a}}$ and $\ket*{\Psi_{\text{Bell}}}$ as abbreviations for quantum states that also include a complete specification of the particle positions which in the context of quantum physics will be a wave function that determines the probability the particles can be detected in a particular region.} Throughout this chapter we will make this assumption, that is, we will assume there are no hidden variables, whether local or otherwise; rather, a quantum state whose evolution is described by the Schr\"{o}dinger equation provides the most complete description possible of a physical system.

In this chapter, we will be discussing the measurement problem in some detail, and in particular, we will be examining the many-worlds interpretation of quantum physics which attempts to sidestep the problem of outcomes by refusing to acknowledge the reality of outcomes. Although the many-worlds interpretation is mathematically appealing, we'll see that from a philosophical point of view, it is woefully inadequate. Hence, any account of the mysterious correlations of the EPR-Bohm paradox (as described in chapter \ref{BellChapter}) that depended on the many-worlds interpretation would be unsatisfactory. Nevertheless, an understanding of the many-worlds interpretation and how it relates to the problem of outcomes will prove helpful when we come to evaluate Kent's theory of quantum physics. 
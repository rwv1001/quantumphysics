\chapter{Evaluating Kent's Solution to the Measurement Problem}


In the previous chapter, we saw how a denial of the Copenhagen interpretation of quantum mechanics and a denial of hidden variables leads fairly naturally to a so-called Many-Worlds interpretation of quantum mechanics. However, the Many-Worlds interpretation seems to be radically opposed to the view that we can make common sense truth claims about the physical world. A strategy among some philosophers of physics who do not wish to endorse the Many-Worlds interpretation is therefore to reexamine the assumptions that lead to Bell's Inequality. One of these assumptions will have to be discarded since Bell's Inequality is experimentally violated. The false assumption that is used to prove Bell's Inequality is sometimes referred to as \textbf{the culprit}. We therefore need to identify the culprit, that is we need to decide which assumption we should discard while keeping in mind that we wish to maintain a theory that is compatible with the experimental findings of quantum physics and special relativity.

Shimony, noticed that there are two key assumptions in the proof of Bell's Inequality that might be identified as the culprit. He refers to one assumption as {Outcome Independence} (OI), and to the other assumption as {Parameter Independence} (PI).\footnote{See \cite[146-147]{Shimony86}.} Shimony argued that if we only denied OI, then the proof of Bell's Inequality would fail to go through. Yet by continuing to assume PI, there is a sense in which Special Relativity is not obviously violated. Shimony therefore thought that denying OI and assuming PI was sufficient to ensure peaceful coexistence between quantum theory and special relativity. In other words, Shimony thought OI was the culprit. Butterfield\footnote{See \cite{Butterfield}.}, however, argues that although PI is a necessary assumption if there is to be peaceful coexistence between quantum theory and special relativity, PI together with the denial of OI independence is not sufficient to ensure peaceful coexistence, the reason being that Shimony offers no account of what an outcome is. As mentioned in the previous chapter, the problem of outcomes is an important issue that d'Espagnat highlights.
It is Shimony's failure to address the problem of outcomes that motivated Butterfield to explore whether Kent's interpretation of quantum physics provides what is lacking in Shimony's account. 

In this chapter, I will present and evaluate Kent's interpretation of quantum physics. In order to provide such an evaluation, it will be necessary show that the predictions of Kent's theory do not contradict the  predictions of quantum theory that have been experimetnally validated. We also need to show that Kent's theory also possesses the symmetries, known as Lorentz invariance, that belong to a theory that is consistent with Einstein's theory of Special Relativity. But we also need to address Shimony's question of identifying the cultprit, and to this end, we will need to explain what is meant by OI and PI. In recent years,  Leegwater et al.\footnote{See \cite{LeegwaterGijs2016Aitf}, \cite{ColbeckRoger2011Neoq}, \cite{ColbeckRoger2012Tcoq}, \cite{LandsmanK2015OtCt}, and \cite{Landsman}.} have also proved an important theorem, the so-called {`Collbeck-Renner Theorem'} regarding PI which suggest that if PI holds together with what is called a {`no conspiracy'} criterion, then this implies standard quantum theory without any additional variables. And finally, we will have to consider the extent to which Kent's interpretation yields a convincing account of what an outcome is.

We will first turn our attention to the notions of OI and PI.

  
    \section{The Problem of Outcomes}\label{probOutcomes}
    In the last two sections we have seen how decoherence theory solves the preferred basis problem and the problem of the nonobservability of interference. However, there is a third fundamental problem in quantum physics which decoherence theory is unable to solve. This is the problem of outcomes. As discussed in subsection \ref{vonNeumannMeasurement}, in the von Neumann measurement scheme, it is supposed that for the measurement of a physical system $\mathcal{S}$ to take place, it must interact with a measuring device $\mathcal{A}$ which together satisfy the conditions \ref{vonNeumannMeasurement1}. and \ref{vonNeumannMeasurement2}. on page \pageref{vonNeumannMeasurement1}. If $\mathcal{S}$ is initially in a superposition of states $\ket{\psi}=\sum_i c_i\ket{s_i}$ then for $\mathcal{A}$ to measure $\mathcal{S}$, it is necessary for the combined system $\mathcal{S}+\mathcal{A}$ to enter into a superposition
    \begin{equation}\label{vNevolution}
    \ket{\psi}\ket{a_r(t_0)}\xrightarrow{\text{time evolution}}\sum_i c_i\ket{s_i}\ket{a_i(t)}.
    \end{equation}
    However, although the evolution described in (\ref{vNevolution}) must take place if $\mathcal{A}$ is to measure $\mathcal{S}$, it is not sufficient. When one takes the partial trace of $\dyad{\psi}$ over $\mathcal{A}$, then according to (\ref{reduced}),
    \begin{equation}\label{vNevolution2}
    \tr_\mathcal{A}(\dyad{\psi})\xrightarrow{\text{time evolution}}\sum_i \abs{c_i}^2\dyad{s_i}.
    \end{equation}
    But as noted on page \pageref{Espagnat}, we cannot give an ignorance interpretation to $\sum_i \abs{c_i}^2\dyad{s_i}$ for as d'Espagnat puts it, this is an improper mixture. When considered together, the system $\mathcal{S}$ and the apparatus $\mathcal{A}$ remain in the superposition described by (\ref{vNevolution}), and so none of the measurement outcomes from the set of possible outcomes $\{\ket{s_i}:i\}$ have actually occurred. The problem of explaining how the composite system   $\mathcal{S}+\mathcal{A}$       goes from being in the state $\sum_i c_i\ket{s_i}\ket{a_i(t)}$ to a state $\ket{s_i}\ket{a_i(t)}$ is known as the \textbf{problem of outcomes}\index{problem of outcomes}.\label{proboutcomes}  
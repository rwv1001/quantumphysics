  
     

      
    \section{Decoherence theory\label{decotheory}\textsuperscript{*}\protect\footnotemark}\renewcommand{\thefootnote}{\fnsymbol{footnote}}\footnotetext[1]{As mentioned in the introduction on page \pageref{asteriskmeaning}, sections marked with an asterisk may be challenging to readers who do not have a mathematics or physics background.}\renewcommand*{\thefootnote}{\arabic{footnote}}\footnotetext{For more details see \cite[ch. 2]{Schlosshauer}.}Before we can show how decoherence theory solves the preferred basis problem, we will first need to look at decoherence theory in general. To understand what's going on in decoherence theory, there are a number of things we need to discuss, namely
    \begin{enumerate}[noitemsep, nosep, topsep=0pt]
    \item composite systems
    \item entanglement
    \item density matrices and traces 
    \item coherence
    \item partial traces and reduced density matrices
    \item the von Neumann measurement scheme
    \item decoherence
    \end{enumerate}
    \subsection{Composite Systems} First we need to consider \textbf{composite systems}\index{composite systems}. We thus assume there is a distinction between what is being measured and the rest of physical reality. We denote the system that is being measured by $\mathcal{S}$ and the rest of physical reality by $\mathcal{E}$. We will refer to $\mathcal{E}$ as the environment, and we will denote the composite system of $\mathcal{S}$ and $\mathcal{E}$ by $\mathcal{U}$. We will often indicate that $\mathcal{U}$ is a composite of systems $\mathcal{S}$ and $\mathcal{E}$ by writing $\mathcal{U}=\mathcal{S}+\mathcal{E}$. The system $\mathcal{S}$ could be something microscopic like a silver atom, or something much bigger such as a cat or even a planet or galaxy.
    
    Now suppose we have an observable (i.e. any Hermitian operator) $\hat{O}_{\mathcal{S}}$ that acts on the Hilbert space $H_\mathcal{S}$ of states of $\mathcal{S}$. As already mentioned, this means that we can find orthonormal eigenstates $\ket{\psi_i}_\mathcal{S}$ of $\hat{O}_{\mathcal{S}}$ and corresponding eigenvalues $o_i$ such that any state $\ket{\psi}_\mathcal{S}\in H_\mathcal{S}$ can be uniquely expressed as a sum $\ket{\psi}_\mathcal{S}=\sum_{i=1}^M\alpha_i\ket{\psi_i}_\mathcal{S}$. We will often include the subscript $\mathcal{S}$ on the ket-vectors in order to make it clear that these ket-vectors belong to the Hilbert space $H_\mathcal{S}$. At other times we will omit these subscripts when it is clear what system we are talking about, but for the time being, we will keep these subscripts in place.
     
      Now let us suppose we have a basis of normalized (but not necessarily orthonormal) states $\{\ket{\chi_i}_\mathcal{E}:i\}$ for the state space $H_\mathcal{E}$ of $\mathcal{E}$. In other words, every state $\ket{\chi}_\mathcal{E}\in H_\mathcal{E}$ can be uniquely expressed as a linear combination $\ket{\chi}_\mathcal{E}=\sum_{i=1}^N\beta_i\ket{\chi_i}_\mathcal{E}$. It is then assumed we will be able to express every state $\ket{\xi}_\mathcal{U}\in H_\mathcal{U}$ of the composite system $\mathcal{U}$ as a linear combination
      \begin{equation}\label{entangled}\ket{\xi}_\mathcal{U}=\sum_{i=1}^M\sum_{j=1}^N\gamma_{i,j}\ket{\psi_i}_\mathcal{S}\ket{\chi_j}_\mathcal{E}.
      \end{equation}
      Thus, we assume there no are emergent physical properties describing the composite system $\mathcal{U}$ that couldn't be expressed in terms of the component subsystems $\mathcal{S}$ and $\mathcal{E}$. 
      The Hilbert space $H_\mathcal{U}$ is  endowed with the bra-ket $\ip{\xi'}{\xi}_\mathcal{U}$ such that if $\ket{\xi}_\mathcal{U}=\ket{\psi}_\mathcal{S}\ket{\chi}_\mathcal{E}$ and $\ket{\xi'}_\mathcal{U}=\ket{\psi'}_\mathcal{S}\ket{\chi'}_\mathcal{E}$, then $\ip{\xi'}{\xi}_\mathcal{U}=\ip{\psi'}{\psi}_\mathcal{S}\ip{\chi'}{\chi}_\mathcal{E}$ where we have again used subscripts to indicate which Hilbert space the bra-ket corresponds to. 
      
    \subsection{Entanglement}  
     By defining the bra-ket on the Hilbert space $H_\mathcal{U}$ as we have done, we are making the assumption that if we define $P(\psi',\chi'|\psi,\chi)$ to be the probability the composite system would be measured to be in the $\ket{\psi'}_\mathcal{S}\ket{\chi'}_\mathcal{E}$-state given that the composite system was known to be in the $\ket{\psi}_\mathcal{S}\ket{\chi}_\mathcal{E}$-state, then $P(\psi',\chi'|\psi,\chi)=P(\psi'|\psi)P(\chi'|\chi).$ A consequence of this assumption is that if the composite system is in the $\ket{\psi}_\mathcal{S}\ket{\chi}_\mathcal{E}$-state, the probability of finding system $\mathcal{S}$ to be in any particular state in $H_\mathcal{S}$ will be independent\footnote{Here we are using the standard probabilistic definition of independence as given by (\ref{indep}) on page \pageref{indep}: two events $X$ and $Y$ are independent if and only if $P(X \text{ and } Y\text{ occur})=P(X\text{ occurs})P(Y\text{ occurs})$} of the state in $H_\mathcal{E}$ describing $\mathcal{E}$. For this reason, when the state $\ket{\xi}_\mathcal{U}\in H_\mathcal{U}$ describing the composite system $\mathcal{U}$ is expressible as a product state $\ket{\xi}_\mathcal{U}=\ket{\psi}_\mathcal{S}\ket{\chi}_\mathcal{E}$,  we say that $\mathcal{S}$ and $\mathcal{E}$ are \textbf{not entangled} with one another. On the other hand, when  the state $\ket{\xi}_\mathcal{U}\in H_\mathcal{U}$ of the composite system $\mathcal{U}$ cannot be expressed as a product state, we say that $\mathcal{S}$ and $\mathcal{E}$ are \textbf{entangled}\index{entangled}.
       For example, if $\ket{\xi}_\mathcal{U}=\frac{1}{\sqrt{2}}(\ket{\psi_1}_\mathcal{S}\ket{\chi_1}_\mathcal{E}+\ket{\psi_2}_\mathcal{S}\ket{\chi_2}_\mathcal{E})$ with $\ket{\psi_1}_\mathcal{S}\not\propto\ket{\psi_2}_\mathcal{S}$ and $\ket{\chi_1}_\mathcal{E}\not\propto\ket{\chi_2}_\mathcal{E}$,\footnote{Here the notation $\ket{\psi_1}_\mathcal{S}\propto\ket{\psi_2}_\mathcal{S}$ means there exists some $\alpha$ such that $\ket{\psi_1}_\mathcal{S}=\alpha\ket{\psi_2}_\mathcal{S}$, in which case $\ket{\xi}_\mathcal{U}=\frac{1}{\sqrt{2}}\ket{\psi_2}_\mathcal{S}(\alpha\ket{\chi_1}_\mathcal{E}+\ket{\chi_2}_\mathcal{E})$. Thus, if  $\ket{\psi_1}_\mathcal{S}\propto\ket{\psi_2}_\mathcal{S}$, then  $\mathcal{S}$ and $\mathcal{E}$ would not be entangled. This is why in the above example, we assume $\ket{\psi_1}_\mathcal{S}\not\propto\ket{\psi_2}_\mathcal{S}$, that is, we assume there is no such $\alpha$ such that $\ket{\psi_1}_\mathcal{S}=\alpha\ket{\psi_2}_\mathcal{S}$, and for the same reason we assume $\ket{\chi_1}_\mathcal{S}\not\propto\ket{\chi_2}_\mathcal{E}$.} then $\mathcal{S}$ and $\mathcal{E}$ would be entangled with one another.
      
     Now given the observable $\hat{O}_{\mathcal{S}}$ acting on $H_\mathcal{S}$, we can  naturally define the observable $\hat{O}_\mathcal{U}$ acting on $H_\mathcal{U}$ so that 
    \begin{equation}\label{extension}\hat{O}_\mathcal{U}\ket{\xi}_\mathcal{U}=\sum_{i=1}^M\sum_{j=1}^N\gamma_{i,j}\hat{O}_\mathcal{S}\ket{\psi_i}_\mathcal{S}\ket{\chi_j}_\mathcal{E}=\sum_{i=1}^M\sum_{j=1}^N\gamma_{i,j}o_i\ket{\psi_i}_\mathcal{S}\ket{\chi_j}_\mathcal{E}.
    \end{equation}
    Just as in equation (\ref{evev}), for a given normalized state $\ket{\xi}_\mathcal{U}\in H_\mathcal{U}$, the expectation value of the observable $\hat{O}_\mathcal{U}$ will be $\ev*{\hat{O}_\mathcal{U}}_\xi=\ev*{\hat{O}_\mathcal{U}}{\xi}_\mathcal{U}$. It is easy to see that if $\ket{\xi}_\mathcal{U}=\ket{\psi}_\mathcal{S}\ket{\chi}_\mathcal{E}$ (i.e. $\mathcal{S}$ and $\mathcal{E}$ are not entangled), then $\ev*{\hat{O}_\mathcal{U}}_\xi=\ev*{\hat{O}_\mathcal{S}}_\psi.$\footnote{\label{untangledobservable}This is because by definition, if $\ket{\xi}_\mathcal{U}=\ket{\psi}_\mathcal{S}\ket{\chi}_\mathcal{E}$ and $\ket{\xi'}_\mathcal{U}=\ket{\psi'}_\mathcal{S}\ket{\chi'}_\mathcal{E},$ then $\ip{\xi'}{\xi}_\mathcal{U}=\ip{\psi'}{\psi}_\mathcal{S}\ip{\chi'}{\chi}_\mathcal{E}$. We will also have $\hat{O}_\mathcal{U}\ket{\xi}=\hat{O}_\mathcal{S}\ket{\psi}_\mathcal{S}\ket{\chi}_\mathcal{E}$. Thus, assuming both $\ket*{\psi}_\mathcal{S}$ and $\ket*{\chi}_\mathcal{E}$ are normalized, we have $\ev*{\hat{O}_\mathcal{U}}_\xi=\ev*{\hat{O}_\mathcal{U}}{\xi}=\ev*{\hat{O}_\mathcal{S}}{\psi}_\mathcal{S}\ip{\chi}{\chi}_\mathcal{E}=\ev*{\hat{O}_\mathcal{S}}_\psi.$ } Thus, when $\mathcal{S}$ and $\mathcal{E}$ are not entangled with one another, it is possible to say things about $\mathcal{S}$ independently of the current state of the environment  $\mathcal{E}$. In this case we need have no knowledge of the information about $\mathcal{E}$ encapsulated in the state $\ket{\chi}_\mathcal{E}$ to determine the expectation value $\ev*{\hat{O}_\mathcal{U}}_\xi. $
    
    However, for a general entangled state $\ket{\xi}_\mathcal{U}=\sum_{i=1}^M\sum_{j=1}^N\gamma_{i,j}\ket{\psi_i}_\mathcal{S}\ket{\chi_j}_\mathcal{E}$,  $\ev*{\hat{O}_\mathcal{U}}_\xi $ will typically depend on the $\ket{\chi_j}_\mathcal{E}$-states and the coefficients $\gamma_{i,j}$. Nevertheless, despite there being a huge amount of information contained within these $\ket{\chi_j}_\mathcal{E}$-states and the  $\gamma_{i,j}$, if we are only interested in making measurements on the system $\mathcal{S}$, nearly all this information can be discarded. In order to see how this is done, we need to generalize the notion of a state to that of a density matrix. 
    \subsection{Density Matrices and Traces}
    Given a normalized state $\ket{\psi}$ in any Hilbert space $H$,  its density matrix will be the operator $\hat{\rho}_\psi\myeq\dyad{\psi}$ which acts on $H$ by sending an arbitrary state $\ket{\psi'}$ to $\ip{\psi}{\psi'}\ket{\psi}.$  Note that $\hat{\rho}_\psi$ is a Hermitian operator.\footnote{This is because for any arbitrary state $\ket{\psi'}$, $\ev*{\hat{\rho}_\psi}{{\psi'}} = \ip*{\psi'}{\psi} \ip*{\psi}{\psi'}=\overline{\ip*{\psi}{\psi'}}\ip*{\psi}{\psi'}=\abs{\ip*{\psi}{\psi'}}^2$, and so $\ev*{\hat{\rho}_\psi}{{\psi'}}$ is real, and from this it follows that $\hat{\rho}_\psi$ is Hermitian. } Also note that if we had a measuring device that returned the output $1$ if a system was in the state $\ket{\psi}$ and $0$ if the system was in a state $\ket{\chi}$ with $\ip{\psi}{\chi}=0$, the density matrix $\hat{\rho}_\psi$ would be the observable corresponding to this measurement. The expectation value of this measurement for an initial normalized state $\ket{\psi'}$ would then be $\ev*{\hat{\rho}_\psi}{\psi'}=\abs{\ip{\psi}{\psi'}}^2=P(\psi|\psi').$ In particular, if the system was initially in the state $\ket{\psi}$, the expectation value of this measurement would be $1$.
    
      Now it turns out that if we have an arbitrary orthonormal basis $\{\ket{\phi_i}:i\}$ of $H$ and any observable $\hat{O}$ on $H$, then 
    \begin{equation}\label{exptrace}\ev*{\hat{O}}_\psi=\sum_{i}\ev*{\hat{\rho}_\psi\hat{O}}{\phi_i}.\protect\footnotemark
    \end{equation}
    \footnotetext{
    To see this, as we saw in footnote \ref{footnote_1}, if we define the mapping $I=\sum_{i=1}^N\dyad{\phi_i}$ then $I\ket{\psi}=\ket{\psi}$. 
    Therefore, $\ev*{\hat{O}}_\psi=\ev*{\hat{O}}{\psi}=\ev*{\hat{O}I}{\psi}=\sum_i\mel{\psi}{\hat{O}}{\phi_i}\ip{\phi_i}{\psi}=\sum_i\ip{\phi_i}{\psi}\mel{\psi}{\hat{O}}{\phi_i}=\sum_i\ev*{\hat{\rho}_\psi\hat{O}}{\phi_i}$.}Since this expression can be shown to be independent of
     which basis we choose,\footnote{To see this, we first note that for any orthonormal basis $\{\ket{\phi_i}:i\}$ of $H$, and any two operators $\hat{A}$ and $\hat{B}$ acting on $H$, using the fact that $I=\sum_i\dyad{\phi_i}$ is the identity operator on $H$, we have the commutativity property $$\sum_i\ev*{\hat{A}\hat{B}}{\phi_i}=\sum_i\ev*{\hat{A}\sum_j\dyad{\phi_j}\hat{B}}{\phi_i}=\sum_{ij}\mel*{\phi_j}{\hat{B}}{\phi_i}\mel*{\phi_i}{\hat{A}}{\phi_j}=\sum_j\ev*{\hat{B}\hat{A}}{\phi_j}.$$ Now suppose that $\{\ket{\phi_i'}:i\}$  is another orthonormal basis of $H$. Then we can define the operator $\hat{U}$ such that $\hat{U}\ket{\phi_i}=\ket{\phi_i'}$. We can also define the operator $\hat{U}^*$ such that $\ip{\phi_i'}{\psi}=\mel*{\phi_i}{\hat{U}^*}{\psi}$ for any state $\ket{\psi}\in H$. Since $\ip{\phi_i'}{\phi_j'}=\mel*{\phi_i}{\hat{U}^*}{\phi_j'}$ for all $i, j$, it will follow that  $\hat{U}^*\ket{\phi_j'}=\ket{\phi_j}$. Therefore, $\hat{U}\hat{U}^*=I$. Using this fact together with the commutativity property, we have 
    $$\sum_i \ev*{\hat{O}}{\phi_i'}= \sum_i\ev*{\hat{U}^*\hat{O}\hat{U}}{\phi_i}
    =\sum_i\ev*{\hat{O}\hat{U}\hat{U}^*}{\phi_i}=\sum_i\ev*{\hat{O}}{\phi_i}.$$} 
     we have a well-defined function called the \textbf{trace}\index{trace}, written as $\Tr(\cdot)$, which maps any operator $\hat{A}$ acting on $H$ to a value in $\mathbb{C}$ according to the formula
    \begin{equation}\Tr (\hat{A})=\sum_i \ev*{\hat{A}}{\phi_i}.\label{tracedef}\end{equation}
    Thus, it follows from equations (\ref{exptrace}) and (\ref{tracedef}) that 
    \begin{equation}
    \ev*{\hat{O}}_\psi=\Tr (\hat{\rho}_\psi\hat{O}\label{traceev}).
    \end{equation}
     So far we have defined the density matrix $\hat{\rho}_\psi$ corresponding to a normalized state $\ket*{\psi}$. More generally, a \textbf{density matrix}\index{density matrix} $\hat{\rho}$ is defined to be a Hermitian operator with positive eigenvalues such that $\Tr(\hat{\rho})=1.$ We will write $M(H)$ for the set of all density matrices on $H$. Since we are assuming\footnote{As mentioned earlier, we are making the assumption that Hermitian operators are compact.} that for any Hermitian operator, there is an orthonormal basis of the Hilbert space consisting of eigenstates of the Hermitian operator, we can find an orthonormal basis $\{\ket{\psi_i}:i\}$ of eigenstates of $\hat{\rho}$ with corresponding eigenvalues $p_i$ such that 
    \begin{equation}\label{rhodiag}
    \hat{\rho}=\sum_i p_i\dyad{\psi_i}.
    \end{equation} 
    The condition $\Tr(\hat{\rho})=1$ will then imply that $\sum_i p_i =1$. Now we could think of the operator $\hat{\rho}$ as corresponding to a measurement which gave the output $p_i$ when the system was in the state $\ket{\psi_i}$. However, we can alternatively think of $\hat{\rho}$ as describing a system which is known to be in one of the $\ket{\psi_i}$-states, but that we only know it is in the $\ket{\psi_i}$-state with probability $p_i$. Then given that $\hat{\rho}$ describes all we know about the system, the expectation value $\ev*{\hat{O}}_\rho$ for an observable $\hat{O}$ on the system can be shown to be 
    \begin{equation}\label{expdensity}\ev*{\hat{O}}_\rho=\Tr(\hat{\rho}\hat{O}),\protect\footnotemark\end{equation}\footnotetext{This follows since $\Tr(\hat{\rho}\hat{O})=\sum_i p_i \Tr(\dyad{\psi_i}\hat{O})=\sum_i p_i \ev*{\hat{O}}_{\psi_i}$ which will be the expectation value of $\hat{O}$ given that $\hat{\rho}$ encapsulates our knowledge of the system.}which is a generalization of (\ref{traceev}). Thus, we can think of a density matrix $\hat{\rho}\in M(H)$ as a generalization \label{genket} of a state ket-vector $\ket*{\psi}\in H$, since for every $\ket*{\psi}\in H$ there corresponds a density matrix $\hat{\rho}=\dyad{\psi}\in M(H)$. Because of this identification, $\hat{\rho}=\dyad{\psi}$ is referred to as a \textbf{pure state}\index{pure state}. On the other hand, the converse does not hold: if $\hat{\rho}=\sum_i p_i\dyad{\psi_i}\in M(H)$ with more than one of the $p_i> 0$, then there will not be a corresponding $\ket*{\psi}\in H$ such that $\hat{\rho}=\dyad{\psi}$. In this case, when $\hat{\rho}$ is interpreted as describing a system that is definitely in one of the $\ket{\psi_i}$-states with probability $p_i$, then we will refer to $\hat{\rho}$ as a \textbf{mixed state}\index{mixed state}.\label{mixedstate}
    \subsection{Coherence}
    Now suppose that the system $\mathcal{S}$ is initially in a superposition state $\ket{\psi}=\sum_i c_i\ket{s_i}$ with $\sum_i\abs{c_i}^2=1.$ Then the corresponding density matrix on $\mathcal{S}$ will be $$\dyad{\psi}=\sum_{ij} c_i\overline{c_j}\dyad{s_i}{s_j}.$$ When a density matrix has non-zero $\dyad{s_i}{s_j}$-components for $i\neq j$, we say that there is \textbf{coherence}\index{coherence} between the  $\ket{s_i}$ and $\ket{s_j}$-states.\footnote{The fact that a density matrix can be written out in terms of $\dyad{s_i}{s_j}$-components explains why we refer to a density matrix as a density \emph{matrix}. For example, if our state space has a basis of just two states $\{\ket{s_1},\ket{s_2}\}$, and if $\hat{\rho}=a\dyad{s_1}+b\dyad{s_1}{s_2}+c\dyad{s_2}{s_1}+d\dyad{s_2}$, then we can identify $\hat{\rho}$ with the matrix $\mqty(a & b \\ c & d)$. If we then identify the state $\ket{\psi}=x\ket{s_1}+y\ket{s_2}$ with the column vector $\mqty(x \\ y)$, then the state $\hat{\rho}\ket{\psi}$ would be identified with the column vector under matrix multiplication $\mqty(a & b \\ c & d) \mqty(x \\ y)=\mqty(ax+by \\ cx+dy)$. The trace of a density matrix is then just the sum of the diagonal elements (top left to bottom right) of the matrix. Decoherence with respect to a particular basis occurs when the off-diagonal elements of the density matrix vanish. } Thus, for the density matrix $\dyad{\psi}$ there will be coherence between the $\ket{s_i}$ and $\ket{s_j}$-states so long as both $c_i$ and $c_j$ are non-zero. Decoherence is a process (to be described shortly) by which the $\dyad{s_i}{s_j}$-components (for $i\neq j$) of a density matrix restricted to a subsystem of a composite system appear to vanish.  
    \subsection{Partial Traces and Reduced Density Matrices}
    As already mentioned, if we have a general entangled state on a composite system $\mathcal{U}=\mathcal{S}+\mathcal{E}$ of the form  $\ket{\xi}_\mathcal{U}=\sum_{i,j}\gamma_{i,j}\ket{\psi_i}_\mathcal{S}\ket{\chi_j}_\mathcal{E}$, there is a huge amount of information in all the $\gamma_{i,j}$. However, most of this information can be discarded if we are only interested in making measurements on the system $\mathcal{S}$. We can't typically encapsulate this information in the form of a state $\ket*{\psi}\in H_\mathcal{S}$, but we can encapsulate this information in the form of a density matrix $\hat{\rho}_\mathcal{S}\in M(H_\mathcal{S})$ which as mentioned on page \pageref{genket} can be thought of as a generalization of a state ket-vector $\ket*{\psi}\in H_\mathcal{S}$. In this subsection, we will show how the density matrix $\hat{\rho}=\dyad{\xi}\in M(H_\mathcal{U})$ can be reduced to a density matrix $\hat{\rho}_\mathcal{S}\in M(H_\mathcal{S})$ which encapsulates all the information needed to calculate expectation values of observables on $\mathcal{S}$. The reduced density matrix $\hat{\rho}_\mathcal{S}$ is derived from $\hat{\rho}$ via an operation called the partial trace.
    
    In the context of a composite system $\mathcal{U}=\mathcal{S}+\mathcal{E}$, 
    when taking traces, we will need to be more specific over which basis we are taking the trace over. 
     If $\{\ket{\psi_i}:i\}$ is an orthonormal basis of $H_\mathcal{S}$ and $\{\ket{\chi_j}:j\}$ 
     is an orthonormal basis of $H_\mathcal{E}$, 
     then $\{\ket{\xi_{ij}}\myeq\ket{\psi_i}\ket{\chi_j}:i,j\}$ 
     will be an orthonormal basis of $H_\mathcal{U}$. 
     For an operator $\hat{A}_\mathcal{S}$ of $H_\mathcal{S}$, we define 
    $$\Tr_\mathcal{S}(\hat{A}_\mathcal{S})=\sum_i \ev*{\hat{A}_\mathcal{S}}{\psi_i},$$
     and for an operator $\hat{A}_\mathcal{U}$ of $H_\mathcal{U}$, we define 
     $$\Tr_\mathcal{U}(\hat{A}_\mathcal{U})=\sum_{ij} \ev*{\hat{A}_\mathcal{U}}{\xi_{ij}}.$$
    This is just what we would expect the traces to be for operators on $H_\mathcal{S}$ and on $H_\mathcal{U}$ respectively. But we also need the notion of a \textbf{partial trace}\index{partial trace} for an operator $\hat{A}_\mathcal{U}$ on $H_\mathcal{U}$: 
    \begin{equation}\label{partialtrace} \Tr_\mathcal{E}(\hat{A}_\mathcal{U})=\sum_j \ev*{\hat{A}_\mathcal{U}}{\chi_j}.\end{equation} 
    Note that whereas $\Tr_\mathcal{U}(\hat{A}_\mathcal{U})$ is just a number, the partial trace $Tr_\mathcal{E}(\hat{A}_\mathcal{U})$  is an operator that acts on $H_\mathcal{S}$. To see why this is, consider the simple example of when $\hat{A}_\mathcal{U}=\dyad{\xi_{ij}}{\xi_{lk}}$. The operator $\hat{A}_\mathcal{U}$ would send the state $\ket{\xi_{lk}}$ to $\ket{\xi_{ij}}$ and all the other $\ket{\xi_{l'k'}}$-states of $H_\mathcal{U}$ to $0$. But in order to define the partial trace as given in equation (\ref{partialtrace}), we need to know what $\hat{A}_\mathcal{U}\ket{\chi}$ is and then what $\ev*{\hat{A}_\mathcal{U}}{\chi}$ is. In the case when $\hat{A}_\mathcal{U}=\dyad{\xi_{ij}}{\xi_{lk}}$, we stipulate that $\hat{A}_\mathcal{U}\ket{\chi}$ is the operator that sends the state $\ket{\psi}\in H_\mathcal{S}$ to the state $\ip{\psi_l}{\psi}\ip{\chi_k}{\chi}\ket{\xi_{ij}}\in H_\mathcal{U}$.  Furthermore, if we stipulate that $\ip{\chi}{\xi_{ij}}=\ip{\chi}{\chi_j}\ket{\psi_i}\in H_\mathcal{S}$, it follows that $\ev*{\hat{A}_\mathcal{U}}{\chi}$ will be the operator $\ip{\chi}{\chi_j}\ip{\chi_k}{\chi}\dyad{\psi_i}{\psi_l}$ that sends  the state $\ket{\psi}\in H_\mathcal{S}$ to the state  $\ip{\chi}{\chi_j}\ip{\chi_k}{\chi}\ip{\psi_l}{\psi}\ket{\psi_i}\in H_\mathcal{S}.$ By (\ref{partialtrace}), we therefore find that
    \begin{equation} \label{partialtrace2}
    \Tr_\mathcal{E}(\dyad{\xi_{ij}}{\xi_{lk}})=
    \begin{cases} \dyad{\psi_i}{\psi_l} & \text{if $j=k$,} \\
    0 & \text{if $j\neq k$}.
    \end{cases}
    \end{equation}
    And since any arbitrary  operator $\hat{A}_\mathcal{U}$ on $H_\mathcal{U}$ can be expressed as a sum $$\hat{A}_\mathcal{U}=\sum_{ijkl}\mu_{ijkl}\dyad{\xi_{ij}}{\xi_{lk}},$$ we can use equation (\ref{partialtrace2}) to find that $$\Tr_\mathcal{E}(\hat{A}_\mathcal{U})=\sum_{ijl}\mu_{ijjl}\dyad{\psi_i}{\psi_l}.$$
    
    Now it turns out that given a density matrix $\hat{\rho}$ on $H_\mathcal{U}$ and an observable $\hat{O}_\mathcal{S}$  of $H_\mathcal{S}$ (which induces an observable $\hat{O}_\mathcal{U}$ on $H_\mathcal{U}$ in the obvious way, e.g. $\hat{O}_\mathcal{U}\ket{\xi_{ij}}= (\hat{O}_\mathcal{S}\ket{\psi_i})\ket{\chi_j}$), we have the important formula 
    \begin{equation}\label{reducedev}
    \ev*{\hat{O}_\mathcal{U}}_\rho=\Tr_\mathcal{S}(\hat{\rho}_\mathcal{S}\hat{O}_\mathcal{S})
    \end{equation}
    where $ \hat{\rho}_\mathcal{S}=\Tr_\mathcal{E}(\hat{\rho})$.\footnotemark\;We \footnotetext{To see this, following \cite[46]{Schlosshauer}, we have
    \begin{align*}\ev*{\hat{O}_\mathcal{U}}_\rho
    &=\Tr_\mathcal{U}(\hat{\rho}\hat{O}_\mathcal{U})
    =\sum_{ij}\ev*{\hat{\rho}\hat{O}_\mathcal{U}}{\xi_{ij}}=\sum_i\bra{\psi_i}\Big(\sum_j\ev*{\hat{\rho}}{\chi_j}\Big)\hat{O}_\mathcal{S}\ket{\psi_i}\\
    &=\sum_i\bra{\psi_i}\hat{\rho}_\mathcal{S}\hat{O}_\mathcal{S}\ket{\psi_i}=\Tr_\mathcal{S}(\hat{\rho}_\mathcal{S}\hat{O}_\mathcal{S}). \end{align*}} refer to $\hat{\rho}_\mathcal{S}$ as the \textbf{reduced density matrix}\index{reduced density matrix} of $\hat{\rho}$. 
    
    Note that if $\hat{\rho}=\dyad{\xi}$ with $\ket{\xi}=\ket{\psi}\ket{\chi}$ so that $\mathcal{S}$ and $\mathcal{E}$ are not entangled, then $\hat{\rho}_\mathcal{S}=\dyad{\psi}$.\footnotemark\;This\footnotetext{\label{untanglepartialtrace}To see this, we recall that the partial trace $\Tr_\mathcal{E}$ is independent of which orthonormal basis $\{\ket{\chi_j}:j\}$ we choose for $\mathcal{E}$. Therefore, if $\hat{\rho}=\dyad{\xi}$ with $\ket{\xi}=\ket{\psi}\ket{\chi}$, we can choose $\ket{\chi_1}=\ket{\chi}$ and all other $\ket{\chi_i}$ such that $\ip{\chi_i}{\chi}=0$. Then $\Tr_\mathcal{E}(\hat{\rho})=\sum_j\ev{\hat{\rho}}{\chi_j}=\dyad{\psi}$. } is what we should expect, since if $\mathcal{S}$ and $\mathcal{E}$ are not entangled, then the expectation values of observables defined on $\mathcal{S}$ should be independent of the state of $\mathcal{E}$, and by equation (\ref{reducedev}), this independence is seen to hold when $\hat{\rho}_\mathcal{S}$ is independent of any states on $\mathcal{E}$.
    
    More generally, \label{subtle}for an entangled state $\ket{\xi}=\sum_{i,j}\gamma_{i,j}\ket{\psi_i}\ket{\chi_j}$, from equations (\ref{expdensity}) and (\ref{reducedev}), we have  $\ev*{\hat{O}_\mathcal{U}}_\rho=\ev*{\hat{O}_\mathcal{S}}_{\rho_\mathcal{S}}$. This means that when it comes to taking expectation values of measurements on a subsystem $\mathcal{S}$ that is part of a composite system $\mathcal{U}=\mathcal{S}+\mathcal{E}$ which is in the state $\ket*{\xi}\in H_\mathcal{U}$, the subsystem $\mathcal{S}$ behaves as though it was described by the density matrix $\hat{\rho}_\mathcal{S}$. However, there is a rather subtle point one needs to be aware of here.\footnote{It is unfortunate that many physicists fail to pick up on this subtlety with the result that they form the erroneous belief that decoherence can by itself solve the measurement problem (of outcomes) when in fact it can't. For a further discussion of the problem of outcomes, see \cite[57-60]{Schlosshauer}.} 
    For in general, as we saw in equation (\ref{rhodiag}),  any density matrix $\hat{\rho}_\mathcal{S}\in M(H_\mathcal{S})$ can be expressed as a sum $\hat{\rho}_\mathcal{S}=\sum_i p_i\dyad{\psi_i}$, and this can be \emph{thought of} as corresponding to the system $\mathcal{S}$ being in one of the $\ket*{\psi_i}$-states, but that we only know it is in the $\ket{\psi_i}$-state with probability $p_i$. If this was the correct interpretation of  $\hat{\rho}_\mathcal{S}$, then as explained on page \pageref{mixedstate}, we would refer to  $\hat{\rho}_\mathcal{S}$ as a \emph{mixed state}. But just because we can think of $\hat{\rho}_\mathcal{S}$ in this way, it doesn't follow that $\mathcal{S}$ really is in one of these $\ket{\psi_i}$-states and that we are only ignorant of which state it is. When $\mathcal{S}$ is entangled with $\mathcal{E}$ there is no fact of the matter regarding which state $\mathcal{S}$ is in. Rather, there are only facts of the matter for the composite system $\mathcal{U}$, e.g. the fact of the matter is that $\mathcal{U}$ is in the state $\ket*{\xi}$ rather than some other state of $H_\mathcal{U}$. Therefore, $\mathcal{U}$ is really in a pure state with density matrix $\dyad{\xi}$. Because we cannot give an ignorance interpretation to  $\hat{\rho}_\mathcal{S}$, d’Espagnat\label{Espagnat} referred to density matrices of this sort as being \textbf{improper mixtures}\index{improper mixtures}.\footnote{See \cite[ch. 6.2]{Espagnat} -- cited in \cite[p. 19]{Butterfield}.} But despite this subtle distinction between mixed states and improper mixtures, we have nevertheless succeeded in showing how a density matrix $\hat{\rho}=\dyad{\xi}\in M(H_\mathcal{U})$ can be reduced to a density matrix $\hat{\rho}_\mathcal{S}\in M(H_\mathcal{S})$ which encapsulates all the information needed to calculate expectation values of observables on $\mathcal{S}$.\label{subtleend} 
    
    
    \subsection{The von Neumann Measurement Scheme}\label{vonNeumannMeasurement}
    \addtocounter{footnote}{-1}
    \addtocounter{footnote}{1}
    
    We are now in a position to consider the \textbf{von Neumann measurement scheme}\index{von Neumann measurement scheme}.\footnote{See \cite[50-53]{Schlosshauer} for more details.} Instead of considering the whole of physical reality, for the time being, we just consider a physical system $\mathcal{S}$ and a measuring device $\mathcal{A}$. This division reflects the fact that a scientist doesn't measure the system $\mathcal{S}$ directly, but rather observes a measuring device $\mathcal{A}$ that is affected by $\mathcal{S}$. The measuring device $\mathcal{A}$ has the characteristic that it has a normalized ready state $\ket{a_r(t_0)}$ at initial time $t_0$ and that there is an orthonormal basis $\{\ket{s_i}:i\}$ of $H_\mathcal{S}$, and normalized states $\ket{a_i(t)}$ of $\mathcal{A}$ such that
    \begin{enumerate}[noitemsep, nosep, topsep=0pt] 
    \item for any $t\geq t_0$ we have the evolution of the states \label{vonNeumannMeasurement1}
    $\ket{s_i}\ket{a_r(t_0)}\xrightarrow{\text{time evolution}}\ket{s_i}\ket{a_i(t)}$ so that $\mathcal{S}$ and $\mathcal{A}$ do not become entangled when $\mathcal{S}$ is initially in state $\ket{s_i}$ and $\mathcal{A}$ is initially in state $\ket{a_i(t_0)}$.
    \item \label{vonNeumannMeasurement2} there exists $\delta>0$ such that if $t> t_0+\delta$, then $\ip{a_i(t)}{a_j(t)}\approx 0$ for $i\neq j$.\footnote{\label{approx}More precisely, we should say that for all $\epsilon >0$ there exists $\delta>0$ such that if $t> t_0+\delta$, then $|\ip{a_i(t)}{a_j(t)}|<\epsilon$ for $i\neq j$.}
    \end{enumerate}
    These two criteria characterize the von Neumann measurement scheme. The orthonormal basis $\{\ket{s_i}:i\}$ of $H_\mathcal{S}$ for which these two criteria hold are called \textbf{pointer states\label{pointer}}\index{pointer state} of $\mathcal{A}$. These pointer states will be determined by the  dynamics of the composite system $\mathcal{S}+\mathcal{A}$ as well as the relative configuration of $\mathcal{S}$ with respect to $\mathcal{A}$. For instance, if $\mathcal{S}$ is a silver atom and $\mathcal{A}$ is a Stern-Gerlach apparatus, then the configuration and dynamics of the system will determine a fixed axis $\uvb{a}$ relative to the Stern-Gerlach configuration $\mathcal{A}$ such that the states $\ket{\uvbp{a}}$ and $\ket{\uvbm{a}}$ of $\mathcal{S}$ don't get entangled with $\mathcal{A}$, that is, there exists $\delta>0$ such that $\ket{\uvbpm{a}}\ket{a_r(t_0)}\xrightarrow{\text{time evolution}}\ket{\uvbpm{a}}\ket{a_\pm(t)}$ with $\ip{a_+(t)}{a_-(t)}\approx 0$ for $t> t_0+\delta$.\footnote{%
    Strictly speaking, we would  need more information to describe states in $H_\mathcal{S}$ besides the spin, so we should really express this scenario in terms of $\{\ket{s_{i,+}}\myeq \ket{\uvbp{a},i}:i\}\cup \{\ket{s_{i,-}}\myeq \ket{\uvbm{a},i}:i\}$ and  $\{\ket{a_{i,+}(t)}:i\}\cup\{\ket{a_{i,-}(t)}:i\}$ where the $i$-indices encode all the additional information beyond spin.
    }\textsuperscript{, }\footnote{\label{nearzero}Although 
    we only require that  $\ip{a_+(t)}{a_-(t)}\approx 0$ for $t> t_0+\delta$ rather than demanding $\ip{a_+(t)}{a_-(t)} = 0$, we can think of the scientist who observes the apparatus as determining whether the apparatus is either in one of two normalized state $\ket*{a_+'(t)}$ or $\ket*{a_-'(t)}$ where 
    $\ip{a_+'(t)}{a_-'(t)} = 0$ 
    and $\ip{a_\pm'(t)}{a_\pm'(t)}\approx 1$, so that the scientist can confidently assert that the particle is in the state $\ket*{\uvbp{a}}$ if for instance the measurement device is found to be in the state $\ket*{a_+'(t)}$. Because $\ip{a_+'(t)}{a_-(t)}$ is only very small, but not identically zero, in theory, the particle could be in the $\ket*{\uvbm{a}}$-state, but we're assuming that such a possibility would be as likely as a violation of the Second Law of Thermodynamics, say.}
    Since no entanglement occurs with the silver atom and the Stern-Gerlach apparatus when the silver atom is in the $\ket{\uvbpm{a}}$-state, then in this situation, we can interact with the apparatus to find out whether the particle is in the $\ket{\uvbp{a}}$-state or the $\ket{\uvbm{a}}$-state without changing the spin state of the silver atom. Indeed, we should expect an experimental apparatus to have this property of non-entanglement with the measurement outcomes it reports, for otherwise, every scientist who looked at the measurement device couldn't be sure that the spin state of the silver atom being measured remained unchanged whenever the apparatus was observed, and so the scientists couldn't expect there to be any agreement among themselves  regarding which spin-state the silver atom was in. Thus, the basis of $H_\mathcal{S}$ 
    for which entanglement doesn't occur is a preferred basis. However, if we were to consider a different basis, say $\{\frac{1}{\sqrt{2}}(\ket{\uvbp{a}}+\ket{\uvbm{a}}), \frac{1}{\sqrt{2}}(\ket{\uvbp{a}}-\ket{\uvbm{a}})\}$,\footnote{According to equation (\ref{spintrans}), this basis would correspond to measuring the spin in an axis at right angles to 
    $\uvb{a}$.} then assuming that the configuration of $\mathcal{A}$ remained unchanged, entanglement between $\mathcal{S}$ and $\mathcal{A}$ would occur since then 
    $\frac{1}{\sqrt{2}}(\ket{\uvbp{a}}\pm\ket{\uvbm{a}})\ket{a_r(t_0)}\xrightarrow{\text{time evolution}}\frac{1}{\sqrt{2}}(\ket{\uvbp{a}}\ket{a_+(t)}\pm\ket{\uvbm{a}}\ket{a_-(t)})$. Thus, $\{\frac{1}{\sqrt{2}}(\ket{\uvbp{a}}+\ket{\uvbm{a}}), \frac{1}{\sqrt{2}}(\ket{\uvbp{a}}-\ket{\uvbm{a}})\}$ would not be a preferred basis. In this case, if  $\mathcal{S}$ was in the $\frac{1}{\sqrt{2}}(\ket{\uvbp{a}}+\ket{\uvbm{a}})$-state, a scientist would measure $\mathcal{A}$ to be in the $\ket{a_+(t)}$-state with probability $\frac{1}{2}$. But having measured $\mathcal{A}$ to be in the $\ket{a_+(t)}$-state, the scientist would continue to observe $\mathcal{A}$ to be in the $\ket{a_+(t)}$-state because of the subsequent non-entanglement of $\mathcal{S}$ with $\mathcal{A}$ when $\mathcal{S}$ is in the   $\ket{\uvbp{a}}$-state and $\mathcal{A}$ is in the $\ket{a_+(t)}$-state. Note that this situation is somewhat analogous to when we have the Bell-state (\ref{bell}), so that when Bob measures his particle to be in the $\ket{\uvbm{a}}$-state, he knows that Alice's particle is in the $\ket{\uvbp{a}}$-state. Likewise, in the von Neumann measurement scheme, if the scientist measures $\mathcal{A}$ to be in the $\ket{a_+(t)}$-state for $t>t_0+\delta$, he will then (almost certainly) know\footnote{We say that Bob is \emph{almost} certain rather than \emph{completely} certain because $\ip{a_+(t)}{a_-(t)}$ will be very nearly zero rather than identically zero as discussed in footnote \ref{nearzero}.} that the system $\mathcal{S}$ will be in the $\ket{\uvbp{a}}$-state.
    
    In the case where $\mathcal{S}$ has more than two states, we can write a generic normalized state of the composite system $\mathcal{U}=\mathcal{S}+\mathcal{A}$  as $\ket{\Psi(t)}=\sum_i c_i\ket{\xi_i(t)}$  where $\ket{\xi_i(t)}=\ket{s_i}\ket{a_i(t)}$. There will then be coherence between $\ket{\xi_i(t)}$ and $\ket{\xi_j(t)}$ for the density matrix $\hat{\rho}(t)\myeq\dyad{\Psi(t)}$ so long as both $c_i$ and $c_j$ are non-zero. However, if we are only interested in observables $\hat{O}_\mathcal{S}$ on $H_\mathcal{S}$, then we only need to consider the reduced density matrix $\hat{\rho}_\mathcal{S}(t)=\Tr_\mathcal{A}(\hat{\rho}(t))$. Initially, at time $t_0$ we have $\ket{a_i(t_0)}=\ket{a_r(t_0)}$ so $\ket{\Psi(t_0)}=\ket{\psi}\ket{a_r(t_0)}$ where $\ket{\psi}=\sum_i c_i\ket{s_i}$ which we assume to be normalized. Thus, initially, $\mathcal{S}$ would not be entangled with $\mathcal{A}$, and therefore the density matrix describing $\mathcal{S}$ would be $\hat{\rho}_\mathcal{S}(t_0)=\dyad{\psi}$.\footnote{Recall if $\ket{\xi}_\mathcal{U}=\ket{\psi}_\mathcal{S}\ket{\chi}_\mathcal{E}$ (i.e. $\mathcal{S}$ and $\mathcal{E}$ are not entangled), then $\ev*{\hat{O}_\mathcal{U}}_\xi=\ev*{\hat{O}_\mathcal{S}}_\psi$ as explained in footnote \ref{untanglepartialtrace}.} Hence, if we consider $\hat{O}_\mathcal{S}=\dyad{\psi}$ as an observable on $\mathcal{S}$ corresponding to a measurement\footnote{This is a measurement we conduct by some means other than looking at the apparatus $\mathcal{A}$.} that records the value $1$ if the system is in the $\ket{\psi}$ and $0$ if the system is in a state $\ket{\psi'}$ with $\ip{\psi'}{\psi}=0$, then both intuitively\footnote{I.e. we would expect the expectation value of $\hat{O}_\mathcal{U}$ to be 1 if we knew that $\mathcal{S}$ was in the state $\ket{\psi}$ with probability $1$. } and by equation (\ref{reducedev}),\footnote{I.e. given that $\hat{\rho}_\mathcal{S}(t_0)=\dyad{\psi}=\hat{O}_\mathcal{S},$ and that $\hat{O}_\mathcal{S}^2= \hat{O}_\mathcal{S}$, and $\Tr_\mathcal{S}(\hat{O}_\mathcal{S}) =1$, it follows that $\ev*{\hat{O}_\mathcal{U}}_{\rho(t_0)}=\Tr_\mathcal{S}(\hat{\rho}_\mathcal{S}(t_0)\hat{O}_\mathcal{S})= \Tr_\mathcal{S}(\hat{O}_\mathcal{S})=1.$} we would have $\ev*{\hat{O}_\mathcal{U}}_{\rho(t_0)}=1$. But if the scientist is to measure $\mathcal{S}$ to be in the $\ket{\psi}$-state, the expectation value $\ev*{\hat{O}_\mathcal{U}}_{\rho(t)}$ would have to be $1$ for times $t$ discernibly greater than $t_0$. 
    
    However, if more than one of the $c_i$ are non-zero, then the scientist will not be able to measure the system $\mathcal{S}$ to be in the $\ket{\psi}$-state for any discernible length of time. To see why this is, we first note that 
    \begin{equation}\label{reduced}\hat{\rho}_\mathcal{S}(t)=\sum_{i}\abs{c_i}^2\dyad{s_i}+\sum_{i\neq j}c_i\overline{c_j}\ip{a_j(t)}{a_i(t)} \dyad{s_i}{s_j}.\protect\footnotemark
    \end{equation}
    \footnotetext{To see this, it is sufficient to show that $\Tr_\mathcal{A}(\dyad{\xi_i(t)}{\xi_j(t)})=\ip{a_j(t)}{a_i(t)} \dyad{s_i}{s_j}$ for then we will obtain the first summand of $\hat{\rho}_\mathcal{S}$ from the fact that $\ket{a_i(t)}$ are normalized, and we will obtain the second summand by linearity of $\Tr_\mathcal{A}(\cdot)$. Well, taking $\{\ket{\phi_k}:k\}$ to be an orthonormal basis of $H_\mathcal{A},$ we have
    \begin{equation*}\begin{split}\Tr_\mathcal{A}(\dyad{\xi_i(t)}{\xi_j(t)})&=\sum_{k}\bra{\phi_k}\Big(\dyad{\xi_i(t)}{\xi_j(t)}\Big)\ket{\phi_k}=\sum_k \ip{\phi_k}{a_i(t)}\ip{a_j(t)}{\phi_k}\dyad{s_i}{s_j}\\ &=\bra{a_j(t)}\Big(\sum_k\dyad{\phi_k}\Big)\ket{a_i(t)}\dyad{s_i}{s_j}= \ip{a_j(t)}{a_i(t)} \dyad{s_i}{s_j},
    \end{split}\end{equation*}
    where we have used the fact that $I=\sum_k\dyad{\phi_k}$ is the identity operator on $H_\mathcal{A}$.
    }Now because $\ip{a_j(t)}{a_i(t)}\approx 0$ for $t>t_0+\delta$, it follows that $\hat{\rho}_\mathcal{S}\approx \sum_i\abs{c_i}^2\dyad{s_i}$ for $t>t_0+\delta$. It will then follow that $\ev*{\hat{O}_\mathcal{U}}_{\rho(t)}=\sum_i\abs{c_i}^4$,\footnote{This is because
    \begin{align*}\Tr_\mathcal{S}(\hat{\rho}_\mathcal{S}\dyad{\psi})&\approx \Tr_\mathcal{S}(\sum_i\abs{c_i}^2\dyad{s_i}\sum_{jk} c_j \overline{c_k}\dyad{s_j}{s_k})\\
    &=\Tr_\mathcal{S}(\sum_{ik}\abs{c_i}^2 c_i\overline{c_k}\dyad{s_i}{s_k})=\sum_l\sum_{ik}\bra{s_l}\abs{c_i}^2 c_i\overline{c_k}\ket{s_i}\ip{s_k}{s_l}=\sum_i\abs{c_i}^4.\end{align*}} and this will only be $1$ if only one of the $c_i$ is $1$ and all the other $c_i$ are 0. Hence, if more than one of the $c_i$ are non-zero, the scientist will not be able to measure the system $\mathcal{S}$ to be in the $\ket{\psi}$-state for any discernible length of time.
    \subsection{Decoherence}
    Note that although for the original density matrix $\dyad{\psi}$ there is coherence between the states $\ket{s_i}$ and $\ket{s_j}$, this coherence effectively  disappears when the system $\mathcal{S}$ interacts with the measuring device $\mathcal{A}$ (i.e. the $\dyad{s_i}{s_j}$-coefficients of $\hat{\rho}_\mathcal{S}$ are approximately zero for $t> t_0+\delta$). This is what we mean by \textbf{decoherence}\index{decoherence}: the coherence has effectively disappeared. The \textbf{decoherence time}\index{decoherence time} $\delta$ which is the time it takes for $\ip{a_i(t)}{a_j(t)}$ to go from $1$ when $t=t_0$ to approximately zero when $t=t_0+\delta$ will depend on what situation we are considering, but very often this time will be extremely small. For instance if we were measuring neurons firing in the brain, the decoherence time will typically be of the order $\delta =\du{e-19}{s}$.\footnote{For details of this estimate see \cite[370]{Schlosshauer}.} It is because of decoherence that we can't expect the system $\mathcal{S}$ to remain in the state $\ket{\psi}=\sum_i c_i\ket{s_i}$ for any discernible length of time, unless $\ket{\psi}$ is proportional to one of the $\ket{s_i}$-states.  
    
    Also note that when decoherence occurs, we say the coherence \emph{effectively} disappears, insofar as the coherence will not be measurable if we only consider observables just acting on $H_\mathcal{S}$. Thus, after decoherence has taken place, if we restrict our attention to the system $\mathcal{S}$ alone, it will be experimentally indistinguishable\footnote{Recall the discussion following equation (\ref{rhodiag}) on page \pageref{rhodiag} as well as the discussion on page \pageref{subtle}.} from the situation where $\mathcal{S}$ is known to be in one of the $\ket{s_i}$-states, but that we only know that it is in the $\ket{s_i}$-state with probability $\abs{c_i}^2$. Nevertheless, the coherence is still there, since if we chose to consider more general observables on the composite $\mathcal{S}+\mathcal{A}$, the $\dyad{\xi_i(t)}{\xi_j(t)}$-coefficients of $\hat{\rho}$ will continue to be $c_i\overline{c_j}$ which will in general will be non-zero, and as a whole, at time $t$ the composite system will be in the state $\ket{\Psi(t)}=\sum_i c_i\ket{\xi_i(t)}$ with probability $1$.
    
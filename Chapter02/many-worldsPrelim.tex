\section{A preliminary consideration of the many-worlds interpretation}
     In the previous chapter, we saw that in order to calculate the probabilities of the various measurement outcomes of an experiment, we can do this by positing that the state describing the item being measured is in a superposition of two states each of which corresponds to a definite measurement outcome. For example, in the case of a spin singlet, the quantum state of the two particles can be expressed as the Bell state
     \begin{equation}\tag{\ref{bell} revisited}          
          \ket*{\Psi_{\text{Bell}}}=\frac{1}{\sqrt{2}}(\ket*{\uvbp{a}}_A\ket*{\uvbm{a}}_B-\ket*{\uvbm{a}}_A\ket*{\uvbp{a}}_B). 
      \end{equation}
     which is a superposition of the two definite measurement outcome states  $\ket*{\uvbp{a}}_A\ket*{\uvbm{a}}_B$ and $\ket*{\uvbm{a}}_A\ket*{\uvbp{a}}_B$.
    But at this stage in our line of reasoning, it is too early to resort to a many-worlds interpretation of the Bell state where  the first component corresponds to a world in which Alice detects her particle at location $\uvbp{a}$ and Bob detects his particle at location $\uvbm{a}$, and where the second component corresponds to a world in which Alice detects her particle at location $\uvbm{a}$ and Bob detects his particle at location $\uvbp{a}$. Such an interpretation would be premature because as mentioned on page \pageref{bellstate2}, for any other axis $\uvb{b}$, the transformation rules in equation (\ref{spintrans}) imply that 
    \begin{equation*}\tag{\ref{bellstate2} revisited}
     \begin{split}
     \frac{1}{\sqrt{2}}(\ket*{\uvbp{a}}_A\ket*{\uvbm{a}}_B&-\ket*{\uvbm{a}}_A\ket*{\uvbp{a}}_B)\\
     &=\frac{1}{\sqrt{2}}(\ket*{\uvbp{b}}_A\ket*{\uvbm{b}}_B-\ket*{\uvbm{b}}_A\ket*{\uvbp{b}}_B).
\end{split}
\end{equation*}
    Similarly, given the transformation rules in equation (\ref{spintrans}), we should resist the temptation  to interpret a state of the form $\frac{1}{\sqrt{2}}(\ket*{\uvbp{a}}+\ket*{\uvbm{a}})$ as representing two worlds, one in which the particle is in the state $\ket*{\uvbp{a}},$ and another in which the particle is in the state $\ket*{\uvbm{a}}.$ For according to equation (\ref{spintrans1}), the much more obvious interpretation is that this state just describes one world in which the particle is in the state $\ket*{\uvbp{b}}$ where the angle between
     the $\uvb{a}$ and the  $\uvb{b}$ axis is $90\degree$.\footnote{This is because when $\theta=90\degree$, $\sin(\theta/2)=\cos(\theta/2)=\frac{1}{\sqrt{2}}$, so  $\ket*{\uvbp{b}}= \frac{1}{\sqrt{2}}(\ket*{\uvbp{a}}+\ket*{\uvbm{a}})$ in equation (\ref{spintrans1}) with $\theta=90\degree$. } 
     
    In order to make a case for a many-worlds interpretation, we need to discuss decoherence theory. Decoherence theory considers how a system interacts with its environment, and it allows us to understand what kinds of measurements can be made on the system. In order to discuss decoherence theory and its relevance to the many-worlds interpretation, we first need to introduce the mathematical formalism of standard quantum theory.
   
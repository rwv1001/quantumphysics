
\section{\label{sectionPreferredBasis}A Solution to the Preferred Basis Problem\protect\footnotemark}
\footnotetext{For more details, see \cite[71-84]{Schlosshauer}}We can now see how decoherence theory solves the preferred basis problem. Although up to this point we have been focusing on how a system $\mathcal{S}$ interacts with a measuring apparatus $\mathcal{A}$, we can generalize to the situation in which a system $\mathcal{S}$ interacts with its environment $\mathcal{E}$. We can still define pointer states in the same way as we did on page \pageref{pointer}. These pointer states will then make up the preferred basis. The two defining criteria of pointer states entail that pointer states will remain stable and immune to decoherence effects. 

Since physicists have a good understanding of how different systems interact, they are able to explain what it is about a basis that makes it a preferred basis.  The details of their analysis need not concern us here, but it's possible to show that for macroscopic and mesoscopic objects, states specified in terms of position decohere with one another very rapidly.\footnote{e.g. See the discussion in \cite[94]{Schlosshauer}.} This explains why we don't detect $\frac{1}{\sqrt{2}}(\ket{\text{Cat Alive}}+\ket{\text{Cat Dead}})$ and $\frac{1}{\sqrt{2}}(\ket{\text{Cat Alive}}-\ket{\text{Cat Dead}})$-states, but we do detect $\ket{\text{Cat Alive}}$ and $\ket{\text{Cat Dead}}$-states. Also note that $\ket{\text{Cat Alive}}$  does indeed have the property that it is immune to decoherence effects, for if we were to express $\ket{\text{Cat Alive}}$ in terms of the basis $\{\ket{\psi_+},\ket{\psi_-} \}$ where $\ket{\psi_\pm}=\frac{1}{\sqrt{2}}(\ket{\text{Cat Alive}}\pm\ket{\text{Cat Dead}})$, then $\ket{\text{Cat Alive}}=\frac{1}{\sqrt{2}}(\ket{\psi_+}+\ket{\psi_-})$. The corresponding density matrix would then be 
\begin{equation}\label{catdensity}\dyad{\text{Cat Alive}}=\frac{1}{\sqrt{2}}(\dyad{\psi_+}+\dyad{\psi_-}+\dyad{\psi_+}{\psi_-}+\dyad{\psi_-}{\psi_+}).
\end{equation} Since in normal situations, the left-hand side of equation (\ref{catdensity}) will remain unperturbed by the environment, the coefficients of the off-diagonal terms $\dyad{\psi_\pm}{\psi_\mp}$ will also remain as they are; that is, $\ket{\psi_\pm}$ and $\ket{\psi_\mp}$ will not decohere with one another. It is only in very contrived situations such as when the cat's environment is a poison releasing device coupled to a radioactive atom that $\ket{\text{Cat Alive}}$ will no longer be a pointer state with respect to this environment.

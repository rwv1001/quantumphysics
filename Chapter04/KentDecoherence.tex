
\section{Kent's Theory\label{Kentdecoherencesection} and the Problem of Outcomes}
\subsection{Overview}
In section \ref{probOutcomes} we saw that decoherence theory by itself does not offer a solution to the problem of outcomes. In this section, we consider how the additional information in Kent's theory is sufficient to address this problem. We will explain this by again considering  Kent's toy model discussed in section \ref{toysection}.

We thus suppose that a system is in a superposition $\psi_0^\text{sys} = c_1\psi_1^\text{sys}+c_2\psi_2^\text{sys}$ of two local states $\psi_1^\text{sys}$ and $\psi_2^\text{sys}$ where $\abs{c_1}^2+\abs{c_2}^2=1$, and that there is a photon coming in from the left that interacts with the system.\footnote{As in footnote \ref{wavefunctionfootnote} on page \pageref{wavefunctionfootnote}, we write $\psi_i^\text{sys}$ for the wave function that corresponds to the state $\ket*{\psi_i^\text{sys}(t)}$ with $\psi_i^\text{sys}(z, t)=\ip{z}{\psi_i^\text{sys}(t)}$.} We also suppose that  $y_1$ is a spacetime location with spatial location $z_1$ between the two hypersurfaces $S_0$ and $S$, and we consider a hypersurface $S_n=S_n(y_1)$  in a sequence of hypersurfaces that each contain $y_1$ as described on page \pageref{siydef}. 

In what follows, I use the equation (\ref{kentconsistency0}) to show how the state of the system in question goes from being an improper mixture  to being a pure state as the additional variables in the measurement on the hypersurface $S$ become available to condition on when calculating $\ev*{T^{\mu\nu}(y)}_{\tau_S}$. This solves the problem of outcomes because when a state is given by an improper mixture, in reality, the system is not in any one of two states described by $\psi_1$ and $\psi_2$; rather, the system together with its environment is in a superposition of two states, the first being the state $\psi_1$ state of the system with its corresponding environment state, and second being the state $\psi_2$ state of the system with its corresponding environment state.  However, once the system is described by a pure state, we can either say that the system is definitely in the state described by $\psi_1$, or say that the system is definitely in the state described by $\psi_2$. Thus, there is a definite outcome to the measurement of which of the two states $\psi_1$ or $\psi_2$ the system is in. Hence, in this example, we can see how Kent's theory is able to solve the problem of outcomes.    

\subsection{Technical Details\textsuperscript{*}}

In order to obtain a sufficiently simple description of the state $\ket{\Psi_n}\in H_{S_n}$ of $S_n$ for which we can use the formula (\ref{kentconsistency0}) to calculate Kent's beable, we will  
use a coarse-grained model so that $S_n$ is treated as a mesh of tiny cells\footnote{For a more detailed discussion of coarse-graining, see pp. \pageref{meshref} ff.} labeled by a sequence $(y_k)_{k=1}^\infty$. %
\nomenclature{$(y_k)_{k=1}^\infty$}{A sequence of tiny cells that defines a mesh over the hypersurface $S_n$, \nomrefpage} %
 Thus, for each cell $y_k$ there will be a Hilbert space $H_k$  %
 \nomenclature{$H_k$}{a Hilbert space for the cell $y_k\in S_n$, \nomrefpage}% 
 describing the state of that cell. We can think of each of these $y_k$ as systems that can become entangled with one another, but we will assume that $y_1$ is entangled with only a finite number $M$ of the other $y_k$ which we label as $y_{k_1}, \ldots, y_{k_M}$. %
 \nomenclature{$y_{k_1}, \ldots, y_{k_M}$}{The finite number of cells that $y_1$ is entangled with, \nomrefpage}%
  What this means is that the most general expression for $\ket{\Psi_n}$ will be of the form
\begin{equation}\label{Sistate}
\ket{\Psi_n}=\Big(\sum_j\sum_{n\in\mathbb{N}^M} c_{j,n}\ket{\xi_{1,j}}\prod_{l=1}^{M}\ket{\xi_{k_{l},n_l}}\Big)\ket*{\Xi}.
\end{equation}
In this expression, $\{\ket{\xi_{1,j}}:j\}$ %
\nomenclature{$\{\ket{\xi_{k,j}}:j\}$}{An orthonormal basis of the Hilbert space $H_k$ that describes the space of states of the cell $y_k\in S_n$, \nomrefpage}%
 is an orthonormal basis of $H_1$, $\mathbb{N}^M$ means the set of all lists $(n_1,\ldots,n_M)$ with each $n_l\in\mathbb{N}$ where $\mathbb{N}$ % 
\nomenclature{$\mathbb{N}$}{ the set of positive integers greater than $0$, \nomrefpage}%
is the set of positive integers greater than 0. The set of states $\{\ket{\xi_{k_{l},m}}:m\in\mathbb{N}\}$ form an orthonormal basis of $H_{k_{l}}$ for each $k_l$, and the $k_{l}$ are all distinct from each other and from $1$. Also, $M$ is chosen to be as small as possible so that any common factors of $\ket{\Psi_n}$ belong to $\ket*{\Xi}$ which is a sum of states of the form $\prod_l\ket{\xi_{\kappa_l}}$ where the states $\ket{\xi_{\kappa_l}}\in H_{\kappa_l}$ range over all the cells of $S_n$ not included in the set $\{k_{l}:l=1,\ldots,M\}.$ We also assume that each summand $c_{j,n}\ket{\xi_{1,j}}\prod_{l=1}^{M}\ket{\xi_{k_{l},n_l}}\ket*{\Xi}$ of $\ket*{\Psi_n}$ contains a state in each $H_k$ for every cell $k$ of $S_n$. In other words, if $k\neq 1$ and does not belong to the set $\{k_{l}:l=1,\ldots,M\}$, then $k$ belongs to the set $\{\kappa_l:\text{ there exists }\ket*{\xi_{\kappa_l}}\in H_{\kappa_l}\text{ appearing in }\ket*{\Xi}.\}$. Also, we will give $H_{S_n}$ an inner product so that if 
$$ \ket{\Psi_n'}=\Big(\sum_j\sum_{n\in\mathbb{N}^M} c'_{j,n}\ket{\xi_{1,j}}\prod_{l=1}^{M}\ket{\xi_{k_{l},n_l}}\Big)\ket*{\Xi'},$$ then
$$\ip{\Psi_n'}{\Psi_n}=\Big(\sum_j\sum_{n\in\mathbb{N}^M} \overline{c'_{j,n}}c_{j,n} \Big) \ip{\Xi'}{\Xi}$$
where $\ip{\Xi'}{\Xi}$ is defined in the obvious way. With this inner product, we will assume that $\ket{\Psi_n}$ is appropriately normalized so that $\ip{\Psi_n}{\Psi_n}=1$. If we also assume that  $\ip{\Xi}{\Xi}=1$, it will follow that $\sum_j\sum_{n\in\mathbb{N}^M}\abs{c_{j,n}}^2=1.$

Now in order to see how Kent's model addresses the problem of outcomes, we will need to consider several scenarios from his toy model. In each scenario, we will use the decomposition (\ref{Sistate}) of $\ket{\Psi_n}$ to calculate the reduced density matrix that encapsulates all the information needed to calculate expectation values at different spacetime locations. 

First, consider Figure \ref{kentdeco1} which depicts the hypersurface $S_n(y^a_1)$ for a spacetime location $y^a_1$ that occurs before the photon has interacted with the system.

\begin{figure}[ht!]
\captionsetup{justification=justified}
\centering
\tikzmath{
\a=5.3;  
\psione=0;
\psitwo=1.5;
\h=1;
\labelx=(\psione+\psitwo)/2;
\labely=-1.5;
\phstartx=-1.7;
\phstarty=\h;
\tbeg=\phstarty;
\ca=\phstartx-\tbeg;
\ttwo=\psitwo-\ca;
\cb=\ttwo+\ca;
\xend=\ttwo-\a+\cb;
\tone=\psione-\ca;
\cc=\tone+\ca;
\xendz=\tone-\a+\cc;
\tthree=\cc+\tone-\psitwo;
\circsize=0.08;
\md = (\a+\h)/2;
\tlen=0.75;
\textscale = 0.7;
\picscale = 0.89;
\nudge=0.1;
\tnudge=(\ttwo-\tone)/2+0.8;
\ttesta=2*\tone-\ttwo-\tnudge;
\ttestb=2*\tone-\ttwo+\tnudge;
\ttestc=\ttwo-\tnudge;
\ttestd=\ttwo+\tnudge;
\sonea=\a+\psione-\ttesta;
\sadd=0.5;
\lrange = -(\psione-\a+\ttesta)+\sadd;
\rrange =\lrange; 
\timex=\rrange-1.5;
\e = 0.4;
\ttestx=2.4;
\at=\a-\ttestx;
\ct=\ttestx;
\lam = 0.87;
\e = 0.2;
\hae=(2*pow(\at,3)*\lam*(2+\lam)+\e*\e*(2*\e-sqrt(4*\e*\e+\at*\at*\lam*\lam))-8*\at*\e*(-2*\e+sqrt(4*\e*\e+\at*\at*\lam*\lam))-2*\at*\at*(2+\lam)*(-2*\e+sqrt(4*\e*\e+\at*\at*\lam*\lam)))/(4*\at*\lam*(2*\at+2*\e-sqrt(4*\e*\e+\at*\at*\lam*\lam)))-(\e*\e/4+\at*\at*(-1+\lam))/(\at*\lam);
\mmu=(\tone-\tbeg)/(\psione-\phstartx);
\nnu=(\psione*\tbeg-\phstartx*\tone)/(\psione-\phstartx);
} 
\tikzstyle{scorestars}=[star, star points=5, star point ratio=2.25, draw, inner sep=1pt]%
\tikzstyle{scoresquare}=[draw, rectangle, minimum size=1mm,inner sep=0pt,outer sep=0pt]%
\begin{tikzpicture}[scale=\picscale,
declare function={
	testxonep(\ps,\t)=\a+\ps-\t;
	testxonem(\ps,\t)=\ps-\a+\t; 
	xa(\m,\n)=(-\ct*\m+\e*\m+\m*\n-sqrt(\ct*\ct*\lam*\lam+\e*\e*\m*\m+2*\e*\lam*\lam*\n+\lam*\lam*\n*\n-2*\ct*\lam*\lam*(\e+\n)))/(\lam*\lam-\m*\m);
	xb(\m,\n)=(-\ct*\m+\e*\m+\m*\n+sqrt(\ct*\ct*\lam*\lam+\e*\e*\m*\m+2*\e*\lam*\lam*\n+\lam*\lam*\n*\n-2*\ct*\lam*\lam*(\e+\n)))/(\lam*\lam-\m*\m);
	bl(\x)=\ct+\at-\e/2-1/2*(sqrt(\lam*\lam*pow(\x+\hae+\at,2)+\e*\e)-\e+\lam*(\x+\hae+\at));
	br(\x)=\ct+\at-\e/2-1/2*(sqrt(\lam*\lam*pow(\x-\hae-\at,2)+\e*\e)-\e-\lam*(\x-\hae-\at));
	bc(\x)=\ct+sqrt(\lam*\lam*\x*\x+\e*\e)-\e;
},
 ] 
\definecolor{tempcolor}{RGB}{250,190,0}
\definecolor{darkgreen}{RGB}{40,190,40}

\draw[->,blue, thick] [domain=-\at/2:-\lrange, samples=150]   plot (\x, {bl(\x)})  ;
\draw[blue, thick] [domain=-\at/2:\at/2, samples=150] plot (\x, {bc(\x)})   ;
\draw[->,blue, thick] [domain=\at/2:\rrange, samples=150]   plot (\x, {br(\x)})   ;
 \node[scale=\textscale]  at (2.7,4.1) {$S_n(y^a_1)$}; 

\draw[<->] (-\lrange, \h) node[left, scale=\textscale] {$S_0$} -- (\rrange, \h) node[right, scale=\textscale] {$S_0$};
\draw[<->] (-\lrange, \a) node[left, scale=\textscale] {$S$} -- (\rrange, \a) node[right, scale=\textscale] {$S$};
              
\draw[->, shorten <= 5pt,  shorten >= 1pt] (\psione,\h) node[below, scale=\textscale]{$\psi_1^{\text{sys}}$} node[above left, scale=\textscale]{$z_1$}-- (\psione,\a) ;
\draw[->,  shorten <= 5pt,  shorten >= 1pt] (\psitwo,\h) node[below, scale=\textscale]{$\psi_2^{\text{sys}}$} node[above right, scale=\textscale]{$z_2$}-- (\psitwo,\a);

\draw[dashed, tempcolor,  thick](\phstartx,\tbeg) node[below left=-2,scale=\textscale]{}--(\psione,\tone) node[above, pos=0.3, rotate=45,scale=0.7,black] {} node[below, pos=0.3, rotate=45,scale=0.7,black] {};
\draw[dashed, tempcolor](\psione,\tone) node[below left=-2,scale=\textscale]{}--(\psitwo,\ttwo );
\draw[dashed, tempcolor](\psitwo,\ttwo)--(\xend,\a); 
\draw[dashed, tempcolor,  thick](\psione,\tone)--(\xendz,\a);



%\draw[dashed, darkgreen, ultra thick](\psione,\tone)--(\psitwo,\tthree);
%\draw [black,fill] (\psitwo,\tthree) circle [radius=\circsize] node [black,below=4,right,scale=\textscale] {$2t_1-t_2$}; 


%\draw [black,fill] (\psione,\tone) circle [radius=\circsize] node [black,left=4,scale=\textscale] {$t_1$}; 
%\draw [black,fill](\psitwo,\ttwo)circle [radius=\circsize] node [black,above right,scale=\textscale, align=left] {$t_2$}; 

%\draw [decorate, decoration = {calligraphic brace}] (\psitwo+\nudge,\ttwo) --  (\psitwo+\nudge,\tone+\nudge/4) node[midway,right, scale=\textscale]{$t_2-t_1$}; 
%\draw [decorate, decoration = {calligraphic brace}] (\psitwo+\nudge,\tone-\nudge/4) --  (\psitwo+\nudge,2*\tone-\ttwo) node[midway,right, scale=\textscale]{$t_2-t_1$}; 

\draw[->] (\timex,\md-\tlen/2) --  (\timex,\md+\tlen/2) node[midway,right, scale=\textscale]{time}; 
 
\draw[dotted](\psione,\ttestx)--({testxonep(\psione,\ttestx)},\a);
\draw[dotted](\psione,\ttestx)--({testxonem(\psione,\ttestx)},\a);
%\draw [black,fill](\psione,\ttestx) circle [radius=\circsize] node [black,below=6,left,scale=\textscale] {$y^a_1$}; 
\draw (\psione,\ttestx) node[ scoresquare, fill=gray]  {} node [black,below=6,right,scale=\textscale] {$y^a_1$};
\draw (\psitwo, {bc(\psitwo)}) node[ scoresquare, fill=gray]  {} node [black,below=6,right,scale=\textscale] {$y^a_2$};
\draw ({xa(\mmu,\nnu)}, {xa(\mmu,\nnu)*\mmu+\nnu}) node[ scoresquare, fill=white]  {} node [black,left=7,below=-4,scale=\textscale] {$y^a_3$};

\draw[magenta,->,ultra thick] ({testxonem(\psione,\ttestx)},\a)--(-\lrange,\a);
\draw[magenta,->,ultra thick] ({testxonep(\psione,\ttestx)},\a)--(\rrange,\a);
%\draw [black,fill] (\xendz,\a) circle [radius=\circsize] node [black,above,scale=\textscale] {$\gamma_1$}; 
\end{tikzpicture}% pic 1

\vspace*{2px}
\caption[Depiction of localized states prior to any photon interaction]{Depiction of a superposition of two local states at $z_1$ and $z_2$ before the photon has interacted with them. The gray squares indicate cells in $S^1(y^a_1)$ whose states are among the summands in (\ref{Sistate}) rather than in $\Xi$. The white square indicates a cell in $S_n(y^a_1)$ whose state is a factor in $\Xi$.}
\label{kentdeco1}
\end{figure}

The gray squares correspond to the summands that appear in (\ref{Sistate}). If the system were in the $\psi_1^\text{sys}$-state, then the state describing $S_n(y^a_1)$ would have a factor $\ket{\psi_1^\text{sys}}\in H_1$ indicating that there is a 
non-zero mass at the $y^a_1$-cell, and there would also be a factor $\ket{0_2}\in H_2$ which we use to indicate that there is zero mass/energy at $y^a_2$. 
There is also an incoming photon at the $y^a_3$-cell, and so we use $\ket{\gamma_3}$ to indicate that there is a photon there.
 Thus, if  the system  were in the $\psi_1^\text{sys}$-state, we would write the state of $S_n(y^a_1)$ 
 as $\ket{\Psi_n}=\ket{\psi_1^\text{sys}}\ket{0_2}\ket{\gamma_3}\ket*{\Xi'}$, where $\ket*{\Xi'}$ describes the states of all the other cells of $S_n(y^a_1)$. In this very simple scenario, $\ket*{\Xi'}=\prod_{k\neq 1,2,3}\ket{0_k}$ indicating that there is zero mass/energy at all the other $y_k$.

 On the other hand, if the system were in the state $\psi_2^\text{sys}$, then the state describing $S_n(y^a_1)$ would have a factor $\ket{\psi_2^\text{sys}}\in H_2$ 
 indicating that there is a non-zero mass at the $y^a_2$-cell, and there would also be a factor $\ket{0_1}\in H_1$ which we use to indicate that there is zero mass at $y^a_1$,
  and again the $y^a_3$-cell would be in the $\ket{\gamma_3}$-state, and every other cell would be described by  $\ket*{\Xi'}$  just as if the system had been in the $\psi_1^\text{sys}$-state. Therefore, when the system is in the state $\psi_2^\text{sys}$, we would write the state of $S_n(y^a_1)$ as $\ket{\Psi_n}=\ket{0_1}\ket{\psi_2^\text{sys}}\ket{\gamma_3}\ket*{\Xi'}$. 
 
 Now since the system is actually in a supposition $\psi_0^\text{sys} = c_1\psi_1^\text{sys}+c_2\psi_2^\text{sys}$, the state of $S_n(y^a_1)$ will be 
 \begin{equation*}
 \ket{\Psi_n}=\big(c_1\ket{\psi_1^\text{sys}}\ket{0_2}+c_2\ket{0_1}\ket{\psi_2^\text{sys}}\big)\ket{\gamma_3}\ket*{\Xi'}=\big(c_1\ket{\psi_1^\text{sys}}\ket{0_2}+c_2\ket{0_1}\ket{\psi_2^\text{sys}}\big)\ket*{\Xi}
 \end{equation*}
where we have absorbed the $\ket{\gamma_3}$-state into $\ket*{\Xi}$ (i.e. $\ket*{\Xi}= \ket{\gamma_3}\ket*{\Xi'}$).

Now as it stands, the state $\ket{\Psi_n}$ describing $S_n(y^a_1)$ has a definite mass-energy density $\tau_S(x)$ for $x\in S_n(y^a_1)\cap S$, namely $0$. Thus, if $\pi_n$ is the operator featuring in (\ref{kentconsistency0}) that corresponds to this definite mass-energy density, then $\pi_n\ket{\Psi_n}=\ket{\Psi_n}$. Therefore, equation (\ref{kentconsistency0}) for Kent's beables tells us that
\begin{equation}\label{kentbeablenoinfo}
\ev*{T^{\mu\nu}(y^a_1)}_{\tau_S}=\ev*{\hat{T}^{\mu\nu}(y^a_1)}{\Psi_n},
\end{equation}
where we have also used the fact that $\ip*{\Psi_n}{\Psi_n}=1.$

Now as we saw in section \ref{decotheory}, if we are interested only in the expectation values of observables for a system $\mathcal{S}$ contained within a universe $\mathcal{U}=\mathcal{S}+\mathcal{E}$, then the information needed to do this can be encapsulated in the reduced density matrix for $\mathcal{S}$. Thus, if the universe is described by a state 
$\ket{\Psi}=\sum_j c_j \ket{\psi_j}_\mathcal{S}\ket{E_j}_\mathcal{E}$ with corresponding density matrix $\hat{\rho}=\dyad{\Psi}\in M(H_\mathcal{U})$, then the reduced density matrix $\hat{\rho}_\mathcal{S}\in M(H_\mathcal{S})$ is the Hermitian operator acting on the state space $H_\mathcal{S}$ with the property that 
\begin{equation}\tag{\ref{reducedev} revisited}
\ev*{\hat{O}_\mathcal{U}}_\rho=\Tr_\mathcal{S}(\hat{\rho}_\mathcal{S}\hat{O}_\mathcal{S})
\end{equation}
where $\hat{O}_\mathcal{S}$ is an observable on $H_\mathcal{S}$,  and $\hat{O}_\mathcal{U}$ is the corresponding observable on $H_\mathcal{U}$. Furthermore, we also have
\begin{equation}\label{reduced2}
\hat{\rho}_\mathcal{S}=\sum_j \abs{c_j}^2\dyad{\psi_j}+\sum_{j\neq k} c_j\overline{c_k}\ip{E_k}{E_j}\dyad{\psi_j}{\psi_k}.\protect\footnotemark
\end{equation}
\footnotetext{cf. (\ref{reduced})}We can thus apply this to the situation at hand by taking $S_n$ to be our universe $\mathcal{U}$ and $y^a_1$ to be the system $\mathcal{S}$, and $S_n\setminus \{y^a_1\}$ to be the environment $\mathcal{E}$. If we assume that $\ip{0_2}{\psi_2^\text{sys}}= 0$, then by (\ref{reduced2}), the corresponding reduced density matrix $\hat{\rho}_{y^a_1}$ takes the form of an improper mixture
\begin{equation}\label{kentred}
\hat{\rho}_{y^a_1}= \abs{c_1}^2\dyad{\psi_1^\text{sys}}+\abs{c_2}^2\dyad{0_1}.
\end{equation}
Therefore, by (\ref{kentbeablenoinfo}) and (\ref{kentred}), Kent's beable at $y^a_1$ will take the form 
\begin{equation}\label{kentbe}
\begin{split}
\ev*{T^{\mu\nu}(y^a_1)}_{\tau_S}&= \ev*{\hat{T}^{\mu\nu}(y^a_1)}{\Psi_n} \\
&=\Tr_{y^a_1}(\hat{\rho}_{y^a_1}\hat{T}^{\mu\nu}(y^a_1))\\
&=\abs{c_1}^2\ev*{\hat{T}^{\mu\nu}(y^a_1)}{\psi_1^\text{sys}}+\abs{c_2}^2\ev*{\hat{T}^{\mu\nu}(y^a_1)}{0_1}.
\end{split}
\end{equation}
But since $\hat{\rho}_{y^a_1}$ is an improper mixture, we cannot give (\ref{kentbe}) an ignorance interpretation -- the universe is still in the superposition state
$$\ket{\Psi_n}=c_1 \ket*{\psi_1^\text{sys}}\ket*{E_1}+c_2 \ket*{0_1}\ket*{E_2}$$ where $\ket{E_1}=\ket{0_2}\ket*{\Xi}$ and $\ket{E_2}=\ket{\psi_2^\text{sys}}\ket*{\Xi}$. 

In order to solve the problem of outcomes, we need to provide a satisfactory explanation (i.e. an explanation that is consistent with special relativity and the predictions of standard quantum theory)  of how the superposition state $\ket{\Psi_n}$ effectively goes to either the state $\ket*{\psi_1^\text{sys}}\ket*{E_1}$ or to the state $\ket*{0_1}\ket*{E_2}$.\footnote{cf. the initial discussion of the problem of outcomes on page \pageref{proboutcomes}. I have used the word `effectively' to qualify this sentence since it is not necessary to prove that there actually is such a transition from the state  $\ket{\Psi_n}$ to either the state $\ket*{\psi_1^\text{sys}}\ket*{E_1}$ or to the state $\ket*{0_1}\ket*{E_2}$. Rather, it is sufficient to show that if we consider any observable $\hat{O}_{\mathcal{S}}$ acting on $\mathcal{S}$, then the expectation value takes the form $\ev*{\hat{O}_{\mathcal{S}}}{\psi_1^\text{sys}}$ or $\ev*{\hat{O}_{\mathcal{S}}}{0_1}$ once there is an outcome, for then the system $\mathcal{S}$ has all the properties consistent with it being in the state $\ket*{\psi_1^\text{sys}}$ or the state $\ket*{0_1}$ respectively.} In terms of density operators, this means we need to show how the improper state (\ref{kentred}), transitions to a pure state of the form $\dyad{\psi_1^\text{sys}}$ or $\dyad{0_1}$.  

To this end, let us consider Kent's beables at the spacetime location $y^b_1$ depicted in figure \ref{kentdecoh2}. 
\begin{figure}[ht!]
\captionsetup{justification=justified}
\centering
\tikzmath{
\a=5.3;  
\psione=0;
\psitwo=1.5;
\h=1;
\labelx=(\psione+\psitwo)/2;
\labely=-1.5;
\phstartx=-1.7;
\phstarty=\h;
\tbeg=\phstarty;
\ca=\phstartx-\tbeg;
\ttwo=\psitwo-\ca;
\cb=\ttwo+\ca;
\xend=\ttwo-\a+\cb;
\tone=\psione-\ca;
\cc=\tone+\ca;
\xendz=\tone-\a+\cc;
\tthree=\cc+\tone-\psitwo;
\circsize=0.08;
\md = (\a+\h)/2;
\tlen=0.75;
\textscale = 0.7;
\picscale = 0.89;
\nudge=0.1;
\tnudge=(\ttwo-\tone)/2+0.8;
\ttesta=2*\tone-\ttwo-\tnudge;
\ttestb=2*\tone-\ttwo+\tnudge;
\ttestc=\ttwo-\tnudge;
\ttestd=\ttwo+\tnudge;
\sonea=\a+\psione-\ttesta;
\sadd=0.5;
\lrange = -(\psione-\a+\ttesta)+\sadd;
\rrange =\lrange; 
\timex=\rrange-1.5;
\e = 0.4;
\ttestx=3.0;
\at=\a-\ttestx;
\ct=\ttestx;
\lam = 0.87;
\e = 0.2;
\hae=(2*pow(\at,3)*\lam*(2+\lam)+\e*\e*(2*\e-sqrt(4*\e*\e+\at*\at*\lam*\lam))-8*\at*\e*(-2*\e+sqrt(4*\e*\e+\at*\at*\lam*\lam))-2*\at*\at*(2+\lam)*(-2*\e+sqrt(4*\e*\e+\at*\at*\lam*\lam)))/(4*\at*\lam*(2*\at+2*\e-sqrt(4*\e*\e+\at*\at*\lam*\lam)))-(\e*\e/4+\at*\at*(-1+\lam))/(\at*\lam);
\mmu=(\tone-\tbeg)/(\psione-\phstartx);
\nnu=(\psione*\tbeg-\phstartx*\tone)/(\psione-\phstartx);
\muf=(\a-\tone)/(\xendz-\psione);
\nuf=(\xendz*\tone-\psione*\a)/(\xendz-\psione);
} 
\tikzstyle{scorestars}=[star, star points=5, star point ratio=2.25, draw, inner sep=1pt]%
\tikzstyle{scoresquare}=[draw, rectangle, minimum size=1mm,inner sep=0pt,outer sep=0pt]%
\begin{tikzpicture}[scale=\picscale,
declare function={
	testxonep(\ps,\t)=\a+\ps-\t;
	testxonem(\ps,\t)=\ps-\a+\t; 
	xa(\m,\n)=(-\ct*\m+\e*\m+\m*\n-sqrt(\ct*\ct*\lam*\lam+\e*\e*\m*\m+2*\e*\lam*\lam*\n+\lam*\lam*\n*\n-2*\ct*\lam*\lam*(\e+\n)))/(\lam*\lam-\m*\m);
	xb(\m,\n)=(-\ct*\m+\e*\m+\m*\n+sqrt(\ct*\ct*\lam*\lam+\e*\e*\m*\m+2*\e*\lam*\lam*\n+\lam*\lam*\n*\n-2*\ct*\lam*\lam*(\e+\n)))/(\lam*\lam-\m*\m);
	bl(\x)=\ct+\at-\e/2-1/2*(sqrt(\lam*\lam*pow(\x+\hae+\at,2)+\e*\e)-\e+\lam*(\x+\hae+\at));
	br(\x)=\ct+\at-\e/2-1/2*(sqrt(\lam*\lam*pow(\x-\hae-\at,2)+\e*\e)-\e-\lam*(\x-\hae-\at));
	bc(\x)=\ct+sqrt(\lam*\lam*\x*\x+\e*\e)-\e;
},
 ] 
\definecolor{tempcolor}{RGB}{250,190,0}
\definecolor{darkgreen}{RGB}{40,190,40}

\draw[->,blue, thick] [domain=-\at/2:-\lrange, samples=150]   plot (\x, {bl(\x)})  ;
\draw[blue, thick] [domain=-\at/2:\at/2, samples=150] plot (\x, {bc(\x)})  node[right=20, above=10,black,scale=\textscale]{}  ;
\draw[->,blue, thick] [domain=\at/2:\rrange, samples=150]   plot (\x, {br(\x)})   ;

 \node[scale=\textscale]  at (2.7,4.6) {$S_n(y^b_1)$}; 
\draw[<->] (-\lrange, \h) node[left, scale=\textscale] {$S_0$} -- (\rrange, \h) node[right, scale=\textscale] {$S_0$};
\draw[<->] (-\lrange, \a) node[left, scale=\textscale] {$S$} -- (\rrange, \a) node[right, scale=\textscale] {$S$};
              
\draw[->, shorten <= 5pt,  shorten >= 1pt] (\psione,\h) node[below, scale=\textscale]{$\psi_1^{\text{sys}}$} node[above left, scale=\textscale]{$z_1$}-- (\psione,\a) ;
\draw[->,  shorten <= 5pt,  shorten >= 1pt] (\psitwo,\h) node[below, scale=\textscale]{$\psi_2^{\text{sys}}$} node[above right, scale=\textscale]{$z_2$}-- (\psitwo,\a);

\draw[dashed, tempcolor,  thick](\phstartx,\tbeg) node[below left=-2,scale=\textscale]{}--(\psione,\tone) node[above, pos=0.3, rotate=45,scale=0.7,black] {} node[below, pos=0.3, rotate=45,scale=0.7,black] {};
\draw[dashed, tempcolor,  thick](\psione,\tone) node[below left=-2,scale=\textscale]{}--(\psitwo,\ttwo );
\draw[dashed, tempcolor,  thick](\psitwo,\ttwo)--(\xend,\a); 
\draw[dashed, tempcolor, thick](\psione,\tone)--(\xendz,\a);





\draw[->] (\timex,\md-\tlen/2) --  (\timex,\md+\tlen/2) node[midway,right, scale=\textscale]{time}; 
 
\draw[dotted](\psione,\ttestx)--({testxonep(\psione,\ttestx)},\a);
\draw[dotted](\psione,\ttestx)--({testxonem(\psione,\ttestx)},\a);
%\draw [black,fill](\psione,\ttestx) circle [radius=\circsize] node [black,below=6,left,scale=\textscale] {$y^b_1$}; 
\draw (\psione,\ttestx) node[ scoresquare, fill=gray]  {} node [black,below=6,right,scale=\textscale] {$y^b_1$};
\draw (\psitwo, {bc(\psitwo)}) node[ scoresquare, fill=gray]  {} node [black,below=6,right,scale=\textscale] {$y^b_2$};
\draw ({xa(\mmu,\nnu)}, {xa(\mmu,\nnu)*\mmu+\nnu}) node[ scoresquare, fill=gray]  {} node [black,below=3,scale=\textscale] {$y^b_4$};
\draw ({xb(\muf,\nuf)}, {xb(\muf,\nuf)*\muf+\nuf}) node[ scoresquare, fill=gray]  {} node [black,below=3,scale=\textscale] {$y^b_3$};

\draw[magenta,->,ultra thick] ({testxonem(\psione,\ttestx)},\a)--(-\lrange,\a);
\draw[magenta,->,ultra thick] ({testxonep(\psione,\ttestx)},\a)--(\rrange,\a);
%\draw [black,fill] (\xendz,\a) circle [radius=\circsize] node [black,above,scale=\textscale] {$\gamma_1$}; 
\end{tikzpicture}% pic 1

\vspace*{2px}
\caption[Depiction of localized states after photon interaction]{Depiction of a superposition of two local states at $z_1$ and $z_2$ with $S_n(y^b_1)$ being after the photon has interacted without the photon intersecting $S_n(y^b_1)\cap S$. The gray squares indicate cells in $S^1(y^b_1)$ whose states are among the summands in (\ref{Sistate}).}
\label{kentdecoh2}
\end{figure}The state of $S_n(y^b_1)$ will then be
 \begin{equation*}
 \ket{\Psi_n}=\big(c_1\ket{\psi_1^\text{sys}}\ket{0_2}\ket{\gamma_3}\ket{0_4}+c_2\ket{0_1}\ket{\psi_2^\text{sys}}\ket{0_3}\ket{\gamma_4}\big)\ket*{\Xi}
 \end{equation*}
 where the notation is analogous to that in the previous example. Since no photon detections are registered on $S_n(y^b_1)\cap S$, we again have $\pi_n\ket{\Psi_n}=\ket{\Psi_n}$ so that the reduced density matrix $\hat{\rho}_{y_1^b}$ will again be given by  (\ref{kentred}) with $y^a_1$ replaced by $y^b_1$. However, in this case, Kent's beables $\ev*{T^{\mu\nu}(y^b_1)}_{\tau_S}$ will not be given by (\ref{kentbe}) because in the limit as $n\rightarrow\infty$, the photon \emph{will} be registered on $S_n(y^b_1)\cap S$. 
 
 To deal with the case when a photon is registered on $S_n(y^b_1)\cap S$, we consider a third example as depicted in figure \ref{kentdecoh3}.


\begin{figure}[ht!]
\captionsetup{justification=justified}
\centering
\tikzmath{
\a=5.3;  
\psione=0;
\psitwo=1.5;
\h=1;
\labelx=(\psione+\psitwo)/2;
\labely=-1.5;
\phstartx=-1.7;
\phstarty=\h;
\tbeg=\phstarty;
\ca=\phstartx-\tbeg;
\ttwo=\psitwo-\ca;
\cb=\ttwo+\ca;
\xend=\ttwo-\a+\cb;
\tone=\psione-\ca;
\cc=\tone+\ca;
\xendz=\tone-\a+\cc;
\tthree=\cc+\tone-\psitwo;
\circsize=0.08;
\md = (\a+\h)/2;
\tlen=0.75;
\textscale = 0.7;
\picscale = 0.89;
\nudge=0.1;
\tnudge=(\ttwo-\tone)/2+0.8;
\ttesta=2*\tone-\ttwo-\tnudge;
\ttestb=2*\tone-\ttwo+\tnudge;
\ttestc=\ttwo-\tnudge;
\ttestd=\ttwo+\tnudge;
\sonea=\a+\psione-\ttesta;
\sadd=0.5;
\lrange = -(\psione-\a+\ttesta)+\sadd;
\rrange =\lrange; 
\timex=\rrange-1.5;
\e = 0.4;
\ttestx=4;
\at=\a-\ttestx;
\ct=\ttestx;
\lam = 0.87;
\e = 0.2;
\hae=(2*pow(\at,3)*\lam*(2+\lam)+\e*\e*(2*\e-sqrt(4*\e*\e+\at*\at*\lam*\lam))-8*\at*\e*(-2*\e+sqrt(4*\e*\e+\at*\at*\lam*\lam))-2*\at*\at*(2+\lam)*(-2*\e+sqrt(4*\e*\e+\at*\at*\lam*\lam)))/(4*\at*\lam*(2*\at+2*\e-sqrt(4*\e*\e+\at*\at*\lam*\lam)))-(\e*\e/4+\at*\at*(-1+\lam))/(\at*\lam);
\mmu=(\tone-\tbeg)/(\psione-\phstartx);
\nnu=(\psione*\tbeg-\phstartx*\tone)/(\psione-\phstartx);
\muf=(\a-\ttwo)/(\xend-\psitwo);
\nuf=(\xend*\ttwo-\psitwo*\a)/(\xend-\psitwo);
} 
\tikzstyle{scorestars}=[star, star points=5, star point ratio=2.25, draw, inner sep=1pt]%
\tikzstyle{scoresquare}=[draw, rectangle, minimum size=1mm,inner sep=0pt,outer sep=0pt]%
\begin{tikzpicture}[scale=\picscale,
declare function={
	testxonep(\ps,\t)=\a+\ps-\t;
	testxonem(\ps,\t)=\ps-\a+\t; 
	xa(\m,\n)=(-\ct*\m+\e*\m+\m*\n-sqrt(\ct*\ct*\lam*\lam+\e*\e*\m*\m+2*\e*\lam*\lam*\n+\lam*\lam*\n*\n-2*\ct*\lam*\lam*(\e+\n)))/(\lam*\lam-\m*\m);
	xb(\m,\n)=(-\ct*\m+\e*\m+\m*\n+sqrt(\ct*\ct*\lam*\lam+\e*\e*\m*\m+2*\e*\lam*\lam*\n+\lam*\lam*\n*\n-2*\ct*\lam*\lam*(\e+\n)))/(\lam*\lam-\m*\m);
	bl(\x)=\ct+\at-\e/2-1/2*(sqrt(\lam*\lam*pow(\x+\hae+\at,2)+\e*\e)-\e+\lam*(\x+\hae+\at));
	br(\x)=\ct+\at-\e/2-1/2*(sqrt(\lam*\lam*pow(\x-\hae-\at,2)+\e*\e)-\e-\lam*(\x-\hae-\at));
	bc(\x)=\ct+sqrt(\lam*\lam*\x*\x+\e*\e)-\e;
},
 ] 
\definecolor{tempcolor}{RGB}{250,190,0}
\definecolor{darkgreen}{RGB}{40,190,40}

\draw[->,blue, thick] [domain=-\at/2:-\lrange, samples=150]   plot (\x, {bl(\x)})  ;
\draw[blue, thick] [domain=-\at/2:\at/2, samples=150] plot (\x, {bc(\x)})  node[right=20, above=10,black,scale=\textscale]{}  ;
\draw[->,blue, thick] [domain=\at/2:\rrange, samples=150]   plot (\x, {br(\x)})   ;

 \node[scale=\textscale]  at (1,4.1) {$S_n(y^c_1)$}; 
\draw[<->] (-\lrange, \h) node[left, scale=\textscale] {$S_0$} -- (\rrange, \h) node[right, scale=\textscale] {$S_0$};
\draw[<->] (-\lrange, \a) node[left, scale=\textscale] {$S$} -- (\rrange, \a) node[right, scale=\textscale] {$S$};
              
\draw[->, shorten <= 5pt,  shorten >= 1pt] (\psione,\h) node[below, scale=\textscale]{$\psi_1^{\text{sys}}$} node[above left, scale=\textscale]{$z_1$}-- (\psione,\a) ;
\draw[->,  shorten <= 5pt,  shorten >= 1pt] (\psitwo,\h) node[below, scale=\textscale]{$\psi_2^{\text{sys}}$} node[above right, scale=\textscale]{$z_2$}-- (\psitwo,\a);

\draw[dashed, tempcolor,  thick](\phstartx,\tbeg) node[below left=-2,scale=\textscale]{}--(\psione,\tone) node[above, pos=0.3, rotate=45,scale=0.7,black] {} node[below, pos=0.3, rotate=45,scale=0.7,black] {};
\draw[dashed, tempcolor,  thick](\psione,\tone) node[below left=-2,scale=\textscale]{}--(\psitwo,\ttwo );
\draw[dashed, tempcolor,  thick](\psitwo,\ttwo)--(\xend,\a); 
\draw[dashed, tempcolor,  thick](\psione,\tone)--(\xendz,\a);

\draw[magenta,->,ultra thick] ({testxonem(\psione,\ttestx)},\a)--(-\lrange,\a);
\draw[magenta,->,ultra thick] ({testxonep(\psione,\ttestx)},\a)--(\rrange,\a);



\draw[->] (\timex,\md-\tlen/2) --  (\timex,\md+\tlen/2) node[midway,right, scale=\textscale]{time}; 
 
\draw[dotted](\psione,\ttestx)--({testxonep(\psione,\ttestx)},\a);
\draw[dotted](\psione,\ttestx)--({testxonem(\psione,\ttestx)},\a);
%\draw [black,fill](\psione,\ttestx) circle [radius=\circsize] node [black,below=6,left,scale=\textscale] {$y^c_1$}; 
\draw (\psione,\ttestx) node[ scoresquare, fill=gray]  {} node [black,below=6,right,scale=\textscale] {$y^c_1$};
\draw (\psitwo, {bc(\psitwo)}) node[ scoresquare, fill=gray]  {} node [black,below=6,right,scale=\textscale] {$y^c_2$};
\draw ({xb(\muf,\nuf)}, {xb(\muf,\nuf)*\muf+\nuf}) node[ scoresquare, fill=gray]  {} node [black,right=7, below=-4,scale=\textscale] {$y^c_4$};
\draw (\xendz,\a) node[ scoresquare, fill=gray]  {} node [black,below=3,scale=\textscale] {$y^c_3$};


%\draw [black,fill] (\xendz,\a) circle [radius=\circsize] node [black,above,scale=\textscale] {$\gamma_1$}; 
\end{tikzpicture}% pic 1

\vspace*{2px}
\caption[Depiction of localized states with reflected photons intersecting $S_n(y^c_1)$]{Depiction of a superposition of two local states at $z_1$ and $z_2$ with $y^c_1$ sufficiently late that the photon intersects $S_n(y^c_1)\cap S$. The gray squares indicate cells in $S^1(y^c_1)$ whose states are among the summands in (\ref{Sistate})}
\label{kentdecoh3}
\end{figure}

In this case, the state of $S_n(y^c_1)$ will be
 \begin{equation*}
 \ket{\Psi_n}=\big(c_1\ket{\psi_1^\text{sys}}\ket{0_2}\ket{\gamma_3}\ket{0_4}+c_2\ket{0_1}\ket{\psi_2^\text{sys}}\ket{0_3}\ket{\gamma_4}\big)\ket*{\Xi}
 \end{equation*}
 but now we have to consider the fact that the photon intersects $S_n(y^c_1)\cap S$. There are two possible (notional) measurement outcomes that can occur on $S_n(y^c_1)\cap S$: either $T_S=\tau_{S,1}$ where $\tau_{S,1}(y^c_3)\neq 0$, or $T_S=\tau_{S,2}$ where $\tau_{S,2}(y^c_3)=0.$ 
 
 The case  $T_S=\tau_{S,1}$ indicates that there is a photon detection at $y^c_3$ so that the local state at the $y^c_3$-cell is $\ket{\gamma_3}$. Therefore, if we write $\pi_{n,1}$ for the operator $\pi_n$, we have 
 $$\pi_{n,1}\ket{\Psi_n}=c_1\ket{\psi_1^\text{sys}}\ket{0_2}\ket{\gamma_3}\ket{0_4}\ket*{\Xi}.$$
 Therefore, 
 $\ev*{\pi_{n,1}\hat{T}^{\mu\nu}(y^c_1)}{\Psi_n}=\abs{c_1}^2\ev*{\hat{T}^{\mu\nu}(y^c_1)}{\psi_1^\text{sys}}$ and  $\ev*{\pi_{n,1}}{\Psi_n}=\abs{c_1}^2$. Hence, by (\ref{kentconsistency0}), Kent's beables at $y^c_1$ will be 
 $$\ev*{T^{\mu\nu}(y^c_1)}_{\tau_{S,1}}=\ev*{\hat{T}^{\mu\nu}(y^c_1)}{\psi_1^\text{sys}}.$$ 
 From this, it follows that the reduced density matrix at $y^c_1$ will take the form of a pure state:
 \begin{equation}\label{purerho}
\hat{\rho}_{y^c_1}= \dyad{\psi_1^\text{sys}}.
\end{equation} 
 On the other hand, for the case when  $T_S=\tau_{S,2}$, this indicates that there is no photon detection at $y^c_3$, so that the local state at the $y^c_3$-cell will be $\ket{0_3}$. So if we now  write $\pi_{n,2}$ for the operator $\pi_n$, we have 
 $$\pi_{n,2}\ket{\Psi_n}=c_2\ket{0_1}\ket{\psi_2^\text{sys}}\ket{0_3}\ket{\gamma_4}\ket*{\Xi}.$$
 Therefore, 
 $\ev*{\pi_{n,2}\hat{T}^{\mu\nu}(y^c_1)}{\Psi_n}=\abs{c_2}^2\ev*{\hat{T}^{\mu\nu}(y^c_1)}{0_1}$ and  $\ev*{\pi_{n,2}}{\Psi_n}=\abs{c_2}^2$,  
  and so by (\ref{kentconsistency0}), Kent's beables at $y^c_1$ will be 
 $$\ev*{T^{\mu\nu}(y^c_1)}_{\tau_{S,2}}=\ev*{\hat{T}^{\mu\nu}(y^c_1)}{0_1}.$$
 In this case, the reduced density matrix at $y^c_1$  will be
  \begin{equation}
\hat{\rho}_{y^c_1}= \dyad{0_1},
\end{equation} 
which is again a pure state.

In these examples we have therefore seen how the additional information concerning photon detection on $S_n(y_1)\cap S$ is able to determine whether the reduced density matrix at $y_1$ is a pure state or an improper mixture. Hence, Kent's theory offers an answer to d'Espagnat's problem of outcomes. As mentioned in section \ref{probOutcomes}, d'Espagnat noticed that with decoherence theory alone, we are not entitled to give an ignorance interpretation to the reduced density matrix for a system that is an improper mixture, and thus we are not able to conclude that an outcome has occurred. However, if the reduced density matrix of a system goes from being an improper mixture to a pure state of the form $\dyad{\psi}$ as it does when Kent's additional information is taken into account, then we can say that an outcome has occurred, namely the outcome of the system being in the state $\ket{\psi}.$  








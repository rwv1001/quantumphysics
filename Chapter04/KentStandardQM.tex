
\section{Kent's Theory and standard quantum theory\textsuperscript{*}\label{KentconsistentQT}}
In this section, I will describe in more detail how to extract the quantum state of a system at a particular time from the universal quantum state $\ket*{\Psi_0}$ and the mass-energy density $\tau_S$, and I will show that Kent's theory does indeed give the same predictions as standard quantum theory in the case of an experimental apparatus $\mathcal{A}$ measuring the properties of a particle $\mathcal{S}$. Again, we will consider a toy model similar to Kent's toy model described in section \ref{toysection} where photons are treated as point particles.

So let $\tau_S$ be the notional mass-energy density measurement on $S$. In order to avoid undue complexity, we will assume that there is no simultaneous $\hat{T}_S$-eigenstate degeneracy so that if $\ket*{\Psi}$ and $\ket*{\Psi'}$ are normalized\footnote{In this section we will assume that all states are normalized.} simultaneous $\hat{T}_S$-eigenstates with simultaneous eigenvalues $\tau_S$ and $\tau_S'$ respectively, then
$$(\forall x\in S) \,(\tau_S(x)=\tau_S'(x))\implies \ket*{\Psi}=\ket*{\Psi'}.$$
This means that corresponding to $\tau_S$, there will be a unique simultaneous $\hat{T}_S$-eigenstate $\ket*{\Psi}$, and according to the Born Rule, the notional mass-energy density measurement $\tau_S$ will have been selected with a probability 
$$P(T_S = \tau_S)=|\mel{\Psi}{U_{SS_0}}{\Psi_0}|^2=|\ip{\Psi}{\Psi_S}|^2$$
where $\ket*{\Psi_0}$ is the state of the initial hypersurface $S_0$, where $U_{SS_0}$ is the unitary operator defined by equation (\ref{SchwingerUnitaryOP}), and where $\ket*{\Psi_S}=U_{SS_0}\ket*{\Psi_0}$.

Now suppose this notional measurement $\tau_S$ indicates that there is some apparatus $\mathcal{A}$ that exists in the vicinity of a spatial location $z_0$ at time $t_i$ as depicted in figure \ref{pisolution}. 
\begin{figure}[ht!]
	\captionsetup{justification=justified}
	\centering
	\tikzmath{
	\a=5.3;  
	\qa=5.3;  
	\psione=0;
	\qpsione=0;
	\psitwo=1.5;
	\h=-1;
	\labelx=(\psione+\psitwo)/2;
	\labely=-1.5;
	\phstartx=-1.1;
	\qphstartx=-3.8;
	\wphstartx=-0.8;
	\phstarty=\h;
	\tbeg=\phstarty;
	\ca=\phstartx-\tbeg;
	\ttwo=\psitwo-\ca;
	\cb=\ttwo+\ca;
	\xend=\ttwo-\a+\cb;
	\tone=\psione-\ca;
	\cc=\tone+\ca;
	\xendz=\tone-\a+\cc;
	\qca=\qphstartx-\tbeg;
	\qtone=\qpsione-\qca;
	\qqcc=\qtone+\qca;
	\qxendz=\qtone-\qa+\qqcc;
	\wa=5.3;  
	\wpsione=0;
	\wttestx=3.7;
	\wca=\wphstartx-\tbeg;
	\wtone=\wpsione-\wca;
	\wwcc=\wtone+\wca;
	\wxendz=\wtone-\wa+\wwcc;
	\wat=\wa-\wttestx;
	\wct=\wttestx;
	\wlam = 0.87;
		\we=0.1;
	\tthree=\cc+\tone-\psitwo;
	\circsize=0.08;
	\md = (\a+\h)/2;
	\tlen=0.75;
	\textscale = 0.7;
	\picscale = 0.78;
	\nudge=0.1;
	\tnudge=(\ttwo-\tone)/2+0.8;
	\ttesta=2*\tone-\ttwo-\tnudge;
	\sonea=\a+\psione-\ttesta;
	\sadd=0.5;
	\lrange = -(\psione-\a+\ttesta)+\sadd;
	\rrange =\lrange; 
	\timex=\rrange-1.5;
	\ttestx=2.4;
	\at=\a-\ttestx;
	\ct=\ttestx;
	\qttestx=1.2;
	\qat=\qa-\qttestx;
	\qct=\qttestx;
	\qlam = 0.87;
	\lam = 0.87;
	\e = 0.1;
	\qe=0.1;
	\hae=(2*pow(\at,3)*\lam*(2+\lam)+\e*\e*(2*\e-sqrt(4*\e*\e+\at*\at*\lam*\lam))-8*\at*\e*(-2*\e+sqrt(4*\e*\e+\at*\at*\lam*\lam))-2*\at*\at*(2+\lam)*(-2*\e+sqrt(4*\e*\e+\at*\at*\lam*\lam)))/(4*\at*\lam*(2*\at+2*\e-sqrt(4*\e*\e+\at*\at*\lam*\lam)))-(\e*\e/4+\at*\at*(-1+\lam))/(\at*\lam);
	\qhae=(2*pow(\qat,3)*\qlam*(2+\qlam)+\qe*\qe*(2*\qe-sqrt(4*\qe*\qe+\qat*\qat*\qlam*\qlam))-8*\qat*\qe*(-2*\qe+sqrt(4*\qe*\qe+\qat*\qat*\qlam*\qlam))-2*\qat*\qat*(2+\qlam)*(-2*\qe+sqrt(4*\qe*\qe+\qat*\qat*\qlam*\qlam)))/(4*\qat*\qlam*(2*\qat+2*\qe-sqrt(4*\qe*\qe+\qat*\qat*\qlam*\qlam)))-(\qe*\qe/4+\qat*\qat*(-1+\qlam))/(\qat*\qlam);
		\whae=(2*pow(\wat,3)*\wlam*(2+\wlam)+\we*\we*(2*\we-sqrt(4*\we*\we+\wat*\wat*\wlam*\wlam))-8*\wat*\we*(-2*\we+sqrt(4*\we*\we+\wat*\wat*\wlam*\wlam))-2*\wat*\wat*(2+\wlam)*(-2*\we+sqrt(4*\we*\we+\wat*\wat*\wlam*\wlam)))/(4*\wat*\wlam*(2*\wat+2*\we-sqrt(4*\we*\we+\wat*\wat*\wlam*\wlam)))-(\we*\we/4+\wat*\wat*(-1+\wlam))/(\wat*\wlam);
	} 
	\tikzstyle{scorestars}=[star, star points=5, star point ratio=2.25, draw, inner sep=1pt]%
	\tikzstyle{scoresquare}=[draw, rectangle, minimum size=1mm,inner sep=0pt,outer sep=0pt]%
	\begin{tikzpicture}[scale=\picscale,
	declare function={
		testxonep(\ps,\t)=\a+\ps-\t;
		testxonem(\ps,\t)=\ps-\a+\t; 
		bl(\x)=\ct+\at-\e/2-1/2*(sqrt(\lam*\lam*pow(\x+\hae+\at,2)+\e*\e)-\e+\lam*(\x+\hae+\at));
		br(\x)=\ct+\at-\e/2-1/2*(sqrt(\lam*\lam*pow(\x-\hae-\at,2)+\e*\e)-\e-\lam*(\x-\hae-\at));
		bc(\x)=\ct+sqrt(\lam*\lam*\x*\x+\e*\e)-\e;
		qbl(\qx)=\qct+\qat-\qe/2-1/2*(sqrt(\qlam*\qlam*pow(\qx+\qhae+\qat,2)+\qe*\qe)-\qe+\qlam*(\qx+\qhae+\qat));
		qbr(\qx)=\qct+\qat-\qe/2-1/2*(sqrt(\qlam*\qlam*pow(\qx-\qhae-\qat,2)+\qe*\qe)-\qe-\qlam*(\qx-\qhae-\qat));
		qbc(\qx)=\qct+sqrt(\qlam*\qlam*\qx*\qx+\qe*\qe)-\qe;
		wbl(\wx)=\wct+\wat-\we/2-1/2*(sqrt(\wlam*\wlam*pow(\wx+\whae+\wat,2)+\we*\we)-\we+\wlam*(\wx+\whae+\wat));
		wbr(\wx)=\wct+\wat-\we/2-1/2*(sqrt(\wlam*\wlam*pow(\wx-\whae-\wat,2)+\we*\we)-\we-\wlam*(\wx-\whae-\wat));
		wbc(\wx)=\wct+sqrt(\wlam*\wlam*\wx*\wx+\we*\we)-\we;
	},
	 ] 
	\definecolor{tempcolor}{RGB}{250,190,0}
	\definecolor{darkgreen}{RGB}{40,190,40}
	
	\draw[->,blue, thick] [domain=-\at/2:-\lrange, samples=150]   plot (\x, {bl(\x)})  ;
	\draw[blue, thick] [domain=-\at/2:\at/2, samples=150] plot (\x, {bc(\x)})   ;
	\draw[->,blue, thick] [domain=\at/2:\rrange, samples=150]   plot (\x, {br(\x)})   ;
	
	 
	 \draw[->,gray, thick] [domain=-\at/2:-\lrange, samples=150]   plot (\x, {qbl(\x)})  ;
	\draw[gray, thick] [domain=-\at/2:\at/2, samples=150] plot (\x, {qbc(\x)})   ;
	\draw[->,gray, thick] [domain=\at/2:\rrange, samples=150]   plot (\x, {qbr(\x)})   ;
	
	\draw[->,black, thick] [domain=-\at/2:-\lrange, samples=150]   plot (\x, {wbl(\x)})  ;
	 \draw[black, thick] [domain=-\at/2:\at/2, samples=150] plot (\x, {wbc(\x)})   ;
	 \draw[->,black, thick] [domain=\at/2:\rrange, samples=150]   plot (\x, {wbr(\x)})   ;

	  
	  
	
	
	 \node[scale=\textscale]  at (2.5,4.1) {$S_{n,f}$}; 
	  \node[scale=\textscale]  at (3.9,4.1) {$S_{n,i}$}; 
	  	  \node[scale=\textscale]  at (1.1,4.1) {$S_{n,m}$}; 
	
	\draw[<->] (-\lrange, \h) node[left, scale=\textscale] {$S_0$} -- (\rrange, \h) node[right, scale=\textscale] {$S_0$};
	\draw[<->] (-\lrange, \a) node[left, scale=\textscale] {$S$} -- (\rrange, \a) node[right, scale=\textscale] {$S$};
				  
	\draw[->, shorten <= 5pt,  shorten >= 1pt] (\psione,\h)  node[above right, scale=\textscale]{$z_0$}-- (\psione,\a) ;
	
	
	\draw[dashed, tempcolor,  ultra thick](\phstartx,\tbeg) node[below left=-2,scale=\textscale]{}--(\psione,\tone) node[above, pos=0.3, rotate=45,scale=0.7,black] {} node[below, pos=0.3, rotate=45,scale=0.7,black] {};
	\draw[dashed, tempcolor,  ultra thick](\psione,\tone)--(\xendz,\a);
	
	
	\draw[dashed, tempcolor, ultra thick](\qphstartx,\tbeg) node[below left=-2,scale=\textscale]{}--(\psione,\qtone) node[above, pos=0.3, rotate=45,scale=0.7,black] {} node[below, pos=0.3, rotate=45,scale=0.7,black] {};
	\draw[dashed, tempcolor, ultra thick](\psione,\qtone)--(\qxendz,\a);

	\draw[dashed, orange](\wphstartx,\tbeg) node[below left=-2,scale=\textscale]{}--(\psione,\wtone) node[above, pos=0.3, rotate=45,scale=0.7,black] {} node[below, pos=0.3, rotate=45,scale=0.7,black] {};
	\draw[dashed, orange](\psione,\wtone)--(\wxendz,\a);
	
	\draw[->] (\timex,\md-\tlen/2) --  (\timex,\md+\tlen/2) node[midway,right, scale=\textscale]{time}; 
	 
	\draw[dotted](\psione,\ttestx)--({testxonep(\psione,\ttestx)},\a);
	\draw[dotted](\psione,\ttestx)--({testxonem(\psione,\ttestx)},\a);
	\draw[dotted](\psione,\qttestx)--({testxonep(\psione,\qttestx)},\a);
	\draw[dotted](\psione,\qttestx)--({testxonem(\psione,\qttestx)},\a);
	
	\draw (\psione,\ttestx) node[ scoresquare, fill=gray]  {} node [black,below=6,right,scale=\textscale] {$t_f$};
	\draw (\psione,\qttestx) node[ scoresquare, fill=gray]  {} node [black,below=6,right,scale=\textscale] {$t_i$};
	\draw (\psione,\wttestx) node[ scoresquare, fill=gray]  {} node [black,below=6,right,scale=\textscale] {$t_m$};
	
%	\draw[magenta,->,ultra thick] ({testxonem(\psione,\qttestx)},\a)--(-\lrange,\a);
%	\draw[magenta,ultra thick] ({testxonem(\psione,\ttestx)},\a)-- ({testxonem(\psione,\qttestx)},\a);
%	\draw[magenta,ultra thick] ({testxonep(\psione,\ttestx)},\a)-- ({testxonep(\psione,\qttestx)},\a);
%	\draw[magenta,->,ultra thick] ({testxonep(\psione,\qttestx)},\a)--(\rrange,\a);
	
	\draw [black,fill](\xendz,\a) circle [radius=\circsize] node [black,above=9,right=-4,scale=\textscale] {$\gamma_i^{(\mathcal{A})}$}; 
	\draw [black,fill](\wxendz,\a) circle [radius=\circsize] node [black,above=9,left=-7,scale=\textscale] {$\gamma_i^{(\mathcal{S})}$}; 
	\draw [black,fill](\qxendz,\a) circle [radius=\circsize] node [black,above,scale=\textscale] {$\gamma_f^{\prime}$}; 
		%\draw [black,fill] (\xendz,\a) circle [radius=\circsize] node [black,above,scale=\textscale] {$\gamma_1$}; 
	\end{tikzpicture}% pic 1
	
	\vspace*{2px}
	\caption{Depicts an experiment where the state of some photons $\gamma_i^{(\mathcal{S})}$ and $\gamma_i^{(\mathcal{A})}$ on the hypersurface $S$ determines the initial conditions of an experimental setup of a particle $\mathcal{S}$ and apparatus $\mathcal{A}$ in the vicinity of the spacetime location $( t_i, z_0)$.  The state of the photons $\gamma_f^{\prime}$ on the hypersurface $S$ determines the final state of the apparatus $\mathcal{A}$ after the particle $ \mathcal{S}$ has finished interacting with it from time $t_f$ onwards so that the apparatus at time $t_m$ displays a definite measurement outcome. It is assumed that no incoming photons have become entangled with the experiment after the $\gamma_i^{(\mathcal{S})}$ and $\gamma_i^{(\mathcal{A})}$ photons and before the $\gamma_f^{\prime}$ photons have become entangled with the experiment.  }
	\label{pisolution}
	\end{figure}
This means that prior to $t_i$, the apparatus will have interacted with many photons so that the photon detections on $S$ outside the light cone of $y_i=(t_i, z_0)$ will indicate via Kent's stress-energy beables that there is some apparatus in the vicinity of $z_0$ that we can identify as $\mathcal{A}$. We further assume that Kent's stress-energy beables determine that $\mathcal{A}$ is in a state $\ket{a}$ which encapsulates among other things the measurement parameters of the apparatus.  We also suppose that the information in $\tau_S$ indicates that there is a particle $\mathcal{S}$ at time $t_i$ in a state
$\ket{s}$ that is heading towards the apparatus so that it will interact with it. This means that all the 
possible\footnote{i.e. possible given $\ket{\Psi_0}$ and the Born rule selection criterion.} 
simultaneous $\hat{T}_S$-eigenstates on $S$ whose simultaneous $\hat{T}_S$-eigenvalues agree with $\tau_S$ for all $x\in S^1(y_i)$\footnote{Recall from page \pageref{S2} that $S^1(y_i)$ is the subset of $S$ that is outside the light cone of $y_i$.} 
 are such that their simultaneous $\hat{T}_S$-eigenvalues within the light cone of $y_i$ indicate that at some time after $t_i$, the particle $\mathcal{S}$ would have interacted
with the apparatus $\mathcal{A}$. 

At this point, it will be helpful to define a \textbf{simultaneous $\hat{T}_S$-eigenstate of a subregion}\index{simultaneous $\hat{T}_S$-eigenstate of a subregion} $U$ of $S$ to be a state $\ket{\Psi_U}\in H_U$  which is an eigenstate of $\hat{T}_S(u)$ for every $u\in U$, where $H_U$ is the Hilbert space of states describing $U$. We will assume that we can write any state of $H_S$ as a superposition of states of the form $\ket*{\Psi_U}\ket*{\Psi_{S\setminus U}}$, where $\ket{\Psi_U}\in H_U $ is  a simultaneous $\hat{T}_S$-eigenstate of the subregion $U$ of $S$, and $\ket*{\Psi_{S\setminus U}}\in H_{S\setminus U}$ is a simultaneous
$\hat{T}_S$-eigenstate of the subregion $S\setminus U$ of $S$.\footnote{We effectively made this assumption in equation (\ref{Sistate}).} We will accordingly let  
$\ket*{\gamma^{(S^1(y_i))}_i}$ denote the simultaneous $\hat{T}_S$-eigenstate of the subregion $S^1(y_i)$ of $S$ with simultaneous $\hat{T}_S$-eigenvalue $\tau_S(x)$ respectively for all $x\in S^1(y_i)$. Then any  simultaneous $\hat{T}_S$-eigenstate over the whole of $S$  whose simultaneous $\hat{T}_S$-eigenvalue agrees with 
$\tau_S$ on the subregion $S^1(y_i)$ will be expressible as
$\ket*{\gamma^{(S^1(y_i))}_i}\ket*{\Psi_{S\setminus S^1(y_i)}}$, 
and any such simultaneous $\hat{T}_S$-eigenstate will determine the apparatus $\mathcal{A}$ and the particle $\mathcal{S}$ to exist in states $\ket{a}$ and $\ket{s}$ respectively given the state $\ket*{\Psi_0}$ of the initial hypersurface $S_0$. In figure \ref{pisolution}, rather than depicting $\ket*{\gamma^{(S^1(y_i))}_i}$, we depict the two components $\ket*{\gamma^{(\mathcal{S})}_i}$ and $\ket*{\gamma^{(\mathcal{A})}_i}$ of $\ket*{\gamma^{(S^1(y_i))}_i}$ which are simultaneous $\hat{T}_S$-eigenstates for subregions $S_{\mathcal{S}}$ and $S_{\mathcal{A}}$ of $S^1(y_i)$ where the states of $\mathcal{S}$ and $\mathcal{A}$ are determined respectively, so that 
\begin{equation}\label{gammadecomp}
\ket*{\gamma^{(S^1(y_i))}_i}=\ket*{\gamma^{(\mathcal{S})}_i}\ket*{\gamma^{(\mathcal{A})}_i}\ket*{\Xi^{(S^1(y_i))}_i}
\end{equation}
where $\ket*{\Xi^{(S^1(y_i))}_i}$ is a simultaneous $\hat{T}_S$-eigenstate over the remainder of $S^1(y_i)$, i.e. for the subregion $S^1(y_i)\setminus(S_{\mathcal{S}}\cup S_{\mathcal{A}})$. Note that it is not necessary that the state $\ket*{\gamma^{(\mathcal{S})}_i}$ is caused by photons that have interacted directly with $\mathcal{S}$. Rather, it is sufficient that the photons have been reflected from some system $\mathcal{B}$, say, that has interacted with $\mathcal{S}$ for which $\ket*{s}$ is a pointer state, and that enough photons from $\mathcal{B}$ have intersected $S$ so as to distinguish its different pointer states. 

We also assume that there are no further interactions of $\mathcal{S}$ with photons that are registered on $S$ until the particle has finished interacting with the apparatus $\mathcal{A}$.  In making this assumption, we suppose that there is a time $t_f>t_i$ such that if the notional measurement on $S$ had resulted in the outcome $\tau_S^{(j)}$ rather than $\tau_S$ where $\tau_S^{(j)}(x)=\tau_S(x)$ for all $x\in S^1(y_i)$, then for all $x \in S^1(y_f,y_i)=S^1(y_f)\setminus S^1(y_i)$,\footnote{i.e $S^1(y_f,y_i)$ is the subset of $S$ that is within the light cone of $y_i=(t_i, z_0)$ but outside the light cone of $y_f=(t_f, z_0)$.} we would be able to say of the mass-energy density $\tau_S^{(j)}(x)$ at $x$ that it wasn't caused by a photon that had been reflected from $\mathcal{S}$ or from something that had become entangled with $\mathcal{S}$.\footnote{This assumption of our toy model assumes that photons can be treated as point-particles so that they have precise trajectories and that different photons will almost certainly be detected at different locations on $S$. In physical reality, we can't treat photons as point-particles, and so there will be some ambiguity about where the photons detected on $S$ came from.}

We will assume that given that the notional measurement restricted to $S^1(y_i)$ is $\tau_S$, there is sufficiently little interaction of the composite system $\mathcal{S}+\mathcal{A}$ with its environment so that we can assume it evolves unitarily in accordance with the Schr\"{o}dinger equation. Therefore, given that the apparatus $\mathcal{A}$ is in a state $\ket{a}$ that encapsulates its parameter settings, and the particle $\mathcal{S}$ is on course to interact with $\mathcal{A}$, we can express the state $\ket{s}$ of $\mathcal{S}$ as a superposition $\ket{s}=\sum_j c_j \ket{s_j}$ where $\{\ket{s_j}:j\}$ is the set of pointer states corresponding to a particular parameter setting of the apparatus $\mathcal{A}$ that are encapsulated in the state $\ket{a}$. As described on page \pageref{pointer}, this means there are future states $\ket{a_j}$ of the apparatus $\mathcal{A}$ such that $\ip{a_j}{a_{j'}}=0$ for $j\neq j'$,\footnote{Strictly speaking, $\ip{a_j}{a_{j'}}\approx 0$ for $j\neq j'$, but we will assume $\ip{a_j}{a_{j'}}=0$ in this toy model in order to avoid undue complexity.} and such that the composite system $\mathcal{S}+\mathcal{A}$ will evolve according to the Schr\"{o}dinger equation as
$$\ket{s_j}\ket{a}\rightarrow \ket{s_j}\ket{a_j} $$
for each pointer state $\ket{s_j}$, and hence
\begin{equation}\label{saevolution}
\ket{s}\ket{a}\rightarrow \sum_j c_j \ket{s_j}\ket{a_j}.
\end{equation}

We will also assume that given that the notional measurement restricted to $S^1(y_i)$ is $\tau_S$, there is a time interval between $t_f$ and $t_m > t_f$ during which photons will reflect off the apparatus $\mathcal{A}$ and ultimately be detected at spacetime locations on the hypersurface $S$ that correspond to one of the definite measurement states $\ket{a_f}$ of the apparatus indicating that $\mathcal{S}$ is in the state $\ket{s_f}$. But whatever the  notional measurement on $S$ is, so long as it results in an outcome $\tau_S^{(j)}$ with $\tau_S^{(j)}(x)=\tau_S(x)$ for all $x\in S^1(y_i)$, then we're assuming that for all $x \in S^1(y_m,y_f),$\footnote{where $y_m=(t_m, z_0)$.} we would be able to say of the mass-energy density $\tau_S^{(j)}(x)$ at $x$ that it was caused by a photon that had been reflected from the apparatus $\mathcal{A}$ being in some state $\ket{a_j}$ rather than one of the other states.\footnote{This is again an assumption of our toy model, so it may not be true in more realistic models where there are many more object off which photons could reflect.} We will let $S_{\mathcal{A}}'$ denote the subregion of $S^1(y_m,y_f)$ where photons coming from the apparatus arrive and hence determine which state the apparatus  is in, and we will let $\ket*{\gamma'_j}$ denote the simultaneous $\hat{T}_S$-eigenstate of the subregion $S_{\mathcal{A}}'$ of $S$ that corresponds to the apparatus $\mathcal{A}$ being in state $\ket{a_j}$.  We will also label the states on the subregion $S\setminus (S^1(y_i)\cup S_{\mathcal{A}}')$ as $\ket{\Xi_j}$ so that we can express the state $\ket*{\Psi_S}=U_{SS_0}\ket*{\Psi_0}$ as a superposition
\begin{equation}\label{PsiSdecomp}
\ket*{\Psi_S}=b\ket*{\gamma_i^{(S^1(y_i))}}\sum_j c_j \ket*{\gamma_j'}\ket*{\Xi_j} +\sum_{k\neq 0} b_k\ket*{\gamma_k^{(S^1(y_i))}}\ket*{\Xi_k'}
\end{equation}
where the $\ket*{\gamma_k^{(S^1(y_i))}}$ for $k\neq i$ are simultaneous $\hat{T}_S$-eigenstates of the subregion $S^1(y_i)$ whose simultaneous $\hat{T}_S$-eigenvalues restricted to $S^1(y_i)$ are distinct from $\tau_S$, where $\ket*{\Xi_k'}$ are states of the subregion $S\setminus S^1(y_i)$, and where $b$  is a complex numbers whose modulus squared gives the probability that the notional measurement on $S^1(y_i)$ will be $\tau_S$, and likewise, the modulus squared of the $b_k$ give the probabilities of the other possible notional measurements on $S^1(y_i).$ 

Now we aim to show that within Kent's theory, we can calculate the probability the particle emerges from the measuring apparatus $\mathcal{A}$ in state $\ket{s_{f}}$ given that it enters $\mathcal{A}$ in state $\ket{s}$, and that this probability is the same as if one ignored $S$ and just applied the Born Rule to $\ket{s}$ and $\ket{s_f}$. 

In order to show this, let us choose a sequence of hypersurfaces $S_{n,i}$ which go through the spacetime location $y_i=(t_i, z_0)$ such that $\lim_{n\rightarrow\infty} S_{n,i}\cap S=S^1(y_i).$\footnote{We understand this limit first by giving $S^1(y_i)$ and $S_{n,i}$  topologies that makes them locally homeomorphic to Euclidean space. Then by saying there is this limit of hypersurfaces, we mean that every point $u\in S^1(y_i)$ has a neighborhood $U\subset S^1(y_i)$ for which there is an integer $N$ such that $U\subset S_{n,i}$ for all $n\geq N$.} Let us assume that $n$ is sufficiently large so that the photons described by $\ket*{\gamma_i^{(\mathcal{S})}}$ and $\ket*{\gamma_i^{(\mathcal{A})}}$ belong to $S_{n,i}$. The hypersurface $S_{n,i}$ and the photons being reflected from the vicinity of $z_0$ just before time $t_i$ are  depicted in figure \ref{pisolution}.

With equation (\ref{PsiSdecomp}) in mind, we can express the quantum state $\ket{\Psi_{n,i}}=U_{S_{n,i},S_0}\ket{\Psi_0}$  of the hypersurface $S_{n,i}$ as a superposition 
\begin{equation}\label{Psinidecomp}
\ket{\Psi_{n,i}}=b\ket*{\gamma_i^{(\mathcal{S})}}\ket*{\gamma_i^{(\mathcal{A})}}\ket{s}\ket{a}\ket{\xi_{n,i}}+\cdots.
\end{equation} 
The $\ket*{\gamma_i^{(\mathcal{S})}}\ket*{\gamma_i^{(\mathcal{A})}}$ component of the first summand of (\ref{Psinidecomp}) is extracted from the $\ket*{\gamma_i^{(S^1(y_i))}}$ component of (\ref{PsiSdecomp}) which we can do because we're assuming $S_{n,i}$ overlaps with $S$ in the subregion $S^1(y_i)$ corresponding to the states $\ket*{\gamma_i^{(\mathcal{S})}}$ and $\ket*{\gamma_i^{(\mathcal{A})}}$. Corresponding to the region in the vicinity of $y_i$ on $S_{n,i}$, we have the components $\ket{s}\ket{a}$ because we're assuming that the measurement of $\tau_S$ on $S^1(y_i)$ guarantees that $\mathcal{S}$ and $\mathcal{A}$ are in the states $\ket{s}$ and $\ket{a}$ respectively. The component $\ket{\xi_{n,i}}$ corresponds to the state of all the other regions of $S_{n,i}$ not determined by $\ket*{\gamma_i^{(\mathcal{S})}}$, $\ket*{\gamma_i^{(\mathcal{A})}}$, $\ket{s}$ or $\ket{a}$, and the ellipses correspond to the summation over $k$ term of (\ref{PsiSdecomp}) suitably modified so that the summands are states on $S_{n,i}$ rather than $S$.

If we now define the projection $\pi_{n,i}$ corresponding to the measurement outcome  $\tau_S(x)$ on $S_{n,i}\cap S$ as in equation (\ref{tauprojection}), then
\begin{equation}\label{piphi}
	\pi_{n,i}\ket{\Psi_{n,i}}\approx b\ket*{\gamma_i^{(\mathcal{S})}}\ket*{\gamma_i^{(\mathcal{A})}}\ket{s}\ket{a}\ket{\xi_{n,i}},
\end{equation}
and the larger $n$ is, the closer (\ref{piphi}) will come to being an equality. The key thing to note about (\ref{piphi}) is that the systems $\mathcal{S}$ and $\mathcal{A}$ are not entangled with each other or with the environment. We can therefore think of the measurement outcome of $\tau_S(x)$ on $S_{n,i}\cap S$ for sufficiently large $n$ as specifying the initial states $\ket{s}$ and $\ket{a}$ of   $\mathcal{S}$ and $\mathcal{A}$ before they interact. If $n$ was too small, $S_{n,i}\cap S$ might not contain the subregion for which $\ket{\gamma_i^{\mathcal{A}}}$ is a state, in which case we could expect the $\mathcal{A}$-component of $\pi_{n,i}\ket{\Psi_{n,i}}$ to be entangled with the environment with different environment states being correlated with different parameter settings of $\mathcal{A}$. Likewise, we could expect the $\mathcal{S}$-component of $\pi_{n,i}\ket{\Psi_{n,i}}$ to be entangled with  the environment with different environment states being correlated with different states of $\mathcal{S}$ if $n$ was too small. In some situations, the rate at which $\mathcal{S}$ interacts with its immediate environment may be greater than the rate at which photons from the immediate environment of $\mathcal{S}$ are registered on $S$, in which case, it would not be possible to disentangle $\mathcal{S}$ from its environment in $\pi_{n,i}\ket{\Psi_{n,i}}$. But in controlled experimental settings in which the system $\mathcal{S}$ is prepared to be in a definite state, it should be possible to disentangle $\mathcal{S}$ from its environment in $\pi_{n,i}\ket{\Psi_{n,i}}$. So in this situation, to extract the state of a system $\mathcal{S}$ in the vicinity of $z_0$ at time $t_i$ from $\ket{\Psi_0}$ and $\tau_S$, we need to take a hypersurface $S_{n,i}$ for sufficiently large $n$ that goes through $y_i=(t_i, z_0)$. The state of $\mathcal{S}$ in the vicinity of $y_i$ will then be the disentangled normalized $\mathcal{S}$-component of $\pi_{n,i}\ket{\Psi_{n,i}}=\pi_{n,i}U_{S_{n,i},S_0}\ket{\Psi_0}$. So in the case of (\ref{Psinidecomp}), the disentangled normalized $\mathcal{S}$-component of $\pi_{n,i}\ket{\Psi_{n,i}}$ will be $\ket{s}$.

\begin{comment}
  For convenience, we will omit the reference to $n$ and write $S_i$ for $S_{n,i}$, $\ket{\Psi_i}$ for $\ket{\Psi_{n,i}}$, and $\pi_i$ for $\pi_{n,i}$. We will also write $\ket{\Phi_i}$ for the normalized state of $\pi_i\ket{\Psi_i}$ so that 
 \begin{equation}\label{phii}
	\ket{\Phi_i}\approx \ket{s}\ket{a}\ket*{\gamma_i^{(\mathcal{S})}}\ket*{\gamma_i^{(\mathcal{A})}}.
\end{equation}

Now we're assuming that the measurement outcome $\tau_S$ on the whole of $S$ will determine the apparatus $\mathcal{A}$ to be in the state $\ket{a_f}$ by time $t_m$, and hence the particle $\mathcal{S}$ to be in the state $\ket{s_f}$, and $\tau_S$ will determine this outcome with probability $1$. However, if the the mass-energy density was only determined to be $\tau_S$ on $S^1(y_i)$, then the probability the apparatus and the particle would be found to be in the states $\ket{a_f}$ and $\ket{s_f}$ respectively at time $t_m$ would typically be less than $1$.
 
 To calculate this probability, we first consider the evolution of $\ket{\Phi_i}$ to $ U_{S_f,S_i}\ket{\Phi_{i}}$. It's possible that there may be photons ``measured'' on $S$ between the hypersurfaces $S_i$ and $S_f$ as indicated in figure \ref{pisolution} (i.e. on $(S\cap S_f)\setminus(S\cap S_i)$), but we are assuming that they do not get entangled with the different pointer states of the apparatus. In other words, $\ket{\Phi_i}$ will evolve to a state of the form 
\begin{equation}\label{USfievolve1}
	U_{S_f,S_i}\ket{\Phi_{i}}\approx \sum_j c_j\ket{s_j}\ket{a_j}\ket*{\gamma_i^{(\mathcal{S})}}\ket*{\gamma_i^{(\mathcal{A})}}\sum_kg_k\ket{\gamma'_k},
\end{equation}
where $\ket{\gamma_k'}$ correspond to the possible measurements of $T_S(x)$  on $(S\cap S_f)\setminus(S\cap S_i)$, and $\sum_k\abs{g_k}^2=1$.

But from time $t_f$ to $t_m$, we assume that the apparatus does get entangled with photons which are measured on $S\cap S_m$. Thus, if $\{\ket{\gamma_j^{\prime\prime}}:j\}$ are the normalized states representing the possible measurements outcomes of these photons such that $\ip{\gamma_j^{\prime\prime}}{\gamma_k^{\prime\prime}}\approx 0$ for $j\neq k$, then 
$$U_{S_m,S_f} U_{S_f,S_i}\ket{\Phi_{i}}\approx \sum_j c_j\ket{s_j}\ket{a_j}\ket*{\gamma_i^{(\mathcal{S})}}\ket*{\gamma_i^{(\mathcal{A})}}\sum_kg_k\ket{\gamma'_k}\ket{\gamma_j^{\prime\prime}}.$$
Since we are assuming that at time $t_m$ a measurement of photons on $S\cap S_m$ is able to determine that the apparatus is in state $\ket{a_f}$, this can only happen if $U_{S_m,S_f} U_{S_f,S_i}\ket{\Phi_{i}}$ is found to be in one of the states $\ket{\Phi_{k,f}}$ for some $k$ where
$$\ket{\Phi_{k,j}}=\ket{s_j}\ket{a_j}\ket*{\gamma_i^{(\mathcal{S})}}\ket*{\gamma_i^{(\mathcal{A})}}\ket{\gamma'_k}\ket{\gamma_j^{\prime\prime}}.$$
By the Born Rule, the probability $\ket{\Phi_i}$ will be found to be in state $\ket{\Phi_{k,j}}$ will be
$$\abs{\ip{\Phi_{k,j}}{\Phi_i}}=\abs{c_j}^2\abs{g_k}^2.$$
Therefore,
$$P(f|\ket{\Phi_i})=\sum_k \abs{\ip{\Phi_{k,f}}{\Phi_i}}=\abs{c_f}^2=\abs{\ip{s_f}{s}}^2.$$
Hence, the probability that a complete measurement of $T_S(x)$ on $S$ will give a measurement outcome of the particle being in state $\ket{s_f}$ given the partial measurement of $T_S(x)$ on $S_i\cap S$ determines the particle to be initially in the state $\ket{s}$ will be the same as the standard Born Rule probability $\abs{\ip{s_f}{s}}^2$ of $\ket{s}$ being found to be in state $\ket{s_f}$.

\end{comment}

Now given that the system $\mathcal{S}$ is in state $\ket{s}$ and that $\ket{s_f}$ was one of the possible measurement outcome states for $\mathcal{S}$, according to standard quantum mechanics, the Born rule would predict that the measurement outcome state $\ket{s_f}$ occurs with a probability of $|\ip{s}{s_f}|^2$. We will now show that we can obtain this probability in Kent's theory as well. To do this, we recall that in standard quantum theory, if  we define the operator $[\psi]=\dyad{\psi}$ for some state $\ket{\psi}$ of a system, then when the system is in some initial state $\ket{\chi}$, the Born Rule implies that $\ev{[\psi]}{\chi}=P(\psi|\chi)$, where $P(\psi|\chi)$ is the probability that the system will be found to be in state $\ket{\psi}$ given that it was initially in state $\ket{\chi}$. But by (\ref{evev}),  $\ev{[\psi]}{\chi}$ is just the expectation $\ev*{[\psi]}_\chi$ of $[\psi]$ when $[\psi]$ is treated as an observable. 

Now in equation (\ref{kentconsistency0}), we saw how to calculate the expectation value $\ev*{T^{\mu\nu}(y)}_{\tau_S}$ of the observable $\hat{T}^{\mu\nu}(y)$ given the notional measurement $\tau_S$ on $S$ outside the light cone of $y$. This suggests that the expectation value of any observable $\hat{O}$ defined at spacetime location $y_i=(t_i,z_0)$ given the notional measurement $\tau_S$ on $S$ outside the light cone of $y_i$ is going to be
\begin{equation}\label{kentevo}
\ev*{\hat{O}}_{\tau_S}=\lim_{n\rightarrow\infty}\frac{\ev*{\pi_{n,i}\hat{O}}{\Psi_{n,i}}}{\ev*{\pi_{n,i}}{\Psi_{n,i}}}.
\end{equation}
By (\ref{piphi}), $\lim_{n\rightarrow\infty}\ev*{\pi_{n,i}}{\Psi_{n,i}}=\abs{b}^2$, and so taking $\hat{O}$ to be $[s_f]$ at spacetime location $y_i$ we have 
$$ \ev*{[s_f]}_{\tau_S}=\frac{\abs{b}^2\abs{\ip{s_f}{s}}^2}{\abs{b}^2}=\abs{\ip{s_f}{s}}^2.$$
Thus, Kent's conditional expectation $\ev*{[s_f]}_{\tau_S}$ at spacetime location $y_i$ gives us the same probability $\abs{\ip{s_f}{s}}^2$ for a particle transitioning from state $\ket{s}$ to state $\ket{s_f}$ as in standard quantum theory.

We can also use Kent's conditional expectation to show that at time $t_m$, $\ket{s_f}$ occurs with probability $1$ given the notional measurement $\tau_S$ on $S$ outside the light cone of $y_m=(t_m,z_0)$ has $\ket{\gamma_f'}$ as a component . To see this, first note that since we are assuming that between times $t_i$ and $t_f$, $\ket{s}\ket{a}$ evolves according to (\ref{saevolution}), we can apply $U_{S_{n,f},S_{n,i}}$ to (\ref{Psinidecomp}) (where $S_{n,f}$ is one of the hypersurfaces that goes through $y_f$ as depicted in figure \ref{pisolution}) to get
\begin{equation}\label{Psinfdecomp}
	\ket{\Psi_{n,f}}=b\ket*{\gamma_i^{(\mathcal{S})}}\ket*{\gamma_i^{(\mathcal{A})}}\sum_j c_j \ket{s_j}\ket{a_j}\ket{\xi_{n,f}}+\cdots.
	\end{equation} 
In (\ref{Psinfdecomp}), the component $\ket{\xi_{n,f}}$ corresponds to the state of all the other regions of $S_{n,f}$ not determined by $\ket*{\gamma_i^{(\mathcal{S})}}$, $\ket*{\gamma_i^{(\mathcal{A})}}$, or the state of $\mathcal{S}$ and $\mathcal{A}$ in the vicinity of $y_f=(t_f, z_0)$, and the ellipses correspond to the ellipses of (\ref{Psinidecomp}) to which $U_{S_{n,f},S_{n,i}}$ has been applied. 

Since we need to calculate Kent's conditional expectation of $[s_f]$ at spacetime location $y_m$, we need to apply  $U_{S_{n,m},S_{n,f}}$ to (\ref{Psinfdecomp}), where $S_{n,m}$ is one of the hypersurfaces that goes through $y_m$ as depicted in figure \ref{pisolution}. To do this, we continue to assume that no photons interact with $\mathcal{S}$ between times $t_f$ and $t_m$. However, we do assume that photons will interact with the apparatus, and for large enough $n$, the $S_{n,m}$ hypersurfaces will contain the subregion $S_{\mathcal{A}}'$ of $S^1(y_m)$ where the photons coming from the apparatus arrive, and so for each $j$, the $\ket{\gamma_j'}$-state that corresponds to the $\ket{a_j}$-state of the apparatus $\mathcal{A}$ and which forms a component of one of the summands of $\ket{\Psi_S}$ as shown in equation (\ref{PsiSdecomp}) will also appear $\ket{\Psi_{n,m}}$. It therefore follows that 
\begin{equation}\label{Psinfdecomp}
	\ket{\Psi_{n,f}}=b\ket*{\gamma_i^{(\mathcal{S})}}\ket*{\gamma_i^{(\mathcal{A})}}\sum_j c_j \ket{s_j}\ket{a_j}\ket{\gamma_j'}\ket{\xi_{n,m}}+\cdots.
\end{equation} 
where the component $\ket{\xi_{n,m}}$ corresponds to the state of all the regions of $S_{n,m}$ not determined by $\ket*{\gamma_i^{(\mathcal{S})}}$, $\ket*{\gamma_i^{(\mathcal{A})}}$,  $S_{\mathcal{A}}'$, or the state of $\mathcal{S}$ and $\mathcal{A}$ in the vicinity of $y_m=(t_m, z_0)$, and the ellipses correspond to the ellipses of (\ref{Psinfdecomp}) to which $U_{S_{n,m},S_{n,f}}$ has been applied. Since we are assuming that the notional measurement restricted to $S^1(y_m)$ will correspond to the apparatus being in the definite measurement state $\ket{a_f}$, then defining the projection corresponding to the measurement outcome $\tau_S$ on $S_{n,m}\cap S$ as in equation (\ref{tauprojection}), we will have
\begin{equation}\label{piphim}
\pi_{n,m}\ket{\Psi_{n,m}}\approx b c_f \ket*{\gamma_i^{(\mathcal{S})}}\ket*{\gamma_i^{(\mathcal{A})}}\ket{s_f}\ket{a_f}\ket{\gamma_f'}\ket{\xi_{n,m}}
\end{equation}
and the larger $n$ is, the closer (\ref{piphim}) will come to being an equality. 

We can now use (\ref{kentconsistency}) to calculate Kent's conditional expectation of $\ev*{[s_f]}_{\tau_S}$ at spacetime location $y_m$. By (\ref{piphim}), we have  $\lim_{n\rightarrow\infty}\ev*{\pi_{n,m}}{\Psi_{n,m}}=\abs{b}^2 \abs{c_f}^2 $, and $\lim_{n\rightarrow\infty}\ev*{[s_f] \pi_{n,m}}{\Psi_{n,m}}=\abs{b}^2 \abs{c_f}^2 $, and so by (\ref{kentevo}), at spacetime location $y_m$ we have
$$ \ev*{[s_f]}_{\tau_S}=1,$$
and so according to Kent's theory, $\mathcal{S}$ will be in state $\ket{s_f}$ by time $t_m$.

Also note that we can typically expect the $\ket*{\gamma_i^{(\mathcal{S})}}$-state to be independent of the $\ket*{\gamma_i^{(\mathcal{A})}}$-state. Therefore, since $\ket*{\gamma_i^{(\mathcal{A})}}$ will determine the measurement choice, and since $\ket*{\gamma_i^{(\mathcal{S})}}$ determines the initial state of the particle, we can expect the state of the particle to be independent of the measurement choice in Kent's theory. Thus, we can fulfil one of the necessary criteria (i.e. criterion \ref{hidden2}) for PI to be a well-defined notion.

We can also choose a set of $\lambda$ so that criterion \ref{hidden5} holds. To do this, we consider equation (\ref{gammadecomp}) which presupposes there is a subregion $S_{\mathcal{S}}$ of $S^1(y_i)$ which determines the state $\mathcal{S}$ and another non-overlapping subregion $S_{\mathcal{A}}$ that determines the state $\mathcal{A}$. In general, we wouldn't be able to make this association between subregions of $S^1(y_i)$ and states of  $\mathcal{S}$  and $\mathcal{A}$ -- after all, $\mathcal{A}$ might not even exist. But we should be able to make this association if an appropriate choice for the state of the remainder of $S^1(y_i)$ is made. In (\ref{gammadecomp}), $\ket*{\Xi_i^{(S^1(y_i))}}$ is able to serve this role. We can then suppose that $\ket*{\gamma_i^{(\mathcal{S})}}$ is from a basis $\Lambda_{\mathcal{S}}=\{\ket*{\gamma_{i,1}^{(\mathcal{S})}},\ket*{\gamma_{i,2}^{(\mathcal{S})}},\ldots\}$ of states that describe all the states of the subregion $S_{\mathcal{S}}$ of $S^1(y_i)$ corresponding to $\mathcal{S}$ that together with $\ket*{\Xi_i^{(S^1(y_i))}}$ and $\ket*{\Psi_S}$ determine $\mathcal{S}$ to be in the state $\ket{s}$. We could then take the $\lambda$ of the system $\mathcal{S}$ in criterion \ref{hidden5} to be one of the basis states in $\Lambda_{\mathcal{S}}$. Given that the interpretation of the states in $\Lambda_{\mathcal{S}}$ presupposes $\ket*{\Xi_i^{(S^1(y_i))}}$, we would take the probability $p_\lambda$ for $\lambda = \ket*{\gamma_i^{(\mathcal{S})}}$ to be the probability that the notional measurement on $S$ given $\ket*{\Xi_i^{(S^1(y_i))}}$ and $\ket*{\Psi_S}$ agreed with the mass-energy density values specified by $ \ket*{\gamma_i^{(\mathcal{S})}}$ on the subregion $S_{\mathcal{S}}$ of $S^1(y_1)$ that $ \ket*{\gamma_i^{(\mathcal{S})}}$ describes. If we let $\{\ket*{Z_1}, \ket*{Z_2},\ldots  \}$ be a basis of simultaneous $\hat{T}_S$-eigenstates for the subregion $(S\setminus S^1(y_1))\cup S_{\mathcal{A}}$ so that states of the form $\ket*{\gamma_{i,l}^{(\mathcal{S})}}\ket*{\Xi_i^{(S^1(y_i))}}\ket*{Z_k}$ will be simultaneous $\hat{T}_S$-eigenstates for the whole of $S$, then the formula for the probability $p_\lambda$ will be 
\begin{equation}\label{plambda}
p_{\lambda}=\frac{\sum_k |\ip*{\Psi_S}{\gamma_i^{(\mathcal{S})}}\ket*{\Xi_i^{(S^1(y_i))}}\ket*{Z_k}|^2}{\sum_{k,l} |\ip*{\Psi_S}{\gamma_{i,l}^{(\mathcal{S})}}\ket*{\Xi_i^{(S^1(y_i))}}\ket*{Z_k}|^2}.
\end{equation}
We can then show that a version of EA analogous to (\ref{adeq}) on page \pageref{adeq} holds. To express EA in this context, we let
$O_{\mathcal{S}}$ be the observable that returns $j$ if the system $\mathcal{S}$ is measured to be in the state $\ket*{s_j}$, and for $\lambda = \ket*{\gamma_i^{(\mathcal{S})}}$, we let $P_\lambda^{\ket{s}}(O_{\mathcal{S}}=j)$ denote the probability that $O_{\mathcal{S}}=j$ given that the hypersurface is in state $\ket*{\Psi_S}$ before the notional mass-energy density measurement on $S$, and that this measurement result is only determined up to the state $\ket*{\gamma_i^{(\mathcal{S})}}\ket*{\Xi_i^{(S^1(y_i))}}$ on the subregion $S^1(y_1)\setminus S_{\mathcal{A}}$ which nevertheless ensures the system $\mathcal{S}$ is in the  state $\ket{s}$ before it interacts with the apparatus $\mathcal{A}$. Then to calculate $P_\lambda^{\ket{s}}(O_{\mathcal{S}}=j)$, we will need to sum over the probabilities that the notional energy density measurement has a $\ket*{\gamma_j'}$ component\footnote{To avoid undue complexity, we assume there is only one $\ket*{\gamma_j'}$-state for each $\ket*{a_j}$-state of the apparatus $\mathcal{A}$.} in its simultaneous $\hat{T}_S$-eigenstate for each state in the basis of states  $\{\ket*{Z_1'},\ket*{Z_2'},\ldots\}$ corresponding to the subregion $S\setminus(S^{1}(y_i)\cup S_{\mathcal{A}}')$. Accordingly, if we slightly redefine our notation by absorbing $\ket*{\gamma_i^{(\mathcal{A})}}$ into the $\ket*{\gamma_l'}$, we will find that   
$$P_\lambda^{\ket{s}}(O_{\mathcal{S}}=j)=\frac{\sum_k|\ip*{\Psi_S}{\gamma_i^{(\mathcal{S})}}\ket*{\Xi_i^{(S^1(y_i))}}\ket*{\gamma_j'}\ket*{Z_k'}|^2}{\sum_{k,l}|\ip*{\Psi_S}{\gamma_i^{(\mathcal{S})}}\ket*{\Xi_i^{(S^1(y_i))}}\ket*{\gamma_l'}\ket*{Z_k'}|^2}.$$ 
Also note that with this slight redefinition of $\ket*{\gamma_l'}$,  equation (\ref{PsiSdecomp}) becomes
$$\ket*{\Psi_S}=b\ket*{\gamma_i^{(\mathcal{S})}}\ket*{\Xi_i^{(S^1(y_i)}}\sum_lc_l\ket{\gamma_l'}\ket{\Xi_l}+\cdots.$$
We will therefore be able to express
$\ket*{\Xi_l}$ in terms of the basis $\{\ket*{Z_1'}, \ket*{Z_2'},\ldots  \}$ so that 
$$\ket{\Xi_l}= \sum_k d_{lk}\ket*{Z_k'}$$
with $\sum_{k}|d_{lk}|^2=1$ since we are assuming all states are normalized. Therefore
$$P_\lambda^{\ket{s}}(O_{\mathcal{S}}=j)=\frac{|b|^2|c_j|^2\sum_k |d_{jk}|^2}{|b|^2\sum_{k,l} |c_l|^2 |d_{lk}|^2}= |c_j|^2.$$
Therefore $$P_\lambda^{\ket{s}}(O_{\mathcal{S}}=j)=|\ip*{s}{s_j}|^2=P^{\ket{s}}(O_{\mathcal{S}}=j).$$ Moreover, since $\sum_\lambda p_\lambda =1$ we have 
$$\sum_\lambda p_\lambda P_\lambda^{\ket{s}}(O_{\mathcal{S}}=j)= P^{\ket{s}}(O_{\mathcal{S}}=j)$$
which is analogous to the EA formula on page \pageref{adeq}.

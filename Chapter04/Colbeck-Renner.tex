
\section{Hidden variables and the Colbeck-Renner theorem\label{colbeckrennerthm}}
In this section we will consider hidden variables in the light of the Colbeck-Renner theorem. Roughly speaking, the Colbeck-Renner theorem says there are no hidden variables theories that are of interest to quantum physicists -- either the hidden variables will be redundant or the hidden variable theory will be incompatible with standard quantum physics. It might therefore seem that the Colbeck-Renner theorem presents a serious challenge to Kent's theory. However, the challenge really hinges on what criteria the data of a theory must satisfy for it to be classified as hidden variable data. A careful analysis of the kind of data Kent's theory relies on reveals that the Colbeck-Renner theoerm is not as serious a challenge to Kent's theory as it first might seem. 

First we need to consider the criteria a set of data should satisfy if it is to be classified as hidden variable data of a physical system $\mathcal{S}$. The criteria we discuss below can be found either explicitly or implicitly in Leegwater's proof of the Colbeck-Renner theorem.\footnote{See \cite{LeegwaterGijs2016Aitf}} The first criterion we will discuss is the following:
\begin{enumerate}
	\item all the information of $\lambda$ is about $\mathcal{S}$ so that a change in $\lambda$ corresponds to a change in the system $\mathcal{S}$.\label{hidden1}
\end{enumerate} 
Notice that Butterfield does not accept this criterion. Butterfield assumes that the hidden variables in Kent's theory consist in the outcome $\tau_S(x)$ of $T_S(x)$ over the whole of $S$. However, this assumption is going to cause difficulties in the context of Shimony's analysis. This is because in Kent's theory, the information in $\tau_S(x)$ over the whole of $S$ clearly would determine which parameters are chosen in a Bell experiment, for this information would determine where a silver atom coming out of a Stern-Gerlach apparatus would be detected on a detection screen (as depicted in figure \ref{rotate}), and from the position of this detection, one could determine the orientation of the magnetic field used in the Stern-Gerlach experiment. So if we stipulated that $\lambda=\tau_S$ is the hidden variable data of every system in Kent's theory, then Kent's theory wouldn't satisfy the preconditions necessary for defining OI and PI. This would make Kent's theory radically different from the pilot wave interpretation where one can define OI and PI because the hidden variables in the pilot wave interpretation, being the positions and momenta of the particles, are independent of the measurement choices. An unfortunate consequence of not being able to define OI and PI is that we wouldn't be able to evaluate Kent's theory in the light of Shimony's analysis of why Bell's inequality fails to hold. 

But it is not obvious that we should stipulate that $\lambda=\tau_S$ is the hidden variable data of every system in Kent's theory. Just because we give $\tau_S$ a single label $\lambda$, it doesn't follow that $\tau_S$ is a single piece of information. There is typically going to be a huge amount of information in $\tau_S$, and so for a given system $\mathcal{S}$, we should carefully discern  what collection of information in $\tau_S$ should be stipulated as being the hidden variable data $\lambda$ of $\mathcal{S}$, hence criterion \ref{hidden1}. Criterion \ref{hidden1} shouldn't be that difficult to satisfy, for if $\lambda$ contained information that could change without this corresponding to any change in $\mathcal{S}$, then we should be able to discard this irrelevant information when considering $\mathcal{S}$ and redefine what $\lambda$ should be for the system. 

In the pilot wave interpretation, the positions and momenta of the particles that constitute a system would fulfil criterion \ref{hidden1}. On the other hand, all the information in $\tau_S$ of Kent's theory would not fulfil this criterion unless, of course, $\mathcal{S}$ was the whole universe. So in order for Kent's theory to satisfy criterion \ref{hidden1}, we just need to discard the information of $\tau_S$ that is irrelevant to the system $\mathcal{S}$ that is being considered in order to obtain the appropriate hidden variable data $\lambda$. 

Note, however, that we don't insist that a difference in $\mathcal{S}$ entails a difference in $\lambda$. This is because a hidden-variables theory is envisaged as augmenting standard quantum theory. So in the case when $\mathcal{S}$ is not entangled with any other system, there will be a quantum state describing $\mathcal{S}$, and this quantum state can be other than it is (indicating that $\mathcal{S}$ can be in a different physical state)  whilst the hidden variable remains the same. We thus impose a second criterion for a hidden-variables theory:
\begin{enumerate}
	\setcounter{enumi}{1}
	\item \label{hidden3} If $\lambda$ is the hidden variable of a system $\mathcal{S}$ and if $\ket{\phi}$ is the quantum state of $\mathcal{S}$ or of some composite system $\mathcal{U}$ that contains $\mathcal{S}$ as a subsystem, then it is possible for there to be a different quantum state $\ket{\phi'}$ of $\mathcal{S}$ (or $\mathcal{U}$) while the hidden variable $\lambda$ remains unchanged, and it is possible for there to be a different hidden variable $\lambda$ while $\ket{\phi}$ remains unchanged.
\end{enumerate} 
This criterion is satisfied in the pilot wave interpretation, since the quantum state is the pilot wave itself. The pilot wave could be other than it is without any of the positions and momenta of the particles changing, but changing the pilot wave would result in a physical change of the system since the pilot wave governs how  the positions and the momenta of the particles subsequently evolve over time. We will discuss the failure of criterion \ref{hidden3} for Kent's theory at the end of this section.

Another criterion implicit in Leegwater's proof for a set of data $\lambda$ to constitute the hidden variable data of a system $\mathcal{S}$ is the following: 
\begin{enumerate}
	\setcounter{enumi}{2}
\item \label{hidden2} it should be possible to change the measurement parameters when measuring $\mathcal{S}$ without this determining what $\lambda$ should be. 
\end{enumerate} 
It is worth noting that  we used criterion \ref{hidden2} when showing that OI implies the negation of PI. If this criterion doesn't hold, we cannot even begin to consider whether PI holds in a given theory. This is because the criterion for PI depends on the probability $P_{\lambda,\bm{\hat{a}},\bm{\hat{b}}}(\uvbpm{a},\uvbpm{b})$ being well-defined, but if the choice of $\bm{\hat{a}}$ or $\bm{\hat{b}}$ determines what $\lambda$ should be, then one wouldn't be able to define $P_{\lambda',\bm{\hat{a}},\bm{\hat{b}}}(\uvbpm{a},\uvbpm{b})$ for $\lambda'\neq\lambda$.\footnote{Cf. footnote \ref{lambdadetermineprob} on page \pageref{lambdadetermineprob}.} But even without any consideration of PI, a rejection of criterion \ref{hidden2} would still seem undesirable, for it would be rather strange if changing the choice of measurement parameters necessarily changed the hidden variable $\lambda$ of the system $\mathcal{S}$, and hence by criterion \ref{hidden1} the system $\mathcal{S}$ itself.\footnote{Of course, in the quantum world, strangeness is not necessarily an argument against something being true.} Moreover, if free will is real and the free choice of the measurement parameter was made before it was determined which system was going to be measured, then it wouldn't even be clear which $\lambda$ of which system we were talking about.\footnote{If the reality of free will is denied, then there wouldn't necessarily be this problem of associating the choice of measurement with the $\lambda$ of the system, because the physical state of the universe could determine which choice of measurement is going to be made as well as which system is going to be measured.}  

In the pilot wave interpretation, the positions and momenta of the particles that constitute a system would fulfil criterion \ref{hidden2}. In the case of Kent's interpretation, we would have to choose carefully the subset of $\tau_S$ for the $\lambda$ of the system being measured if criterion \ref{hidden2} is to hold. But it seems that this should be possible, for one can imagine just changing the part of $\tau_S$ that corresponded to where the measurement parameter dial/knob was without this changing Kent's stress-momentum tensor for the system being measured immediately before it interacts with the measurement device.   

Closely related to criterion \ref{hidden2} is the following criterion: 
\begin{enumerate}
	\setcounter{enumi}{3}
	\item \label{hidden5} There is a range of possible values $\lambda$ for a system $\mathcal{S}$, and for each possible value, we can assign a probability $p_\lambda$ that $\lambda$ obtains, and we can do so in such a way that $p_\lambda$ is independent of any choice of measurement that is to be made on $\mathcal{S}$.
\end{enumerate} 
Butterfield refers to criterion \ref{hidden5} as the `no-conspiracy' assumption, and it is one of the criteria assumed in the Colbeck-Renner theorem. However, Butterfield adds that `no-conspiracy' is a rather unfair label for this criterion since there wouldn't necessarily be anything conspiratorial if this assumption was violated.\footnote{See \cite[34]{Butterfield}.} But in saying this, Butterfield is envisaging $\lambda$ to be the whole of $\tau_S$, and this is in violation of criterion \ref{hidden1}. Since we can expect the whole of $\tau_S$ (together with the universal state $\ket*{\Psi_0}$) to determine Kent's stress-momentum tensor for the measurement parameter dial/knob, then we wouldn't expect criterion \ref{hidden5} to hold. But on the other hand, if we adopt criterion \ref{hidden1} and assume that only information relevant to $\mathcal{S}$ is included in the $\lambda$ for $\mathcal{S}$, then criterion \ref{hidden5} is more plausible. It ties in with common sense intuitions that scientists have free will and can make choices about which measurements they make, and that these choices are statistically independent of the states of the physical systems they are measuring. However, not everyone shares this intuition. If criterion \ref{hidden5} doesn't hold, then what a physical system does will depend on what property of this system one is about to measure. This kind of dependence is referred to as \textbf{superdeterminism}\index{superdeterminism}. Bell coined the term superdeterminism in a 1983 BBC interview, where he said:  
\begin{adjustwidth}{1cm}{}
	\begin{displayquote}
		There is a way to escape the inference of superluminal speeds and spooky action at a distance. But it involves absolute determinism in the universe, the complete absence of free will. Suppose the world is superdeterministic, with not just inanimate nature running on behind-the-scenes clockwork, but with our behavior, including our belief that we are free to choose to do one experiment rather than another, absolutely predetermined, including the ``decision'' by the experimenter to carry out one set of measurements rather than another, the difficulty disappears.
	\end{displayquote}
\end{adjustwidth}
However, Sabine Hossenfelder disputes Bell's argument that the only way to escape the inference of superluminal speeds and spooky action at a distance is to violate free will. Rather, all that is needed to escape this inference is a violation of the statistical independence of the choice of measurement parameters and the state of the system being measured, and this violation is what Hossenfelder takes to be superdeterminism.\footnote{ See \cite{superdeterminism}.   }


\begin{comment}

more reasonable. For example, in the case where the choice is made before it is determined which  system is to be measured, and the system  if wouldn't be clear $\lambda$ for which system would be affected
For if this probability $p_\lambda$ did depend on the choice that was to be made on the system, then we could  envisage a system $\mathcal{S}$ consisting of two entangled particles which Alice and Bob measure separately. If Bob's choice of measurement affected the probability distribution of $p_\lambda$, then it would seem that the effect of Bob's measurement could propagate to Alice faster than the speed of light.\footnote{Recall from the discussion on page \pageref{lambdaknowledge}, that in considering any possible violation of relativity,   it is irrelevant whether Alice knows the hidden variable $\lambda$ and hence whether Alice could work out which measurement Bob had made, because Alice's state of knowledge shouldn't have any effect on the speed at which Bob's effect is propagated. But if she did know $\lambda$, after enough measurements, she would be able to work out the distribution $p_\lambda$, and so if this distribution depended on Bob's choice, she would be able to say something about the choice he made.} Adherents of relativity theory would therefore not want to reject criterion \ref{hidden5}.


We should also state explicitly a fifth criterion for hidden variables as a condition for the possibility :
\begin{enumerate}
	\setcounter{enumi}{4}
	\item \label{hidden5} Suppose  $\mathcal{A}$ is any system that is entangled with $\mathcal{S}$, and that the quantum state of the composite system $\mathcal{S}+\mathcal{A}$  is $\ket{\phi}_{\mathcal{S}+\mathcal{A}}$. Then for any measurement $O_\mathcal{S}$ on $\mathcal{S}$ and $O_\mathcal{A}$ on $\mathcal{A}$, there is a probability for the joint measurement of $O_\mathcal{S}$ and $O_\mathcal{A}$ on $\mathcal{S}+\mathcal{A}$ that is a function of $\lambda$  despite $\lambda$ only referring to the system $\mathcal{S}$.	
\end{enumerate}
\end{comment}

In addition to these four criteria for the hidden variable data $\lambda$ of a system $\mathcal{S}$, it is also desirable for a hidden-variables theory to satisfy PI and empirical adequacy. We defined PI for a two-outcome measurement on page \pageref{PIdef}, but it is easy to generalize the definition of PI for measurements with more than two outcomes. In this more generalized setting, we suppose that $\mathcal{A}$ is any system that is entangled with the system $\mathcal{S}$, and that the quantum state of the composite system $\mathcal{S}+\mathcal{A}$  is $\ket{\phi}_{\mathcal{S}+\mathcal{A}}$. We also suppose that $O_{\mathcal{S}}$ and $O_{\mathcal{A}}$ represent measurement procedures on $\mathcal{S}$ and $\mathcal{A}$ respectively, and that $o_{\mathcal{S}}$ and $o_{\mathcal{S}}$ represent particular measurement outcomes respectively. For a hidden variable $\lambda$ for the system $\mathcal{S}$,\footnote{Although Leegwater, in his proof of the Colbeck-Renner theorem assumes that $\lambda$ is a hidden variable for the system $\mathcal{S}$ only, it looks like the proof would still go through if $\lambda$ was a hidden variable for the composite system $\mathcal{S}+\mathcal{A}$.} there will be a probability $P_\lambda^{\ket{\phi}_{\mathcal{S}+\mathcal{A}}}(O_\mathcal{S}=o_\mathcal{S}\, \& \, O_\mathcal{A}=o_\mathcal{A})$ (understandable in a frequentist sense) that the measurement outcomes of $O_{\mathcal{S}}$ and $O_{\mathcal{A}}$ will be $o_{\mathcal{S}}$  and $o_{\mathcal{A}}$ respectively. Given a second measurement procedure $O_{\mathcal{A}}'$ on $\mathcal{A}$, PI states that
\begin{equation}\tag{PI}
	\sum_{\substack{o_\mathcal{A}\text{ an }\\ \text{outcome }\\ 
	 \text{of }O_{\mathcal{A}}}}P_\lambda^{\ket{\phi}_{\mathcal{S}+\mathcal{A}}}(O_\mathcal{S}=o_\mathcal{S}\, \& \,O_\mathcal{A}=o_\mathcal{A}) = \sum_{\substack{o_\mathcal{A}'\text{ an}\\ \text{outcome }\\ 
	 \text{of }O_{\mathcal{A}}'}}P_\lambda^{\ket{\phi}_{\mathcal{S}+\mathcal{A}}}(O_\mathcal{S}=o_\mathcal{S}\, \& \,O'_\mathcal{A}=o'_\mathcal{A}). 
\end{equation}
If PI holds, then we can define the probability
\begin{equation}
	P^{\ket{\phi}_{\mathcal{S}+\mathcal{A}}}_{\lambda}(O_\mathcal{S}=o_\mathcal{S})=\sum_{\substack{o_\mathcal{A}\text{ an }\\ \text{outcome }\\ 
	\text{of }O_{\mathcal{A}}}}P_\lambda^{\ket{\phi}_{\mathcal{S}+\mathcal{A}}}(O_\mathcal{S}=o_\mathcal{S}\, \& \,O_\mathcal{A}=o_\mathcal{A})
\end{equation}
that is independent of the measurement procedure $O_\mathcal{A}$ on $\mathcal{A}$.\footnote{Cf. (\ref{PIone}).}


As for the definition of \textbf{empirical adequacy}\index{empirical adequacy!formal definition} (EA),  this states that
\begin{equation}\tag{EA}\label{adeq}
	\sum_{\lambda\in\Lambda}p_\lambda P_\lambda^{\ket{\phi}_{\mathcal{S}+\mathcal{A}}}(O_\mathcal{S}=o_\mathcal{S}\, \& \,O_\mathcal{A}=o_\mathcal{A})=P^{\ket{\phi}_{\mathcal{S}+\mathcal{A}}}(O_\mathcal{S}=o_\mathcal{S}\, \& \,O_\mathcal{A}=o_\mathcal{A})
\end{equation}
where $\Lambda$ is the set of all hidden variables so that $\sum_{\lambda\in\Lambda} p_\lambda = 1$, and where 
$$P^{\ket{\phi}_{\mathcal{S}+\mathcal{A}}}(O_\mathcal{S}=o_\mathcal{S}\, \& \,O_\mathcal{A}=o_\mathcal{A})$$
 is the standard probability calculated using the Born Rule with the eigenstates of the observables $\hat{O}_\mathcal{S}$ and $\hat{O}_\mathcal{A}$ and the quantum state $\ket{\phi}_{\mathcal{S}+\mathcal{A}}$. EA is essentially the same as equation (\ref{bohmconsistency}). It also has some similarities with (\ref{kentconsistency}), though the main difference is the range of the summation -- the index of the summands of (\ref{kentconsistency}) does not parametrize hidden variables that satisfy criteria \ref{hidden1} to \ref{hidden5} above.

Now, as I've been alluding to, criteria \ref{hidden1} to \ref{hidden5} together with the conditions of PI and EA are very restrictive. Leegwater proves a version of the Colbeck-Renner theorem\footnote{See \cite{LeegwaterGijs2016Aitf}.} which takes the following form: if one defines hidden variables according to criteria \ref{hidden1} to \ref{hidden5}, then in any hidden-variables theory for which PI and EA hold, the hidden variables are redundant. In other words, 
\begin{equation}\label{colbeckrenner}
P_\lambda^{\ket{\phi}_{\mathcal{S}+\mathcal{A}}}(O_\mathcal{S}=o_\mathcal{S}\, \& \,O_\mathcal{A}=o_\mathcal{A})=P^{\ket{\phi}_{\mathcal{S}+\mathcal{A}}}(O_\mathcal{S}=o_\mathcal{S}\, \& \,O_\mathcal{A}=o_\mathcal{A})
\end{equation}
for any measurement $O_\mathcal{S}$ on $\mathcal{S}$ and $O_\mathcal{A}$ on $\mathcal{A}$.\footnote{Strictly speaking, we should say that equation (\ref{colbeckrenner}) holds for almost all $\lambda$, but we need not concern ourselves here with the details of measure theory that would be needed to make sense of this qualification.} 

Thus, the Colbeck-Renner theorem means that we cannot hope to make Kent's theory into a hidden-variables theory that satisfies criterion \ref{hidden1} to \ref{hidden5} as well as  PI and EA, for the information in Kent's theory is clearly non-redundant.

But nevertheless, it still seems that we should be able to make some kind of sense of PI and EA in Kent's theory and that we should be able to evaluate Kent's theory on the basis of whether these notions of PI and EA are true in this context. To achieve this aim, one strategy would be to relax or drop one or more of the 
four criteria for a hidden variable. Since we still want to be able to make sense of PI and EA, we won't want to relax criteria \ref{hidden2} or \ref{hidden5}. That leaves the possibility of relaxing or droping criteria \ref{hidden1} or \ref{hidden3}. 

Now clearly, we wouldn't be able to drop criterion \ref{hidden1} entirely, for otherwise $\lambda$ would contain information that would determine the choice of measurement made on $\mathcal{S}$. But it doesn't seem problematic if we relax criterion \ref{hidden1} so that there can be information that can change without this corresponding to a change in the system $\mathcal{S}$ so long as this information doesn't determine the measurement choice that is to be made. 

As for criterion \ref{hidden3}, there doesn't seem to be any problem with dropping it entirely. On doing this, then instead of thinking of $\tau_S$ as an augmentation of standard quantum theory, we can think of $\tau_S$ as a rather elaborate way of stipulating the initial quantum states of experiments as well as the quantum states of measurement outcomes. In the next section, we will describe in some more detail how to extract the quantum state of a system from the universal quantum state $\ket*{\Psi_0}$ and $\tau_S$, but roughly speaking, if we consider an experimental setup including some measurement apparatus $\mathcal{A}$ and an object to be measured $\mathcal{S}$, then the information in  $\tau_S$ outside the light cone of the spacetime location of $\mathcal{S}+\mathcal{A}$ before they have interacted will determine the initial quantum states of $\mathcal{S}$ and $\mathcal{A}$ before they interact, and likewise, the information in  $\tau_S$ outside the light cone of the spacetime location of $\mathcal{S}+\mathcal{A}$ after they have interacted will determine the quantum outcome states of $\mathcal{S}$ and $\mathcal{A}$ after they have interacted. This will mean that when the information in $\tau_S$ that is about $\mathcal{S}$ changes, the quantum state of $\mathcal{S}$ extracted from $\tau_S$ and   $\ket*{\Psi_0}$ will also change, and hence criterion \ref{hidden3} will fail to hold. But the information of $\tau_S$ will be non-redundant, for without this information, we would only have the evolution of universal quantum state $\ket{\Psi_0}$ which would continually branch into many worlds. In the many worlds that resulted, the energy density on the hypersurface $S$ would not be in a definite state, but rather would be in a superposition of definite states. But with the information of $\tau_S$ one of these many states in this superposition is selected as actual. If we could then appropriately partition the information in $\tau_S(x)$ on the basis of whether it determined the quantum state of $\mathcal{S}$, or the quantum state of the apparatus $\mathcal{A}$, or the quantum state of the rest of the universe, we could then consider whether Kent's theory gave the same predictions as standard quantum theory. If it did, then PI and EA would hold in Kent's theory, since these both hold in standard quantum theory. And since Kent's theory is formulated in the Lorentz invariant setting of Schwinger and Tomonaga, this would mean that Kent's theory is a solution to the measurement problem! In other words, we would have a one-world interpretation of quantum physics which gave the same probabilities for experimental outcomes that standard quantum physics predicts, and under this interpretation, the physical world would possess the necessary symmetries that guarantee whatever frame of reference one was in, the speed of light would be constant.  



\section{Kent's Theory and Parameter Independence\label{kentpi}}
  In the previous section, we saw how we can generalize Kent's beable $\ev*{\hat{T}^{\mu\nu}(y)}_{\tau_S}$ to calculate conditional expectations $\ev*{\hat{O}}_{\tau_S}$ for any observable $\hat{O}$ defined at a particular spacetime location $(t_i, z_0)$, and that in the case of the observable $[s_f]=\dyad{s_f}$, this expectation yields the same probability as standard quantum theory for the outcome $\ket*{s_f}$ given the initial state $\ket{s}$ of the system.  We also saw that  EA holds in Kent's theory, and to see this, it was necessary to calculate the probability $P_\lambda^{\ket{s}}(O_{\mathcal{S}}=j)$ appropriately conditioned on the mass-energy density measurement determined on a subregion of $S$. 
  
  Now the expectation  $\ev*{[s_f]}_{\tau_S}$ and the probability $P_\lambda^{\ket{s}}(O_{\mathcal{S}}=j)$ depend on just one observable for just one spacetime location. However, in order to consider whether PI holds, we need to consider two observables corresponding to two different spacetime locations. In order to do this, we need to make a further adaption to Kent's theory. In this section, we will describe this adaption and show that with it, Kent's theory allows us to calculate probabilities for Bell-type experiments, and that these probabilities are the same as in standard quantum theory. Since PI holds in standard quantum theory,  a consequence of Kent's theory agreeing with standard quantum theory is that PI will also hold in Kent's theory.


So let's consider figure \ref{bellsolution} which depicts a one-dimensional view of a Bell-type experiment.
\begin{figure}[ht!]
	\captionsetup{justification=justified}
	\centering
	\tikzmath{
	\a=5.3;  
	\qa=5.3;  
	\ja=5.3;  
	\psione=-1;
	\jpsione=1;
	\qpsione=-1;
	\psitwo=1.5;
	\h=-1;
	\labelx=(\psione+\psitwo)/2;
	\labely=-1.5;
	\phstartx=-2.1;
	\offset=-\psione;
	\joffset=-\jpsione;
	\qphstartx=-3.8;
	\jphstartx=-3.8;
	\wphstartx=-1.8;
	\phstarty=\h;
	\tbeg=\phstarty;
	\ca=\phstartx-\tbeg;
	\ttwo=\psitwo-\ca;
	\cb=\ttwo+\ca;
	\xend=\ttwo-\a+\cb;
	\tone=\psione-\ca;
	\cc=\tone+\ca;
	\xendz=\tone-\a+\cc;
	\qca=\qphstartx-\tbeg;
	\qtone=\qpsione-\qca;
	\qqcc=\qtone+\qca;
	\qxendz=\qtone-\qa+\qqcc;
	\jca=\jphstartx-\tbeg;
	\jtone=\jpsione-\jca;
	\jqcc=\jtone+\jca;
	\jxendz=\jtone-\ja+\jqcc;
	\wa=5.3;  
	\wpsione=-1;
	\wttestx=3.7;
	\wca=\wphstartx-\tbeg;
	\wtone=\wpsione-\wca;
	\wwcc=\wtone+\wca;
	\wxendz=\wtone-\wa+\wwcc;
	\wat=\wa-\wttestx;
	\wct=\wttestx;
	\wlam = 0.87;
		\we=0.1;
	\tthree=\cc+\tone-\psitwo;
	\circsize=0.08;
	\md = (\a+\h)/2;
	\tlen=0.75;
	\textscale = 0.7;
	\picscale = 0.78;
	\nudge=0.1;
	\tnudge=(\ttwo-\tone)/2+0.8;
	\ttesta=2*\tone-\ttwo-\tnudge;
	\sonea=\a+\psione-\ttesta;
	\sadd=0.5;
	\lrange = 8.7;
	\rrange =\lrange; 
	\timex=\rrange-1.5;
	\ttestx=2.4;
	\at=\a-\ttestx;
	\ct=\ttestx;
	\qttestx=1.2;
	\qat=\qa-\qttestx;
	\qct=\qttestx;
	\qlam = 0.87;
	\jttestx=1.2;
	\jat=\ja-\jttestx;
	\jct=\jttestx;
	\jlam = 0.87;
	\lam = 0.87;
	\e = 0.1;
	\qe=0.1;
	\je=0.1;
	\hae=(2*pow(\at,3)*\lam*(2+\lam)+\e*\e*(2*\e-sqrt(4*\e*\e+\at*\at*\lam*\lam))-8*\at*\e*(-2*\e+sqrt(4*\e*\e+\at*\at*\lam*\lam))-2*\at*\at*(2+\lam)*(-2*\e+sqrt(4*\e*\e+\at*\at*\lam*\lam)))/(4*\at*\lam*(2*\at+2*\e-sqrt(4*\e*\e+\at*\at*\lam*\lam)))-(\e*\e/4+\at*\at*(-1+\lam))/(\at*\lam);
	\qhae=(2*pow(\qat,3)*\qlam*(2+\qlam)+\qe*\qe*(2*\qe-sqrt(4*\qe*\qe+\qat*\qat*\qlam*\qlam))-8*\qat*\qe*(-2*\qe+sqrt(4*\qe*\qe+\qat*\qat*\qlam*\qlam))-2*\qat*\qat*(2+\qlam)*(-2*\qe+sqrt(4*\qe*\qe+\qat*\qat*\qlam*\qlam)))/(4*\qat*\qlam*(2*\qat+2*\qe-sqrt(4*\qe*\qe+\qat*\qat*\qlam*\qlam)))-(\qe*\qe/4+\qat*\qat*(-1+\qlam))/(\qat*\qlam);
		\whae=(2*pow(\wat,3)*\wlam*(2+\wlam)+\we*\we*(2*\we-sqrt(4*\we*\we+\wat*\wat*\wlam*\wlam))-8*\wat*\we*(-2*\we+sqrt(4*\we*\we+\wat*\wat*\wlam*\wlam))-2*\wat*\wat*(2+\wlam)*(-2*\we+sqrt(4*\we*\we+\wat*\wat*\wlam*\wlam)))/(4*\wat*\wlam*(2*\wat+2*\we-sqrt(4*\we*\we+\wat*\wat*\wlam*\wlam)))-(\we*\we/4+\wat*\wat*(-1+\wlam))/(\wat*\wlam);
			\jhae=(2*pow(\jat,3)*\jlam*(2+\jlam)+\je*\je*(2*\je-sqrt(4*\je*\je+\jat*\jat*\jlam*\jlam))-8*\jat*\je*(-2*\je+sqrt(4*\je*\je+\jat*\jat*\jlam*\jlam))-2*\jat*\jat*(2+\jlam)*(-2*\je+sqrt(4*\je*\je+\jat*\jat*\jlam*\jlam)))/(4*\jat*\jlam*(2*\jat+2*\je-sqrt(4*\je*\je+\jat*\jat*\jlam*\jlam)))-(\je*\je/4+\jat*\jat*(-1+\jlam))/(\jat*\jlam);
	} 
	\tikzstyle{scorestars}=[star, star points=5, star point ratio=2.25, draw, inner sep=1pt]%
	\tikzstyle{scoresquare}=[draw, rectangle, minimum size=1mm,inner sep=0pt,outer sep=0pt]%
	\begin{tikzpicture}[scale=\picscale,
	declare function={
		testxonep(\ps,\t)=\a+\ps-\t;
		testxonem(\ps,\t)=\ps-\a+\t; 
		bl(\x)=\ct+\at-\e/2-1/2*(sqrt(\lam*\lam*pow(\x+\hae+\at,2)+\e*\e)-\e+\lam*(\x+\hae+\at));
		br(\x)=\ct+\at-\e/2-1/2*(sqrt(\lam*\lam*pow(\x-\hae-\at,2)+\e*\e)-\e-\lam*(\x-\hae-\at));
		bc(\x)=\ct+sqrt(\lam*\lam*\x*\x+\e*\e)-\e;
		qbl(\qx)=\qct+\qat-\qe/2-1/2*(sqrt(\qlam*\qlam*pow(\qx+\qhae+\qat,2)+\qe*\qe)-\qe+\qlam*(\qx+\qhae+\qat));
		qbr(\qx)=\qct+\qat-\qe/2-1/2*(sqrt(\qlam*\qlam*pow(\qx-\qhae-\qat,2)+\qe*\qe)-\qe-\qlam*(\qx-\qhae-\qat));
		qbc(\qx)=\qct+sqrt(\qlam*\qlam*\qx*\qx+\qe*\qe)-\qe;
		wbl(\wx)=\wct+\wat-\we/2-1/2*(sqrt(\wlam*\wlam*pow(\wx+\whae+\wat,2)+\we*\we)-\we+\wlam*(\wx+\whae+\wat));
		wbr(\wx)=\wct+\wat-\we/2-1/2*(sqrt(\wlam*\wlam*pow(\wx-\whae-\wat,2)+\we*\we)-\we-\wlam*(\wx-\whae-\wat));
		wbc(\wx)=\wct+sqrt(\wlam*\wlam*\wx*\wx+\we*\we)-\we;
		jbl(\jx)=\jct+\jat-\je/2-1/2*(sqrt(\jlam*\jlam*pow(\jx+\jhae+\jat,2)+\je*\je)-\je+\jlam*(\jx+\jhae+\jat));
		jbr(\jx)=\jct+\jat-\je/2-1/2*(sqrt(\jlam*\jlam*pow(\jx-\jhae-\jat,2)+\je*\je)-\je-\jlam*(\jx-\jhae-\jat));
		jbc(\jx)=\jct+sqrt(\jlam*\jlam*\jx*\jx+\je*\je)-\je;
		rr(\x)=2*(pow(\x,4)/4-pow(\offset*\x,2)/2+pow(\offset,4)/4)+\qttestx;
	},
	 ] 
	\definecolor{tempcolor}{RGB}{250,190,0}
	\definecolor{darkgreen}{RGB}{40,190,40}
	

	
	 
	 \draw[->,gray, thick] [domain=-\at/2-\offset:-\lrange, samples=150]   plot (\x, {qbl(\x+\offset)})  ;
	\draw[gray, thick] [domain=-\at/2-\offset:0-\offset, samples=150] plot (\x, {qbc(\x+\offset)})   ;
%	\draw[->,gray, thick] [domain=\at/2-\offset:\rrange, samples=150]   plot (\x, {qbr(\x+\offset)})   ;
	\draw[gray, thick] [domain=-\offset:-\joffset, samples=150] plot (\x, {\qttestx})   ;
	\draw[gray, thick, densely dashed] [domain=-\offset:-\joffset, samples=150] plot (\x, {rr(\x)})   ;
	
	
	 \draw[->,blue, thick] [domain=-\at/2-\offset:-\lrange, samples=150]   plot (\x, {wbl(\x+\offset)})  ;
	 \draw[blue, thick] [domain=-\at/2-\offset:0-\offset, samples=150] plot (\x, {wbc(\x+\offset)})   ;
	 \draw[blue, thick] [domain=-\offset:-\joffset, samples=150] plot (\x, {\wttestx})   ;
	 \draw[blue, thick, densely dashed] [domain=-\offset:-\joffset, samples=150] plot (\x, {rr(\x)+\wttestx-\qttestx})   ;
	 
	 \draw[->,blue, thick] [domain=\at/2+\offset:\lrange, samples=150]   plot (\x, {wbl(-\x+\offset)})  ;
	\draw[blue, thick] [domain=\at/2+\offset:\offset, samples=150] plot (\x, {wbc(-\x+\offset)})   ;

 %   \draw[->,gray, thick] [domain=-\at/2-\joffset:-\lrange, samples=150]   plot (\x, {jbl(\x+\joffset)})  ;
	\draw[gray, thick] [domain=-\joffset:\at/2-\joffset, samples=150] plot (\x, {jbc(\x+\joffset)})   ;
	\draw[->,gray, thick] [domain=\at/2-\joffset:\rrange, samples=150]   plot (\x, {jbr(\x+\joffset)})   ;

	  \node[scale=\textscale]  at (0,\qttestx-0.23) {$S_{n,i}$}; 
	  \node[scale=\textscale]  at (0,\wttestx-0.23) {$S_{n,m}$}; 

	
	\draw[<->] (-\lrange, \h) node[left, scale=\textscale] {$S_0$} -- (\rrange, \h) node[right, scale=\textscale] {$S_0$};
	\draw[<->] (-\lrange, \a) node[left, scale=\textscale] {$S$} -- (\rrange, \a) node[right, scale=\textscale] {$S$};
				  
	\draw[->, shorten <= 5pt,  shorten >= 1pt] (\psione,\h)  node[above right, scale=\textscale]{$z_L$}-- (\psione,\a) ;
	\draw[->, shorten <= 5pt,  shorten >= 1pt] (\jpsione,\h)  node[above right, scale=\textscale]{$z_R$}-- (\jpsione,\a) ;
	
	
	\draw[dashed, tempcolor,  ultra thick](\phstartx,\tbeg) node[below left=-2,scale=\textscale]{}--(\psione,\tone) node[above, pos=0.3, rotate=45,scale=0.7,black] {} node[below, pos=0.3, rotate=45,scale=0.7,black] {};
	\draw[dashed, tempcolor,  ultra thick](\psione,\tone)--(\xendz,\a);
	
		\draw[dashed, tempcolor,  ultra thick](\qphstartx,\tbeg) node[below left=-2,scale=\textscale]{}--(\psione,\qtone) node[above, pos=0.3, rotate=45,scale=0.7,black] {} node[below, pos=0.3, rotate=45,scale=0.7,black] {};
	\draw[dashed, tempcolor,  ultra thick](\psione,\qtone)--(\qxendz,\a);
	
			\draw[dashed, tempcolor,  ultra thick](-\qphstartx,\tbeg) node[below left=-2,scale=\textscale]{}--(-\psione,\qtone) node[above, pos=0.3, rotate=45,scale=0.7,black] {} node[below, pos=0.3, rotate=45,scale=0.7,black] {};
	\draw[dashed, tempcolor,  ultra thick](-\psione,\qtone)--(-\qxendz,\a);
	
		\draw[dashed, tempcolor,  ultra thick](-\phstartx,\tbeg) node[below left=-2,scale=\textscale]{}--(-\psione,\tone) node[above, pos=0.3, rotate=45,scale=0.7,black] {} node[below, pos=0.3, rotate=45,scale=0.7,black] {};
	\draw[dashed, tempcolor,  ultra thick](\jpsione,\tone)--(-\xendz,\a);



	
	\draw[->] (\timex,\md-\tlen/2) --  (\timex,\md+\tlen/2) node[midway,right, scale=\textscale]{time}; 
	 

	\draw[dotted](\psione,\qttestx)--({testxonep(\psione,\qttestx)},\a);
	\draw[dotted](\psione,\qttestx)--({testxonem(\psione,\qttestx)},\a);
	\draw[dotted](\jpsione,\jttestx)--({testxonep(\jpsione,\jttestx)},\a);
	\draw[dotted](\jpsione,\jttestx)--({testxonem(\jpsione,\jttestx)},\a);

	\draw (\psione,\qttestx) node[ scoresquare, fill=gray]  {} node [black,below=6,right,scale=\textscale] {$t_i$};
    \draw (\jpsione,\jttestx) node[ scoresquare, fill=gray]  {} node [black,below=6,right,scale=\textscale] {$t_i$};
\draw (\psione,\wttestx) node[ scoresquare, fill=gray]  {} node [black,below=6,right,scale=\textscale] {$t_m$};
    \draw (\jpsione,\wttestx) node[ scoresquare, fill=gray]  {} node [black,below=6,right,scale=\textscale] {$t_m$};



	
	\draw [black,fill](\xendz,\a) circle [radius=\circsize] node [black,above=9,right=-4,scale=\textscale] {$\gamma_i^{(\mathcal{A}_L)}$}; 
	\draw [black,fill](-\xendz,\a) circle [radius=\circsize] node [black,above=9,right=-4,scale=\textscale] {$\gamma_i^{(\mathcal{A}_R)}$}; 
	\draw [black,fill](\qxendz,\a) circle [radius=\circsize] node [black,above=9,right=-4,scale=\textscale] {$\gamma_{m,+}^{(\mathcal{A}_L)}$}; 
	\draw [black,fill](-\qxendz,\a) circle [radius=\circsize] node [black,above=9,right=-4,scale=\textscale] {$\gamma_{m,+}^{(\mathcal{A}_R)}$}; 
	
	%\draw [black,fill] (\xendz,\a) circle [radius=\circsize] node [black,above,scale=\textscale] {$\gamma_L$}; 
	\end{tikzpicture}% pic 1
	
	\vspace*{2px}
	\caption[Toy model for Bell-type experiment with measurement choice]{Depicts a Bell-type experiment where the state of some photons $\gamma_i^{(\mathcal{A}_L)}$ and $\gamma_i^{(\mathcal{A}_R)}$ on the hypersurface $S$ determines the choice of measurement parameters of the left wing and right wing of the experiment respectively, and some photons $\gamma_{m,+}^{(\mathcal{A}_L)}$ and $\gamma_{m,+}^{(\mathcal{A}_R)}$ on the hypersurface $S$ determine the measurement outcome of the experiment on the left wing and the right wing respectively. The dashed lines on the hypersurfaces $S_{n,m}$ and $S_{n,i}$ indicate other choices for the hypersurfaces, but they still lead to the same probability being calculated.   }
	\label{bellsolution}
	\end{figure}
There is a left wing of the experiment located in the vicinity of $z_L$, and a right wing of the experiment located in the vicinity of $z_R$. Shortly before time $t_i$, photons interact with a Stern-Gerlach apparatus on the left wing and a Stern-Gerlach apparatus on the right wing, and some of these photons eventually intersect the hypersurface $S$ so that there are subregions $S_{\mathcal{A}_L}$ and $S_{\mathcal{A}_R}$ %
\nomenclature{$S_{\mathcal{A}_L}, S_{\mathcal{A}_R}$}{Subregions of $S$ such that if the notional measurement of the mass-energy density corresponds to simultaneous $\hat{T}_S$-eigenstates $\ket*{\gamma_i^{(\mathcal{A}_L)}}$ and $\ket*{\gamma_i^{(\mathcal{A}_R)}}$ respectively on these subregions, \nomrefpage}%
 of $S$ such that if the notional measurement of the mass-energy density corresponds to simultaneous $\hat{T}_S$-eigenstates $\ket*{\gamma_i^{(\mathcal{A}_L)}}$ and $\ket*{\gamma_i^{(\mathcal{A}_R)}}$ %
 \nomenclature{$\ket*{\gamma_i^{(\mathcal{A}_L)}},\ket*{\gamma_i^{(\mathcal{A}_R)}}$}{simultaneous $\hat{T}_S$-eigenstates on the subregions $S_{\mathcal{A}_L}, S_{\mathcal{A}_R}$ that are sufficient to determine the measurement parameters of the apparatuses on the left wing and the right wing of the experiment respectively, \nomrefpage}%
  respectively on these subregions, then this will be sufficient to determine the measurement parameters of the apparatuses on the left wing and the right wing of the experiment respectively. 

Now in order to consider whether PI holds, we will need to adapt Kent's sequences of hypersurfaces so that they can be used to calculate conditional expectation values for observables that depend on two spacetime locations %
\nomenclature{$y_L, y_R$}{Spacetime locations of the two Stern-Gerlach apparatuses $y_L=(t_i, z_L)$ and $y_R=(t_i,z_R) $ respectively, \nomrefpage}% 
$y_L=(t_i, z_L)$ and $y_R=(t_i,z_R)$.%
\nomenclature{$z_L, z_R$}{Spatial locations of the two Stern-Gerlach apparatuses, \nomrefpage}%
  We therefore require that sequences of hypersurfaces $S_{n,i}$ are chosen so that they all contain the spacetime locations $y_L$ and $y_R$, and that in the limit, $\lim_{n\rightarrow\infty}S_{n,i}$ contains as much of $S^1(y_L)$ and $S^1(y_R)$ as possible, where as usual, $S^1(y)$ denotes the subset of $S$ lying outside the light cone of $y$. Ultimately, this limit (unlike the limit of Kent's hypersurfaces) will not contain the whole of $S^1(y_L)$ or $S^1(y_R)$, but only serves to guarantee that we use as much of the information in $S$ as possible in calculating the expectation values of observables at $y_L$ and $y_R$. There will be some degree of freedom in what we choose for the hypersurface between $y_L$ and $y_R$ as depicted by the dashed line in figure \ref{bellsolution}. However, such freedom will have no effect on the probabilities calculated, because under the assumption that the hypersurface is very far into the future, there will be no choice of hypersurface in this region that would give us more information in $S$ to condition on. Also, we recall that the stress-energy operators of the form $\hat{T}^{\mu\nu}(y)$ in the Tomonaga-Schwinger formulation of relativistic quantum physics are defined so that they are invariant under any perturbation of the hypersurface (so long as the hypersurface continues to contain $y$), so under the assumption that all physical observables will be ultimately expressible in terms of the stress-energy operators, the arbitrary choice of the hypersurfaces in regions that can't intersect with $S$ will have no effect of the probabilities calculated. 

On the hypersurface $S_{n,i}$, we assume that $n$ is sufficiently large that enough of the subregions $S_{\mathcal{A}_L}$ and $S_{\mathcal{A}_R}$ are contained within $S_{n,i}$ so that the simultaneous $\hat{T}_S$-eigenstates $\ket*{\gamma_i^{(\mathcal{A}_L)}}$ and $\ket*{\gamma_i^{(\mathcal{A}_R)}}$ restricted to $S_{\mathcal{A}_L}\cap S_{n,i}$ and $S_{\mathcal{A}_R}\cap S_{n,i}$ are still able to determine the choice of measurement axes for the left and right wings of the experiment respectively. In order to avoid introducing too much extra notation, we will shrink the subregions $S_{\mathcal{A_L}}$ and $S_{\mathcal{A_R}}$ so that they are contained within $S_{n,i}$ for sufficiently large $n$, but we only shrink them slightly so that the mass-energy density measurement on them is still sufficient to determine the choice of measurement axes for the left and right wings of the experiment.\footnote{For more realistic models, we wouldn't expect the mass-energy density measurement on $S_{n,i}$ to determine the choice of measurement axes with 100\% certainty, since there will be a degree of overlap of the different $\ket*{\gamma_i^{(\mathcal{A}_L)}}$ for different measurement choices (and likewise for the different $\ket*{\gamma_i^{(\mathcal{A}_R)}}$). But this overlap will get smaller and smaller the more that photons interacting with the apparatus intersect $S_{n,i}$, and so the certainty of which measurement is being made will approach 100\% as long as there is sufficient enough time from the time the measurement parameters are chosen to time $t_i$, and as long as $n$ is large enough so that there are enough photon interactions with the apparatus that intersect $S_{n,i}$. Nevertheless, the fact that we never reach 100\% certainty should not worry us too much in the context of Kent's theory, since it just means that Kent's $\ev*{\hat{T}^{\mu\nu}(y)}_{\tau_S}$-beables will be perturbed by a very small amount in the vicinity of the apparatus caused by the very small amount of overlap between the the different $\ket*{\gamma_i^{(\mathcal{A}_L)}}$ for different measurement choices. }   

Let us now assume that the axis of orientation of the right wing Stern-Gerlach apparatus makes an angle $\theta$ with the axis of the left wing apparatus. We also assume that there are two particles that together form a Bell-state \begin{equation}\label{bellstatePI}
	\frac{1}{\sqrt{2}}(\ket*{\uvbp{s}}_L\ket*{\uvbm{s}}_R-\ket*{\uvbm{s}}_L\ket*{\uvbp{s}}_R).
\end{equation}
We saw  in footnote \ref{bellstate2pf} on page \pageref{bellstate2pf} that a Bell state does not depend on the orientation of $\uvb{s}$, so without loss of generality, we can suppose that the $\ket*{\uvbp{s}}_L$ and $\ket*{\uvbm{s}}_L$ %
\nomenclature{$\ket*{\uvbp{s}}_L, \ket*{\uvbm{s}}_L$}{Pointer states for the apparatus on the left-wing of the experiment, \nomrefpage}%
 are pointer states for the apparatus on the left-wing of the experiment. This means there will be a ready state $\ket{a}_L$  %
 \nomenclature{$\ket{a}_L$}{Ready state of the left wing apparatus, \nomrefpage}%
  as well as two states $\ket{a+}_L$ and $\ket{a-}_L$ %
  \nomenclature{$\ket{a+}_L, \ket{a-}_L$}{States of the left wing apparatus corresponding to the pointer states $\ket*{\uvbp{s}}_L$ and $\
  ket*{\uvbm{s}}_L$ respectively, \nomrefpage}%
   of the left wing apparatus such that 
$$\ket*{\uvbpm{s}}_L\ket{a}_L\rightarrow\ket*{\uvbpm{s}}_L\ket{a\pm}_L.$$

As for the right wing of the experiment, we let  $\ket*{\bm{\hat{s}_\theta+}}_R$ and $\ket*{\bm{\hat{s}_\theta-}}_R$ %
\nomenclature{$\ket*{\bm{\hat{s}_\theta+}}_R, \ket*{\bm{\hat{s}_\theta-}}_R$}{Pointer states for the apparatus on the right-wing of the experiment, \nomrefpage}%
 be pointer states for the apparatus so that there is a ready state $\ket{a}_R$  %
 \nomenclature{$\ket{a}_R$}{Ready state of the right wing apparatus, \nomrefpage}%
   as well as two states $\ket{a_\theta+}_R$ and $\ket{a_\theta-}_R$ %
   \nomenclature{$\ket{a_\theta+}_R, \ket{a_\theta-}_R$ }{States of the right wing apparatus corresponding to the pointer states $\ket*{\bm{\hat{s}_\theta+}}_R$ and $\ket*{\bm{\hat{s}_\theta-}}_R$ respectively, \nomrefpage}%
   of the right wing apparatus such that 
$$\ket*{\bm{\hat{s}}_\theta\pm}_R\ket{a}_R\rightarrow\ket*{\bm{\hat{s}}_\theta\pm}_R\ket{a_\theta\pm}_R.$$
In a manner similar to equation (\ref{Psinidecomp}), we can express $\ket*{\Psi_{n,i}}=U_{S_{n,i},S_0}\ket*{\Psi_0}$ as a superposition
\begin{equation}\label{bellstatePI2}
	\ket{\Psi_{n,i}}= \frac{b}{\sqrt{2}}\big(\ket*{\uvbp{s}}_L\ket*{\uvbm{s}}_R-\ket*{\uvbm{s}}_L\ket*{\uvbp{s}}_R\big)\ket{a}_L\ket{a}_R\ket*{\gamma_i^{(\mathcal{A}_L)}}\ket*{\gamma_i^{(\mathcal{A}_R)}}\ket*{\xi_{n,i}}+\cdots.
\end{equation}
where $\ket*{\xi_{n,i}}$ corresponds to the state of the subregion of $S_{n,i}$ not determined by $\ket*{\uvbpm{s}}_L\ket*{\uvbmp{s}}_R$, $\ket{a}_L$, $\ket{a}_R$, $\ket*{\gamma_i^{(\mathcal{A}_L)}}$ or $\ket*{\gamma_i^{(\mathcal{A}_R)}}$.
As in equations (\ref{spintrans1}) and (\ref{spintrans2}), we have
\begin{align*}
\ket*{\uvbp{s}}_R&= \alpha_\theta\ket*{\bm{\hat{s}_\theta+}}_R+\beta_\theta \ket*{\bm{\hat{s}_\theta-}}_R,\\
\ket*{\uvbm{s}}_R&= \alpha_\theta\ket*{\bm{\hat{s}_\theta-}}_R-\beta_\theta \ket*{\bm{\hat{s}_\theta+}}_R,
\end{align*}
where $\alpha_\theta=\cos(\theta/2)$, and $\beta_\theta=\sin(\theta/2).$
Substituting this into (\ref{bellstatePI2}), we can express the state of the hypersurface $S_{n,i}$ that goes through the two particles at spacetime locations $y_L$ and $y_R$ as
\begin{equation}\label{bellstatePI3}
	\begin{split}
	\ket{\Psi_{n,i}}&=\frac{b}{\sqrt{2}}\big(\alpha_\theta\ket*{\uvbp{s}}_L\ket*{\bm{\hat{s}_\theta-}}_R
	-\beta_\theta\ket*{\uvbp{s}}_L\ket*{\bm{\hat{s}_\theta+}}_R-\alpha_\theta\ket*{\uvbm{s}}_L\ket*{\bm{\hat{s}_\theta+}}_R\\
	&\quad
	-\beta_\theta\ket*{\uvbm{s}}_L\ket*{\bm{\hat{s}_\theta-}}_R\big)\ket{a}_L\ket{a}_R\ket*{\gamma_i^{(\mathcal{A}_L)}}\ket*{\gamma_i^{(\mathcal{A}_R)}}\ket*{\xi_{n,i}}+\cdots.
	\end{split}
\end{equation}
We now let $\pi_{n,i}$ be the projection as defined in equation \ref{tauprojection} that corresponds to the measurement outcome $\tau_S(x)$ on $S_{n,i}\cap S$. If we apply $\pi_{n,i}$ to $\ket*{\Psi_{n,i}}$ as we did in (\ref{piphi}) then we will get the approximation
\begin{equation}\label{piphiPI}
	\begin{split}
		\pi_{n,i}\ket{\Psi_{n,i}}&\approx\frac{b}{\sqrt{2}}\big(\alpha_\theta\ket*{\uvbp{s}}_L\ket*{\bm{\hat{s}_\theta-}}_R
		-\beta_\theta\ket*{\uvbp{s}}_L\ket*{\bm{\hat{s}_\theta+}}_R-\alpha_\theta\ket*{\uvbm{s}}_L\ket*{\bm{\hat{s}_\theta+}}_R\\
		&\quad
		-\beta_\theta\ket*{\uvbm{s}}_L\ket*{\bm{\hat{s}_\theta-}}_R\big)\ket{a}_L\ket{a}_R\ket*{\gamma_i^{(\mathcal{A}_L)}}\ket*{\gamma_i^{(\mathcal{A}_R)}}\ket*{\xi_{n,i}},
		\end{split}
\end{equation}
and the larger $n$ is, the closer (\ref{piphiPI}) will come to being an equality, though in the case of our toy model where we treat photons as point particles, we can expect (\ref{piphiPI}) to become an equality for sufficiently large $n$.



Now in the previous section, we calculated Kent's conditional expectation value of $[s_f]$ and argued that this would give the probability that the measurement outcome of the system $\mathcal{S}$ would be $\ket*{s_f}$ given a particular mass-energy density measurement on $S^1(y_i)$. In the situation at hand in which we wish to know the probability of two measurements, we need to consider conditional expectation values of observables such as  $[\bm{\hat{s}+}]_L[\bm{\hat{s}_\theta+}]_R$ where the observable $[\uvbp{s}]_L=\ket{\uvbp{s}}_L\prescript{}{L}{\bra{\uvbp{s}}}$ depends on spacetime location $y_L$, and where the observable $[\bm{\hat{s}_\theta+}]_R=\ket{\bm{\hat{s}_\theta+}}_R\prescript{}{R}{\bra{\bm{\hat{s}_\theta+}}}$ 
\nomenclature{$[\uvbp{s}]_L, [\bm{\hat{s}_\theta+}]_R$}{The observables $\ket{\uvbp{s}}_L\prescript{}{L}{\bra{\uvbp{s}}}$ and $\ket{\bm{\hat{s}_\theta+}}_R\prescript{}{R}{\bra{\bm{\hat{s}_\theta+}}}$ respectively, \nomrefpage}%
depends on spacetime location $y_R.$ Since we are choosing the sequence of hypersurfaces $S_{n,i}$ so that both $y_L$ and $y_R$ belong to $S_{n,i}$, and since any two observables for locations that are spacelike separated commute, we can easily see what the eigenvalues for  $[\bm{\hat{s}+}]_L[\bm{\hat{s}_\theta+}]_R$ must be: they are 1 and 0, where the eigenvalue of $1$ corresponds to all the states of $S_{n,i}$ in which the particle about to be measured by the left wing apparatus is in the $\ket*{\bm{\hat{s}+}}_L$-state and in which the particle about to be measured by the right wing apparatus is in the $\ket*{\bm{\hat{s}_\theta+}}_R$-state, and where the eigenvalue of $0$ corresponds to all the states of $S_{n,i}$ in which either the particle about to be measured by the left wing apparatus is in a pointer state of the apparatus that is not the $\ket*{\bm{\hat{s}+}}_L$-state or the particle about to be measured by the right wing apparatus is a pointer state of the apparatus that is not the $\ket*{\bm{\hat{s}_\theta+}}_R$-state. It therefore follows from the definition of expectation that given the mass-energy density measurement on the subregion $S_{\mathcal{A}_L}$ is $\ket*{\gamma_i^{(\mathcal{A}_L)}}$ and the mass-energy density measurement on the subregion $S_{\mathcal{A}_R}$ is $\ket*{\gamma_i^{(\mathcal{A}_R)}}$, the probability that the left wing will be measured to be in state  $\ket*{\bm{\hat{s}+}}_L$ and that the right wing will be measured to be in $\ket*{\bm{\hat{s}_\theta+}}_R$ will be
\begin{equation}\label{slsrev}
	\begin{split}
	\ev*{[\bm{\hat{s}+}]_L[\bm{\hat{s}_\theta+}]_R}_{\tau_S}
	&=\lim_{n\rightarrow\infty}\frac{\ev*{\pi_{n,i}[\bm{\hat{s}+}]_L[\bm{\hat{s}_\theta+}]_R}{\Psi_{n,i}}}{\ev*{\pi_{n,i}}{\Psi_{n,i}}}\\
	&=\frac{\prescript{}{R}{\bra*{\bm{\hat{s}_\theta+}}}\prescript{}{L}{}\ev*{\overline{\frac{b}{\sqrt{2}}\beta_{\theta}}[\bm{\hat{s}+}]\prescript{}{L}{}[\bm{\hat{s}_\theta+}]\prescript{}{R}{}\frac{b}{\sqrt{2}}\beta_{\theta}}{\hat{s}+}_L\ket*{\bm{\hat{s}_\theta+}}_R}{|b|^2}\\
	&=\frac{|\beta_\theta|^2}{2}=\frac{1}{2}\sin^2(\theta/2),
	\end{split}	
\end{equation}
where we have used (\ref{piphiPI}), and this is the same as the probability that standard quantum theory predicts as given in equation (\ref{bellsin}). We therefore see that the mass-energy density measurement on $S$ allows us to determine the state of the particles and the apparatus before the particles are measured in an EPR-Bohm type experiment, and furthermore, the adaption I've made to Kent's model so that it can compute probabilities of the measurement outcomes of this experiment produces the same probabilities as standard quantum theory. Therefore, since PI hold's in standard quantum theory, it must also hold Kent's adapted model as well.\label{kentpiend}


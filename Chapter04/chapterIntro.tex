\chapter{Evaluating Kent's Theory\label{KentEval}}
In order to evaluate Kent's theory of quantum physics, it will first be helpful to remind ourselves of the problem we are trying to solve. 

In chapter \ref{BellChapter}, we discussed the EPR-Bohm paradox and the problem of explaining  the mysterious correlations between measurement outcomes of spin singlets in a way consistent with special relativity and the predictions of standard quantum theory. We saw that the Copenhagen interpretation does not seem to be consistent with special relativity. We also discussed Shimony's distinction between Outcome Independence (OI) and Parameter Independence (PI) and Shimony's idea that we should only accept a theory in which OI is false and PI is true. Since PI is false in the pilot-wave theory, we should reject it according to Shimony's criterion. 

But although Shimony's criterion is a promising line of inquiry, by itself, it is not sufficient to resolve the  EPR-Bohm paradox. This is because Shimony's criterion doesn't address the controversial issue of what is meant by an outcome. In chapter \ref{measprobchap}, we discussed this controversy over outcomes and why the many-worlds interpretation that denies the reality of outcomes is unsatisfactory. This motivated the discussion of Kent's theory in chapter \ref{kentchapterdesc} in the hope that it might provide a satisfactory solution to the EPR-Bohm paradox. In the previous chapter, we only got as far as describing the key features of Kent's theory such as it being a one-world theory which posits additional variables to standard quantum theory. 

So having now reminded ourselves of the problem at hand, we see that there are several issues that we need to consider in order to evaluate whether Kent's theory provides a satisfactory solution to this problem. Firstly, we should consider whether the predictions of Kent's theory are consistent with the  predictions of standard quantum theory. Since standard quantum theory (that is a theory of states whose evolution is determined by the Schr\"{o}dinger equation) predicts the correlations that have been experimentally observed in the EPR-Bohm paradox, then if Kent's theory makes the same predictions as standard quantum theory, these EPR-Bohm correlations will also be exhibited in Kent's theory.  In section \ref{kentinterpretationconsistency}, I will show that Kent's theory agrees with standard quantum theory given the universal quantum state $\ket*{\Psi_n}$ on a hypersurface $S_n$. However, since in practice, scientists do not base their calculations on universal quantum states, in section \ref{KentconsistentQT}, I will explain how to extract what I call a conditioned quantum state from $\ket*{\Psi_n}$, and I will show that  Kent's theory agrees with standard quantum theory when we work with conditioned quantum states.

In section \ref{LorentzInvariance}, I will consider Lorentz invariance. The importance of Lorentz invariance lies in the fact that a necessary condition for a theory to be consistent with special relativity is that it be invariant under a group of symmetries called Lorentz transformations. Since a satisfactory solution to our problem must be consistent with special relativity, we therefore need to consider whether Kent's theory satisfies Lorentz invariance. 

In section \ref{Kentdecoherencesection}, I will consider the Problem of Outcomes in the light of Kent's theory. Since a satisfactory theory must be one in which there are outcomes, we need to consider whether Kent succeeds in giving us a convincing account of what an outcome is. In doing this, I will examine how Kent's theory ties in with decoherence theory and d'Espagnat's objection about improper mixtures. 

Several sections will be needed to address Parameter Independence in Kent's theory. In section \ref{OISec}, we saw that if OI is true, then PI must be false. However, Butterfield, in his extension of Kent's toy model argues that OI holds in Kent's theory. This  conclusion could therefore seem rather worrying given that we want PI to be true. This is a worry that I will address in section  \ref{butterfieldtoy}.

Butterfield has also emphasized the need to understand Kent's theory in the light of an important theorem proved in recent years, the so-called Colbeck-Renner Theorem.\footnote{See \cite{LeegwaterGijs2016Aitf}, \cite{ColbeckRoger2011Neoq}, \cite{ColbeckRoger2012Tcoq}, \cite{LandsmanK2015OtCt}, and \cite{Landsman}.} This theorem states that if a theory satisfies PI together with what is called a `no conspiracy' criterion, then this theory is reducible to standard quantum theory without any hidden variables. This result could seem rather worrying since in the absence of any additional variables, standard quantum theory is unable to resolve the EPR-Bohm paradox. This  suggests that any theory that satisfactorily addresses the  EPR-Bohm paradox can't be a hidden-variables theory according the the assumptions of the Colbeck-Renner theorem. In section \ref{colbeckrennerthm}, I will therefore consider whether it would be both possible and acceptable to relax or drop any of these assumptions in the context of Kent's theory.

Having addressed the concerns Butterfield raises,  I will argue in section \ref{kentpi} that when we adapt Kent's theory to use conditioned quantum states, PI is true. The conclusion of this chapter is that we therefore have good reason to think that Kent's theory provides a satisfactory solution to the EPR-Bohm paradox. 

There are nevertheless, some remaining questions concerning Kent's theory. Firstly, there is the question of how plausible it is for the  beables in reality to be what Kent suggests they are. Secondly, Kent's theory may also strike us as rather counter-intuitive given that his theory posits that present events should be conditioned on far-distant future states of affairs. There is also the question of whether Kent's theory is deterministic. I will therefore conclude this chapter with section \ref{beablesandtime} where I discuss these questions.

Since the arguments in this chapter are rather technical, several sections are divided into two subsections where the first subsection will contain a non-technical overview, and the second subsection will contain the technical details and can be skipped if desired.  
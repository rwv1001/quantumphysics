\section{Kent's Theory and Lorentz Invariance\label{LorentzInvariance}}
\subsection{Overview}
Lorentz invariance is the fundamental property of a physical theory that must be satisfied if it is to be consistent with special relativity. Experimental observations show that light always travels at the same speed in a vacuum no matter how fast the source of light is travelling. This experimental fact leads to some surprising consequences regarding how different observers will measure the same physical events. For instance, it means that velocities don't add up in the way one would naturally expect. We can understand why velocities don't add up in this way by considering a simple example: suppose Alice is travelling on a train at 100 mph, and she has a ball with her that she kicks down the aisle of the passenger car at 20 mph in the direction of travel. Then naturally, we would expect the kicked ball to be travelling at (100+20)=120 mph with respect to Bob who happens to be standing by the railroad track.  But now suppose Alice shines a beam of light in the direction of travel. If we ignore the fact that light travels slightly more slowly in air than in a vacuum, then whilst on the train, Alice would measure the beam of light to travel at 670,616,629 mph. So by analogy with the previous example, we would expect the stationary observer, Bob to measure the beam of light that Alice shines to travel at (100+670,616,629)=670,616,729 mph. However, it turns out that Bob measures the beam of light to have exactly the same speed as Alice measures. Thus, the manner in which Alice and Bob's velocity measurements differ is not what we would expect -- we can't just add or subtract velocities. Instead we have to use what's known as a \emph{Lorentz transformation}\index{Lorentz transformation!overview} to work out how their respective measurements transform. 

Lorentz transformations have the property that they approximately behave in the way one would expect such transformations to behave for situations involving small velocities like Alice kicking a ball on a train. But in situations which involve very high velocities such as is the case with the beam of light Alice shines on the train, then the Lorentz transformations must always be consistent with Alice and Bob measuring light to have the same speed. 

Now we are interested in physical theories that make predictions about quantifiable physical properties. The simplest kind of quantifiable physical property is called a \textbf{scalar}\index{scalar}.  A scalar consists of a single number typically associated with a particular spacetime location, and it has the same value no matter what coordinates an observer uses to describe the spacetime location. One example of a scalar is an object's \textbf{rest mass}\index{rest mass} which is the mass an object would have if it had no velocity. 

Now if two observers, Alice and Bob travel at a constant velocity relative to one another, there will be a Lorentz transformation that maps the coordinates Alice uses to describe a spacetime location to the coordinates  Bob uses to describe the same spacetime location.  But despite them using different coordinates to describe the same spacetime location, if what they are measuring is a scalar, then they will attribute the same scalar value to the same spacetime location.

As another example of a scalar, consider the mass-energy density $T_S(x)$ that we introduced on page \pageref{massenergydensity} which specifies a single number $T_S(x)$ that is associated with every $x$ belonging to the hypersurface $S$. In the next subsection, we will show that if the coordinates Alice and Bob use are related to each other via a Lorentz transformation, then their measurement of $T_S(x)$ will only depend on the spacetime location $x$ and not on the coordinate system used to describe $x$ meaning that $T_S(x)$ also has the defining property of a scalar.

Slightly more complicated than a scalar is a \emph{four-vector}\index{four-vector!overview}. Rather than associating a single number with a spacetime location, a four-vector associates four numbers with a spacetime location. Four-vectors can be used to describe the velocity of an object, and also the object's energy and momentum. With their different coordinates systems, Alice and Bob will typically describe the same four-vector with four different numbers, but the fundamental property that four-vectors must satisfy is that they transform under Lorentz transformations in such a way that any two four-vectors at a spacetime location can be combined into a product that is a scalar. It is because of this fundamental property that a four-vector can be associated with a photon to describe its velocity in such a way as to guarantee that any observer will always measure it to have the same speed.

Besides scalars and four-vectors, the final kind of quantifiable physical properties that will concern us are \emph{rank-two tensors}.\index{rank-two tensor!overview} A rank-two tensor associated with a spacetime location is a four by four array of sixteen numbers which has the property that it can be ``contracted'' with any two four-vectors associated with the same spacetime location to form a scalar. The stress-energy tensor $T^{\mu\nu}(y)$ introduced on page \pageref{stressenergy} is an example of a rank-two tensor. It is because the stress-energy tensor $T^{\mu\nu}(x)$ is a rank-two tensor for every spacetime location $x$ belonging to the hypersurface $S$ that enables us to define the scalar $T_S(x)$ by twice contracting $T^{\mu\nu}(x)$ with the four-vector that is perpendicular to the hypersurface $S$.

Now if a physical theory makes predictions about quantifiable physical properties such that these predictions transform in the same way as the quantifiable physical properties that the theory predicts, then we say that the theory is \emph{Lorentz invariant}\index{Lorentz invariant}. So in a Lorentz invariant theory, if the theory makes predictions concerning scalars, four-vectors, or rank-two tensors,  then these predictions should transform like scalars, four-vectors, or rank-two tensors respectively. 

Now the aim of this section is to show that Kent's theory is Lorentz invariant. Since Kent's theory makes predictions concerning the stress-energy tensor by predicting that ${T^{\mu\nu}(y)}$ will take the value $\ev*{\hat{T}^{\mu\nu}(y)}_{\tau_S}$, it follows that for Kent's theory to be Lorentz invariant, it is necessary that  $\ev*{\hat{T}^{\mu\nu}(y)}_{\tau_S}$ transforms as a rank-two tensor. 

The proof of Lorentz invariance involves a study of the Hilbert space of states two observers will use to describe the state of a hypersurface $S_n$ as well as the observables they use in standard quantum theory to make predictions about the mass-energy density on  $S_n\cap S$ and the stress-energy tensor on $S_n$. In particular, there will be an operator $U(\Lambda)$ dependent on the Lorentz transformation $\Lambda$ between the coordinates of the two observers that maps the state $\ket*{\Psi_n}$ one observer ascribes to $S_n$ to the state $\ket*{\Psi_n'}$ the other observer describes to $S_n$. Using this operator $U(\Lambda)$, we can find expressions\footnote{such as equations (\ref{TUrelation}) and (\ref{invariantTShat})} that relate the observables of the two observers to each other. We can also use $U(\Lambda)$ to show how the states $\pi_n\ket*{\Psi_n}$ and $\pi_n'\ket*{\Psi_n'}$ relate to one another.\footnote{How they related to one another is easily seen from equation (\ref{pipidash}) and the fact that $\ket*{\Psi_n'}=U(\Lambda)\ket*{\Psi_n'}$.} By using these relationships in the formula (\ref{kentconsistency0}), it follows that $\ev*{\hat{T}^{\mu\nu}(y)}_{\tau_S}$ transforms as a rank-two tensor, and hence that Kent's theory is Lorentz invariant.


\subsection{Technical Details\textsuperscript{*}}
In this subsection, we provide the technical details that show that Kent's theory is Lorentz invariant. Up until now we have been representing a spacetime location by a four-tuple $(x^0,$ $x^1,$ $x^2,$ $x^3)$ where $(x^i)_{i=1}^3$ are spatial coordinates, and where $x^0=ct$ with $c$ being equal to the speed of light and $t$ being the time. It will be convenient to represent spacetime locations using a more concise notation. So we let $(1,0,0,0)$ correspond to the spacetime location $\hat{e}_0$, $(0,1,0,0)$ %
\nomenclature{$\hat{e}_0, \hat{e}_1, \hat{e}_2, \hat{e}_3$}{Spacetime locations corresponding to $(1,0,0,0)$,$(0,1,0,0)$, $(0,0,1,0)$, and $(0,0,0,1)$ respectively, \nomrefpage}%
 correspond to the spacetime location $\hat{e}_1$, etc.. Then we can express any other spacetime location  as a sum $\sum_{\mu=0}^3x^\mu\hat{e}_\mu$. We will use the so-called \textbf{Einstein summation convention}\index{Einstein summation convention}\label{Einsteinsum} of dropping the summation sign and implicitly assuming that there is a summation whenever an upper index and a lower index are the same so that we can write $x^\mu\hat{e}_\mu$ %
 \nomenclature{$x^\mu\hat{e}_\mu$}{The sum $\sum_{\mu=0}^3x^\mu\hat{e}_\mu$ given by the Einstein summation convention, \nomrefpage}%
  instead of $\sum_{\mu=0}^3x^\mu\hat{e}_\mu$. We also use the convention of letting Greek letters range over $0$, $1$, $2$, and $3$ (e.g. the $\mu$ in $x^\mu\hat{e}_\mu$), and of letting Roman letters range over $1$, $2$, and $3$ (e.g. the $i$ in $(x^i)_{i=1}^3$). %
  \nomenclature{$i,j,k$}{Roman letters range over $1$, $2$, and $3$ (e.g. the $i$ in $(x^i)_{i=1}^3$), \nomrefpage}%

Now suppose an observer $\mathcal{O}$ %
\nomenclature{$\mathcal{O}, \mathcal{O}'$}{Two observers, \nomrefpage}%
 expresses spacetime locations in terms of $\{\hat{e}_\mu:\mu=0,\ldots,3\}$ and hence uses the coordinates $(x^0, x^1, x^2, x^3)$ to describe various spacetime locations. For another observer $\mathcal{O}'$, it may be more natural to express spacetime locations in terms of a different set $\{\hat{e}'_\mu:\mu=0,\ldots,3\}$ so that the location described by $\mathcal{O}$ as $(x^0, x^1, x^2, x^3)$ would be described by $\mathcal{O}'$ as $({x'}^0, {x'}^1, {x'}^2, {x'}^3)$ where ${x'}^\mu{\hat{e}'}_\mu=x^\mu\hat{e}_\mu$.  For instance if $\mathcal{O}$ and $\mathcal{O}'$ are moving with respect to each other, they may both want to use coordinates in which their own spatial coordinates are fixed and in which the spatial coordinates of the other observer are changing. As another example, figure \ref{rotfigure} shows how the $(x^1, x^2)$-coordinates transform under a spatial rotation. 


\begin{figure}[ht!]
\captionsetup{justification=justified}
\centering
\tikzmath{	
\lrange = 4;
\rrange=2;	
\textscale = 0.7;
\picscale = 0.58;
\circsize=3.5;
\px=3;
\py=3.5;
\pr=sqrt(\px*\px+\py*\py);
\th=33;
\thd=atan(\py/\px)-\th;
\pyd=\pr*sin(\thd);
\pxd=\pr*cos(\thd);
\rrange =7; 
\labelx= \rrange/2;
\labely=-0.3;
\vr=3;
\vth=75;
\vx=\vr*cos(\vth);
\vy=\vr*sin(\vth);
\vxt=\vx+\px;
\vyt=\vy+\py;
\vrt=sqrt(\vxt*\vxt+\vyt*\vyt);
\vthd=atan(\vyt/\vxt)-\th;
\vxtd=\vrt*cos(\vthd);
\vytd=\vrt*sin(\vthd);
\vxd=\vr*cos(\vth-\th);
\vyd=\vr*sin(\vth-\th);
} 
\begin{tikzpicture}[scale=\picscale] 
    \draw[-latex] (0,0) -- (\rrange,0)  node[right,scale=\textscale, text width=5em] {$\hat{e}_1$};
    \draw[-latex] (0,0) -- (0,\rrange)  node[right,scale=\textscale, text width=5em] {$\hat{e}_2$};
    \draw[dotted, thick] (\px,0) --  (\px,\py) node[midway,right, scale=\textscale]{${x}^2$}; 
     \draw[dotted, thick] (0,\py) --  (\px,\py)  node[midway,above, scale=\textscale]{${x}^1$};  

    \begin{scope}[rotate=\th,draw=red, text=red]
       \draw[-latex] (0,0) -- (\rrange,0)  node[right,scale=\textscale, text width=5em]  {$\hat{e}'_1$};
   	   \draw[-latex] (0,0) -- (0,\rrange)  node[right,scale=\textscale, text width=5em] {$\hat{e}'_2$};
   	   \draw[dotted, thick] (\pxd,0) --  (\pxd,\pyd) node[midway, above=4,right=1,scale=\textscale]{${x'}^2$}; 
   	   \draw[dotted, thick] (0,\pyd) --  (\pxd,\pyd)  node[midway,above, scale=\textscale]{${x'}^1$};  
    \end{scope}

  \draw [black,fill] (\px,\py) circle [radius=\circsize pt] ;
    
    \coordinate[label = below: (a)]  (D) at (\labelx,\labely); 
\end{tikzpicture}% pic 1
\vspace*{2px}
\caption{Shows how a location (marked as $\bullet$) can be expressed either  in coordinates $(x^1, x^2)$ with respect to the basis $\{\hat{e}_1,\hat{e}_2\}$ or in coordinates $({x'}^1,{x'}^2)$ with respect to the basis $\{\hat{e}'_1,\hat{e}'_2\}$.}\label{rotfigure}
\end{figure}

Now the key fact about all observers is that they must always observe light in a vacuum to have a constant speed $c$. Thus,  for a photon that goes through the spacetime locations $(0,0,0,0)$ and $(x^0, x^1, x^2, x^3)$ in the coordinates of $\mathcal{O}$, we must have $(x^0, x^1, x^2, x^3)=(ct,tv^1,tv^2,tv^3)$ where 
$$\sqrt{(v^1)^2 +(v^2)^2+(v^3)^2}=c.$$ But if $(0,0,0,0)$ and $(x^0, x^1, x^2, x^3)$ corresponds to $(0,0,0,0)$ and $({x'}^0, {x'}^1, {x'}^2, {x'}^3)$ respectively in the coordinates of another observer $\mathcal{O}'$, then we must also have $({x'}^0, {x'}^1, {x'}^2, {x'}^3)=(ct',t'{v'}^1,t'{v'}^2,t'{v'}^3)$ where 
$$\sqrt{({v'}^1)^2 +({v'}^2)^2+({v'}^3)^2}=c.$$ 
In either case, we must have 
\begin{equation}\label{invariant}
(x^0)^2- (x^1)^2- (x^2)^2 - (x^3)^2=({x'}^0)^2- ({x'}^1)^2- ({x'}^2)^2 - ({x'}^3)^2=0.
\end{equation}
If we define $\eta_{00}=1$, $\eta_{ii}=-1$ for $i=1,2,3$ and $\eta_{\mu\nu}=0$ %
\nomenclature{$\eta_{\mu\nu}$}{$\eta_{00}=1$, $\eta_{ii}=-1$ for $i=1,2,3$ and $\eta_{\mu\nu}=0$ for $\mu\neq\nu$, \nomrefpage}%
 for $\mu\neq\nu$, then using the Einstein summation convention as well as the convention of lowering indices so that we define $x_\mu\myeq\eta_{\mu\nu}x^\nu$, %
 \nomenclature{$x_\mu$}{$x_\mu\myeq\eta_{\mu\nu}x^\nu$, \nomrefpage}%
  then (\ref{invariant}) is equivalent to 
$$x_\mu x^\mu={x'}_\mu {x'}^\mu=0.$$ 
Thus, for any coordinate transformation $x\rightarrow x'$ such that  $x_\mu x^\mu={x'}_\mu{x'}^\mu$,  if the speed of light is $c$ in the $x$-coordinates, then the speed of light is also guaranteed to be $c$ in the $x'$-coordinates.  A \textbf{Lorentz transformation}\index{Lorentz transformation!technical definition} $\Lambda$ %
\nomenclature{$\Lambda$}{A Lorentz transformation, \nomrefpage}%
 is any coordinate  %
 \nomenclature{$Lambda\indices{^\mu_\nu}x^\nu$}{The components of a Lorentz Transformation, \nomrefpage}%
 transformation of the form ${x'}^\mu=\Lambda\indices{^\mu_\nu}x^\nu$\label{coordtansformation} such that $x_\mu x^\mu={x'}_\mu{x'}^\mu$. Since a Lorentz transformation must satisfy
$$x_\mu x^\mu=\eta_{\mu\rho}\Lambda\indices{^\rho_\sigma}x^\sigma\Lambda\indices{^\mu_\nu}x^\nu$$
for all $x$, it follows that  
\begin{equation}\label{lorentztrans}
\Lambda\indices{^\rho_\mu}\eta_{\rho\sigma}\Lambda\indices{^\sigma_\nu}=\eta_{\mu\nu}.\protect\footnotemark
\end{equation}
\footnotetext{To see why this is, note that if  $x_\mu x^\mu={x'}_\mu{x'}^\mu$ for all $x$, then  for any other spacetime location $y$, we have $(x+y)_\mu (x+y)^\mu={(x'+y')}_\mu{(x'+y')}^\mu$. If we expand this out and cancel $x_\mu x^\mu$ with ${x'}_\mu{x'}^\mu$ and cancel $y_\mu y^\mu$ with ${y'}_\mu{y'}^\mu$, and using the fact that $y_\mu x^\mu=x_\mu y^\mu$, etc. we find that  $x_\mu y^\mu={x'}_\mu{y'}^\mu$ for all $x$ and $y$. Hence,
$$\eta_{\nu\mu}x^\mu y^\nu = x_\mu y^\mu=\eta_{\sigma\rho}\Lambda\indices{^\rho_\mu}\Lambda\indices{^\sigma_\nu}x^\mu y^\nu.$$  
Since we can choose $x$ such that $x^\mu=1$ and $x^\alpha = 0$ for $\alpha\neq\mu$, and can choose $y$ such that $y^\nu=1$ and $y^\beta=0$ for $\beta\neq\nu$, it follows that
$$\eta_{\mu\nu} = \eta_{\sigma\rho}\Lambda\indices{^\rho_\mu}\Lambda\indices{^\sigma_\nu}, $$ which is the result we wanted to prove.}Having considered how the coordinates of a spacetime location viewed by one observer relate to the coordinates of the same spacetime location viewed by a different observer, we can now consider how physical quantities viewed by different observers relate to each other. So suppose $\phi(x)\myeq\phi(x^0, x^1, x^2, x^3)$ is the value of a scalar defined at the spacetime location $(x^0, x^1, x^2, x^3)$ as described by an observer $\mathcal{O}$. Then another observer $\mathcal{O}'$ using a different set of coordinate $({x'}^0, {x'}^1, {x'}^2, {x'}^3)$ to describe the spacetime location $(x^0, x^1, x^2, x^3)$ will describe this same scalar as $\phi'(x')\myeq\phi'({x'}^0, {x'}^1, {x'}^2, {x'}^3)$  where $\phi'(x')=\phi(x)$. Therefore,
\begin{equation}\label{lorentzscalar}
\phi'(x')=\phi(\Lambda^{-1}x')
\end{equation} 
where $\Lambda^{-1}$ is the inverse Lorentz transformation that takes the coordinates $x'=({x'}^0, {x'}^1, {x'}^2, {x'}^3)$ of a spacetime location to the coordinates $x=(x^0, x^1, x^2, x^3)$ describing that spacetime location. Thus, equation (\ref{lorentzscalar}) shows us how a scalar transforms under a Lorentz transformation $\Lambda$. 

Many physical quantities, however, are not scalars and so will look different to different observers. For instance, the energy of an object has a kinetic component that depends on the velocity the object has relative to an observer. However, it turns out that if an observer $\mathcal{O}$ considers an object's energy $E$ 
 %
 \nomenclature{$E$}{An object's energy, \nomrefpage}%
 together with its three components of momentum $p^1, p^2$, and $p^3$ (in the directions $\hat{e}_1$, $\hat{e}_2$, and $\hat{e}_3$ respectively) to form the four-tuple $p\myeq(E/c, p^1, p^2, p^3)$ %
\nomenclature{$p$}{The four-momentum $p=(E/c, p^1, p^2, p^3)$ where $E$ is an object's energy, and $p^1, p^2$, and $p^3$ %
\nomenclature{$p^1, p^2$, and $p^3$}{The three components of an object's momentum in the directions $\hat{e}_1$, $\hat{e}_2$, and $\hat{e}_3$ respectively, \nomrefpage}%
 are the three components of momentum in the directions $\hat{e}_1$, $\hat{e}_2$, and $\hat{e}_3$ respectively, \nomrefpage} %
 known as the object's \textbf{four-momentum}\index{four-momentum}, then $p$ transforms in the same way as spacetime coordinates transform between different observers. In other words, a different observer $\mathcal{O}'$ whose coordinates are given by ${x'}^\mu=\Lambda\indices{^\mu_\nu}x^\nu$ would observe the object's four-momentum to be ${p'}^\mu=\Lambda\indices{^\mu_\nu}p^\nu$.\footnote{\label{srdefivation}In order for $p$ to transform in this way, we have to redefine what we mean by energy and momentum. In classical mechanics, the momentum of an object is the product of the object's mass and its velocity. In the context of special relativity, however, the four-momentum of an object is defined to be the product of its rest mass $m_0$ and its \textbf{four-velocity}\index{four-velocity} where the four velocity of an object is a four-tuple $(u^0, u^1, u^2, u^3)$ with $u_\mu u^\mu=c^2$ such that the object's velocity (in the classical sense) is the vector $(c \frac{u^1}{u^0}, c \frac{u^2}{u^0}, c\frac{u^3}{u^0})$. The motivation for this definition can be seen by considering an object whose classical velocity is $\vb{v}=(v^1,v^2,v^3)$ that goes through $(0,0,0,0)$. It will have a spacetime trajectory $x(t)=(ct, v^1 t, v^2 t, v^3 t)$. $u$ is just the four-vector proportional to $x(1)$ with $u_\mu u^\mu=c^2$, so if $\gamma$ is the constant of proportionality such that $u^0=\gamma c$ and $u^i=\gamma v^i$, then by eliminating $\gamma$ we get $v^i=c\frac{u^i}{u^0}$.  We can then easily work out the constant of proportionality $\gamma$ and hence the
four-velocity $u$ of an object whose classical velocity is $\vb{v}$. For we must have $u^i=\frac{v^i u^0}{c}$, for $i=1$ to $3$. Therefore, since $u_\mu u^\mu=c^2$, we must have $(u^0)^2\big(1-\frac{v^2}{c^2}\big)=c^2$ where $v=\sqrt{({v}^1)^2+({v}^2)^2+({v}^3)^2}$. Thus, if we define $\beta={v}/{c}$ and $\gamma=\frac{1}{\sqrt{1-\beta^2}}$, then $u^0=\gamma c$ and $u^i=\gamma v^i$ for $i=1$ to $3$, and hence the four-velocity of the object must be $u=\gamma(c,v^1,v^2,v^3).$ From this, we see that the object's four-momentum will be $\gamma m_0(c,v^1,v^2,v^3).$ If the object's velocity is very small compared to the speed of light, then $\gamma\approx 1+\frac{v^2}{2c^2}$, and hence the object's four-momentum $(E/c, p^1, p^2, p^3)$ will be approximately $(m_0c+\frac{1}{2}m_0{v^2}/c, m_0v^1,m_0v^2,m_0v^3)$. Therefore, $(p^1, p^2, p^3)$ is approximately equal to the classical momentum. However, the energy is now $E=m_0c^2+\frac{1}{2}m_0{v^2}$. Thus, in addition to the kinetic energy term $\frac{1}{2}m_0{v^2}$, there is a rest mass energy $m_0c^2$. If we define the \textbf{relativistic mass}\index{relativistic mass} $m=\gamma m_0$, then we obtain Einstein's famous formula $E=mc^2$.  } More generally, any list of four physical quantities $(\varphi^0, \varphi^1, \varphi^2, \varphi^3)$ that transforms as $\varphi\rightarrow\varphi'$ with  ${\varphi'}^\mu=\Lambda\indices{^\mu_\nu}\varphi^\nu$ is called a \textbf{four-vector}\index{four-vector}.   
\begin{figure}[ht!]
	\captionsetup{justification=justified}
	\centering
	\tikzmath{
	\textscale = 0.7;
	\picscale = 0.58;
	\circsize=3.5;
	\px=3;
	\py=3.5;
	\pr=sqrt(\px*\px+\py*\py);
	\th=33;
	\thd=atan(\py/\px)-\th;
	\pyd=\pr*sin(\thd);
	\pxd=\pr*cos(\thd);
	\rrange =7; 
	\labelx= \rrange/2;
	\labely=-0.3;
	\vr=3;
	\vth=75;
	\vx=\vr*cos(\vth);
	\vy=\vr*sin(\vth);
	\vxt=\vx+\px;
	\vyt=\vy+\py;
	\vrt=sqrt(\vxt*\vxt+\vyt*\vyt);
	\vthd=atan(\vyt/\vxt)-\th;
	\vxtd=\vrt*cos(\vthd);
	\vytd=\vrt*sin(\vthd);
	\vxd=\vr*cos(\vth-\th);
	\vyd=\vr*sin(\vth-\th);
	} 
	\begin{tikzpicture}[scale=\picscale ] 
		\draw[-latex] (0,0) -- (\rrange,0)  node[right,scale=\textscale, text width=5em] {$\hat{e}_1$};
		\draw[-latex] (0,0) -- (0,\rrange)  node[right,scale=\textscale, text width=5em] {$\hat{e}_2$};
		\draw[dotted, thick] (\px,0) --  (\px,\py) node[midway,right, scale=\textscale]{${x}^2$}; 
		\draw[dotted, thick] (0,\py) --  (\px,\py)  node[midway,above, scale=\textscale]{${x}^1$};  
		
		\draw[-latex] (\px,\py)  -- (\px+\vx,\py+\vy)  node[right=22,above=-4,scale=\textscale, text width=5em] {${\varphi}$};
		\draw[dotted, thick] (\px,\py) --  (\px,\py+\vy) node[midway,left, scale=\textscale]{${\varphi}^2$}; 
		\draw[dotted, thick] (\px,\py+\vy) --  (\px+\vx,\py+\vy) node[midway,above, scale=\textscale]{${\varphi}^1$};  
	
		\begin{scope}[rotate=\th,draw=red, text=red]
		   \draw[-latex] (0,0) -- (\rrange,0)  node[right,scale=\textscale, text width=5em]  {$\hat{e}'_1$};
			  \draw[-latex] (0,0) -- (0,\rrange)  node[right,scale=\textscale, text width=5em] {$\hat{e}'_2$};
		\draw[dotted, thick] (\pxd,\pyd) --  (\vxtd,\pyd) node[midway,right=3,below=-3, scale=\textscale]{${\varphi'}^1$}; 
		\draw[dotted, thick] (\vxtd,\pyd) --  (\vxtd,\vytd) node[midway,right=6,above=-4, scale=\textscale]{${\varphi'}^2$};  
		\end{scope}
	
	  \draw [black,fill] (\px,\py) circle [radius=\circsize pt];    
		\coordinate[label = below: (b)]  (D) at (\labelx,\labely); 
	\end{tikzpicture}% pic 1
	\vspace*{2px}
	\caption{Shows how a four-vector ${\varphi}$ (of which only two components are shown) defined at a spacetime location (indicated by $\bullet$) can be expressed either as $(\varphi^1, \varphi^2)$ with respect to the basis $\{\hat{e}_1,\hat{e}_2\}$ or as $({\varphi'}^1,{\varphi'}^2)$ with respect to the basis $\{\hat{e}'_1,\hat{e}'_2\}$.}\label{rotfigure2}
	\end{figure}
	Figure \ref{rotfigure2} shows how (two of) the components of a four-vector $\varphi$ at a particular location will differ for different observers under a spatial rotation of the coordinates. A four-vector $\varphi^\mu(x)$ defined at every spacetime location $x$ is called a \textbf{four-vector field}\index{four-vector field}. Thus,  at each spacetime location $x$, the four-vector field $\varphi^\mu(x)$ assigns four numbers, $\varphi^0(x)$, $\varphi^1(x)$, $\varphi^2(x)$, and $\varphi^3(x)$. If $\mathcal{O}$ observes this vector-field $\varphi^\mu(x)$, and $\mathcal{O}'$ is another observer whose coordinates are related to the coordinates $\mathcal{O}$ via the Lorentz transformation $\Lambda$, then $\mathcal{O}'$ will describe the same physical reality that $\mathcal{O}$ describes by assigning four numbers ${\varphi'}^0(x')$, ${\varphi'}^1(x')$, ${\varphi'}^2(x')$, and ${\varphi'}^3(x')$ at every spacetime location $x'$, and the relationship between the description $\mathcal{O}$ gives and the  description $\mathcal{O}'$ gives will be given by the formula
	$${\varphi'}^\mu(x')=\Lambda\indices{^\mu_\nu}\varphi^\nu(x).$$
	Hence, under the Lorentz transformation $\Lambda$,  a vector field $\varphi^\mu(x)$ transforms as ${\varphi}^\mu(x)\rightarrow {\varphi'}^\mu(x')$ where
	\begin{equation}\label{lorentzvector}
	{\varphi'}^\mu(x')=\Lambda\indices{^\mu_\nu}\varphi^\nu(\Lambda^{-1}x').
	\end{equation} 



From a four-vector $\varphi^\mu$, we can also define the so-called \textbf{four-covector}\index{four-covector}: 
\begin{equation}\label{covector}
	\varphi_\mu\myeq\eta_{\mu\nu}\varphi^\nu.
\end{equation}
To see how four-covectors transform under a Lorentz transformation $\Lambda$, it will be helpful to define 
\begin{equation}\label{colambda}
	\Lambda\indices{_\mu^\nu}\myeq\eta_{\mu\rho}\eta^{\nu\sigma}\Lambda\indices{^\rho_\sigma}
\end{equation}
where %
\nomenclature{$\Lambda\indices{_\mu^\nu}$}{The components of an inverse Lorentz transformation $\Lambda^{-1}$, see equation (\ref{colambda}), \nomrefpage}%
 $\eta^{\nu\sigma}=\eta_{\nu\sigma}$. %
 \nomenclature{$\eta^{\nu\sigma}$}{Defined so that  $\eta_{\mu\rho}\eta^{\nu\rho}=\delta^\nu_\mu$, \nomrefpage}%
 If we also define the \textbf{Kronecker-delta}\index{Kronecker-delta} $\delta^\nu_\mu$ such%
\nomenclature{$\delta^\nu_\mu$}{The {Kronecker-delta} given by $\delta^\nu_\mu=1$ when $\mu=\nu$ and $\delta^\nu_\mu=0$ otherwise, \nomrefpage}%
that $\delta^\nu_\mu=1$ when $\mu=\nu$ and $\delta^\nu_\mu=0$ otherwise, then using the fact that $\eta_{\mu\rho}\eta^{\nu\rho}=\delta^\nu_\mu$ together with equation (\ref{lorentztrans}), we have 
\begin{equation}\label{lambdainverse}
\Lambda\indices{^\rho_\mu}\Lambda\indices{_\rho^\nu}=\delta^\nu_\mu.
\end{equation}
Since by definition, the inverse of $\Lambda^{-1}$ satisfies 
$(\Lambda^{-1})\indices{^\nu_\rho}\Lambda\indices{^\rho_\mu}=\delta^\nu_\mu,$
 we have $(\Lambda^{-1})\indices{^\nu_\rho}=\Lambda\indices{_\rho^\nu}.$  From (\ref{lorentzvector}), (\ref{covector}), and (\ref{colambda}), we therefore see that under a Lorentz transformation $\Lambda$, a four-covector field $\varphi_\mu(x)$ transforms as $\varphi_\mu(x)\rightarrow\varphi'_\mu(x')$
where
\begin{equation}\label{lorentzcovector}
\varphi'_\mu(x')=\Lambda\indices{_\mu^\nu}\varphi_\nu(\Lambda^{-1}x')
\end{equation}
 

Besides scalars, four-vectors, and four-covectors, we also need to consider physical quantities called rank-two tensors. The defining property of a rank-two tensor field $\varphi^{\mu\nu}(x)$ is that under a Lorentz transformation $\Lambda$, it transforms as $\varphi^{\mu\nu}(x)\rightarrow{\varphi'}^{\mu\nu}(x')$ where
\begin{equation}\label{lorentztensor}
{\varphi'}^{\mu\nu}(x')=\Lambda\indices{^\mu_\rho}\Lambda\indices{^\nu_\sigma}\varphi^{\rho\sigma}(\Lambda^{-1}x').
\end{equation}
On page \pageref{massenergydensity}, we introduced the mass-energy density $T_S(x)$ on a hypersurface $S$. As explained in section \ref{massenergydensity}, the values of $T_S(x)$ for all $x\in S$ are the additional variables that Kent uses to supplement standard quantum theory.  It was mentioned in passing that $T_S(x)$ does not depend on which frame of reference one is in. In other words, $T_S(x)$ is a scalar. I will now explain why this is so. 

We first need to consider the precise definition of $T_S(x)$. At each spacetime location on the hypersurface $S$ which an observer $\mathcal{O}$ describes as having coordinates $x=(x^\mu)_{\mu=0}^3$, we define  $\eta^\mu(x)$ %
\nomenclature{$\eta^\mu(x)$}{The future-directed  unit four-vector at $x$ that is orthogonal to $S$., \nomrefpage}%
 to be the future-directed  unit four-vector at $x$ that is orthogonal to $S$. In other words, $\eta^0(x)>0$, $\eta_\mu(x)\eta^\mu(x)=1$, and if $y\in S$ is very close to $x$, then 
\begin{equation}\label{etaorthog}
\frac{(x-y)_\mu\eta^\mu(x)}{\sqrt{(x-y)_\nu(x-y)^\nu}}\approx 0.
\end{equation}$T_S(x)$ is then given by the formula 
\begin{equation}\label{TSdef}
T_S(x)=T^{\mu\nu}(x)\eta_{\mu}(x)\eta_{\nu}(x).
\end{equation}%
\nomenclature{$T_S(x)$  }{The mass-energy density on a hypersurface $S$, p. \pageref{firstTS}. Also see equation \ref{TSdef}, \nomrefpage }%
For example, if $S$ was the hypersurface consisting of all spacetime locations $x = (0,x^1,x^2,x^3)$, then $\big(\eta^{0}(x),\eta^{1}(x),\eta^{3}(x),\eta^{3}(x)\big) =(1,0,0,0),$ and hence $T_S(x)=T^{00}(x)$ which is the density of relativistic mass at $x$, i.e. the energy density at $x$ divided by $c^2$. Note that the condition that $y\in S$ is very close to $x$ in (\ref{etaorthog}) means that $T_S(x)$ only has a local dependence \label{localdependenceS} on $S$ in the vicinity of $x$. i.e. if $S'$ only differs from $S$ outside the vicinity of $x$, then  $T_{S'}(x)=T_S(x).$

To see why $T_S(x)$ is a scalar, suppose that $\Lambda$ is a Lorentz transformation such that $\Lambda\indices{^0_\mu}\eta^\mu>0$ for any future-directed  unit four-vector vector $\eta^\mu$. We refer to a $\Lambda$ with this property as an \textbf{orthochronous}\index{Lorentz transformation!orthochronous} Lorentz transformation. Also, suppose that $\mathcal{O}$ and $\mathcal{O}'$ are two observers such that spacetime locations that observer $\mathcal{O}$ describes as having coordinates $x=(x^\mu)_{\mu=0}^3$ are described by $\mathcal{O}'$ as having coordinates $x'=(\Lambda\indices{^\mu_\nu}x^\nu)_{\mu=0}^3$. %
\nomenclature{$x'$}{The coordinate of a spacetime location described by observer $\mathcal{O}'$ so that $x'=(\Lambda\indices{^\mu_\nu}x^\nu)_{\mu=0}^3$ where $x$ is a spacetime location described by observer $\mathcal{O}$, \nomrefpage}%
 Then since ${x'}_\mu{y'}^\mu= x_\mu y^\mu$, it follows that the future-directed unit four-vector orthogonal to $S$ at $x$ which $\mathcal{O}$ describes as $\eta^\mu(x)$ will be described by $\mathcal{O}'$ as  ${\eta'}^\mu(x')=\Lambda\indices{^\mu_\nu}\eta^\nu(x)$. %
 \nomenclature{${\eta'}^\mu(x')$}{The future-directed unit four-vector orthogonal to $S$ as described by observer $\mathcal{O}'$, \nomrefpage}%
  Thus, for any spacetime location in $S$ that $\mathcal{O}'$ describes as having coordinates $x'$ with corresponding  future-directed $S$-orthogonal unit four-vector ${\eta'}^\mu(x')$, $\mathcal{O}'$ can construct a function $T'_S(x')$  with 
\begin{equation}\label{TSprimedef}
T'_S(x')=T'^{\mu\nu}(x')\eta'_{\mu}(x')\eta'_{\nu}(x').
\end{equation} %
\nomenclature{$T'_S(x')$}{The mass-energy density as described by an observer $\mathcal{O}'$, see equation (\ref{TSprimedef}), \nomrefpage}%
Then using  (\ref{lorentzcovector}) and (\ref{lorentztensor}) on the right-hand side of (\ref{TSprimedef}),  we have
\begin{equation}\label{invariantTS1}
\begin{split}
T'_S(x')&=\Lambda\indices{^\mu_\rho}\Lambda\indices{^\nu_\sigma}T^{\rho\sigma}(x)\Lambda\indices{_\mu^\alpha}\eta_{\alpha}(x)\Lambda\indices{_\nu^\beta}\eta_{\beta}(x)\\
&=\Lambda\indices{^\mu_\rho}\Lambda\indices{_\mu^\alpha} \Lambda\indices{^\nu_\sigma}\Lambda\indices{_\nu^\beta}T^{\rho\sigma}(x)\eta_{\alpha}(x)\eta_{\beta}(x)\\
&=\delta^\alpha_\rho\delta^\beta_\sigma T^{\rho\sigma}(x)\eta_{\alpha}(x)\eta_{\beta}(x)\\
&=T^{\alpha\beta}(x)\eta_{\alpha}(x)\eta_{\beta}(x)\\
&=T_S(x)
\end{split}
\end{equation}
where on the third line we have used (\ref{lambdainverse}), and on the last line we have used (\ref{TSdef}). To obtain (\ref{invariantTS1}), we assumed that $\Lambda$ is orthochronous because definition (\ref{TSdef}) assumes that $\eta^\mu(x)$ is future-directed. But if $\Lambda$ is non-orthochronous, we would need to take the negations of ${\eta'}^\mu(x')$ to get the future-directed $S$-orthogonal unit four-vector. But clearly this will not affect the equality in (\ref{invariantTS1}), so (\ref{invariantTS1}) holds for all Lorentz transformations, whether they are orthochronous or non-orthochronous.  We thus see that   $T_S(x)$ is a scalar.

Let us now consider the Hilbert space $H_{S_n}$  for a hypersurface $S_n$ as defined on page \pageref{HSidef}.\footnote{Also see page \pageref{HSdef} for the definition of $H_S$.} 
Given that $\hat{T}^{\mu\nu}(x)$ is the observable in the Tomonaga-Schwinger picture whose eigenstates with eigenvalues $\tau$ are the states of $S_n$ for which an observer $\mathcal{O}$ observes the stress-energy tensor $T^{\mu\nu}(x)$ to take the value $\tau$ at $x$, it follows from (\ref{TSdef}) that 
\begin{equation}\label{TShat}
	\hat{T}_S(x)\myeq \hat{T}^{\mu\nu}(x)\eta_{\mu}(x)\eta_{\nu}(x)
	\end{equation}
will %
\nomenclature{$\hat{T}_S(x)$}{The observable corresponding to $T_S(x)$, p. \pageref{firstHatTS}. Also see equation (\pageref{TShat}), \nomrefpage}%
 be the observable whose eigenstates with eigenvalues $\tau_S(x)$ are the states of $S_n$ for which an observer $\mathcal{O}$ observes $T_S(x)$ to take the value $\tau_S(x)$ at $x$ when $x\in S_n\cap S$.\footnote{So long as $S_n$ and $S$ are tangential at $x$ as noted in footnote \footreference{tangentialnote}.}
 
Now two observers $\mathcal{O}$ and $\mathcal{O}'$ will typically assign different physical states to $S_n$ based on their frame of reference. E.g. if $\mathcal{O}$ and $\mathcal{O}'$ are traveling at different speeds, they will attribute different energy levels and momenta to the spacetime locations of $S_n$. To understand the relationship between the states $\mathcal{O}$ assigns to $S_n$ and the states $\mathcal{O}'$ assigns, suppose $\ket*{\psi}$ and  $\ket*{\chi}$ are two states that an observer $\mathcal{O}$ might judge $S_n$ to be in. As usual, we suppose the coordinates $x^\mu$ of observer $\mathcal{O}$ transform to coordinates ${x'}^\mu=\Lambda\indices{^\mu_\nu}x^\nu$ of observer $\mathcal{O}'$ for some Lorentz transformation $\Lambda.$ We also suppose that the states $\ket*{\psi}$ and $\ket*{\chi}$ that $\mathcal{O}$ observes will transform to states $\ket*{\psi'}$  and $\ket*{\chi'}$ that $\mathcal{O}'$ observes. %
\nomenclature{$\ket*{\psi'}, \ket*{\chi'}$}{States that an observer $\mathcal{O}'$ corresponding to state $\ket*{\psi}$ and $\ket*{\chi}$ that an observer $\mathcal{O}$ observer, \nomrefpage}%
 We will denote the Hilbert space of the states on $S_n$ that $\mathcal{O}'$ can observe as $H_{S_n}'$.\label{Hprimespace}\footnote{As we will discuss shortly, there is an inner product preserving map $U(\Lambda)$ via which there is a one-to-one correspondence between states in $H_{S_n}$ and states in $H_{S_n}'$, so we can identify $H_{S_n}'$ with $H_{S_n}$ in which case $U(\Lambda)$ will be a unitary operator.}%
\nomenclature{$H_{S_n}'$}{The states on $S_n$ that an observer $\mathcal{O}'$ can observe, \nomrefpage}%

Now if  $\mathcal{O}$ judged $S_n$ to be in the superposition state $\ket{\psi}+\ket{\chi}$, then $\mathcal{O}'$ would judge $S_n$ to be proportional to the superposition state $\ket{\psi'}+\ket{\chi'}$. Also recall that $\ket*{\psi'}$ and $\lambda\ket*{\psi'}$ represent the same physical state for any complex number $\lambda$, so there is sufficient flexibility as to which state we deem $\ket*{\psi}$ transforms to that we can deem the transformation $\ket{\psi}\rightarrow\ket*{\psi'}$ to be a linear transformation. But also note that if observer $\mathcal{O}$ uses the Born rule to calculate the transition probability from state $\ket*{\psi}$ to state $\ket*{\chi}$, and  observer $\mathcal{O}'$ uses the Born rule to calculate the transition probability from state $\ket*{\psi'}$ to state $\ket*{\chi'}$, then they should calculate the same probabilities. We must therefore have
$$|\ip*{\chi}{\psi}|^2=|\ip*{\chi'}{\psi'}|^2.$$
Using this fact together with the fact that $\ket*{\psi}$ and $\lambda\ket*{\psi}$ represent the same physical states, it can be shown that there is a unitary operator $U(\Lambda)$ %
\nomenclature{$U(\Lambda)$}{The unitary operator which relates the state $\ket*{\psi}$ observer $\mathcal{O}$ observes to the state $\ket*{\psi'}$ observer $\mathcal{O}'$ observes, i.e. $\ket*{\psi'}=U(\Lambda)\ket*{\psi}$, \nomrefpage}%
 which relates the states $\ket*{\psi}$ and $\ket*{\psi'}$ via the formulated
$$\ket*{\psi'}=U(\Lambda)\ket*{\psi}.\protect\footnotemark$$\footnotetext{For more details, see \cite{Wigner1939}. Since a unitary operator maps a Hilbert space to itself, we first need to identify $H_{S_n}$ and $H_{S_n}'$ in order for $U(\Lambda)$ to be unitary.}At this point, it is worth clarifying the different meanings of $T^{\mu\nu}(x)$, $T^{\prime\mu\nu}(x')$, $\tau^{\mu\nu}(x)$, $\tau^{\prime\mu\nu}(x')$,  $\hat{T}^{\mu\nu}(x)$ and $\hat{T}^{\prime\mu\nu}(x')$. 
\begin{itemize}
\item We use $T^{\mu\nu}(x)$ to refer to the description of the physical quantity that is being observed by $\mathcal{O}$. Thus, $T^{\mu\nu}(x)$ is shorthand for the description ``the $\mu\nu$-component of the stress-energy tensor that $\mathcal{O}$ observes at the spacetime location belonging to $S$ that $\mathcal{O}$ describes as  $x$.'' 
\item Similarly,  $T^{\prime\mu\nu}(x')$ is shorthand for the description ``the $\mu\nu$-component of the stress-energy tensor that $\mathcal{O}'$ observes at a spacetime location  belonging to $S$ that $\mathcal{O}'$ describes as $x'$.'' %
\nomenclature{$T^{\prime\mu\nu}(x')$}{The $\mu\nu$-component of the stress-energy tensor that $\mathcal{O}'$ observes at a spacetime location  belonging to $S$ that $\mathcal{O}'$ describes as $x'$, \nomrefpage}% 
\item $\tau^{\mu\nu}(x)$ stands for a particular (real) value of the physical quantity described by $T^{\mu\nu}(x)$ that $\mathcal{O}$ observes, and 
\item $\tau^{\prime\mu\nu}(x')$ stands for a particular (real) value of the physical quantity described by $T^{\prime\mu\nu}(x')$ that $\mathcal{O}'$ observes.%
\nomenclature{$\tau^{\prime\mu\nu}(x')$}{A particular (real) value of the physical quantity described by $T^{\prime\mu\nu}(x')$ that $\mathcal{O}'$ observes., \nomrefpage}%
\item $\hat{T}^{\mu\nu}(x)$ for $x\in S_n$ is the Tomonaga-Schwinger observable acting on $H_{S_n}$ such that if observer $\mathcal{O}$ deemed $S_n$ to be in an eigenstate $\ket*{\psi}$ of $\hat{T}^{\mu\nu}(x)$ with eigenvalue $\tau$ (a real number), then observer $\mathcal{O}$ would observe the physical quantity described by  $T^{\mu\nu}(x)$ to have the value $\tau$.\footnote{Note that we consider a real number $\tau$ here rather than a real valued function $\tau^{\mu\nu}(x)$ of spacetime locations $x\in S$ and indices $\mu$ and $\nu$ since $\hat{T}^{\mu\nu}(x)$ will not in general commute for different values of $\mu$, $\nu$, so we won't be able to find a state which is a simultaneous eigenstate for all the different observables $\hat{T}^{\mu\nu}(x)$, though we may find a state which is very close to being an eigenstate of all the  $\hat{T}^{\mu\nu}(x)$ for different $\mu$ and $\nu$.} 
\item $\hat{T}^{\prime\mu\nu}(x')$ is the Tomonaga-Schwinger observable acting on $H_{S_n}'$ such that if observer $\mathcal{O}'$ deemed $S_n$ to be in an eigenstate $\ket*{\psi'}$ of  $\hat{T}^{\prime\mu\nu}(x')$ with eigenvalue $\tau'$, then observer $\mathcal{O}'$ would observe the physical quantity described by  $T^{\prime\mu\nu}(x')$ to have the value $\tau'$.  %
\nomenclature{$\hat{T}^{\prime\mu\nu}(x')$}{The Tomonaga-Schwinger observable acting on $H_{S_n}'$ such that if observer $\mathcal{O}'$ deemed $S_n$ to be in an eigenstate $\ket*{\psi'}$ of  $\hat{T}^{\prime\mu\nu}(x')$ with eigenvalue $\tau'$, then observer $\mathcal{O}'$ would observe the physical quantity described by  $T^{\prime\mu\nu}(x')$ to have the value $\tau'$., \nomrefpage}%
\item $T_S(x)={T}^{\mu\nu}(x)\eta_{\mu}(x)\eta_{\nu}(x)$ is shorthand for the description ``the mass-energy density of the hypersurface $S$ observed by observer $\mathcal{O}$ at a spacetime location that $\mathcal{O}$ describes as $x$''. 
\item $T'_S(x')={T}^{\prime\mu\nu}(x')\eta'_{\mu}(x')\eta'_{\nu}(x')$ is shorthand for the description ``the mass-energy density of the hypersurface $S$ observed by observer $\mathcal{O}'$ at a spacetime location that $\mathcal{O}'$ describes as $x'$''. %
\nomenclature{$T'_S(x')$}{The mass-energy density of the hypersurface $S$ observed by observer $\mathcal{O}'$ at a spacetime location that $\mathcal{O}'$ describes as $x'$, \nomrefpage}% 
\item The function $\tau_S(x)$ stands for a particular range of values for each $x\in S$ of the physical quantity described by $T_S(x)$ observed by $\mathcal{O}$. 
\item The function $\tau'_S(x')$ stands for a particular range of values for each $x'\in S$ of the physical quantity described by $T'_S(x')$ observed by $\mathcal{O}'$.  %
\nomenclature{ $\tau'_S(x')$}{A particular range of values for each $x'\in S$ of the physical quantity described by $T'_S(x')$ observed by $\mathcal{O}'$, \nomrefpage}%
\item For each $x\in S$, $\hat{T}_S(x)=\hat{T}^{\mu\nu}(x)\eta_{\mu}(x)\eta_{\nu}(x)$ is the Tomonaga-Schwinger observable  such that if an observer $\mathcal{O}$ deems $S$ to be in an eigenstate $\ket*{\psi}\in H_{S_n}$ of  $\hat{T}_S(x)$ with eigenvalue $\tau$ (a real number), then  $\mathcal{O}$ would observe the physical quantity described by  $T_S(x)$ to have the value $\tau$. 
\item For each $x'\in S$, $\hat{T}'_S(x')=\hat{T}^{\prime\mu\nu}(x)\eta'_{\mu}(x')\eta'_{\nu}(x')$ is the Tomonaga-Schwinger observable  such that if an observer $\mathcal{O}'$ deems $S$ to be in an eigenstate $\ket*{\psi'}\in H'_{S_n}$ of  $\hat{T}'_S(x')$ with eigenvalue $\tau'$ (a real number), then  $\mathcal{O}'$ would observe the physical quantity described by  $T'_S(x')$ to have the value $\tau'$.  %
\nomenclature{$\hat{T}'_S(x')$}{The Tomonaga-Schwinger observable  such that if an observer $\mathcal{O}'$ deems $S$ to be in an eigenstate $\ket*{\psi'}\in H'_{S_n}$ of  $\hat{T}'_S(x')$ with eigenvalue $\tau'$ (a real number), then  $\mathcal{O}'$ would observe the physical quantity described by  $T'_S(x')$ to have the value $\tau'$., \nomrefpage}%
\end{itemize}



Having clarified this terminology, we see that if $\ket*{\psi}\in H_{S_n}$ is a state for which $T^{\mu\nu}(x)\approx \tau^{\mu\nu}(x)$,\footnote{We say approximately ($\approx$) here since the operators $\hat{T}^{\mu\nu}$ will not in general commute, so we won't typically be able to find a state $\ket*{\psi}$ which is an eigenstate for all the observables $\hat{T}^{\mu\nu}$. It is the non-commutativity of observables that is responsible for Heisenberg's uncertainty principle.} and if $\ket*{\psi'}\in H_{S_n}'$ is a state for which $T^{\prime\mu\nu}(x')\approx \tau^{\mu\nu}(x')$, then
\begin{subequations}
\begin{align}
\hat{T}^{\mu\nu}(x)\ket{\psi}&\approx\tau^{\mu\nu}(x)\ket{\psi}, \text{ and }\label{stres1}\\ 
\hat{T}^{\prime\mu\nu}(x')\ket{\psi'}&\approx{\tau'}^{\mu\nu}(x')\ket{\psi'}.\label{stres1}
\end{align}
\end{subequations}
It then follows from (\ref{stres1}) that if $\ket{\psi'}=U(\Lambda)\ket{\psi}$, then
\begin{equation}\label{stress3}
    U(\Lambda)^{-1}\hat{T}^{\prime\mu\nu}(x')U(\Lambda)\ket{\psi}\approx{T'}^{\mu\nu}(x')\ket{\psi}.
\end{equation}
Therefore, in order for (\ref{lorentztensor}) to hold for ${T'}^{\mu\nu}(x')$ in the classical limit (where we treat the stress-energy observables as though they commute with each other and replace the approximations by equalities), by plugging an operator form of  (\ref{lorentztensor}) into (\ref{stress3}), we see that we must have
\begin{equation}\label{TUrelation}
U(\Lambda)^{-1}\hat{T}^{\prime\mu\nu}(x')U(\Lambda)=\Lambda\indices{^\mu_\rho}\Lambda\indices{^\nu_\sigma}\hat{T}^{\rho\sigma}(\Lambda^{-1}x').
\end{equation}



Now to say that Kent's model is Lorentz invariant, is to say that (\ref{kentconsistency0}) defines a rank-two tensor, for then this quantity and the quantities on which it depends will transform in the way that physical quantities should transform under a Lorentz transformation $\Lambda$ when the spacetime coordinates of two observers $\mathcal{O}$ and $\mathcal{O}'$ are related by the formula ${x'}^\mu=\Lambda\indices{^\mu_\nu}x^\nu$.\footnote{Note that having a privileged hypersurface $S$ in which a notional measurement of $T_S$ is made does not of itself break Lorentz invariance. Just because we are privileging a hypersurface $S$, we are not making any assumptions about simultaneity being defined by $S$. Because spacelike separation is a Lorentz invariant property, both $\mathcal{O}$ and $\mathcal{O}'$ will deem the spacetime locations on $S$ to be spacelike separated. The Lorentz transformation itself has absolutely no effect on what $S$ is. It is just that $\mathcal{O}$ and $\mathcal{O}'$ will use different coordinates to describe a particular spacetime location on $S$. It maybe that in the coordinate system of $\mathcal{O}$, some of the spacetime locations of $S$ are simultaneous (i.e. have the same $x^0$ value), but there is no requirement of simultaneity, and there is no claim that a reference frame in which the spacetime locations of $S$ are simultaneous is particularly special. In Kent's theory, it is sufficient for there to be just one hypersurface on which  $T_S$ has a determinate value. But if another hypersurface were to be chosen instead, it would make no difference to empirical adequacy since (\ref{kentconsistency}) will hold regardless of what hypersurface $S$ is chosen.} Thus, in order to show that Kent's model is Lorentz invariant, we need to show that if $\{\ket{\xi_j}:j\}$ is an % 
\nomenclature{$\{\ket{\xi_j}:j\}$}{Orthonormal basis of the Hilbert space of states $H_{S_n,\tau_S}$ for which $\mathcal{O}$ observes $T_S(x)$ to be $\tau_S(x)$ for all $x\in S_n(y)\cap S$, \nomrefpage}%
orthonormal basis of the Hilbert space of states\footnote{Thus, $H_{S_n,\tau_S}$ is the subspace of states $\ket{\xi}\in H_{S_n}$ for which  $\hat{T}_S(x)\ket{\xi}=\tau_S(x)\ket{\xi}$  for all $x\in S_n\cap S$ as mentioned on page \pageref{HStau}.} $H_{S_n,\tau_S}$ for which $\mathcal{O}$ observes $T_S(x)$ to be $\tau_S(x)$ for all $x\in S_n(y)\cap S$, and if $\{\ket{\xi_j'}:j\}$ is %
\nomenclature{ $\{\ket{\xi_j'}:j\}$}{Orthonormal basis of the Hilbert space of states $H'_{S_n,\tau_S'}$ for which $\mathcal{O}'$ observes $T'_S(x')$ to be $\tau'_S(x')$ for all $x'\in S_n(y')\cap S$, \nomrefpage}%
 an orthonormal basis of the Hilbert space %
 \nomenclature{ $H'_{S_n,\tau_S'}$}{The Hilbert space of states for which $\mathcal{O}'$ observes $T'_S(x')$ to be $\tau'_S(x')$ for all $x'\in S_n(y')\cap S$, \nomrefpage}%
  of states\footnote{Thus,  $H'_{S_n,\tau'_S}$ is the subspace of states $\ket{\xi'}\in H'_{S_n}$ for which  $\hat{T}'_S(x')\ket{\xi'}=\tau'_S(x')\ket{\xi'}$  for all $x'\in S_n\cap S$. } $H'_{S_n,\tau_S'}$ for which $\mathcal{O}'$ observes $T'_S(x')$ to be $\tau'_S(x')$ for all $x'\in S_n(y')\cap S$, then
\begin{equation}\label{kentlorentz}
\lim_{n\rightarrow\infty}\frac{\ev{\pi_n'\hat{T}^{\prime\mu\nu}(y')}{\Psi_n'}}{\ev{\pi_n'}{\Psi_n'}}=\Lambda\indices{^\mu_\rho}\Lambda\indices{^\nu_\sigma} \lim_{n\rightarrow\infty}\frac{\ev{\pi_n\hat{T}^{\rho\sigma}(y)}{\Psi_n}}{\ev{\pi_n}{\Psi_n}}
\end{equation}
where $\pi_n=\sum_j\dyad{\xi_j}$, $\pi_n'=\sum_j\dyad{\xi_j'}$, and $\ket{\Psi_n'}=U(\Lambda)\ket{\Psi_n}$.  %
\nomenclature{$\pi_n'$}{The projection $\pi_n'=\sum_j\dyad{\xi_j'}$, \nomrefpage}%

To see why (\ref{kentlorentz}) holds, we first recall that $\pi_n'$ will be independent of which orthonormal basis we choose for $H_{S_n,\tau_S'}$.\footnote{We showed this was the case for $\pi_n$ in footnote \ref{priproof} on page \pageref{priproof}.} Therefore, if we can show that $\{\ket{\xi_j'}\myeq U(\Lambda)\ket{\xi_j}:j\}$ is an orthonormal basis of $H_{S_n,\tau_S'}$, it will follow that $\pi_n'=U(\Lambda)\pi_nU(\Lambda)^{-1}$. 

That the elements of $\{U(\Lambda)\ket{\xi_j}:j\}$ are orthonormal follows from the unitarity of $U(\Lambda)$ together with the orthonormality of  $\{\ket{\xi_j}:j\}$. 
It remains for us to show that each $U(\Lambda)\ket*{\xi_j}\in H'_{S_n,\tau_S'},$ and that any $\ket*{\xi'}\in H'_{S_n,\tau_S'}$ can be expressed as a linear combination of the $U(\Lambda)\ket*{\xi_j}$.

Well, first note that by (\ref{TUrelation}) and a calculation similar to (\ref{invariantTS1})
\begin{equation}
\begin{split}\label{invariantTShat}
U(\Lambda)^{-1}\hat{T}'_S(x')U(\Lambda)&=U(\Lambda)^{-1}\hat{T}^{\prime\mu\nu}(x')\eta'_{\mu}(x')\eta'_{\nu}(x')U(\Lambda)\\
&=\Lambda\indices{^\mu_\rho}\Lambda\indices{^\nu_\sigma}\hat{T}^{\rho\sigma}(x)\Lambda\indices{_\mu^\alpha}\eta_{\alpha}(x)\Lambda\indices{_\nu^\beta}\eta_{\beta}(x)\\
&=\Lambda\indices{^\mu_\rho}\Lambda\indices{_\mu^\alpha} \Lambda\indices{^\nu_\sigma}\Lambda\indices{_\nu^\beta}\hat{T}^{\rho\sigma}(x)\eta_{\alpha}(x)\eta_{\beta}(x)\\
&=\delta^\alpha_\rho\delta^\beta_\sigma \hat{T}^{\rho\sigma}(x)\eta_{\alpha}(x)\eta_{\beta}(x)\\
&=\hat{T}^{\alpha\beta}(x)\eta_{\alpha}(x)\eta_{\beta}(x)\\
&=\hat{T}_S(x)
\end{split}
\end{equation}
From (\ref{invariantTShat}), we see that  $U(\Lambda)\hat{T}_S(x)=\hat{T}'_S(x')U(\Lambda)$, so
$$
\hat{T}_S'(x')U(\Lambda)\ket{\xi_j}=U(\Lambda)\hat{T}_S(x)\ket{\xi_j}=\tau_S(x)U(\Lambda)\ket{\xi_j}=\tau_S'(x')U(\Lambda)\ket{\xi_j}
$$
for all $x'\in S_n(y')\cap S$, where we have used the fact $\tau_S'(x')=\tau_S(x)$ since $T_S(x)$ is a scalar. Therefore, $U(\Lambda)\ket{\xi_j}\in H'_{S_n,\tau_S'}$. 

Now suppose that $\ket{\xi'}$ is a state for which $\mathcal{O}'$ observes $T'_S(x')$ to be $\tau'_S(x')$ for all $x'\in S_n(y')\cap S$, i.e.  $\hat{T}_S'(x')\ket{\xi'}=\tau_S'(x')\ket{\xi'}$. From (\ref{invariantTShat}) we see that  $\hat{T}_S(x)U(\Lambda)^{-1}=U(\Lambda)^{-1}\hat{T}'_S(x')$, so
\begin{equation}\label{TSUxi}
\begin{split}
\hat{T}_S(x)U(\Lambda)^{-1}\ket{\xi'}&= U(\Lambda)^{-1} \hat{T}_S'(x')\ket{\xi'}\\
&=\tau_S'(x')U(\Lambda)^{-1}\ket{\xi'}\\
&=\tau_S(x)U(\Lambda)^{-1}\ket{\xi'}
\end{split}
\end{equation}
where on the last line we have used the fact that $T_S(x)$ is a scalar. Therefore, $U(\Lambda)^{-1}\ket{\xi'}$ can be expressed as a linear combination of the basis elements $\{\ket{\xi_j}:j\}$ of $H_{S_n,\tau_S}$, and hence  $\ket{\xi'}$  can be expressed as a linear combination of $\{U(\Lambda)\ket{\xi_j}:j\}$. 




Thus, we see that $\{\ket{\xi_j'}\myeq U(\Lambda)\ket{\xi_j}:j\}$ is a spanning orthonormal subset of $H'_{S_n,\tau_S'}$, so it must therefore be an orthonormal basis of $H'_{S_n,\tau_S'}$. From this it follows that 
\begin{equation}\label{pipidash}
\pi_n'=U(\Lambda)\pi_nU(\Lambda)^{-1}.
\end{equation}
Therefore, 
\begin{equation}\label{kentlorentz2}
\begin{split}
\frac{\ev{\pi_n'\hat{T}^{\prime\mu\nu}(y')}{\Psi_n'}}{\ev{\pi_n'}{\Psi_n'}}&=\frac{\ev{U(\Lambda)^{-1}U(\Lambda)\pi_nU(\Lambda)^{-1}\hat{T}^{\prime\mu\nu}(y')U(\Lambda)}{\Psi_n}}{\ev{U(\Lambda)^{-1}U(\Lambda)\pi_n U(\Lambda)^{-1}U(\Lambda)}{\Psi_n}}\\
&=\frac{\ev{\pi_nU(\Lambda)^{-1}\hat{T}^{\prime\mu\nu}(y')U(\Lambda)}{\Psi_n}}{\ev{\pi_n}{\Psi_n}}\\
&=\frac{\ev{\pi_n\Lambda\indices{^\mu_\rho}\Lambda{^\nu_\sigma}\hat{T}^{\rho\sigma}(y)}{\Psi_n}}{\ev{\pi_n }{\Psi_n}}
\end{split}
\end{equation}
where on the last line we have used (\ref{TUrelation}). Thus, equation (\ref{kentlorentz}) holds, and hence Kent's model is Lorentz invariant.

Note that in this proof of Lorentz invariance, we don't need to take the limit of $S_n$ as $n\rightarrow\infty$. That is, we could remove the $\lim_{n\rightarrow\infty}$ from equation (\ref{kentconsistency0}) and consider a particular $S_n$, and the corresponding $\ev*{T^{\mu\nu}(y)}_{\tau_S}$ would still be a rank-two tensor. Butterfield tells us that Kent's theory is  Lorentz invariant because his algorithm respects the light cone structure of $y$.\footnote{See \cite[30]{Butterfield}.} However, this statement could be slightly misleading because we don't need to consider the subset $S^1(y)\subset S$ of locations outside the light cone of $y$ in order to obtain a Lorentz invariant model. Doing the calculation on any Tomonaga-Schwinger hypersurface is sufficient to guarantee Lorentz invariance since any such hypersurface (e.g. $S_n$) is not altered at all by a Lorentz transformation -- only its coordinate description changes under a Lorentz transformation, and so the additional information of the scalar $\tau_S(x)$ on $S_n\cap S$ is Lorentz invariant. The only reason we need to consider the limit $\lim_{n\rightarrow \infty}S_n$ and hence $S^1(y)=\lim_{n\rightarrow \infty}S_n\cap S$ is that it is only in the limit that we use all the available information in $\tau_S(x)$ to calculate $\ev*{T^{\mu\nu}(y)}_{\tau_S}$. 




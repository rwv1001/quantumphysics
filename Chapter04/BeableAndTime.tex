\section{Beables and Time\label{beablesandtime}}
In previous sections I've tried to show that Kent's theory is a one-world theory which makes the same predictions as standard quantum theory and is consistent with special relativity (subject to a minor alteration which allows us to attribute PI to Kent's theory). There are however a number of issues that might still make people reluctant to take Kent's theory seriously. For instance, it may strike many people as wildly speculative to suggest that the current state of reality is specified in terms of expectation values of stress-momentum operators conditioned on some notional measurement made in the far distant future. Also, many people, myself included, do not find determinism philosophically attractive, yet on the face of it, Kent's theory does seem to be deterministic. It also looks like Kent's theory relies on there being backwards in time causality, an idea which to many people will seem just as absurd as the many-worlds interpretation. In this final section, I will discuss these concerns and how they might be addressed.

To begin with, let's consider the criticism that Kent's theory is too wildly speculative for anyone to take seriously. In response to this criticism, I think the best way to see Kent's theory is not as a claim of how physical reality must be, but more like a thought experiment that opens up the logical space of how physical reality might be if one accepts special relativity and standard quantum theory. For instance, the Colbeck-Renner theorem might lead one to conclude that there are no interesting hidden variables theories that make the same predictions as quantum theory and satisfy PI. However, as I've tried to show in section \ref{colbeckrennerthm}, the information additional to the quantum state that is contained in a distant future mass-energy density measurement is not constrained by the underlying assumptions of Colbeck, Renner, et al., and hence their conclusion about the redundancy of hidden variables does not apply to Kent's theory. 

Another assumption that is often made about hidden variables theories, but which Kent's theory calls into question is the assumption that a hidden variables theory requires there to be new laws of physics beyond standard quantum theory.  Maudlin prefers to speak of \emph{additional variables theories}\index{additional variables theories} rather than of hidden variables theories: according to Maudlin, an additional variables theory is just a theory in which there is more to physical reality than a single quantum state describing the universe. However, Maudlin goes on to claim that if one adopts  an additional variables theory, one must not only specify what the additional variables are, but one must also state the laws governing them.\footnote{see \cite[9]{MAUDLINT1995Tmp}. } It is as though Maudlin  imagines that the additional variables must be dynamical quantities like particle positions that need laws of motion to describe how they evolve overtime. Now Kent's theory is an additional variables theory according to Maudlin's criterion. However, just because Kent's theory is an additional variables theory, it doesn't follow that Kent is proposing some radically new physical theory which specifies new laws of motion: standard quantum theory will do. Kent's theory does need the Born Rule in order to specify the probability with which the mass-energy density measurement $\tau_S$ is selected on the distant future hypersurface $S$, but the Born Rule is just a part of standard quantum theory. No new physics is required to describe the additional variables (i.e. the values of $\tau_S$ over $S$)  of Kent's theory since the mass-energy density over $S$ has just as much right to be described by a quantum state as the initial state  of the universe $\ket*{\Psi_0}$ has. Also, once the mass-energy density $\tau_S$ on the hypersurface $S$ has been selected, $\tau_S$ doesn't change, and so it makes no sense to ask by what laws the mass-energy density evolves on this hypersurface. So insofar as Kent's additional variables don't require any new physics to describe them, his theory is far less speculative than many other extensions to standard quantum theory. 

Nevertheless, the manner in which one might use the additional variables of Kent's theory is something one could speculate about, and it is not obvious that Kent's proposal is the only way or the most natural way of using these additional variables; that is to say, it is not obvious that expectation values of stress-energy tensors conditioned on a future mass-energy density measurement are the most natural candidates for beables that determine the state of physical reality.  

As mentioned on page \pageref{beabledef}, Bell introduced the term beable due to his dissatisfaction with standard quantum theory which is only a theory of observables -- a satisfactory theory shouldn't just tell us how a physical system appears to be, but it should also tell us how the physical system actually is. In other words, a satisfactory physical theory should describe beables which encapsulate the physical system's state of actuality and which can account for why a system appears to be the way it is. 

Now in Kent's theory, there are two kinds of beables. Firstly, there are the beables corresponding to the mass-energy density measurement on $S$. The mass-energy density measurement $\tau_S$ tells us how each spacetime location of $S$ actually is. Secondly, there are the beables for all the spacetime locations prior to $S$ and after $S_0$. According to Kent, for a spacetime location $y$ prior to $S$ and after $S_0$, the conditional expectation value  $\ev*{\hat{T}^{\mu\nu}(y)}_{\tau_S}$ for all the different combinations of $\mu$ and $\nu$ tells us how the state of physical reality at spacetime location $y$ actually is. 

Now it seems that one can accept the first kind of beable in Kent's theory without accepting the second kind. In some situations, the definite values of $\tau_S(x)$ on $S$ (the first kind of beable) will give rise to $\hat{T}^{\mu\nu}(y)$-eigenstates at $y$, in which case the conditional expectation $\ev*{\hat{T}^{\mu\nu}(y)}_{\tau_S}$ (the second kind of beable) will be identical to a definite value for $T^{\mu\nu}(y)$ as normally understood in standard quantum theory. But there will inevitably be situations in which the definite values of $\tau_S(x)$ on $S$ will not give rise to  $\hat{T}^{\mu\nu}(y)$-eigenstates at $y$. In other words, in the notation defined in section \ref{LorentzInvariance} on page \pageref{tauprojection}, we will inevitably find that the state $\pi_n\ket{\Psi_n}$ at $y$ is in a superposition of eigenstates of $\hat{T}^{\mu\nu}(y)$ for some choice of $\mu$ and $\nu$. But in that case, it seems rather dubious to claim that the expectation value $\ev*{\hat{T}^{\mu\nu}(y)}_{\tau_S}$ is the true state of reality and hence the beable at $y$. As an analogy, it seems a bit like saying that if the throw of a six-sided dice was described quantum mechanically, then the dice could yield a beable of 3.5 for its outcome since 3.5 is the expectation value for the throw of a six-sided dice. In Kent's theory, there would of course nearly always be sufficient information in $\tau_S$ on the hypersurface $S$  to determine that a real dice yielded an integer outcome between 1 and 6.  But it is still the case that the information in $\tau_S$ outside the light cone of $y$ will be insufficient to determine all possible measurable quantities at $y$ to have definite values, and so it seems unnecessary to insist that Kent's conditioned expectation values of all the observables corresponding to these measurable quantities must give the state of physical reality at $y$. 

It might be more reasonable just to say that there is no fact of the matter about the definite value of some measurable quantities in cases where the information in $\tau_S$ is insufficient to determine them.  In the context of Kent's theory, one could still suppose that there were beables corresponding to the mass-energy density $\tau_S(x)$ on $S$, and that some definite facts about physical reality flowed from the definite facts about $S$. But there could also be degrees of indefiniteness about physical reality. This would occur whenever the information in $\tau_S(x)$ was insufficient to determine whether the state $\pi_n\ket{\Psi_n}$ was an eigenstate of some observable. If light were to interact with the location $y$ in a different way, then this might be able to settle the question of which eigenstate $\pi_n\ket{\Psi_n}$ was in, but this definite outcome would then result in $y$ being indefinite with respect to other observables. But even though there would inevitably be many physical quantities of a system that lacked definiteness, we could typically expect the state $\pi_n\ket{\Psi_n}$ to be very nearly an eigenstate for these indefinite physical quantities. For instance, there might be real numbers $t^{\mu\nu}(y)$ %
\nomenclature{$t^{\mu\nu}(y)$}{Approximate eigenvalues of  $\hat{T}^{\mu\nu}(y)$ for all $\mu, \nu$, \nomrefpage}%
 for all $\mu, \nu$ such that 
\begin{equation}\label{tapproxmunuy}
\hat{T}^{\mu\nu}(y)\pi_n\ket{\Psi_n}\approx t^{\mu\nu}(y)\pi_n\ket{\Psi_n}.
\end{equation}
For some values of $\mu$ and $\nu$, we might even obtain equality in (\ref{tapproxmunuy}), but it is not going to be possible to obtain equality in (\ref{tapproxmunuy}) for every value of $\mu$ and $\nu$ -- there must be cases in which $\hat{T}^{\mu\nu}(y)\pi_n\ket{\Psi_n}$ and $t^{\mu\nu}(y)\pi_n\ket{\Psi_n}$ are not quite parallel, for otherwise all the $\hat{T}^{\mu\nu}(y)$  acting on $\pi_n\ket{\Psi_n}$ would commute, and this is not possible. 

The suggestion that there are degrees of indefiniteness to measurable quantities is not going to appeal to everyone, especially people who are seeking a perfect mathematical account of physical reality. But in a context in which we use the information of $\tau_S$ to calculate the state $\pi_n\ket{\Psi_n}$ of a hypersurface $S_n$, the degrees of indefiniteness are going to be very small, so for instance, we don't need to worry about whether a cat could be in a superposition of dead and alive states. The values of $t^{\mu\nu}(y)$ in the vicinity of a cat for which (\ref{tapproxmunuy}) is a very good approximation are going to correspond either to the cat being dead or the cat being alive -- these values won't correspond to the cat being both dead and alive.

Now if one is unwilling to accept Kent's proposal that the expectation values $\ev*{\hat{T}^{\mu\nu}(y)}_{\tau_S}$ are beables, there is still the question of whether there are any better alternatives. In answering this question, I will  draw some inspiration from Maudlin's views on entanglement where he says ``The physical state of a complex whole cannot always be reduced to those of its parts, or to those of its parts together with their spatiotemporal relations \ldots. The result of the most intensive scientific investigations in history is a theory that contains an ineliminable holism.''\footnote{See \cite[56]{Maudlin2}.}  

With a view to defending monism, Schaffer also argues that entangled systems constitute irreducible wholes.\footnote{See \cite{SchafferJonathan2010MTPo}.} Schaffer's argument relies on the principle that ``the basic entities must be \emph{complete}\index{completeness of basic entities}, in the sense of providing a blueprint for reality. More precisely, a plurality of entities is complete if and only if duplicating all these entities, while preserving their fundamental relations, metaphysically suffices to duplicate the cosmos and its contents.''\footnote{\cite[39]{SchafferJonathan2010MTPo}.} Now in the case of entangled systems, duplicating the particles of the system and their spatiotemporal relations will not be sufficient to duplicate the system and its content. Supplementing the particles with entanglement relations doesn't help either in our efforts to duplicate the system because an entangled system will typically have intrinsic properties such as spin which are independent of the number of particles in a system. For instance, both a complex system like a silver atom and a simple system like an electron have an identical spin of 1/2. But if the spin of an entangled system is a certain kind of entanglement relation between the system's component parts, then it wouldn't be possible to say that a complex system like a silver atom and a simple system like an electron had the same spin. Schaffer thus concludes that an entangled system is a fundamental unit, and that the basic entities in any complete description of the cosmos cannot be components of entangled systems.\footnote{\cite[54]{SchafferJonathan2010MTPo}.}  

Schaffer also argues that in the absence of any principle that promotes disentanglement, there will only be one entangled system for the whole cosmos. This is because according to our current understanding of physics, the universe began with a Big Bang which would have resulted in every component of the universe getting entangled with every other component.\footnote{\cite[52]{SchafferJonathan2010MTPo}.}  

Now we can apply much of what Maudlin and Schaffer say about entanglement to posit alternative beables to the ones Kent suggests without having to embrace Schaffer's monism. The alternative I'm suggesting is to take  the beables to be the states of the entangled subsystems of $S_n$ that make up the components of $\pi_n\ket*{\Psi_n}$. In other words, I'm suggesting we take the beables to be the conditioned quantum states as defined in section \ref{KentconsistentQT}. 

To explain this idea in more detail, it will be convenient to change our notation. Instead of considering a series of hypersurfaces $S_n$ as defined in section \ref{kentcalculation} on page \pageref{siydef}, we just consider a single generic hypersurface $R$ %
\nomenclature{$R$}{a generic hypersurface which intersects $S$ and in which $R\setminus S$ is a bounded region that is prior to $S$, \nomrefpage}%
which intersects $S$ and in which $R\setminus S$ is a bounded region that is prior to $S$. We also let $\ket*{\Psi_R}=U_{RS_0}\ket*{\Psi_0}$, %
\nomenclature{$\ket*{\Psi_R}$}{the state given by $U_{RS_0}\ket*{\Psi_0}$ that describes the hypersurface $R$ , \nomrefpage}%
and we let $\pi_R$ %
\nomenclature{$\pi_R$}{the projection analogous to $\pi_n$ for the hypersurface $R$ rather than the hypersurface $S_n$, \nomrefpage}%
be the projection analogous to the projection  $\pi_n$ for the hypersurface $R$ rather than the hypersurface $S_n$ as defined in equation (\ref{tauprojection}). Then if we can express $\pi_R\ket*{\Psi_R}$ as a product 
\begin{equation}\label{beablesprod}
\pi_R\ket*{\Psi_R}=b\prod_i \ket*{\Phi_{Ri}}
\end{equation}%
\nomenclature{$\ket*{\Phi_{Ri}}$}{entangled component of $\pi_R\ket*{\Psi_R}$ and beable candidate\nomrefeq, \nomrefpage}%
where the states $\ket*{\Phi_{Ri}}$ cannot be further decomposed into products of states, and where the regions $R_{i}$ %
\nomenclature{$R_{i}$}{Disjoint region of the hypersurface $R$ which  $\ket*{\Phi_{Ri}}$ describes and such that $R=\bigcup_i R_{i}$, \nomrefpage}%
that these states describe are disjoint regions of $R$ with $R=\bigcup_i R_{i}$, then the states $\ket*{\Phi_{Ri}}$ will be the beables of the cosmos.

In making the beables relative to a hypersurface, I am making explicit an assumption that is implicit in Schaffer's definition of completeness. The cosmos that can be duplicated by duplicating all the basic entities is duplicated at a particular time, or more generally on a particular hypersurface. It doesn't seem any more problematic to consider all the beables on a particular hypersurface than it does to consider all the beables at a particular time. But if we want to avoid monism and maximize\footnote{I will discuss below why we would want to maximize the number of beables.} the number of beables in a spacelike region  $R_i$, then $R_i$ must be embedded in a hypersurface $R$ that intersects as much of $S$ as possible. If we were just to consider the state $\ket*{\Psi_R}$ of $R$, then as Schaffer suggests, we have good reason to expect there to be only one entangled system over the whole of $R$. However, if we deem the mass-energy density on $S\cap R$ to be given by $\tau_S$, then this will correspond to applying the projection $\pi_R$ to $\ket*{\Psi_R}$, and as indicated by  the transition from equation (\ref{Psinidecomp}) to equation (\ref{piphi}), $\pi_R\ket*{\Psi_R}$ can be approximated as a product of disentangled states. The possibility that the decomposition (\ref{beablesprod}) might only be approximate rather than exact could be a slight worry, but I'm assuming that the error would be so small as not to make any discernible difference to what any intelligent beings living in the cosmos might observe.\footnote{If we were to speculate further on how an equality in (\ref{beablesprod}) might come about, one possibility might be to suppose that the universe was placed in a huge box with periodic boundary conditions. In this case, energy levels would be quantized and so it would be more plausible for  $\pi_R\ket*{\Psi_R}$ to be expressed exactly as a product of disentangled states.} 

There is also the question of why we might want to choose a hypersurface which maximizes the number of beables. The reason for wanting to do this is that it allows us to conceive of physical reality in a way that is maximally analyzable. When we can express the state $\ket*{\Psi_R}$ as a product of states $\ket*{\Phi_{Ri}}$ as in (\ref{beablesprod}),  we can then consider the properties of the spacelike region $R_i$ that $\ket*{\Phi_{Ri}}$ describes independently of the properties of  spacelike regions $R_{i'}$ that the $\ket*{\Phi_{Ri'}}$-states describe for $i'\neq i$. To see why this is so, suppose $\hat{O}_i$ is an observable that corresponds to the statement $o_i$ that a measurable property of $R_{i}$ lies within a certain range,\footnote{i.e. $\hat{O}_i$ is a Hermitian operator acting on the Hilbert space of states for $R_i$ such that the eigenvalues of $\hat{O}_i$ are either $1$ or $0$, where the eigenstates with eigenvalue $1$ correspond to states of $R_i$ for which the statement $O_i$ is true, and  where the eigenstates with eigenvalue $0$ correspond to states of $R_i$ for which the statement $O_i$ is false, that is, the measurable property of $R_i$ lies within some other range.} and likewise, suppose $\hat{O}_{i'}$ is an observable that corresponds to the statement $o_{i'}$ that a measurable property of $R_{i'}$ lies within a certain range. Then if we write $P(q)$ for the probability $q$ is true, where $q$ is either the statement $o_i$, $o_{i'}$, or $o_i\, \&\, o_{i'}$, and if we assume $i\neq i'$, then from (\ref{beablesprod}) and (\ref{kentevo}) we get
\begin{equation}
\begin{split}
P(o_i\, \&\, o_{i'})&=\frac{\ev*{\pi_R\hat{O}_i\hat{O}_{i'}}{\Psi_R}}{\ev*{\pi_R}{\Psi_R}}=\bra*{\Phi_{Ri'}}\ev*{\hat{O}_i\hat{O}_{i'}}{\Phi_{Ri}}\ket*{\Phi_{Ri'}}\\
&=\ev*{\hat{O}_i}{\Phi_{Ri}}\ev*{\hat{O}_{i'}}{\Phi_{Ri'}}=P(o_i)P(o_{i'}).
\end{split}
\end{equation}
Thus, the properties of $R_i$ and $R_{i'}$ will be statistically independent of one another.


Another issue that might be of concern to some people (though not to others) is that Kent's theory is deterministic. In other words, given an initial state $\ket*{\Psi_0}$ of $S_0$ and a final mass-energy density measurement $\tau_S$ on $S$, the stress energy tensors for every spacetime location between $S_0$ and $S$ are completely determined according to Kent's theory. There is however a way in which one could reintroduce some indeterminism into Kent's theory if one so desired. For Kent's theory says nothing about how the final mass-energy density measurement $\tau_S$ on $S$ is selected apart from the requirement that it is selected with a probability given by the Born Rule. There could, however, be many possible choices for $\tau_S$ consistent with this requirement. 

With this thought in mind,  we can see how Kent's theory could be viewed in a non-deterministic manner by considering a beable $\ket*{\Phi_{Ri}}$ which describes some region $R_i$ belonging to a hypersurface $R$, where $R\setminus S$ is a finite region that contains $R_i$. The $\ket*{\Phi_{Ri}}$-beable will be determined by $\ket*{\Psi_0}$, the hypersurface $R$, and the mass-energy density measurement on $S\cap R$ with corresponding state denoted by $\ket*{\Gamma}$; that is to say, $\ket*{\Gamma}$  will be a simultaneous $\hat{T}_S$-eigenstate of the region $S\cap R\subset S$ with eigenvalue $\tau_S(x)$ for all $x\in S\cap R$. But since $\ket*{\Gamma}$ only specifies the mass-energy density for the region  $S\cap R$ which lies outside the light cone of $R_i$, there can still be many possible mass-energy density measurements within the light cone of $R_i$ consistent with the Born Rule and the mass-energy density measurement being $\tau_S(x)$ for $x\in S\cap R$. Hence, there can be many possible futures for the $\ket*{\Phi_{Ri}}$-beable given $\ket*{\Gamma}$. For example, we can imagine a scenario in which some photons belonging to $R$ and described by the state $\ket*{\gamma}$ are approaching the region $R_i$. We further suppose that there is a hypersurface $R'$ lying between the hypersurfaces $R$ and $S$ such that as the universal quantum state $\ket*{\Psi_{R}}=U_{RS_0}\ket*{\Psi_0}$ evolves to $\ket*{\Psi_{{R'}}}=U_{{R'}S_0}\ket*{\Psi_0}$, the photons interact with the $\ket*{\Phi_{Ri}}$-beable and subsequently intersect $S\cap {R'}$. We can thus depict this interaction as
\begin{equation}\label{psirtopsirprime}
\begin{split}
\ket*{\Psi_{R}}&= b\ket*{\Phi_{Ri}}\ket*{\gamma}\ket*{\Gamma}\ket*{\xi}+c\ket*{\Xi}\\
&\rightarrow \ket*{\Psi_{{R'}}}\\
&=b\sum_j \lambda_j \ket*{\Phi_{R'i, j}}\ket*{\gamma_{j}'}\ket*{\Gamma}\ket*{\xi'}+c\ket*{\Xi'}
\end{split}
\end{equation}
where $\pi_R\ket*{\Phi_{Ri}}\ket*{\gamma}\ket*{\Gamma}\ket*{\xi}=\ket*{\Phi_{Ri}}\ket*{\gamma}\ket*{\Gamma}\ket*{\xi}$,\footnote{This equation is due to  $\ket*{\Gamma}$ being a simultaneous $\hat{T}_S$-eigenstate of the region $S\cap R$ with eigenvalue $\tau_S(x)$ for all $x\in S\cap R$.}  $\ket*{\Xi}$ is a state that describes the whole of $R$ for which $\pi_R\ket*{\Xi}=0$,\footnote{This will ensure that $\pi_R\ket*{\Psi_{R}}=b\ket*{\Phi_{Ri}}\ket*{\gamma}\ket*{\Gamma}\ket*{\xi}$ so that
$\ket*{\Phi_{Ri}}$ will be a beable on $R$.}  $\ket*{\Xi'}=U_{R'R}\ket*{\Xi}$,  $\ket*{\xi}$ describes all the regions of $R$ 
not described by $\ket*{\Phi_{Ri}}$, $\ket*{\gamma}$, or $\ket*{\Gamma}$, and where $b$, $c$, and $\lambda_j$ are normalization factors with $|b|^2+|c|^2=1$ and $\sum_j |\lambda_j|^2=1$ so that $\ket*{\Psi_{R}}$ and $\ket*{\Psi_{R'}}$ are normalized to $1$. 
For simplicity, we also assume that the beables contained within $\ket*{\xi}$ do not become entangled with the $\ket*{\Phi_{Ri}}$-beable, so that under the action of $U_{R'R}$, the $\ket*{\xi}$-component of $\ket*{\Psi_{R}}$ evolves to the $\ket*{\xi'}$-component of $\ket*{\Psi_{R'}}$. We only assume that the beables $\ket*{\gamma}$ and $\ket*{\Phi_{Ri}}$ become entangled as $\ket*{\Psi_{R}}$ evolves to $\ket*{\Psi_{R'}}$ so that  that the $\ket*{\Phi_{R'i, j}}$-component of $\ket*{\Psi_{R'}}$ is correlated with the $\ket*{\gamma_{j}'}$-component of $\ket*{\Psi_{R'}}$ as seen on the final line of (\ref{psirtopsirprime}). Again for simplicity, we assume that the $\ket*{\gamma_{j}'}$ are mutually orthogonal simultaneous $\hat{T}_S$-eigenstates\footnote{In more realistic models where we don't treat photons as point particles, we would only expect $\ip{\gamma_{j}'}{\gamma_{l}'}\approx 0$ for $j\neq l$, and we wouldn't expect the $\ket*{\gamma_{j}'}$ to be simultaneous $\hat{T}_S$-eigenstates on a region of $S\cap R'$  due to the spreading of light.} which  describe a region  $S_{\gamma'}\subset(S\cap R')\setminus R$ which includes all the possible regions the photons initially in the state $\ket*{\gamma}$ can end up after being reflected from the $\ket*{\Phi_{Ri}}$-beable. Now among the $\ket*{\gamma_{j}'}$, there should be one of them, $\ket*{\gamma_{j_0}'}$ say, such that the simultaneous $\hat{T}_S$-eigenstate of $\ket*{\gamma_{j_0}'}$ is given by $\tau_S(x)$ for all $x\in S_{\gamma'}$. Then if $\pi_{R'}$ is the projection on $H_{R',\tau_S}$ as defined in equation (\ref{tauprojection}), we will have 
$$\pi_{R'}\ket*{\Psi_{R'}}=b\lambda_{j_0}\ket*{\Phi_{R'i,j_0}}\ket*{\gamma_{j_0}'}\ket*{\Gamma}\ket*{\xi'}.$$
In other words, the $\ket*{\Phi_{Ri}}$-beable transitions to the $\ket*{\Phi_{R'i,j_0}}$-beable on $R'$ when the available information in the mass-energy density $\tau_S$ is used. However, given $\ket*{\Gamma}$, there could have been a different mass-energy density measurement on $S_{\gamma'}$ with corresponding simultaneous $\hat{T}_S$-eigenstate $\ket*{\gamma_{j_1}'}$, say, in which case, we would have had the $\ket*{\Phi_{Ri}}$-beable transitioning to the $\ket*{\Phi_{R'i,j_1}}$-beable on $R'$ instead.
So in the sense that given $\ket*{\Gamma}$, there are many possible futures for the $\ket*{\Phi_{Ri}}$-beable on $R'$,  Kent's theory is not deterministic.


One may however object to viewing Kent's theory in this non-deterministic manner because it now looks as though the distant future event of selecting the mass-energy density measurement $\tau_S$  on $S_{\gamma'}$ affects which beable $\ket*{\Phi_{Ri}}$ transitions to, and so it could seem that backwards in time causality is at play, a possibility that may strike many people as absurd. Such people may be more comfortable with the idea that the mass-energy density measurement $\tau_S$ on the whole of $S$ is selected before the state $\ket*{\Psi_0}$ has started to evolve, so that when we speak of a mass-energy density measurement on $S$, we are not really speaking of some future act of determination; rather we are speaking of some primordial act of determination. This primordial act of determining $\tau_S$ would then deterministically affect how the beables evolve so as to guarantee consistency with the mass-energy density on $S$ having the value $\tau_S$ that was primordially selected.

One way to address this objection is to re-imagine what we mean by time. Although in contemporary physics, time is thought of as just one of four parameters in four-dimensional spacetime, this is not the only way people have conceived of time. For instance, according to Aristotle, ``time is just this -- number of motion in respect of `before' and `after'.''\footnote{\cite[ Book IV. Chapter 11 (219b 1-5)]{AristotlePhysics}.} This Aristotelian definition of time presupposes an Aristotelian definition of motion which Aristotle defines as follows: ``The fulfilment of what exists potentially, in so far as it exists potentially, is motion.''\footnote{\cite[ Book III. Chapter 1 (201a10)]{AristotlePhysics}.} So according to Aristotle, time presupposes that there are degrees of potentiality and degrees of actuality -- before the motion, the state of fulfilment of something is potential, whereas after the motion, the state of fulfilment is actual. This means that for there to be a duration of time, something potential must be actualized. Now when one considers the transition from the state $\ket*{\Psi_0}$ to the state  $\ket*{\Psi_S}$, it is not obvious that this is a case of some potency being actualized, for $\ket*{\Psi_0}$ is a state of one thing, the hypersurface $S_0$,  whereas $\ket*{\Psi_S}$ is the state of something else, the hypersurface $S$. It is not as though we must assume the state $\ket*{\Psi_0}$ becomes the state $\ket*{\Psi_S}$ or that the hypersurface $S_0$ becomes the hypersurface $S$. Without any further assumptions, a particular state or a particular hypersurface is no more potential or actual than any other state or hypersurface. 

We could however think of the determination of $\tau_S$ on $S$ as coming about via a process of actualization. As an analogy, one could think of the determination of $\tau_S$ as being like the weaving of a tapestry. The color at a particular location of the tapestry isn't determined until the colored thread is weaved into place. It's also the case that some areas of the tapestry must be completed before other areas, though there will also be some freedom as to which area is completed first. So looking at the completed tapestry, there will typically be some parts in which we can discern the order in which the colors were added, whereas there will be other parts in which we can't discern the order of completion. 

In an analogous fashion, we might be able to discern a partial order on $S$ on the basis of whether one region of $S$ necessarily had its mass-energy density determined before another region. We wouldn't expect this order to be complete, so for any two regions, we might not be able to say which came first. But there may nevertheless be a sufficient degree of ordering on $S$ to suggest a (non-universal) notion of time similar to that of Aristotle's notion where time is defined in terms of the numbering of motion in respect of before or after. 

To consider how we might work out this temporal ordering on $S$, we will work with a toy model in which photons of light are treated as point particles, and in which the light detected on $S$ can always be traced back to the object prior to $S$ off which the light was reflected. We suppose $R$ is a hypersurface, and that there is a simultaneous $\hat{T}_S$-eigenstate $\ket*{\Gamma}$ for a region $S_{\Gamma}$ of $S\cap R$ with simultaneous $\hat{T}_S$-eigenvalue $\tau_S(x)$ for all $x\in S_{\Gamma}$ such that the mass-energy density being $\tau_S(x)$ for all $x\in S_\Gamma$ guarantees that a system $\mathcal{S}$ exists in a region $R_{\mathcal{S}}\subset R$, and we suppose that given the inital state $\ket*{\Psi_0}$, the mass-energy density has the value $\tau_S(x)$ for all $x\in S_\Gamma$ with probability $|b|^2$ for some complex number $b$. Furthermore, given $\ket*{\Gamma}$, we suppose that  $\mathcal{S}$ can be described by a beable $\ket*{\Phi_{R, j_0}}$ from a set of possible\footnote{that is possible with respect to $\ket*{\Psi_0}$ and $\ket*{\Gamma}$.} beables $\{ \ket*{\Phi_{R, j}}: j\in \mathbb{N}\}$, where each of these beables is a state of the region $R_{\mathcal{S}}$, and where the probability $\mathcal{S}$ could be in the state $\ket*{\Phi_{R, j}}$ given $\ket*{\Gamma}$ is $|\lambda_j|^2$ for complex numbers $\lambda_j$ with $\sum_j|\lambda_j|^2=1$. We also assume that each $\ket*{\Phi_{R, j}}$-beable is correlated with a photon state $\ket*{\gamma_{j}}$ for a region $S_{\gamma}\subset R\cap S$, where $\ip*{\gamma_{j}}{\gamma_{j'}}$ for $j\neq j'$. We can then express the state $\ket*{\Psi_R}=U_{RS_0}\ket*{\Psi_0}$ as 
\begin{equation}
\ket*{\Psi_R}=b\ket*{\Gamma}\sum_j \lambda_j \ket*{\gamma_{j}}\ket*{\Phi_{R, j}}\ket*{\xi}+c\ket*{\Xi}
\end{equation}
where $\pi_R\ket*{\Gamma}\ket*{\gamma_{j_0}}\ket*{\Phi_{R,j_0}}\ket*{\xi}=\ket*{\Gamma}\ket*{\gamma_{j_0}}\ket*{\Phi_{R,j_0}}\ket*{\xi}$,  
$\pi_R\ket*{\Gamma}\ket*{\gamma_{j}}\ket*{\Phi_{R,j}}\ket*{\xi}=0$
for $\ket*{\gamma_{j}}\neq \ket*{\gamma_{j_0}}$, 
$\ket*{\Xi}$ is  a state that describes the whole of $R$ for which $\pi_R\ket*{\Xi}=0$, $\ket*{\xi}$ describes all the regions of $R$ not described by the $\ket*{\Phi_{R,j}}$, the $\ket*{\gamma_{j}}$ or by $\ket*{\Gamma}$, and where $c$ is a complex number such that $|b|^2+|c|^2=1$.

We also suppose that there is a hypersurface $R'$ lying between $R$ and $S$, and that as the state $\ket*{\Psi_R}$ evolves to $\ket*{\Psi_{R'}}$ under the action of $U_{R'R}$, photons described by a state $\ket*{\gamma_R}$ (which is a component of $\ket*{\xi}$ so that $\ket*{\xi}=\ket*{\gamma_R}\ket*{\eta}$ for some state $\ket*{\eta}$) become entangled with the system $\mathcal{S}$ so that
\begin{equation}
\ket*{\Psi_{R'}}=b\ket*{\Gamma}\sum_j \lambda_j \ket*{\gamma_{j}} \sum_k\lambda_{jk} \ket*{\Phi_{R', jk}}\ket*{\gamma_{jk}'}\ket*{\eta'}+c\ket*{\Xi'}
\end{equation}
where $\ket*{\gamma_{jk}'}$ describe the photon states for a region $S_{\gamma'}\subset R'\cap S$ that have become entangled with $\mathcal{S}$ and which satisfy  $\ip*{\gamma_{jk}'}{\gamma_{j'k'}'}=0$ for either $j\neq j'$ or $k\neq k'$. The complex numbers $\lambda_{jk}$ satisfy $\sum_k|\lambda_{jk}|^2=1$ for each $k$, and $\ket*{\Xi'}$ is the state that evolves from $\ket*{\Xi}$ so that $\ket*{\Xi'}=U_{R'R}\ket*{\Xi}.$ Finally, we suppose that the value of $\tau_S(x)$ for $x\in S_{\gamma'}$ corresponds to the simultaneous $\hat{T}_S$-eigenstate $\ket*{\gamma_{j_0k_0}'}$, so that
$$\pi_{R'}\ket*{\Psi_{R'}}= b\ket*{\Gamma} \lambda_{j_0} \ket*{\gamma_{j_0}} \lambda_{j_0k_0} \ket*{\Phi_{R', j_0k_0}}\ket*{\gamma_{j_0k_0}'}\ket*{\eta'}$$
where $\pi_{R'}$ is the projection corresponding to the measurement $\tau_S(x)$ for $x\in R'\cap S$ analogous to the projection defined in equation (\ref{tauprojection}).

Now given all these assumptions, we see that the state of $S_{\gamma}'$ being $\ket*{\gamma_{j_0k_0}'}$ implies that the state of $S_{\gamma}$ is $\ket*{\gamma_{j_0}}.$ On the other hand, the state of $S_{\gamma}$ being $\ket*{\gamma_{j_0}}$ does not imply the state of $S_{\gamma}'$ is $\ket*{\gamma_{j_0k_0}'}$. There is therefore an asymmetry in the order of determination of $\tau_S$ whereby $S_{\gamma}$ has to be determined before $S_{\gamma}'$ is determined, and this asymmetry corresponds to the evolution of the system $\mathcal{S}$ whereby it transitions from the state $\ket*{\Phi_{R,j_0}}$ to the state $\ket*{\Phi_{R',j_0k_0}}$, and not vice versa. This ordering also presupposes the state $\ket*{\Gamma}$ has been determined, for without this presupposition, the regions  $S_{\gamma}$ and $S_{\gamma}'$ might describe a completely different system in which the order of determination was reversed. We therefore ought to include $\ket*{\Gamma}$ when specifying the order of determination so that we can say $\ket*{\Gamma}\ket*{\gamma_{j_0k_0}'}$ comes after $\ket*{\Gamma}\ket*{\gamma_{j_0}}$ but not vice versa. If we let $\ket*{\gamma}=\ket*{\Gamma}\ket*{\gamma_{j_0}}$ and $\ket*{\gamma'}=\ket*{\Gamma}\ket*{\gamma_{j_0k_0}'}$, then the following statement holds
$$\big((\ket{\gamma'}\Rightarrow\ket{\gamma})\,\& \,(\ket{\gamma}\not\Rightarrow\ket{\gamma'})\big).$$
This suggests we define a partial order $>$ 
on the states of regions of $S$ according to the rule
\begin{equation}\label{timeordering}
\ket{\gamma'} > \ket{\gamma}
\Longleftrightarrow 
\big((\ket{\gamma'}\Rightarrow\ket{\gamma})\,\& \,(\ket{\gamma}\not\Rightarrow\ket{\gamma'})\big).
\end{equation}%
\nomenclature{$>$}{a partial order on states of regions of $S$ expressing the notion `occurs after'\nomrefeqpage}%
where $\ket*{\gamma}$ and $\ket*{\gamma'}$ are any two simultaneous $\hat{T}_S$-eigenstates for two (possibly overlapping) regions $S_\gamma$ and $S_{\gamma'}$ of $S$. When $\ket{\gamma'} > \ket{\gamma}$ we can think of this in a temporal sense and speak of $\ket{\gamma'}$ as occurring after $\ket{\gamma}$.

It is also possible to extend this partial order to states of spacelike regions in between $S_0$ and $S$. We do this by first deeming a state $\ket*{\phi}$ of a spacelike region $R_\phi$ to be simultaneous with $\ket*{\gamma}$ if there exists a hypersurface $R$ with $R_\phi\subset R$ and $S_\gamma \subset R\cap S$ such that 
$$\pi_R\ket*{\Psi_R}=b\ket*{\phi}\ket*{\gamma}\ket*{\xi}$$ 
for some complex number $b$ and state $\ket*{\xi}$ of $R\setminus(R_\phi\cup S_\gamma)$. When $\ket*{\phi}$  is simultaneous with  $\ket*{\gamma}$, we will denote this by writing 
$\ket*{\phi}\sim \ket*{\gamma}$. %
\nomenclature{$\sim$}{we write $\ket*{\phi}\sim \ket*{\gamma}$ to mean that  $\ket*{\phi}$ and $\ket*{\gamma}$ are simultaneous, \nomrefpage}%
Then for two states $\ket*{\phi}$ and $\ket*{\phi'}$, we can define on them a partial order $>$ according to the rule
\begin{equation}
\ket{\phi'}>\ket{\phi}\Longleftrightarrow 
(\ket{\phi'}\sim\ket{\gamma'}) \,\&\, (\ket{\phi}\sim\ket{\gamma})\, \&\, (\ket{\gamma'}>\ket{\gamma}).
\end{equation}
Note that according to this ordering, $\ket*{\Phi_{R', j_0k_0}}>\ket*{\Phi_{R,j_0}}$ so it respects the intuition that the state $\ket*{\Phi_{R', j_0k_0}}$ occurs after the state $\ket*{\Phi_{R,j_0}}$. Also, if we think of time in terms of this ordering rather than the ordering given by the $t$ parameter in four dimensional spacetime, we don't have the problem of backwards in time causality since the determination of $\ket*{\Phi_{R', j_0k_0}}$ and the determination of $\ket*{\Gamma}\ket*{\gamma_{j_0k_0}'}$ will be simultaneous. 

Another advantage of re-imagining what is meant by time is that we don't need to worry about what happens after $S$ or the asymptotic behavior of Kent's theory as later and later hypersurfaces are considered on which to make the notional measurement $\tau_S$. As I've presented Kent's theory so far, it is natural to ask what happens after $S$. It looks like once $\tau_S(x)$ is fixed on $S$, there should be a corresponding simultaneous $\hat{T}_S$-eigenstate $\ket*{\Gamma_{\tau_S}}$ %
\nomenclature{$\ket*{\Gamma_{\tau_S}}$}{simultaneous $\hat{T}_S$-eigenstate of the hypersurface $S$ with simultaneous $\hat{T}_S$-eigenvalue $\tau_S(x)$ where $x$ ranges over $S$, \nomrefpage}%
that we can evolve to an even later hypersurface $S'$ after $S$. But if there are no more notional mass-energy density measurements to condition on, then $U_{S'S}\ket*{\Gamma_{\tau_S}}$ will evolve in a many-worlds fashion when $S'$ is sufficiently far in the future. However, Kent would reply to this concern by emphasizing that the measurement $\tau_S$ on $S$ is only notional -- the measurement on $S$ would have to have been $\tau_S$ given how the history of the universe has actually turned out up to a hypersurface $S_1$ prior to $S$. Kent explains that as long as we restrict our attention to describing physics between two hypersurfaces $S_0$ and $S_1$, we can hope that it won't matter too much which hypersurface $S$ we choose after $S_1$ on which to make the notional measurement of $T_S$ -- given another $S'$ we should be just as likely to choose a measurement outcome for $T_{S'}$ that would produce approximately the same history between $S_0$ and $S_1$.\footnote{See  \cite[3]{Kent2014}. For a discussion of this point, also see footnote \ref{asymptoticfootnote} on page \pageref{asymptoticfootnote}.} But this answer now leaves us asking what happens after the hypersurface $S_1$, since in order to make sense of the asymptotic behavior of Kent's theory, it looks as though we are not allowed to ask about what physical reality is like after $S_1$.  


But with the Aristotelian-inspired notion of time I've been discussing, the question of what happens after $S$ doesn't arise, for the whole of time is understood in terms of the actualization of $S$. With this understanding of time, we can speak of the state $\ket*{\Gamma_{\tau_S}}$ being at the end of time insofar as there is no other state $\ket{\gamma}$  on a region of $S$ with $\ket{\gamma}>\ket*{\Gamma_{\tau_S}}$ -- the state $\ket*{\Gamma_{\tau_S}}$ is like the completed tapestry, so to speak. 

One final point worth considering is why one energy-density measurement on $S$ is selected rather than another. According to Kent's theory, the measurement $\tau_S$ is randomly selected according to the Born Rule. However, it is not obvious that this selection has to be random. Instead, we could conjecture that the state $\ket*{\Psi_S}$ only specifies a realm of possibilities in accordance with the Born Rule, but that there are complementary principles that determine what actually happens within this realm of possibilities. So for example, when the beable $\ket*{\Phi_{R,j_0}}$ transitions to $\ket*{\Phi_{R', j_0k_0}}$ rather than $\ket*{\Phi_{R', j_0k_1}}$, we might suppose it does so not due to pure randomness, but because it is in its nature to do so. It would be an interesting topic for further study to investigate the plausibility of whether there could be such laws of nature that complemented quantum physics rather than contradicted it.



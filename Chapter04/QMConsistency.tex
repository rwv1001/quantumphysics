\section{Consistency of Kent's Theory with Standard Quantum Theory\label{kentinterpretationconsistency}}
\subsection{Overview}
If we are to take Kent's theory seriously, it should be just as good at making predictions as standard quantum theory, and it had better not contradict empirical observations. Over the last century, standard quantum theory has been firmly established, and so far, it has not been contradicted by any experimental observations. Standard quantum theory allows us to form a quantum state description of a physical system based on how the system was set up in an experimental environment, and then Schr\"{o}dinger's equation can be used to evolve this state forwards in time, and finally, we can calculate expectation values for various physical quantities belonging to this physical system, and these agree with the average values measured on the system when the experiment is performed many times. In other words, standard quantum theory is \textbf{empirically adequate}\index{empirical adequacy!informal definition}\footnote{See p. \pageref{adeq} for a more formal definition of empirical adequacy.} in its domain of applicability. Thus, if we can show that Kent's theory is just as good at making predictions as standard quantum theory, then it too will be empirically adequate to the same degree. This doesn't necessarily mean that Kent's theory will make exactly the same predictions as standard quantum theory, for the additional information  beyond standard quantum theory that Kent's theory requires may alter these predictions. Indeed, if this additional information made absolutely no difference to the predictions of standard quantum theory, then it would seem rather redundant. But we should nevertheless be able to derive the predictions of standard quantum theory from Kent's theory by averaging over the unknown variables that describe the additional information in Kent's theory. 

In order to show that Kent's theory is just as good as standard quantum theory, I will consider two cases. In this section I will focus on the predictions Kent's theory makes given the universal state $\ket*{\Psi_0}$. I will postpone until section \ref{KentconsistentQT} a consideration of the predictions Kent's theory makes given a state that is more local and relevant to the kinds of situations that concern experimentalists. 

In this current section I will show that if we calculate Kent's beable $\ev*{T^{\mu\nu}(y)}_{\tau_S}$ for every possible measurement outcome that $\tau_S$ could be,  and if we then take the weighted sum of each of these beables weighted according to how likely $\tau_S$ was to occur given $\ket*{\Psi_0}$, then this weighted sum will give the same value $\ev*{T^{\mu\nu}(y)}$ that standard quantum theory would predict given $\ket*{\Psi_0}$. The intuition is that Kent's theory posits the existence of additional unknown variables, namely those specified by a particular outcome $\tau_S$, and if we had known this information, we would be able to calculate the expectation value of the stress-energy tensor $T^{\mu\nu}(y)$ more accurately than standard quantum theory would allow. However, in reality, we don't know what this extra information is; we can only work out how likely the values constituting this extra information will be based on the state $\ket*{\Psi_0}$ and the use of standard quantum theory. Therefore, the best we can do is to guess what this extra information is going to be and then weight the calculation we do with this extra information according to the likelihood of our guess being correct. But since we believe standard quantum theory is the best theory there is given the information in the quantum state $\ket*{\Psi_0}$, and since we have no information beyond the quantum state when performing this weighted calculation, we wouldn't expect the result of the weighted calculation to be any better than what standard quantum theory predicts. But if we are to take Kent's theory seriously, then Kent's theory shouldn't be any worse than standard quantum theory either. Thus, by showing that the weighted calculation of expectation values that Kent's theory predicts is equal to the expectation values standard quantum theory predicts, we see that Kent's theory is no worse than standard quantum theory.

To show that this is indeed the case, we consider a hypersurface $S_n$ that contains the point $y$ at which we wish to calculate the stress-energy tensor's expectation value. Now, given the universal quantum state $\ket*{\Psi_0}$ describing the initial hypersurface $S_0$, it is possible to work out how this quantum state evolves to a quantum state $\ket*{\Psi_n}$ describing the hypersurface $S_n$. The hypersurface $S_n$ is assumed to intersect the distant future hypersurface $S$ where the notional mass-energy density measurement is made. 

Now we would not expect $\ket*{\Psi_n}$ to describe the hypersurface $S_n$ as having a definite mass-energy density at the locations where $S_n$ intersects $S$. But according to standard quantum theory, we should be able to express $\ket*{\Psi_n}$ as a superposition (that is, a weighted sum) of states for which the mass-energy density on $S\cap S_n$ does have a definite value. In particular, we can express $\ket*{\Psi_n}$ as a sum of two states: the first state which we denote as $\pi_n \ket*{\Psi_n}$ \label{overviewpi}
is the (weighted) sum of all the states in the superposition of $\ket*{\Psi_n}$ for which the mass-energy density has a definite value that is equal to the notional mass-energy density measurement $\tau_S(x)$ for all $x$ in the intersection $S\cap S_n$; and the second state which we denote as $(I- \pi_n) \ket*{\Psi_n}$ is the (weighted) sum of all the other states included in the superposition of $\ket*{\Psi_n}$. Since all the states that contribute to the sum defining $\pi_n \ket*{\Psi_n}$ are mutually exclusive, the probability given $\ket*{\Psi_n}$ (and hence  the probability given $\ket*{\Psi_0}$) that the notional mass-energy density measurement made on $S\cap S_n$ is equal to $\tau_S$ will be equal to the sum of the probabilities of the states that contribute to $\pi_n \ket*{\Psi_n}$. This fact enables us to calculate a simple formula\footnote{See equation (\ref{Prn}) below.} for the probability given $\ket*{\Psi_n}$ that the notional mass-energy density measurement made on $S\cap S_n$ is equal to $\tau_S$. In a similar manner, we can also calculate a simple formula\footnote{See equation (\ref{pqtauri2}) below.} for the probability given $\ket*{\Psi_n}$ that both the notional mass-energy density measurement made on $S\cap S_n$ is equal to $\tau_S$ and that the stress-energy tensor $T^{\mu\nu}(y)$ at the location $y$ takes a particular value $\tau^{\mu\nu}$. %
\nomenclature{$\tau$}{The particular value the stress-energy tensor $T^{\mu\nu}(y)$ takes when a measurement of it is performed at $y\in S_n$, \nomrefpage}%
 From these two formulae, we can calculate the conditional probability that the $T^{\mu\nu}(y)$ takes a particular value $\tau$ given that the notional mass-energy density measurement made on $S\cap S_n$ is equal to $\tau_S$. This in turn enables us to calculate Kent's beable $\ev*{T^{\mu\nu}(y)}_{\tau_S}$ at $y$.\footnote{See equation (\ref{kentconsistency0}) below.} With these formulae, it is then relatively easy to calculate the weighted sum of expectation values that Kent's theory predicts and show that this sum equals the expectation value predicted by standard quantum theory.

\subsection{Technical Details\textsuperscript{*}}
In order to show that Kent's theory is just as good as standard quantum theory given the universal state $\ket*{\Psi_0}$ of the initial hypersurface $S_0$,  recall the preliminary calculation of $\ev*{T^{\mu\nu}(y)}_{\tau_S}$ given in section \ref{kentcalculation}:
\begin{equation}\tag{\ref{beable1} revisited}
	\ev*{T^{\mu\nu}(y)}_{\tau_S}=\sum_\tau P(q(\tau) | r(\tau_S,y))\tau=\lim_{n\rightarrow\infty}\sum_\tau\frac{P(q(\tau) \, \& \, r_n)\tau}{P(r_n)}.
\end{equation}
We proceed to calculate $P(r_n)$ and $P(q(\tau)\, \& \,r_n)$ in order to calculate  $\ev*{T^{\mu\nu}(y)}_{\tau_S}$. We first note that since $S_n$ is a hypersurface,   there will exist a unitary operator $U_{S_nS_0}$ defined by equation (\ref{SchwingerUnitaryOP}) which maps the Hilbert space\footnote{The Hilbert space $H_S$ for any hypersurface $S$ is defined on page \pageref{HSdef}.} of states $H_{S_0}$ describing $S_0$ to the Hilbert space of states $H_{S_n}$\label{HSidef} describing $S_n$ in accord with how the states of $H_{S_0}$  evolve over time. Now let $H_{S_n,\tau_S}\label{HStau}\subset H_{S_n}$ %
\nomenclature{$H_{S_n,\tau_S}$}{The subspace of states $\ket*{\xi}\in  H_{S_n}$ for which  $\hat{T}_S(x)\ket*{\xi}=\tau_S(x)\ket*{\xi}$  for all $x\in S_n\cap S$, \nomrefpage}%
 be the subspace of states $\ket*{\xi}$ for which  $\hat{T}_S(x)\ket*{\xi}=\tau_S(x)\ket*{\xi}$  for all $x\in S_n\cap S$, and  let $\{\ket*{\xi_1},\ket*{\xi_2},\ldots\}$  %
 \nomenclature{$\{\ket*{\xi_j}:j\in\mathbb{N}\}$}{An orthonormal basis of $H_{S_n,\tau_S}$, \nomrefpage}%
 be an orthonormal basis of $H_{S_n,\tau_S}$. Given that the initial state of the world is $\ket*{\Psi_0}$, the probability $P(r_n)$ of ``measuring'' the value of $T_S(x)$ on $S_n\cap S$ to be $\tau_S(x)$ will be 
\begin{equation}\label{Pri}
P(r_n)=\sum_j \abs{ \ip{\xi_j}{\Psi_n}}^2,
\end{equation}
where $\ket*{\Psi_n}=U_{S_nS_0}\ket*{\Psi_0}$, %
\nomenclature{$\ket*{\Psi_n}$}{$\ket*{\Psi_n}=U_{S_nS_0}\ket*{\Psi_0}$, \nomrefpage}%
 and this probability will be independent of the particular orthonormal basis  $\{\ket*{\xi_j}:j\in\mathbb{N}\}$ of $H_{S_n,\tau_S}$.\footnote{To see why this is, we note that we can extend the orthonormal set $\{\ket*{\xi_1},\ket*{\xi_2},\ldots\}$ to an orthonormal basis  $\{\ket*{\xi_1},\ket*{\xi_2},\ldots\}\cup\{\ket*{\zeta_1},\ket*{\zeta_2},\ldots\}$ of $H_{S_n}$ which consists entirely of $\hat{T}_S(x)$-eigenstates for all $x\in S_n\cap S$. We can think of each of the states of this orthonormal basis as the possible measurement outcomes when making the notional measurement of $T_S(x)$ on $S_n\cap S$. By the Born Rule, it therefore follows that $P(r_n)=\sum_j \abs{ \ip{\xi_j}{\Psi_n}}^2$. But to see that this probability is independent of the particular basis, we can uniquely write $\ket*{\Psi_n}$ as a sum $\ket*{\Psi_n}=\ket*{\xi}+\ket*{\zeta}$ where $\ket*{\xi}$ belongs to the span of $\{\ket*{\xi_j}:j\in\mathbb{N}\}$ and $\ket*{\zeta}$ belongs to the span of $\{\ket*{\zeta_j}:j\in\mathbb{N}\}$.  Then since $\ket*{\xi}=\sum_j\ip{\xi_j}{\Psi_n}\ket*{\xi_j}$, it follows that $$\ip{\xi}{\xi}=\sum_j\abs{ \ip{\xi_j}{\Psi_n}}^2=P(r_n).$$ Therefore, since  $\ip{\xi}{\xi}$ is independent of the particular basis chosen of $H_{S_n,\tau_S}$, so is $P(r_n)$.  \label{priproof} } If we define 
\begin{equation}\label{tauprojection}
\pi_n=\sum_j\dyad*{\xi_j},
\end{equation}%
\nomenclature{$\pi_n$}{$\pi_n \ket*{\Psi_n}$ is the (weighted) sum of all the states in the superposition of $\ket*{\Psi_n}$ for which the mass-energy density has a definite value that is equal to the notional mass-energy density measurement $\tau_S(x)$ for all $x\in S\cap S_n$, p. \pageref{overviewpi}, see also equation (\ref{tauprojection}), \nomrefpage}%
then it is easy to see that
\begin{equation}\label{Prn}
	P(r_n)=\ev*{\pi_n}{\Psi_n}.
\end{equation}
We also see that $\pi_n$ is Hermitian (i.e. has real eigenvalues) and that $\pi_n \pi_n = \pi_n$. Any Hermitian operator $\pi$ with $\pi^2=\pi$ is called a \textbf{projection}\index{projection}. We thus see that $\pi_n$ is a projection.

Turning to the calculation of $P(q(\tau)\, \& \, r_n)$, note that for the Tomonaga-Schwinger formulation of relativistic quantum physics, the operators $\hat{T}_S(x)$ and $\hat{T}^{\mu\nu}(y)$ for fixed $\mu,\nu$ commute when $x$ and $y$ are spacelike-separated. It therefore follows that we can express any state of $H_{S_n}$ as a superposition of simultaneous eigenstates of $\hat{T}^{\mu\nu}(y)$ and $\hat{T}_S(x)$ for $x\in S_n\cap S$.\footnote{We make the same approximation as depicted in figure \ref{tauSapprox} on page \pageref{tauSapprox}.}  For a particular choice of $\mu,\nu$, we can then form an orthonormal basis $\{\ket*{\eta_j^{\mu\nu}}:j\in\mathbb{N}\}$ of %
\nomenclature{$\{\ket*{\eta_j^{\mu\nu}}:j\in\mathbb{N}\}$}{An orthonormal basis of $H_{S_n}$ consisting of simultaneous $\hat{T}^{\mu\nu}(y)$, $\hat{T}_S(x)$-eigenstates so that $\hat{T}^{\mu\nu}(y)\ket*{\eta_j^{\mu\nu}}=\tau_{j}\ket*{\eta_j^{\mu\nu}}$ and $\hat{T}_{S}(x)\ket*{\eta_j^{\mu\nu}}=\tau_{S,j}(x)\ket*{\eta_j^{\mu\nu}}$ for $x\in S_n\cap S$, where $\tau_{j}$ and $\tau_{S,j}(x)$ are the corresponding eigenvalues, \nomrefpage}%
 $H_{S_n}$ consisting of simultaneous $\hat{T}^{\mu\nu}(y)$, $\hat{T}_S(x)$-eigenstates so that $\hat{T}^{\mu\nu}(y)\ket*{\eta_j^{\mu\nu}}=\tau_{j}\ket*{\eta_j^{\mu\nu}}$ and $\hat{T}_{S}(x)\ket*{\eta_j^{\mu\nu}}=\tau_{S,j}(x)\ket*{\eta_j^{\mu\nu}}$ for $x\in S_n\cap S$, where $\tau_{j}$ and $\tau_{S,j}(x)$ are the corresponding eigenvalues.\footnote{We include the indices   $\mu,\nu$ in the specification of the basis $\{\ket*{\eta_j^{\mu\nu}}:j\in\mathbb{N}\}$ since in general, we can't assume that $\hat{T}^{\mu\nu}(y)$ and $\hat{T}^{\mu'\nu'}(y)$ commute for $\mu\neq \mu'$ or $\nu\neq \nu'$, and so we have to fix $\mu,\nu$ and consider a basis $\{\ket*{\eta_j^{\mu\nu}}:j\in\mathbb{N}\}$ of $H_n$ which depends on the particular $\mu,\nu$ so that all the elements of this basis can be eigenstates of  $\hat{T}^{\mu\nu}(y)$.} If we define %
 \nomenclature{$\pi_{n,\tau}^{\mu\nu}$}{The projection $\pi_{n,\tau}^{\mu\nu}=\sum_j\dyad*{\chi_{j,\tau}^{\mu\nu}}$, \nomrefpage}%
   $\pi_{n,\tau}^{\mu\nu}=\sum_j\dyad*{\chi_{j,\tau}^{\mu\nu}}$ where $\{\ket*{\chi_{j,\tau}^{\mu\nu}}:j\in\mathbb{N}\}$ %
 \nomenclature{$\{\ket*{\chi_{j,\tau}^{\mu\nu}}:j\in\mathbb{N}\}$}{The subset of $\{\ket*{\eta_j^{\mu\nu}}:j\in\mathbb{N}\}$ such that $\hat{T}^{\mu\nu}(y)\ket*{\chi_{j,\tau}^{\mu\nu}}=\tau\ket*{\chi_{j,\tau}^{\mu\nu}}$ and $\hat{T}_S(x)\ket*{\chi_{j,\tau}^{\mu\nu}}=\tau_S(x)\ket*{\chi_{j,\tau}^{\mu\nu}}$ for all $x\in S_n\cap S$, \nomrefpage}%
  is the subset of $\{\ket*{\eta_j^{\mu\nu}}:j\in\mathbb{N}\}$ such that $\hat{T}^{\mu\nu}(y)\ket*{\chi_{j,\tau}^{\mu\nu}}=\tau\ket*{\chi_{j,\tau}^{\mu\nu}}$ and $\hat{T}_S(x)\ket*{\chi_{j,\tau}^{\mu\nu}}=\tau_S(x)\ket*{\chi_{j,\tau}^{\mu\nu}}$ for all $x\in S_n\cap S$, then 
\begin{equation}\label{pqtauri}
P(q(\tau)\, \& \,r_n)=\sum_j \abs{ \ip{\chi_{j,\tau}^{\mu\nu}}{\Psi_n}}^2=\ev*{\pi_{n,\tau}^{\mu\nu}}{\Psi_n}.\protect\footnotemark
\end{equation}
\footnotetext{The proof of this is very similar to the proof given in footnote \footreference{priproof}.}But %
\nomenclature{$\pi_\tau^{\mu\nu}$}{The projection $\pi_\tau^{\mu\nu}=\sum_j\dyad*{\eta_{j,\tau}^{\mu\nu}}$, \nomrefpage}%
 if we define $\pi_\tau^{\mu\nu}=\sum_j\dyad*{\eta_{j,\tau}^{\mu\nu}}$ where $\{\ket*{\eta_{j,\tau}^{\mu\nu}}:j\in\mathbb{N}\}$ is %
 \nomenclature{$\{\ket*{\eta_{j,\tau}^{\mu\nu}}:j\in\mathbb{N}\}$}{The subset of  $\{\ket*{\eta_j^{\mu\nu}}:j\in\mathbb{N}\}$ with $\hat{T}^{\mu\nu}(y)\ket*{\eta_{j,\tau}^{\mu\nu}}=\tau\ket*{\eta_{j,\tau}^{\mu\nu}}$, \nomrefpage}%
  the subset of  $\{\ket*{\eta_j^{\mu\nu}}:j\in\mathbb{N}\}$ with $\hat{T}^{\mu\nu}(y)\ket*{\eta_{j,\tau}^{\mu\nu}}=\tau\ket*{\eta_{j,\tau}^{\mu\nu}}$, then we also have  $\pi_{n,\tau}^{\mu\nu}=\pi_n\pi_\tau^{\mu\nu}$.\footnote{To see why this is, 
we first show that $\pi_n=\sum_j\dyad*{h_{n,j}^{\mu\nu}}$ where $\{\ket*{h_{n,j}^{\mu\nu}}:j\in\mathbb{N}\}$ is the subset of $\{\ket*{\eta_j^{\mu\nu}}:j\in\mathbb{N}\}$  for which $\ket*{h_{n,j}^{\mu\nu}}\in H_{S_n,\tau_S}$. 
Note that $\pi_n\ket*{h_{n,j}^{\mu\nu}}=\ket*{h_{n,j}^{\mu\nu}}$    since $\{\ket*{\xi_j}:j\in\mathbb{N}\}$ is a basis for $H_{S_n,\tau_S}$ and $\ket*{h_{n,j}^{\mu\nu}}\in H_{S_n,\tau_S}$. 
Therefore, $\pi_n\pi_{n,h}^{\mu\nu}=\pi_{n,h}^{\mu\nu}$  where  $\pi_{n,h}^{\mu\nu}=\sum_j\dyad*{h_{n,j}^{\mu\nu}}$. 
But  $\pi_{n,h}^{\mu\nu}\ket*{\xi_j}=\ket*{\xi_j}$ since $\{\ket*{h_{n,j}^{\mu\nu}}:j\in\mathbb{N}\}$ is a basis for $H_{S_n,\tau_S}$ and $\ket*{\xi_j}\in H_{S_n,\tau_S}$. 
Therefore, $\pi_{n,h}^{\mu\nu}\pi_n=\pi_n.$ But $\pi_{n,h}^{\mu\nu}\pi_n= \pi_n\pi_{n,h}^{\mu\nu}$ since $\pi_n$ and $\pi_{n,h}^{\mu\nu}$ are Hermitian/self-adjoint. Hence, $\pi_n= \pi_{n,h}^{\mu\nu}$. Now the summands of $\pi_n\pi_\tau^{\mu\nu}$ are only going to consist of those $\dyad*{\eta_j^{\mu\nu}}$ for which $\hat{T}^{\mu\nu}(y)\ket*{\eta_j^{\mu\nu}}=\tau\ket*{\eta_j^{\mu\nu}}$ and for which $\hat{T}_S(x)\ket*{\eta_j^{\mu\nu}}=\tau_S(x)\ket*{\eta_j^{\mu\nu}}$ for all $x\in S_n\cap S$, and these are just the $\dyad*{\chi_{j,\tau}^{\mu\nu}}$ which are the summands of  $\pi_{n,\tau}^{\mu\nu}$. Hence,  $\pi_n\pi_\tau^{\mu\nu}=\pi_{n,\tau}^{\mu\nu}.$} 
  Hence,
\begin{equation}\label{pqtauri2}
P(q(\tau)\, \& \, r_n)=\ev*{\pi_n\pi_\tau^{\mu\nu}}{\Psi_n}.
\end{equation}
But clearly $\hat{T}^{\mu\nu}(y)=\sum_\tau \tau \pi_\tau^{\mu\nu}.$ Therefore, combining (\ref{beable1}), (\ref{Prn}), and (\ref{pqtauri2}), we have 
\begin{equation}\label{kentconsistency0}
\ev*{T^{\mu\nu}(y)}_{\tau_S}=\lim_{n\rightarrow\infty}\frac{\sum_\tau \ev*{\pi_n\pi_\tau^{\mu\nu}}{\Psi_n}\tau}{\ev*{\pi_n}{\Psi_n}}=\lim_{n\rightarrow\infty}\frac{\ev*{\pi_n\hat{T}^{\mu\nu}(y)}{\Psi_n}}{\ev*{\pi_n}{\Psi_n}}.
\end{equation}\nomenclature{$\ev*{T^{\mu\nu}(y)}_{\tau_S}$}{Kent's proposed beable, see equation (\ref{beable1}), p. \pageref{beable1}, see also equation (\ref{kentconsistency0}), p. \pageref{kentconsistency0}.}We are now in a position to show that Kent's theory is consistent with standard quantum theory. First let us consider what we need to show. 

In the Bohmian interpretation, consistency with standard quantum theory requires that if one averages the expectation values of an observable over the hidden variables (i.e. the positions and the momenta of all the particles) then one obtains the expectation value of the observable given by standard quantum theory as indicated in equation (\ref{bohmconsistency}).\footnote{For a physical system $\mathcal{S}$ whose quantum wave function is $\psi$ and in which the particles have precise but unknown locations given by a probability distribution $\rho$, then the Bohmian interpretation will give the same predictions as standard quantum theory when averaged over the particle positions so long as $\rho$ is given by $|\psi|^2$. The probability distribution $\rho=|\psi|^2$ is called the \emph{quantum equilibrium distribution}, and we say that a system $\mathcal{S}$  with quantum wave function $\psi$ is in \emph{quantum equilibrium} when the probability distribution for the particles of $\mathcal{S}$  is given by the quantum equilibrium distribution $|\psi|^2$. The question of why the Bohmian interpretation should give the same predictions as standard quantum theory is therefore a question of why we should expect a system to be in quantum equilibrium. An answer to this question can be found in \cite[34--51]{DurrDetlef2013Qpwq}. The authors provide an answer by first supposing there is a universal wave function $\Psi_0$ defined at an initial time, and that the particles $Q$ of the universe are distributed according to the distribution $|\Psi_0|^2$ at the initial time. Then given the Bohmian dynamics of the universal wave function $\Psi_t$ (where $t$ is time) and the relationship between the velocities of the particles and $\Psi_t$ at time $t$, the authors show that the distribution of $Q$ at any time $t$ is given by $|\Psi_t|^2$. The authors then consider an ensemble of systems each of which is described by a wave function $\psi$ (suitably translated to where each system is), and they show that given the distribution of $Q$, the probability for the particles in each system of the ensemble (in a frequentist sense) is given by $|\psi|^2$, hence establishing that a system with wave function $\psi$ will be in quantum equilibrium. Note that the assumption that the particles $Q$ of the universe are initially distributed according to the distribution $|\Psi_0|^2$ is not begging the question, for we can first assume there is an initial probability distribution for $Q$, and from this we can define a wave function $\Psi_0$ so that $|\Psi_0|^2$ gives the probability distribution of $Q$ at the initial time. There still remains the question of where the initial distribution of $Q$ comes from, but we could probably speculate about this endlessly in a way that is analogous to the speculation regarding $T_S$ on pages \pageref{selectionmeaning} to \pageref{selectionmeaningend}. But the fact that we could speculate about this is not an obvious problem for the Bohmian interpretation.} 

Now in Kent's theory, the hidden variables on which his beables $\ev*{T^{\mu\nu}(y)}_{\tau_S}$ depend are the values $\tau_S(x)$ of $T_S(x)$ for $x\in S^1(y)\cap S$. The operator $\pi_n$ in equation (\ref{kentconsistency0}) in the limit as $n\rightarrow\infty$ encapsulates this hidden information. To remind ourselves of $\pi_n$'s dependency on $\tau_S$ restricted to $S_n\cap S$, we will now write $\pi_n(\tau_{S_n\cap S})$ for %
\nomenclature{$\tau_{S_n\cap S}$}{The function $\tau_S$ restricted to $S_n\cap S$, \nomrefpage}%
 $\pi_n$ where $\tau_{S_n\cap S}$ is %
\nomenclature{$\pi_n(\tau_{S_n\cap S})$}{The projection $\pi_n$ where $\tau_{S_n\cap S}$ is the function $\tau_S$ restricted to $S_n\cap S$, \nomrefpage}%
the function $\tau_S$ restricted to $S_n\cap S$. Likewise, we will write  $r_n(\tau_{S_n\cap S})$ for  %
\nomenclature{$r_n(\tau_{S_n\cap S})$}{The statement $r_n$  that $T_S(x)=\tau_S(x)$ for all $x\in S_n(y)\cap S$,  \nomrefpage}%
$r_n$, the statement that $T_S(x)=\tau_S(x)$ for all $x\in S_n(y)\cap S$. If we let $j$ index all possible functions $\tau_{S_n\cap S, j}$ taking real values on $S_n\cap S$,\footnote{Recall that we are implicitly using an approximation scheme described by equation (\ref{tauapproxformula}), so we are really considering functions on the cells of a mesh over $S_n\cap S $ taking values from a finite pool of possible values. Also see footnote \footreference{snapprox}. This is why we can use an index $j$ to index all the $\tau_{S_n\cap S, j}.$} then the analogue of (\ref{bohmconsistency}) requires us to show that 
\begin{equation}\label{kentconsistency}
\ev*{\hat{\bm{T}}^{\mu\nu}(y)}{\Psi_0}=\lim_{n\rightarrow\infty}\sum_{j}P\big(r_n(\tau_{S_n\cap S, j})\big)\ev*{T^{\mu\nu}(y)}_{\tau_{S_n\cap S, j}}
\end{equation}
for all $y$ lying between $S_0$ and $S$, where the left-hand side of (\ref{kentconsistency}) is just the expectation value of $\hat{\bm{T}}^{\mu\nu}(y)$ in the Heisenberg picture as predicted by standard quantum theory.\footnote{Where we are using Schwinger's bold typeface convention as described in footnote \footreference{boldref}.} Equation (\ref{kentconsistency}) is sufficient to establish consistency with standard quantum theory because ultimately, all observables are going to be reducible to expressions dependent on $\hat{T}^{\mu\nu}(y)$, since once we know what to expect for $\hat{T}^{\mu\nu}(y)$, we will know what to expect for the energy and momentum densities for all measuring apparatus readouts etc., and hence what to expect for all measurement outcomes. But from (\ref{Prn}) and (\ref{kentconsistency0}), we have 
\begin{equation}\label{kentconsistency1}
\lim_{n\rightarrow\infty}\sum_jP\big(r_n(\tau_{S_n\cap S, j})\big)\ev*{T^{\mu\nu}(y)}_{\tau_{S_n\cap S, j}}=\lim_{n\rightarrow\infty}\sum_{j}{\ev*{\pi_{n}(\tau_{S_n\cap S, j})\hat{T}^{\mu\nu}(y)}{\Psi_n}}
\end{equation}
Since there is an orthonormal basis $\{\ket*{\eta_j^{\mu\nu}}:j\in\mathbb{N}\}$ of $H_{S_n}$ consisting of simultaneous $\hat{T}_S(x)$-eigenstates so that $\hat{T}_S(x)\ket*{\eta_j^{\mu\nu}}=\tau_{S_n\cap S, j}(x)\ket*{\eta_j^{\mu\nu}}$ for all $x\in S_n\cap S$, it follows that $\sum_j \pi_{n}(\tau_{S_n\cap S, j})=I$, where $I$ is the identity operator on $H_n$. Combining this with (\ref{kentconsistency1}) we get
\begin{equation}\label{kentconsistency2}
	\lim_{n\rightarrow\infty}\sum_jP\big(r_n(\tau_{S_n\cap S, j})\big)\ev*{T^{\mu\nu}(y)}_{\tau_{S_n\cap S, j}}=\lim_{n\rightarrow\infty}\ev*{\hat{T}^{\mu\nu}(y)}{\Psi_n}
	\end{equation}
In the notation of equation (\ref{schwingerformula}), we have $\ket*{\Psi_n}=\ket*{\Psi[S_n]}$, %
\nomenclature{$\ket*{\Psi_n}$}{The state $\ket*{\Psi_n}=\ket*{\Psi[S_n]}$ on $S_n$, \nomrefpage}%
and according to (\ref{schwingerformula}) the expectation value $\ev*{\hat{T}^{\mu\nu}(y)}{\Psi[S_n]}$ will be independent of the hypersurface $S_n$ so long as it contains $y$, and the result will be equal to the expectation value $\ev*{\hat{\bm{T}}^{\mu\nu}(y)}{\Psi_0}$ in the Heisenberg picture which is the value that is predicted by standard quantum theory. Therefore, equation (\ref{kentconsistency}) follows from (\ref{schwingerformula}) and (\ref{kentconsistency2}) which is what we were aiming to show.

